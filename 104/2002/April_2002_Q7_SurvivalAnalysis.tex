\documentclass[a4paper,12pt]{article}

%%%%%%%%%%%%%%%%%%%%%%%%%%%%%%%%%%%%%%%%%%%%%%%%%%%%%%%%%%%%%%%%%%%%%%%%%%%%%%%%%%%%%%%%%%%%%%%%%%%%%%%%%%%%%%%%%%%%%%%%%%%%%%%%%%%%%%%%%%%%%%%%%%%%%%%%%%%%%%%%%%%%%%%%%%%%%%%%%%%%%%%%%%%%%%%%%%%%%%%%%%%%%%%%%%%%%%%%%%%%%%%%%%%%%%%%%%%%%%%%%%%%%%%%%%%%

\usepackage{eurosym}
\usepackage{vmargin}
\usepackage{amsmath}
\usepackage{graphics}
\usepackage{epsfig}
\usepackage{enumerate}
\usepackage{multicol}
\usepackage{subfigure}
\usepackage{fancyhdr}
\usepackage{listings}
\usepackage{framed}
\usepackage{graphicx}
\usepackage{amsmath}
\usepackage{chngpage}

%\usepackage{bigints}
\usepackage{vmargin}

% left top textwidth textheight headheight

% headsep footheight footskip

\setmargins{2.0cm}{2.5cm}{16 cm}{22cm}{0.5cm}{0cm}{1cm}{1cm}

\renewcommand{\baselinestretch}{1.3}

\setcounter{MaxMatrixCols}{10}

\begin{document}
\begin{enumerate} 
 
104 A2002—47
An investigation into the risk factors associated with mortality from lung cancer
among men was undertaken. The purpose of the investigation was to establish whether a new treatment was effective in prolonging survival. Two groups of patients were identified. 

One group was given the “new” treatment and the other was given the “existing” treatment. Other factors taken into consideration were the patients’general state of health at time of diagnosis (recorded as “able to care for self” or “unable to care for self”), and the type of tumour (recorded as “large”, “squamous”, “small” or “adeno”).

A Cox proportional hazards model of the hazard of death was estimated. The table below shows an extract from the results.

Covariate
General state of health at time of diagnosis
Able to care for self
Unable to care for self
Treatment
New
Existing
Type of tumour
Large
Squamous
Small
Adeno
Parameter Standard error
- 0.60
0.00 0.05
0.25
0.00 0.25
0.00
- 0.40
0.45
0.75
0.28
0.26
0.28

%%%%%%%%%%%%%%%%%%%%%%%%%%%%%%%%%%%%%%%%%%%%%%%%%%%%
\begin{enumerate}[(a)]
\item 
(i) Defining all the terms you use, write down a general expression for the Cox
proportional hazards model in terms of a set of covariates, their associated
parameters and a baseline hazard function.

\item 
(ii) In the context of the investigation described above, state the class of men to
which the baseline hazard refers.

\item 
(iii) Compare the new treatment with the previous one. Does it improve the
chances of survival, make them worse, or is it not possible to say? Justify
your answer.

\item 
(iv) Calculate the proportion by which the risk of death for men with “adeno” type
tumours who were “able to care for themselves” at the time of diagnosis is
greater than that for men with “large” type tumours who were “unable to care
for themselves” at the time of diagnosis.
\end{enumerate}

%%%%%%%%%%%%%%%%%%%%%%%%%%%%%%%%%%%%%%%%%%%%%%%%%%%%%%%%%%%%%%%%%%%%%%%%%%%%5
\newpage
\large


7
(i) 

\textbf { part (a) }

\[h(x, t) = h 0 (t)exp( b 1 x 1 + b 2 x 2 + ... + b k x k )\]
where $h(x, t)$ is the hazard at duration t, $h 0 (t)$ is some unspecified baseline
hazard, x 1 ... x k are covariates and $b_1, \ldots ... b_k$ are their associated parameters.

%%%%%%%%%%%%%%%%%%%%%%%%%%%%%%%%%%%%%%%%%%%%%%5
\newpage
\large

(ii) Men who were “unable to care for themselves” at the time of diagnosis, who
were given the “existing” treatment, and whose tumours were of the “large”
type.

%%%%%%%%%%%%%%%%%%%%%%%%%%%%%%%%%%%%%%%%%%%%%%5
\newpage
\large

(iii)

\begin{itemize}

\item The value of the parameter associated with the new treatment is 0.25. This
implies that the ratio of the hazards of death for two otherwise identical
patients, one of whom is given the new treatment and the other the existing
treatment is exp(0.25) = 1.28. 
\item Thus the new treatment appears to increase the
risk of death.
\item However, the standard error associated with the parameter is 0.25. The
approximate 95\% confidence interval is therefore $0.25 \pm 1.96(0.25) = ( - 0.24,
0.74)$, which includes 0. 
\item Therefore, the value of the parameter is not significantly different from zero at the 5\% level, so it is not possible to say
with the available data whether the new treatment affects the risk of death.
\end{itemize}
%%%%%%%%%%%%%%%%%%%%%%%%%%%%%%%%%%%%%%%%%%%%%%5
\newpage
\large


(iv)
\begin{itemize}
\item 
The hazard for men with “adeno” type tumours who were “able to care for
themselves” at the time of diagnosis is h 0 (t)exp( - 0.60 + 0.75).
\item The hazard for men with “large” type tumours who were “unable to care for
themselves” at the time of diagnosis is h 0 (t), since this is the baseline category.
\item The ratio is thus
\[h 0 ( t ) exp( - 0.60 + 0.75)
= exp(0.15) = 1.16
h 0 ( t )
\]
so the risk of death is 16% greater for men with “adeno” type tumours who
were “able to care for themselves” at the time of diagnosis.
\end{itemize}
\end{document}
