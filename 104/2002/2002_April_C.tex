\documentclass[a4paper,12pt]{article}

%%%%%%%%%%%%%%%%%%%%%%%%%%%%%%%%%%%%%%%%%%%%%%%%%%%%%%%%%%%%%%%%%%%%%%%%%%%%%%%%%%%%%%%%%%%%%%%%%%%%%%%%%%%%%%%%%%%%%%%%%%%%%%%%%%%%%%%%%%%%%%%%%%%%%%%%%%%%%%%%%%%%%%%%%%%%%%%%%%%%%%%%%%%%%%%%%%%%%%%%%%%%%%%%%%%%%%%%%%%%%%%%%%%%%%%%%%%%%%%%%%%%%%%%%%%%

\usepackage{eurosym}
\usepackage{vmargin}
\usepackage{amsmath}
\usepackage{graphics}
\usepackage{epsfig}
\usepackage{enumerate}
\usepackage{multicol}
\usepackage{subfigure}
\usepackage{fancyhdr}
\usepackage{listings}
\usepackage{framed}
\usepackage{graphicx}
\usepackage{amsmath}
\usepackage{chngpage}

%\usepackage{bigints}
\usepackage{vmargin}

% left top textwidth textheight headheight

% headsep footheight footskip

\setmargins{2.0cm}{2.5cm}{16 cm}{22cm}{0.5cm}{0cm}{1cm}{1cm}

\renewcommand{\baselinestretch}{1.3}

\setcounter{MaxMatrixCols}{10}

\begin{document}
\begin{enumerate}

PLEASE TURN OVER5
A life insurance company issued a 10-year temporary immediate annuity of £5,000
per annum to a life aged 44 exact on 1 January 1990. Annuity payments were
deferred for 10 years, so that the first payment was made on 1 January 2001.
Premiums were payable annually in advance until the end of the deferred period or
earlier death.
\begin{enumerate}[(i)]
\item (i) State the Principle of Equivalence.
\item (ii) Calculate the level annual premium payable.
\item (iii) Calculate the Expected Death Strain for the calendar year 2004.
[4]
\item 
(iv) If the annuitant died during the calendar year 2004, calculate the Actual Death
Strain for this calendar year.
[1]

\end{enumerate}

Basis: Mortality: A1967–70 Ultimate Males
Interest: 5% per annum interest effective throughout
Expenses are ignored

%%%%%%%%%%%%%%%%%%%%%%%%%%%%%%%%%%%%%%%%%%%%%%%%%%%%%%%%%%%%%%%%%%%%%%%%%%%%%%%

%% Question 6

A life insurance company has investigated the recent mortality experience of its male
annuitants. The following is an extract from the results.
Age
x Exposed
to risk
E x Observed
deaths
q x
70
71
72
73
74
75 600
750
725
650
700
675 23
31
33
29
35
39
\begin{enumerate}[(i)]
\item 
(i) Use the Chi-squared goodness of fit test to compare this experience with the
a(55) Ultimate Mortality Table for Male Annuitants. This was the mortality
basis used to determine the price of these annuities. State the null hypothesis
you are testing and comment on the results of your test.
\item
(ii) Comment on the financial impact on the company if it continues to sell these
annuities with an unchanged mortality basis.
\item
(iii) State how your test in (i) would be varied if you were testing graduated
o
mortality rates for adherence to the above data. The graduated rates, q x , were
determined by fitting the relationship
o
q x = a + bq x s
where q x s are rates from the a(55) Ultimate Mortality Table for Male
Annuitants. No further calculations are required.
\end{enumerate}
%%%%%%%%%%%%%%%%%%%%%%%%%%%%%%%%%%%%%%%%%%%
5
(i)
The Principle of Equivalence says that
Expected Present Value
of Income to a Policy
(ii)
o
Expected Present Value
of Outgo from a Policy
Value of premiums
æ
ö
l
= Pa && 44:10 = P  ́ ç a && 44 - v 10  ́ 54  ́ a && 54 ÷
l 44
è
ø
31,926.430
æ
ö
= P ç 15.859 - (1.05 - 10 )  ́
 ́ 13.357 ÷
33,309.271
è
ø
= P(15.859 - 0.5884265  ́ 13.357)
= 7.99939P
Value of annuity
= 5,000  ́
D 54
 ́ a
D 44 54:10
æ
l ö
= 5,000 ç v 10  ́ 54 ÷ {(1.05 - 1 )  ́ p 54  ́ a && 55:10 }
l 44 ø
è
= 5,000  ́ 0.5884265  ́ {(1.05 -1 ) (1 - 0.00755572)  ́ 7.741}
%%--- Page 6Subject 104 (Survival Models) — April 2002 — Examiners’ Report
= 5,000  ́ 0.5884265  ́ 7.316677
= 21,526.6341
Equation of value:
7.99939P = 21,526.6341
Þ
(iii)
P = £2,691.03 per annum
Policy value at end of 2004
15 V
= 5, 000 a 59:5 - P  ́ 0
(after deferred period, so no further premiums)
= 5,000 {(1.05 -1 )  ́ p 59  ́ a && 60:5 }
= 5,000 {(1.05 -1 ) (1 - 0.01299373)  ́ 4.409}
= 5,000  ́ 4.1445
= 20,722.4316
Expected Death Strain for year 2004
= - q 58 (5,000 + 15 V)
= - 0.01168566 (5,000 + 20,722.4316)
= - £300.58 i.e. a release of reserves
(iv)
If the annuitant died, the actual death strain is
= - (5,000 + 15 V)
= - £25,722.43
Other methods of evaluation are possible particularly in (ii) and (iii).
%%--- Page 7Subject 104 (Survival Models) — April 2002 — Examiners’ Report
\newpage
%%%%%%%%%%%%%%%%%%%%%%%%%%%%%%%%%%%%%%%%%
6 (i)
Age
x Exposed Observed q x (from
to Risk E x Deaths q x tables)
70
71
72
73
74
75
H 0 : the true underlying mortality of the annuitants is that of the standard table.
600
750
725
650
700
675
23
31
33
29
35
39
0.03776
0.04170
0.04602
0.05075
0.05595
0.06164
E x q x
q x - E x q x
E x q x (1 - q x )
0.073676
- 0.050232
- 0.064608
- 0.712583
- 0.684965
- 0.417228
Total
z x =
22.656
31.275
33.3645
32.9875
39.165
41.607
z x 2
0.00543
0.00252
0.00417
0.50778
0.46918
0.17408
1.1632
Degrees of freedom for c 2 test = number of ages = 6.
One tailed test as large values of
å z x 2
indicate excessive deviations.
2
c 6,0.95
= 12.59
å z x 2 < 12.59 , so, there is no evidence to reject H 0 .
All deviations but one are negative, which could indicate the true mortality is lighter than a(55). This is not detected by the Chi-squared test as the statistic is based on squared deviations.

(ii) If the true mortality is lighter than a(55), the company will charge inadequate premiums and will suffer a loss on the policies.
(iii) For testing adherence to data, the test statistic is unchanged, but the number of degrees of freedom is reduced.
In fitting the relationship two parameters have been estimated so the number of degrees of freedom will be reduced from 6 to 4, with a further deduction of 2 or 3 degrees of freedom for the constraints imposed by the choice of the
standard table.
A solution to (i) using E x q x as an estimate of the variance is also acceptable,
provided the assumption 1 - q x ¦ 1 is stated. Candidates who modified their
answers because the data was an extract from the whole experience received credit.
\end{document}
