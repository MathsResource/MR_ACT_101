\documentclass[a4paper,12pt]{article}

%%%%%%%%%%%%%%%%%%%%%%%%%%%%%%%%%%%%%%%%%%%%%%%%%%%%%%%%%%%%%%%%%%%%%%%%%%%%%%%%%%%%%%%%%%%%%%%%%%%%%%%%%%%%%%%%%%%%%%%%%%%%%%%%%%%%%%%%%%%%%%%%%%%%%%%%%%%%%%%%%%%%%%%%%%%%%%%%%%%%%%%%%%%%%%%%%%%%%%%%%%%%%%%%%%%%%%%%%%%%%%%%%%%%%%%%%%%%%%%%%%%%%%%%%%%%

\usepackage{eurosym}
\usepackage{vmargin}
\usepackage{amsmath}
\usepackage{graphics}
\usepackage{epsfig}
\usepackage{enumerate}
\usepackage{multicol}
\usepackage{subfigure}
\usepackage{fancyhdr}
\usepackage{listings}
\usepackage{framed}
\usepackage{graphicx}
\usepackage{amsmath}
\usepackage{chngpage}

%\usepackage{bigints}
\usepackage{vmargin}

% left top textwidth textheight headheight

% headsep footheight footskip

\setmargins{2.0cm}{2.5cm}{16 cm}{22cm}{0.5cm}{0cm}{1cm}{1cm}

\renewcommand{\baselinestretch}{1.3}

\setcounter{MaxMatrixCols}{10}

\begin{document}
\begin{enumerate}
In a clinical trial, 50 patients are observed for two years following treatment with a
new drug. The following data show the period in complete months from the initial
treatment to the end of observation for those patients who died or withdrew from the
trial before the end of the two year period.
Deaths
6, 6, 12, 15, 20, 20, 23
Withdrawals 1, 3, 5, 8, 10, 18
10
(i) Calculate the Nelson-Aalen estimate of the integrated hazard function, L t . [6]
(ii) Hence, or otherwise, estimate the probability of a patient surviving for at least
18 months after the initial treatment.
[2]
[Total 8]
(i) Explain the rationale behind the use of the Poisson distribution to model the
number of deaths among a group of lives. Include in your explanation a
discussion of why the Poisson Model is not always an exact model.
[4]
(ii) A group of N lives is observed for some finite period between the ages of x
and x + 1. Let n i be the observed waiting time for life i. Assuming a constant
force of mortality m between ages x and x + 1, derive the maximum likelihood
estimator m̂ of this constant force under the Poisson model.
[6]
(iii) Write down the expected value and the variance of m̂ .
(iv) (a)
Write down an approximate 95% confidence interval for the constant
force, m.
(b)
Explain how this confidence interval could be used to help in the
graphical graduation of a set of estimated mortality rates, m ˆ x (x = 50,
51, 52, ..., 98).
[6]
[Total 18]
104 S2002—4
[2]

%%%%%%%%%%%%%%%%%%%%%%%%%%%%%%%%%%%%%%%%%%%%%%%%%%%%%%%%%%%%%%%%%%%%%%%%%%%%%%%%%%%%%%%%%%%%%%%%%%%%%%%%%%%%%%%%%%
9
(i)
(ii)
Time t j c j d j n j d j /n j L t = S d j /n j
0 £ t < 6
6 £ t < 12
12 £ t < 15
15 £ t < 20
20 £ t < 23
t 3 23 3
2
0
1
0
0 0
2
1
1
2
1 50
47
43
42
40
38 0
2/47
1/43
1/42
2/40
1/38 0
0.04255
0.06581
0.08962
0.13962
0.16593
F(t) = 1 - exp( -L t )
S(18) = 1 - F(18) = exp( -L 18 ) = exp( - 0.08962) = 0.91428
10
(i)
The Poisson distribution is used to model the number of “rare” events
occurring during some period of time.
Since death is a rare event, the Poisson distribution can thus be used to model
the number of deaths among a group of lives, given the time spent exposed to
risk and assuming that the force of mortality for lives aged x is constant over
(x, x + 1) and over time during the period of observation/investigation.
The Poisson model is not always an “exact” model because, under some
observational plans, it allows a non-zero probability of more than N deaths
among N lives. However, observational plans can, at least in theory, be
adjusted to overcome this problem.
(ii)
Let d be the total number of deaths we observe among the N individuals. This
value d is a sample value of a random variable D .
The maximum likelihood estimate of the force of mortality, m̂ , is the value
which maximises the probability that D = d.
Page 9Subject 104 (Survival Models) — September 2002 — Examiners’ Report
N
The total waiting time is equal to å v i . The Poisson model assumes that D
i = 1
N
has a Poisson distribution with parameter m å v i . The Poisson likelihood is
i = 1
N
e
-m å v i
i = 1
L = Pr[ D = d] =
d
æ N ö
ç ç m å v i ÷ ÷
è i = 1 ø .
d !
The maximum likelihood estimate of m maximises this. To find it, we
differentiate ln L with respect to m and set the derivative equal to zero. Thus
N
æ N ö
ln L = -m å v i + d ln ç m å v i ÷ - ln d !
ç
÷
i = 1
è i = 1 ø
and
N
N
d ln L
= - å v i +
d m
i = 1
d å v 1
i = 1
N
m å v i
N
= - å v i +
i = 1
d
.
m
i = 1
Setting this equal to zero produces the maximum likelihood estimate of m ,
d
m ˆ =
.
N
å v i
i = 1
2
Since
(iii)
d ln L
d m 2
E [ m ˆ ] = m .
Var[ m ˆ ] =
m
N
å v i
i = 1
Page 10
= -
.
d
m 2
is negative, we have a maximum.Subject 104 (Survival Models) — September 2002 — Examiners’ Report
(iv)
95% confidence interval for m̂ , estimate of constant force at age x
m̂ ± 1.96
m ˆ
N
å v i
i = 1
The forces would be estimated separately for each age classification x.
The estimates m ˆ x would be plotted against x.
A confidence band would be plotted around each estimate.
The end points of the confidence bands would be joined to form a “tunnel”.
This “tunnel” would be used as a guide in drawing the smooth curve to
produce the graphically graduated rates. The “tunnel” would be wider at ages
where there were few deaths (say in 50s) or little exposed to risk (say in the
90s).
We would expect the graduation curve to stay within the “tunnel” for 19 out of
20 ages on average, i.e. all but 2 to 4 of the ages 50 to 98 say.
