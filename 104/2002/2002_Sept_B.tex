\documentclass[a4paper,12pt]{article}

%%%%%%%%%%%%%%%%%%%%%%%%%%%%%%%%%%%%%%%%%%%%%%%%%%%%%%%%%%%%%%%%%%%%%%%%%%%%%%%%%%%%%%%%%%%%%%%%%%%%%%%%%%%%%%%%%%%%%%%%%%%%%%%%%%%%%%%%%%%%%%%%%%%%%%%%%%%%%%%%%%%%%%%%%%%%%%%%%%%%%%%%%%%%%%%%%%%%%%%%%%%%%%%%%%%%%%%%%%%%%%%%%%%%%%%%%%%%%%%%%%%%%%%%%%%%

\usepackage{eurosym}
\usepackage{vmargin}
\usepackage{amsmath}
\usepackage{graphics}
\usepackage{epsfig}
\usepackage{enumerate}
\usepackage{multicol}
\usepackage{subfigure}
\usepackage{fancyhdr}
\usepackage{listings}
\usepackage{framed}
\usepackage{graphicx}
\usepackage{amsmath}
\usepackage{chngpage}

%\usepackage{bigints}
\usepackage{vmargin}

% left top textwidth textheight headheight

% headsep footheight footskip

\setmargins{2.0cm}{2.5cm}{16 cm}{22cm}{0.5cm}{0cm}{1cm}{1cm}

\renewcommand{\baselinestretch}{1.3}

\setcounter{MaxMatrixCols}{10}

\begin{document}
\begin{enumerate}
3
The Australian National Life Table for males 1980–1982 was estimated using the
death registrations for males during the three calendar years 1980, 1981 and 1982
(184,197 deaths) and the male population enumerated at the census held on
30 June 1981 (7,416,090 lives).
A student has argued that with such a large number of lives and deaths the standard
errors of the estimated mortality rates will be very small and that graduation of the
rates is not necessary.
Explain why this conclusion is untrue.
4
[5]
(i) Explain the differences between random censoring and Type I censoring in the
context of an investigation into the mortality of life insurance policyholders.
Include in your explanation a statement of the circumstances in which the
censoring will be random, and the circumstances in which it will be Type I. [4]
(ii) Explain what is meant by non-informative censoring in the investigation in (i).
Describe a situation in which censoring might be informative in this
investigation.


%%%%%%%%%%%%%%%%%%%%%%%%%%%%
\newpge
3
Points to be made:
Yes, there are lots of deaths and many years of exposed to risk but these will be spread across about 110 years of age, so numbers at any particular age will be much
smaller.
The numbers will not be divided evenly across ages. At high ages (say > 80) there will be little exposed to risk and at younger ages (say < 30) there will be few deaths.
So there will be ages at which the standard errors are substantial, particularly when compared to the estimated value of the rates at these ages.
The rate for each age is estimated independently of the rates at all other ages. There is nothing in the estimation process that ensures that rates increase smoothly with age as
we would expect a priori.

If there is any irregularity in the published table that would cause difficulties when the table is used for financial calculations, e.g. state pension liabilities.
Page 4Subject 104 (Survival Models) — September 2002 — Examiners’ Report
4
(i)
Random censoring occurs when the time at which the ith life is censored is a random variable. The observation will be censored if the censoring time is
less than the (random) lifetime of the life. In an investigation into the mortality of life insurance policyholders, a life will be censored if either the
period of investigation ends before the life dies, or the life withdraws from the
investigation while still alive (perhaps because the policy lapses). Both these mechanisms will generate random censoring.
Type I censoring occurs if the censoring times are known in advance. In most investigations withdrawals do occur, and if it is not known in advance whether
(let alone when) a life will withdraw, then the censoring is not Type I.
If the period of investigation is known in advance and if there are no withdrawals from the investigation while still alive, the censoring will be
Type I.



\end{document}
