\documentclass[a4paper,12pt]{article}

%%%%%%%%%%%%%%%%%%%%%%%%%%%%%%%%%%%%%%%%%%%%%%%%%%%%%%%%%%%%%%%%%%%%%%%%%%%%%%%%%%%%%%%%%%%%%%%%%%%%%%%%%%%%%%%%%%%%%%%%%%%%%%%%%%%%%%%%%%%%%%%%%%%%%%%%%%%%%%%%%%%%%%%%%%%%%%%%%%%%%%%%%%%%%%%%%%%%%%%%%%%%%%%%%%%%%%%%%%%%%%%%%%%%%%%%%%%%%%%%%%%%%%%%%%%%

\usepackage{eurosym}
\usepackage{vmargin}
\usepackage{amsmath}
\usepackage{graphics}
\usepackage{epsfig}
\usepackage{enumerate}
\usepackage{multicol}
\usepackage{subfigure}
\usepackage{fancyhdr}
\usepackage{listings}
\usepackage{framed}
\usepackage{graphicx}
\usepackage{amsmath}
\usepackage{chngpage}

%\usepackage{bigints}
\usepackage{vmargin}

% left top textwidth textheight headheight

% headsep footheight footskip

\setmargins{2.0cm}{2.5cm}{16 cm}{22cm}{0.5cm}{0cm}{1cm}{1cm}

\renewcommand{\baselinestretch}{1.3}

\setcounter{MaxMatrixCols}{10}

\begin{document}7
(i) Derive a formula to estimate the constant force of mortality for lives aged x
nearest birthday. State any assumptions you make in deriving your formula.
[5]
(ii) Use your formula in (i) to calculate numerical estimates of m 41 and m 42 .
[2]
[Total 7]


%%%%%%%%%%%%%%%%%%%%%%%%%%%%%%%%%%
7
(i) Essential data is:
· date of birth (or date of x th birthday or exact age)
· multiple policy indicator
Either
· date of purchase of term assurance policy (*)
· date of policy lapse, date of policy expiry or date of death (*)
· if occurred between 1.1.95 and 31.12.98
Or
· Date of entry into investigation
· Date of exit from investigation
· Reason for exit
(ii) Data for all lives that had died during the Period of Investigation (1.1.95 to
31.12.98) would be tabulated by age last birthday, x at date of death. Counts
of the number of deaths at each age x, q x would be recorded for all x .
For each age x and for each life two dates would be calculated.
START DATE
The latest date of
1 January 1995
date of purchase of policy
date of xth birthday
END DATE
The earliest date of
31 December 1998
date of death or date of leaving (if any)
date of x + 1th birthday
Then calculate END DATE - START DATE (if this is > 0) and total these
values for all lives. Record this answer in years. This is the Central Exposed
to Risk at age, x, E x c .
Tabulate these values for all x.
Then: m ˆ x =
q x
is an estimate of the force of mortality at age x + 1⁄2, assuming
E x c
that the force is constant over the year of age x to x + 1.
ALTERNATIVELY
Tabulations could be produced for age nearest birthday, in which
START DATE must be amended to use “date of attaining x nearest birthday”,
and END DATE to use “date of attaining x + 1 nearest birthday”.
Page 7Subject 104 (Survival Models) — September 2002 — Examiners’ Report
Then: m ˆ x =
q x
is an estimate of the force of mortality at age x, assuming that
E x c
the force is constant over the year of age (x - 1⁄2, x + 1⁄2).
