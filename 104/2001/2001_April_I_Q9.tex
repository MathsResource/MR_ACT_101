\documentclass[a4paper,1pt]{article}

%%%%%%%%%%%%%%%%%%%%%%%%%%%%%%%%%%%%%%%%%%%%%%%%%%%%%%%%%%%%%%%%%%%%%%%%%%%%%%%%%%%%%%%%%%%%%%%%%%%%%%%%%%%%%%%%%%%%%%%%%%%%%%%%%%%%%%%%%%%%%%%%%%%%%%%%%%%%%%%%%%%%%%%%%%%%%%%%%%%%%%%%%%%%%%%%%%%%%%%%%%%%%%%%%%%%%%%%%%%%%%%%%%%%%%%%%%%%%%%%%%%%%%%%%%%%

\usepackage{eurosym}
\usepackage{vmargin}
\usepackage{amsmath}
\usepackage{graphics}
\usepackage{epsfig}
\usepackage{enumerate}
\usepackage{multicol}
\usepackage{subfigure}
\usepackage{fancyhdr}
\usepackage{listings}
\usepackage{framed}
\usepackage{graphicx}
\usepackage{amsmath}
\usepackage{chngpage}

%\usepackage{bigints}
\usepackage{vmargin}

% left top textwidth textheight headheight

% headsep footheight footskip

\setmargins{.0cm}{.5cm}{16 cm}{cm}{0.5cm}{0cm}{1cm}{1cm}

\renewcommand{\baselinestretch}{1.}

\setcounter{MaxMatrixCols}{10}

\begin{document}

%%---  Question 9
\item 
(a)
Show that at age x if 0 \leq  a < b \leq  1 then
b − a q x + a
(b)
p x
a p x
b
Hence, or otherwise, show that if deaths are uniformly distributed
over the year of age (x, x + 1) then
b − a q x + a
\item 
= 1 −
=
( b − a ) q x
1 − aq x

In a mortality investigation, the following data have been recorded for six
independent lives observed between exact age 70 and exact age 71.
a i the time in years after exact age 70 when the ith life came under
observation;
b i the time in years after exact age 70 when ith life was censored;
d i = 1 if the ith life died before x + b i ;
= 0 if the ith life survived to x + b i ;
t i if d i = 1, then x + t i is the age at which the ith life died.
(a)
Life i a i b i d i t i
1
2
3
4
5
6 0
0.3
0.5
0
0
0 1
0.9
1
0.4
0.9
1 0
0
1
0
1
1 –
−
0.9
–
0.7
0.8
Using the Binomial Model of Mortality write down the likelihood of
these observations.
If deaths are assumed to be uniformly distributed over (70, 71)
express this likelihood as a function of q 70 .
(b)
Using the Poisson Model of Mortality and assuming a constant
force of mortality, \mu 70 , over (70, 71) write down the likelihood of
these observations.
Calculate the maximum likelihood estimate of \mu 70 and hence

obtain the maximum likelihood estimate of q 70 .
[Total 14]
%%%%%%%%%%%%%%%%%%%%%%%%%%%%%%%%%%%%%%%%%%%%%%%%%%%%%%%%%%%%%%%%%%%%%%%%%5
9
(i)
(a)
b − a q x+a
= 1 − b − a p x+a
Now b p x = a p x b − a p x+a
p x
a p x
b
So b − a q x+a = 1 −
(b)
If deaths are uniformly distributed over (x, x + 1) then
t q x
(ii)
(a)
= t . q x
So t
p x = 1 − t . q x
Then b − a q x+a
= 1 −
1 − bq x
( b − a ) q x
=
1 − aq x
1 − aq x
The likelihood for each life is
Life i
Likelihood
1
p 70
2
p
0.6 70.3
3
q
0.5 70.5
4
0.4 p 70
5
0.9 q 70
6
q 70
And the total likelihood is the product
(1 − q 70 ) (1 − 0.6 q 70.3 ) 0.5 q 70.5 (1 − 0.4 q 70 ) 0.9 q 70 q 70
Using (i)(b) we can write
æ
0.6 q 70 ö æ 0.5 q 70 ö
(1 − q 70 ) ç 1 −
֍
÷ (1 − 0.4 q 70 ) 0.9 q 70 q 70
1 − 0.3 q 70 ø è 1 − 0.5 q 70 ø
è
=
(b)
3
(1 − q 70 )(1 − 0.9 q 70 ) (1 − 0.4 q 70 ) 0.45 q 70
(1 − 0.3 q 70 )(1 − 0.5 q 70 )
The likelihood for each life is proportional to (assuming constant
force \mu 70 ) .
Life i
Likelihood
e
1 2
−\mu 70 − 0.6 \mu 70
e
3
e
− 0.4 \mu 70
4
\mu 70
e
− 0.4 \mu 70
And the total likelihood is the product
L ∝ e − 3.9 \mu 70 ( \mu 70 ) 3
Then
Page 12
∂ L
2
= − 3.9 e − 3.9 \mu 70 ( \mu 70 ) 3 + e − 3.9 \mu 70 3 \mu 70
∂\mu 70
5
e
− 0.7 \mu 70
6
\mu 70
e
− 0.8 \mu 70
\mu 70 %%%%%%%%%%%%%%%%%%%%%%%%%%%%%%%%%%%%% — April 2001 — Examiners’ Report
Then − 3.9 \hat{\mu} 70 + 3 = 0
\hat{\mu} 70 =
3
= 0.7692
3.9
The same result can be obtained using the log likelihood.
Then ML estimator of q 70 ; q̂ 70 = 1 − e − 0.7692 = 0.5366
