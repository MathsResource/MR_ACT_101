\documentclass[a4paper,1pt]{article}

%%%%%%%%%%%%%%%%%%%%%%%%%%%%%%%%%%%%%%%%%%%%%%%%%%%%%%%%%%%%%%%%%%%%%%%%%%%%%%%%%%%%%%%%%%%%%%%%%%%%%%%%%%%%%%%%%%%%%%%%%%%%%%%%%%%%%%%%%%%%%%%%%%%%%%%%%%%%%%%%%%%%%%%%%%%%%%%%%%%%%%%%%%%%%%%%%%%%%%%%%%%%%%%%%%%%%%%%%%%%%%%%%%%%%%%%%%%%%%%%%%%%%%%%%%%%

\usepackage{eurosym}
\usepackage{vmargin}
\usepackage{amsmath}
\usepackage{graphics}
\usepackage{epsfig}
\usepackage{enumerate}
\usepackage{multicol}
\usepackage{subfigure}
\usepackage{fancyhdr}
\usepackage{listings}
\usepackage{framed}
\usepackage{graphicx}
\usepackage{amsmath}
\usepackage{chngpage}

%\usepackage{bigints}
\usepackage{vmargin}

% left top textwidth textheight headheight

% headsep footheight footskip

\setmargins{.0cm}{.5cm}{16 cm}{cm}{0.5cm}{0cm}{1cm}{1cm}

\renewcommand{\baselinestretch}{1.}

\setcounter{MaxMatrixCols}{10}

\begin{document}

%% ---- 104 A2001—610
The table gives the data for a small sample of employees in a factory. It shows
the time in months until the first absence from work. Observations marked +
show the time of leaving for those employees who left employment without being
absent from work.
Male employees
Female employees
6+
2+
11
4
13+
7
15
8+
16+
10+
19+
12+
20
17
21+
A Cox Proportional Hazards Model
\lambda(tx) = \lambda_0(t) exp(\beta x)
is to be fitted to these data where t is the time until the first absence from work,
\lambda_0(t) is the baseline hazard and x = 0 for males, = 1 for females.
\item 
Show that the partial log-likelihood for these data can be written
l(\beta) = 3\beta − 4 \log_{e}(1 + e \beta ) − 2 \log_{e}(2 + e \beta ) + c
where c is a constant that does not depend on \beta.

\item  Calculate the maximum partial likelihood estimate of \beta.

\item  Calculate the asymptotic standard error of this estimate.

\item  Test the hypothesis that female employees experience higher first absence
rates than male employees. Explain the steps in your argument and state
your conclusions.

[Total 17]
%% ---- 104 A2001—7

%%%%%%%%%%%%%%%%%%%%%%%%%%%%%%%%%%%%%%%%%%%%%%%%%%%%%%%%%%%%%%%%%%%%%5
10
(i)
Write times in rank order, label groups and label first absences from
work.
Determine cumulative probabilities and contributions to partial likelihood
from first absences.
2+ F e \beta 7 + 8e \beta
4 F e \beta 7 + 7e \beta
6+ M 1 7 + 6e \beta
7 F e \beta 6 + 6e \beta
8+ F e \beta 6 + 5e \beta
10+ F e \beta 6 + 4e \beta
11 M 1 6 + 3e \beta
12+ F e \beta 5 + 3e \beta
13+ M 1 5 + 2e \beta
15 M 1 4 + 2e \beta
16+ M 1 3 + 2e \beta
17 F e \beta 2 + 2e \beta
19+ M 1 2 + e \beta
20 M 1 1 + e \beta
21+ F e \beta e \beta
e \beta
7 + 7 e \beta
e \beta
6 + 6 e \beta
1
6 + 3 e \beta
1
4 + 2 e \beta
e \beta
2 + 2 e \beta
1
1 + e \beta
Page 13 %%%%%%%%%%%%%%%%%%%%%%%%%%%%%%%%%%%%% — April 2001 — Examiners’ Report
Partial Likelihood, L
e \beta
e \beta
1
1
e \beta
1
.
.
.
.
.
\beta
\beta
\beta
\beta
\beta
7(1 + e ) 6(1 + e ) 3(2 + e ) 2(2 + e ) 2(1 + e ) 1 + e \beta
=
e 3 \beta
84(1 + e \beta ) 4 6(2 + e \beta ) 2
So \log_{e}L = l (\beta) = 3\beta − 4 \log_{e}(1 + e \beta ) − 2 \log_{e}(2 + e \beta ) + c
(ii)
Then
∂ l
4 e \beta
2 e \beta
= 3 −
−
∂\beta
1 + e \beta 2 + e \beta
Let x = e \beta then x̂ is given by
3 −
4 x ˆ
2 x ˆ
= 0
−
1 + x ˆ 2 + x ˆ
3(1 + x ˆ )(2 + x ˆ ) − 4 x ˆ (2 + x ˆ ) − 2 x ˆ (1 + x ˆ ) = 0
6 + 9 x ˆ + 3 x ˆ 2 − 8 x ˆ − 4 x ˆ 2 − 2 x ˆ − 2 x ˆ 2 = 0
3 x ˆ 2 + x ˆ − 6 = 0
x̂ =
=
− 1 \pm  1 − 4 ( − 6.3)
6
− 1 \pm  73
6
Estimate must be +ve, so
73 − 1
= 1.25733
6
So \betâ = \log_{e}1.25733 = 0.2290
(iii)
Now
æ − 4 e \beta . e \beta + (1 + e \beta ) . 4 e \beta ö æ − 2 e \beta . e \beta + (2 + e \beta ) . 2 e \beta ö
∂ 2 l
=
−
ç
÷ − ç
÷
(1 + e \beta ) 2
(2 + e \beta ) 2
∂\beta 2
è
ø è
ø
æ 4 e \beta
4 e \beta ö
= − ç
+
÷
\beta 2
(2 + e \beta ) 2 ø
è (1 + e )
ˆ
At e \beta = 1.25733, this has value − (0.98700 + 0.47401) = −1.46101
Page 14 %%%%%%%%%%%%%%%%%%%%%%%%%%%%%%%%%%%%% — April 2001 — Examiners’ Report
So asymptotic standard error = +
1
1.46101
= 0.8273
(iv)
95\% Confidence Interval for \betâ is approximately
0.2290 \pm  2 × 0.8273
i.e.
(−1.43, 1.88)
This interval includes \beta = 0, so the model
\lambda(tx) = \lambda_0(t) exp{0 × x} = \lambda_0(t)
is compatible with the observed data i.e. same model applies to males and
females. There is no significant difference between rates.
OR
Using a one sided alternative hypothesis and testing
H 0 : \beta = 0 against H 1 : \beta > 0 we have:
Under the Null hypothesis
0.2290 − 0
= 0.28
0.8273
should be N(0, 1).
This value is not in the critical region (1.64, ∞) and so there is no reason to
reject the null hypothesis. Conclusions as above.
OR
A likelihood test can be used
− 2( l (0) − l ( \beta ˆ )) = − 2( − 4.9698 + 4.9315) = 0.0765
which is χ 1 2 if H 0 : \beta=0 is true. This value is not in the critical region so
there is no reason to reject H 0 .
Page 15
