\documentclass[a4paper,1pt]{article}

%%%%%%%%%%%%%%%%%%%%%%%%%%%%%%%%%%%%%%%%%%%%%%%%%%%%%%%%%%%%%%%%%%%%%%%%%%%%%%%%%%%%%%%%%%%%%%%%%%%%%%%%%%%%%%%%%%%%%%%%%%%%%%%%%%%%%%%%%%%%%%%%%%%%%%%%%%%%%%%%%%%%%%%%%%%%%%%%%%%%%%%%%%%%%%%%%%%%%%%%%%%%%%%%%%%%%%%%%%%%%%%%%%%%%%%%%%%%%%%%%%%%%%%%%%%%

\usepackage{eurosym}
\usepackage{vmargin}
\usepackage{amsmath}
\usepackage{graphics}
\usepackage{epsfig}
\usepackage{enumerate}
\usepackage{multicol}
\usepackage{subfigure}
\usepackage{fancyhdr}
\usepackage{listings}
\usepackage{framed}
\usepackage{graphicx}
\usepackage{amsmath}
\usepackage{chngpage}

%\usepackage{bigints}
\usepackage{vmargin}

% left top textwidth textheight headheight

% headsep footheight footskip

\setmargins{.0cm}{.5cm}{16 cm}{cm}{0.5cm}{0cm}{1cm}{1cm}

\renewcommand{\baselinestretch}{1.}

\setcounter{MaxMatrixCols}{10}

\begin{document}

Consider a “limited-premium” whole-life assurance, i.e. a whole-life assurance for
which premiums are payable until death but for at most m years. The sum
assured is \$1, payable at the end of the year of death, the age at entry is x, and
the premiums are payable annually in advance.
\item 
Give expressions, in terms of standard actuarial functions, for:
(a)
(b)
(c)
\item 
the net premium
the prospective net premium policy value, at (integer) time t > m
the retrospective net premium policy value, at (integer) time t > m

Hence show that, if all three of the expressions in \item  are calculated on the
same mortality and interest basis, the prospective and retrospective net
premium policy values are equal.

[Total 8]
%%%%%%%%%%%%%%%%%%%%%%%%%%%%
Page 4 %%%%%%%%%%%%%%%%%%%%%%%%%%%%%%%%%%%%% — April 2001 — Examiners’ Report
3
(i)
(iii)
(a) P = A x / a  x : m
(b) A x + t − 0
(c) ( Pa 
x : m
)
− A x 1 : t / v t t p x
Pa  x : m = A x = A x 1: t + v t t p x A x + t
(
Þ A x+t = Pa  x : m − A 1 x : t
)
v t t p x , as required.
A solution using commutation functions is also possible.
