\documentclass[a4paper,1pt]{article}

%%%%%%%%%%%%%%%%%%%%%%%%%%%%%%%%%%%%%%%%%%%%%%%%%%%%%%%%%%%%%%%%%%%%%%%%%%%%%%%%%%%%%%%%%%%%%%%%%%%%%%%%%%%%%%%%%%%%%%%%%%%%%%%%%%%%%%%%%%%%%%%%%%%%%%%%%%%%%%%%%%%%%%%%%%%%%%%%%%%%%%%%%%%%%%%%%%%%%%%%%%%%%%%%%%%%%%%%%%%%%%%%%%%%%%%%%%%%%%%%%%%%%%%%%%%%

\usepackage{eurosym}
\usepackage{vmargin}
\usepackage{amsmath}
\usepackage{graphics}
\usepackage{epsfig}
\usepackage{enumerate}
\usepackage{multicol}
\usepackage{subfigure}
\usepackage{fancyhdr}
\usepackage{listings}
\usepackage{framed}
\usepackage{graphicx}
\usepackage{amsmath}
\usepackage{chngpage}

%\usepackage{bigints}
\usepackage{vmargin}

% left top textwidth textheight headheight

% headsep footheight footskip

\setmargins{.0cm}{.5cm}{16 cm}{cm}{0.5cm}{0cm}{1cm}{1cm}

\renewcommand{\baselinestretch}{1.}

\setcounter{MaxMatrixCols}{10}

\begin{document}
A special endowment assurance with a term of n years is issued to lives aged x.
It is special because the death benefit at time t (where 0 < t < n) is restricted to
the net premium policy value at that time t V, payable immediately on death. The
net premium is P p.a., payable continuously. The survival benefit at time n is S.
\item  Write down Thiele’s differential equation for this policy.
\item  Show that the solution to the equation in \item  is t V = Ps t at the appropriate
rate of interest.
\item 

Suppose that x = 45, n = 20, the rate of interest at all times is 4% p.a., and
the survival benefit S is sufficient to purchase an annuity, payable
annually in arrear, of \$20,000 p.a. Find P.
Mortality basis: a(55) Ultimate Mortality Table (males).
3
%%---  Question 

[Total 9]
%%%%%%%%%%%%%%%%%%%%%%%%%%%%%%%%%%%%%%%%%%%%%%%%%%%%%%%%5
2
(i)
In this model the probabilities are equal because there is no way of
returning to state I once a life has left this state.
∂ ( t V )
= δ t V + P − \mu x + t ( t V − t V) = δ t V + P
∂ t
i.e. t V ′ − δ t V = P.
(ii)
To solve, multiply by e −δ t to get:
( e
−δ t
t V
) ′ = Pe
−δ t
,
e −δ t t V = (−P/δ) e −δ t + c
and 0 V = 0 Þ c = P/δ
Hence t V = e δ t (P/δ) (1 − e −δ t )
= P(e δ t − 1) / δ = Ps t i , with 1 + i = e δ .
(Alternatively verify directly that Ps t i , satisfies the equation and the
initial condition, 0 V = 0.)
(iii)
Ps 20 0.04 = S = 20,000a 65 = 20,000 × 9.790902
Þ P = 6,447.80
