\documentclass[a4paper,1pt]{article}

%%%%%%%%%%%%%%%%%%%%%%%%%%%%%%%%%%%%%%%%%%%%%%%%%%%%%%%%%%%%%%%%%%%%%%%%%%%%%%%%%%%%%%%%%%%%%%%%%%%%%%%%%%%%%%%%%%%%%%%%%%%%%%%%%%%%%%%%%%%%%%%%%%%%%%%%%%%%%%%%%%%%%%%%%%%%%%%%%%%%%%%%%%%%%%%%%%%%%%%%%%%%%%%%%%%%%%%%%%%%%%%%%%%%%%%%%%%%%%%%%%%%%%%%%%%%

\usepackage{eurosym}
\usepackage{vmargin}
\usepackage{amsmath}
\usepackage{graphics}
\usepackage{epsfig}
\usepackage{enumerate}
\usepackage{multicol}
\usepackage{subfigure}
\usepackage{fancyhdr}
\usepackage{listings}
\usepackage{framed}
\usepackage{graphicx}
\usepackage{amsmath}
\usepackage{chngpage}

%\usepackage{bigints}
\usepackage{vmargin}

% left top textwidth textheight headheight

% headsep footheight footskip

\setmargins{.0cm}{.5cm}{16 cm}{cm}{0.5cm}{0cm}{1cm}{1cm}

\renewcommand{\baselinestretch}{1.}

\setcounter{MaxMatrixCols}{10}

\begin{document}
%%---  Question 9
A life insurance company has carried out a mortality investigation. It followed a
sample of independent policyholders aged between 40 and 45 years.
Policyholders were followed from their 40 th birthday until either they died, or
they withdrew from the investigation while still alive, or they celebrated their
45 th birthday (whichever of these events occurred first).
\item  Describe the types of censoring that are present in this investigation.
\item  An extract from the data for 20 policyholders is shown in the table below.
Use these data to calculate the Kaplan-Meier estimate of the survival
function. Determine an approximate 95\% confidence interval for your
estimate.

\begin{center}
\begin{tabular}{ccc}
Person number	&	Last age at which person		&	Outcome	\\	
	&	was observed		&		\\	 \hline
		(years and months)					
1	&	40	6	&	Died	\\	 \hline
2	&	40	6	&	Withdrew	\\	 \hline
3	&	41	0	&	Died	\\	 \hline
4	&	41	0	&	Died	\\	 \hline
5	&	41	6	&	Withdrew	\\	 \hline
6	&	42	3	&	Died	\\	 \hline
7	&	42	3	&	Withdrew	\\	 \hline
8	&	42	3	&	Died	\\	 \hline
9	&	42	6	&	Withdrew	\\	 \hline
10	&	43	0	&	Withdrew	\\	 \hline
11	&	43	3	&	Died	\\	 \hline
12	&	43	3	&	Withdrew	\\	 \hline
13	&	44	3	&	Withdrew	\\	 \hline
14	&	44	6	&	Withdrew	\\	 \hline
15	&	44	9	&	Died	\\	 \hline
16	&	45	0	&	Survived	\\	 \hline
17	&	45	0	&	Survived	\\	 \hline
18	&	45	0	&	Survived	\\	 \hline
19	&	45	0	&	Survived	\\	 \hline
20	&	45	0	&	Survived	\\	 \hline
\end{tabular}
\end{center}
\item 
Plot clearly on a suitably labelled graph the Kaplan-Meier estimate of the
survival function and its associated 95\% confidence band.

[Total 15]
104 S2001—8
%%%%%%%%%%%%%%%%%%%%%%%%%%%%%%%%%%%%%%%%%%%%%%%%%%%%%%%%%
9
(i)
There will be right censoring of all the lives that survive to age 45 or
who withdraw before age 45.
There will be random censoring of all the lives that withdraw before
death or attaining age 45.
(ii)
Person Duration (Months)
1
2
3
4
5 6
6+
12
12
18+
6
7
8
9
10 27
27+
27
30+
36+
11
12
13
14
15 39
39+
51+
54+
57
16
17
18
19
20 60+
60+
60+
60+
60+
where + = censored
Page 15 %%%%%%%%%%%%%%%%%%%%%%%%%%%%%%%%%%%%% — 
So the times of death are 6 12 27 39 57 and initial risk set is 20.
(1) (2) (3) (4) (5)
j t j d j C j n j = n j - 1 -d j - 1 -C j - 1
deaths censored j > 0
risk set
deaths times
0 0 0 0 20
1 6 1 1 20
2 12 2 1 18
3 27 2 3 15
4 39 1 3 10
5 57 1 5 6
(6)
S j =
20 - 0
20
(20 - 1)
20
(18 - 2)
18
15 - 2
15
(10 - 1)
10
(6 - 1)
6
( n j - d j )
n j
= 1
= 0.950
= 0.889
= 0.867
= 0.900
= 0.833
Then the estimate of the survival function is:
time
probability
Page 16
0 \$ t < 6 6 \$ t < 12
1
12 \$ t < 27
27 \$ t < 39
39 \$ t < 57
57 \$ t
1  ́ 0.950 0.950  ́ 0.889 0.844  ́ 0.867 0.732  ́ 0.900 0.659  ́ 0.833
= 0.950
= 0.844
= 0.732
= 0.659
= 0.549 %%%%%%%%%%%%%%%%%%%%%%%%%%%%%%%%%%%%% — 
Standard error of these estimates will be given by Greenwood’s Formula:
Var( \hat{S}(t)) ; ( \hat{S}(t)) 2
å
t j \$ t
(1) (2)
j t j
0 0
1 6
2 12
3 27
4 39
5 57
d j
n j ( n j - d j )
(7)
(8)
d j
d j
å n ( n
n j ( n j - d j )
0
= 0
20(20 - 0)
1
20(20 - 1)
2
18(18 - 2)
2
15(15 - 2)
1
10(10 - 1)
1
6(6 - 1)
j
t j \$ t
j
- d j )
0
= 0.00263 0.00263
= 0.00694 0.00957
= 0.01026 0.01983
= 0.01111 0.03094
= 0.03333 0.06427
Then the standard errors are
time 0 \$ t < 6 6 \$ t < 12 12 \$ t < 27 27 \$ t < 39 39 \$ t < 57 57 \$ t
probability 1 0.950 0.844 0.732 0.659 0.549
standard
error
1  ́
= 0
0 0.950  ́ 0.00263 0.844  ́ 0.00957 0.732  ́ 0.01983 0.659  ́ 0.03094 0.549  ́ 0.06427
= 0.0487
= 0.0826
= 0.1031
= 0.1159
0.1392
Then maximum likelihood estimate of Survival Function with
approximate 95\% confidence intervals:
\hat{S}(t)
	&		&		Approx. 95\% Confidence Interval	&		\\ \hline
0 \$ t < 6	&	1	&		1 \pm  0	&	1	\\ \hline
6 \$ t < 12	&	0.95	&		0.950 \pm  0.096	&	0.854, 1*	\\ \hline
12 \$ t < 27	&	0.844	&		0.844 \pm  0.162	&	0.682, 1*	\\ \hline
27 \$ t < 39	&	0.732	&		0.732 \pm  0.202	&	0.530, 0.934	\\ \hline
39 \$ 4 < 57	&	0.659	&		0.659 \pm  0.227	&	0.432, 0.886	\\ \hline
57 \$ t	&	0.549	&		0.549 \pm  0.273	&	0.276, 0.822	\\ \hline

* Since \hat{S}(t) cannot exceed 1.
Page 17 %%%%%%%%%%%%%%%%%%%%%%%%%%%%%%%%%%%%% — 
Examiners’ Comment: Confidence intervals calculated on log or log-log of
survival probabilities also gained full credit.
(iii)
S(t)
1.2
1.0
0.8
0.6
0.4
0.2
t
10
Page 18
20
30
40
50
60
