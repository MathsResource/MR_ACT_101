\documentclass[a4paper,1pt]{article}

%%%%%%%%%%%%%%%%%%%%%%%%%%%%%%%%%%%%%%%%%%%%%%%%%%%%%%%%%%%%%%%%%%%%%%%%%%%%%%%%%%%%%%%%%%%%%%%%%%%%%%%%%%%%%%%%%%%%%%%%%%%%%%%%%%%%%%%%%%%%%%%%%%%%%%%%%%%%%%%%%%%%%%%%%%%%%%%%%%%%%%%%%%%%%%%%%%%%%%%%%%%%%%%%%%%%%%%%%%%%%%%%%%%%%%%%%%%%%%%%%%%%%%%%%%%%

\usepackage{eurosym}
\usepackage{vmargin}
\usepackage{amsmath}
\usepackage{graphics}
\usepackage{epsfig}
\usepackage{enumerate}
\usepackage{multicol}
\usepackage{subfigure}
\usepackage{fancyhdr}
\usepackage{listings}
\usepackage{framed}
\usepackage{graphicx}
\usepackage{amsmath}
\usepackage{chngpage}

%\usepackage{bigints}
\usepackage{vmargin}

% left top textwidth textheight headheight

% headsep footheight footskip

\setmargins{.0cm}{.5cm}{16 cm}{cm}{0.5cm}{0cm}{1cm}{1cm}

\renewcommand{\baselinestretch}{1.}

\setcounter{MaxMatrixCols}{10}

\begin{document}
%%---  Question 6
A mortality investigation has been carried out over the three calendar years;
1997, 1998 and 1999.
Deaths during the Period of Investigation, \theta x have been classified by age x at the
date of death, where
x = calendar year of death − calendar year of birth
\item  State the rate year implied by this classification, and give the age range of
the lives at the beginning of the rate year.

\item  Censuses of the numbers alive on 1 July 1997, 1 July 1998 and 1 July
1999 have been tabulated and denoted by P x (1⁄2), P x (11⁄2) and P x (21⁄2)
respectively, where x is the age determined at the date of each census.
The force of mortality at age x + f is to be estimated using the formula
\hat{\mu} x + f =
7
\theta x
P x (1⁄2) + P x (11⁄2) + P x (21⁄2)
(a) Determine the age definition x in P x (t), t = 1⁄2, 11⁄2, 21⁄2 if this
formula is correct.
(b) Determine the value of f stating clearly all the assumptions you
have made.

%%%%%%%%%%%%%%%%%%%%%%%%%%%%%%%%%%
6
(i)
x = CY of death − CY of birth
= age, x, on birthday in CY of death
= age next, x, on 1 January in CY of death
= age next, x, on 1 January before date of death
So Calendar Year Rate Interval starting, for lives classified x, on
1 January on which the life is aged x next birthday.
Age range at start of calendar year x − 1 to x.
(ii)
(a)
P x (t) census at t of those x next on previous 1 January would
correspond to the classification of deaths
but ages in the censuses used are ages on 1 July
So (x − 1, x) on 1 January
is (x − 1⁄2, x + 1⁄2) on 1 July = date of census
So required x in P x (1⁄2), P x (11⁄2), P x (21⁄2) is x nearest birthday at date
of census
(b)
Need Birthdays uniform over calendar year
to get average age at start of rate interval, x − 1⁄2
Need force constant over (x − 1⁄2, x + 1⁄2)
So \hat{\mu} x + f will be x + 0, f = 0
\end{document}
