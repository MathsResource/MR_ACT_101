\documentclass[a4paper,1pt]{article}

%%%%%%%%%%%%%%%%%%%%%%%%%%%%%%%%%%%%%%%%%%%%%%%%%%%%%%%%%%%%%%%%%%%%%%%%%%%%%%%%%%%%%%%%%%%%%%%%%%%%%%%%%%%%%%%%%%%%%%%%%%%%%%%%%%%%%%%%%%%%%%%%%%%%%%%%%%%%%%%%%%%%%%%%%%%%%%%%%%%%%%%%%%%%%%%%%%%%%%%%%%%%%%%%%%%%%%%%%%%%%%%%%%%%%%%%%%%%%%%%%%%%%%%%%%%%

\usepackage{eurosym}
\usepackage{vmargin}
\usepackage{amsmath}
\usepackage{graphics}
\usepackage{epsfig}
\usepackage{enumerate}
\usepackage{multicol}
\usepackage{subfigure}
\usepackage{fancyhdr}
\usepackage{listings}
\usepackage{framed}
\usepackage{graphicx}
\usepackage{amsmath}
\usepackage{chngpage}

%\usepackage{bigints}
\usepackage{vmargin}

% left top textwidth textheight headheight

% headsep footheight footskip

\setmargins{.0cm}{.5cm}{16 cm}{cm}{0.5cm}{0cm}{1cm}{1cm}

\renewcommand{\baselinestretch}{1.}

\setcounter{MaxMatrixCols}{10}

\begin{document}

%%---  Question 5
The following diagram represents a 3-state continuous-time Markov Model, in
which states 1 and 2 represent the two stages of a terminal disease and
transitions from state 2 to state 1 are not possible.
\sigma  x + t
1: Healthy
2: Ill
v x + t
\mu x + t
3: Dead
The symbols \sigma  x+t , \mu x+t and v x+t represent the forces of transition. The probability
t
p x ab is defined as
t
p x ab = P(S(t) = bS(0) = a),
where S(t) is the state of the life at time t, and x is the age at time 0.
\item  For a = 1, 2 and 3, write down expressions for t p x aa in terms of the forces of
transition.

\item  Determine an expression for t p x 23 in terms of the forces of transition.
\item  Derive, from first principles, the Kolmogorov forward differential equation
12
for t p 12
x , and state the relevant initial condition, i.e. the value of 0 p x . 
\item  If the forces of transition are assumed constant, show that:
t
p 12
x =
\sigma 
e − ( \sigma +\mu ) t − e − vt
v −\sigma −\mu
(
)

%%%%%%%%%%%%%%%%%%%%%%%%%%%%%%%%%%%%%%%%%%
5
(i)
t æ
p 11
x = exp ç -
è
t p x 22 = exp æ ç -
è
t
(ii)
t
ò
t
( s x + s + m x + s ) ds ö ÷ ;
0
ø
ò
t
0
v x + s ds ö ÷ ;
ø
p x 33 = 1.
p x 23 = 1 - t p x 22 = 1 - exp æ ç -
è
ò
t
0
v x + s . ds ö ÷
ø
Other expressions are possible.
(iii)
12
=
t + h p x
3
å
k = 1
t
p 1 x k h p x k + 2 t with Markov assumption
12
12
22
= t p 11
x h p x + t + t p x h p x + t model constraints
We know that
h p 12
= hs x+t + o(h)
x + t definition
h p x 23 + t = hv x+t + o(h) definition
h p x 21 + t + h p x 22 + t + h p x 23 + t = 1 law of total probability
Substituting and using model constraints we obtain
t + h
11
12
p 12
x = t p x ( h s x + t + o ( h )) + t p x (1 - hv x + t + o ( h ))
Page 5 %%%%%%%%%%%%%%%%%%%%%%%%%%%%%%%%%%%%% — 
Then
Lim
12
p 12
¶
x - t p x
=
h
¶ t
t + h
h ® 0
0
(iv)
(
t
)
12
p 12
= s x + t t p 11
x
x - v x + t t p x
p 12
x = 0
Differential equation is
Hence
(
t
vt
p 12
x e
(
t
p 12
x
) ¢ + v
12
t p x
= s exp(-(s + m)t)
) ¢ = s × exp((v - s - m)t)
ù
s
- vt é
exp{( v - s - m ) t } + c ú ,
Þ t p 12
x = e
ê
ë v - s - m
û
and 0 p 12
x = 0 Þ c =
-s
. Hence result.
v -s-m
[Direct verification is a satisfactory alternative, but the solution must
verify also that 0 p 12
x = 0.]
