\documentclass[a4paper,1pt]{article}

%%%%%%%%%%%%%%%%%%%%%%%%%%%%%%%%%%%%%%%%%%%%%%%%%%%%%%%%%%%%%%%%%%%%%%%%%%%%%%%%%%%%%%%%%%%%%%%%%%%%%%%%%%%%%%%%%%%%%%%%%%%%%%%%%%%%%%%%%%%%%%%%%%%%%%%%%%%%%%%%%%%%%%%%%%%%%%%%%%%%%%%%%%%%%%%%%%%%%%%%%%%%%%%%%%%%%%%%%%%%%%%%%%%%%%%%%%%%%%%%%%%%%%%%%%%%

\usepackage{eurosym}
\usepackage{vmargin}
\usepackage{amsmath}
\usepackage{graphics}
\usepackage{epsfig}
\usepackage{enumerate}
\usepackage{multicol}
\usepackage{subfigure}
\usepackage{fancyhdr}
\usepackage{listings}
\usepackage{framed}
\usepackage{graphicx}
\usepackage{amsmath}
\usepackage{chngpage}

%\usepackage{bigints}
\usepackage{vmargin}

% left top textwidth textheight headheight

% headsep footheight footskip

\setmargins{.0cm}{.5cm}{16 cm}{cm}{0.5cm}{0cm}{1cm}{1cm}

\renewcommand{\baselinestretch}{1.}

\setcounter{MaxMatrixCols}{10}

\begin{document}

[Total 8]
A life insurance company wishes to test if the recent mortality experience of its
annuitants is consistent with the mortality basis used in the calculation of the
annuity rates.
If q o x and q x p denote the mortality rates for the experience and the premium basis
respectively:
\item  State the null hypothesis for the investigation.
\item  For each of
•
•
%%---  Question 
the Chi-Squared Test
the Grouping of Signs Test
(a) State the test statistic.
(b) Derive the sampling distribution of the test statistic, when the null
hypothesis is true.
(c) Indicate the observed values of the test statistic that would lead to
the rejection of the null hypothesis.
Define clearly all the symbols that you use.
%% ---- 104 A2001—4
%%%%%%%%%%%%%%%%%%%%%%%%%%%%%%%%%%%%%%%%%%%%%%%%%%%%%%%%%%%%%%%%%%%%%%%%%%%%%
Page 7 %%%%%%%%%%%%%%%%%%%%%%%%%%%%%%%%%%%%% — April 2001 — Examiners’ Report
7
(i)
H 0 : The observed rates q o x are consistent with coming from a population in
which the premium rate basis q x p are the true rates.
(ii)
E x initial exposed to risk at age x in the investigation
Chi-squared
(a)
The test statistic is
å
(
E x q o x − E x q x p
E x q x p
x
)
2
where E x is the initial exposed to risk at age x.
(b)
If H 0 is true, then
E x q o x − E x q x p
E x q x p
~
approx.
N (0,1)
for each age x using the Central Limit Theorem and assuming that
1 − q x p @ 1.
Then if the observations at each of the n ages are independent
å
x
(c)
( E x q o x − E x q x p ) 2
E x q x p
~
approx.
χ n 2
Reject H 0 if the Observed Value of the test statistic > χ 2 n (1 − α) at
α% level of significance. This is a one-tailed test.
Grouping of Signs
(a)
The test statistic is G, the observed number of positive groups of
signs among n 1 positive and n 2 negative deviations, where the
deviation at age x = E q o − E q p .
x
(b)
x
x
x
If H 0 is true, then the deviations will be allocated randomly to
groups.
Then n 1 positive deviations can be divided into g groups in
æ n 1 − 1 ö
ç
÷ ways
è g − 1 ø
Page 8 %%%%%%%%%%%%%%%%%%%%%%%%%%%%%%%%%%%%% — April 2001 — Examiners’ Report
These g groups of positive signs can be allocated amongst the n 2
negative deviations in
æ n 2 + 1 ö
ç
÷ ways
è g ø
The unrestricted number of ways of arranging n 1 + n 2 positive and
negative deviations is
æ n 1 + n 2 ö
ç
÷
è n 1 ø
So the sampling distribution of the test statistic, G, is
æ n 2 + 1 ö æ n 1 − 1 ö
ç
֍
÷
g ø è g − 1 ø
è
P[G = g] =
g = 1, 2, 3, ...n 2 − 1
æ n 1 + n 2 ö
ç
÷
è n 1 ø
Alternatively, if n 1 + n 2 > 20 (approx.) then
G
æ n ( n + 1) ( n 1 n 2 ) 2 ö
N ç 1 2
,
3 ÷
approx.
è ( n 1 + n 2 ) ( n 1 + n 2 ) ø
~
using the Central Limit Theorem and E[G] and Var[G].
(c)
Reject H 0 if the Observed Value of the test statistic, G \leq  k* where
k* is smallest integer such that
P[G \leq  k*] ≥ 0.05
(or as similar statement based on the N(0, 1) distribution). This is
a one-tailed test.
\end{document}
