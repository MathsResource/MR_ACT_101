\documentclass[a4paper,12pt]{article}

%%%%%%%%%%%%%%%%%%%%%%%%%%%%%%%%%%%%%%%%%%%%%%%%%%%%%%%%%%%%%%%%%%%%%%%%%%%%%%%%%%%%%%%%%%%%%%%%%%%%%%%%%%%%%%%%%%%%%%%%%%%%%%%%%%%%%%%%%%%%%%%%%%%%%%%%%%%%%%%%%%%%%%%%%%%%%%%%%%%%%%%%%%%%%%%%%%%%%%%%%%%%%%%%%%%%%%%%%%%%%%%%%%%%%%%%%%%%%%%%%%%%%%%%%%%%

\usepackage{eurosym}
\usepackage{vmargin}
\usepackage{amsmath}
\usepackage{graphics}
\usepackage{epsfig}
\usepackage{enumerate}
\usepackage{multicol}
\usepackage{subfigure}
\usepackage{fancyhdr}
\usepackage{listings}
\usepackage{framed}
\usepackage{graphicx}
\usepackage{amsmath}
\usepackage{chngpage}

%\usepackage{bigints}
\usepackage{vmargin}

% left top textwidth textheight headheight

% headsep footheight footskip

\setmargins{2.0cm}{2.5cm}{16 cm}{22cm}{0.5cm}{0cm}{1cm}{1cm}

\renewcommand{\baselinestretch}{1.3}

\setcounter{MaxMatrixCols}{10}

\begin{document}


[Total 7]8
In a mortality investigation the total number of deaths at age x during the period
of investigation is θ x . Age x is defined as:
x = [age next birthday at the start of the policy]
+ [curtate duration at the date of death]
(i) State, precisely, the rate interval implied by this definition.
(ii) Describe, precisely, the central exposed to risk E x c that would correspond
to this definition.

(iii) Give the age, x + f , for which the estimate ç ç \mu o ( x ) =
æ
è
θ x
E x c
ö
÷
÷
ø
estimates \mu x + f .
State and explain any assumptions you make in determining f .
(iv)
9


Explain how your answer to (iii) above would change if an investigation
showed that, on average, lives purchased their policies two months before
a birthday.

[Total 8]
Life insurance companies, investigating the mortality experience of their
policyholders, usually count the number of policies in force and the number of
policies giving rise to death claims, rather than the number of lives assured in
force and the number of lives who die during the period of investigation.
(i) Explain why this is done. Describe the statistical problems that might
arise from an investigation based on policies rather than on the lives
assured.

(ii) If N identical and independent lives aged x exact are observed until age
x + 1 (or until earlier death), and π i is the proportion of those lives having
exactly i policies ( i = 1, 2, 3, ...), then show algebraically how the
statistical problems you described in part (i) arise.

[Total 8]
104(S)—5
%%%%%%%%%%%%%%%%%%%%%%%%%%
8
(i) Policy year rate interval starting (for deaths classified x) on the policy
anniversary on which lives were x next birthday.
(iii) The total number of life years for which lives were exposed to the risk of
dying while aged x next birthday on the immediately preceding policy
anniversary.
or
Let P x ( t ) be a census at time t after the start of the period of investigation of those
lives aged x next birthday on the policy anniversary immediately prior to t, then
central exposed to risk is
t = T
ò P x ( t ). dt
t = 0
where the period of investigation is (0,T).
(iii) x+f = average age of lives half-way through the rate interval
Thus, if birthdays uniformly distributed over the policy year, lives will be
on average x-1⁄2 at the start of the rate interval and x+f = x.
(iv) The assumption of uniform birthdays over the policy year is violated.
o
Lives now x − 1 6 on average at the start of the rate interval and hence \mu x
estimates \mu x + 1 3 .
9
(i) Done because life assurance companies have data in this form (and hence
far easier than working with lives) and because the result is an unbiased
estimate of the rate, provided there is no correlation between deaths and
the number of policies covering a life. However, the variance of the
estimate is increased. Graduation tests etc will therefore need to be
adjusted to allow for this.
(ii) Let D i be the number of deaths among the π i N lives each with i polices,
and let C i be the number of claims among the same lives (i.e. C i = i D i ).
We can say that:
D i
~
Binomial( π i N, q x )
Page 7%%%%%%%%%%%%%%%%%%%%%%%%%%%%%%%%%%%%%%%%%%%%%%5for September 2000
because we have independence of deaths . Therefore:
å C ]
Var[ å i D ]
å i Var[ D ]
å i π Nq (1 − q ) .
Var[ C ] =
Var[
i
i
=
i
i
2
=
i
(independence of deaths)
i
2
=
i
x
x
å i π N independent lives (or policies) we would have:
E[ C ] = å i π Nq
Var[C] = å i π Nq ( 1 − q ) .
i
If we observed
i
i
i
x
i
i
x
x
i
So, the effect of duplicate policies is to increase the variance of the number of
claims, in the ratio
å i π
r =
å i π
2
i
i
i
i
ˆ ( t) =
The Nelson-Aalen estimator is Λ


\end{document}
