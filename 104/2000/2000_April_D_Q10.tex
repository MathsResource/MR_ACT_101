\documentclass[a4paper,12pt]{article}
%%%%%%%%%%%%%%%%%%%%%%%%%%%%%%%%%%%%%%%%%%%%%%%%%%%%%%%%%%%%%%%%%%%%%%%%%%%%%%%%%%%%%%%%%%%%%%%%%%%%%%%%%%%%%%%%%%%%%%%%%%%%%%%%%%%%%%%%%%%%%%%%%%%%%%%%%%%%%%%%%%%%%%%%%%%%%%%%%%%%%%%%%%%%%%%%%%%%%%%%%%%%%%%%%%%%%%%%%%%%%%%%%%%%%%%%%%%%%%%%%%%%%%%%%%%%
\usepackage{eurosym}
\usepackage{vmargin}
\usepackage{amsmath}
\usepackage{graphics}
\usepackage{epsfig}
\usepackage{enumerate}
\usepackage{multicol}
\usepackage{subfigure}
\usepackage{fancyhdr}
\usepackage{listings}
\usepackage{framed}
\usepackage{graphicx}
\usepackage{amsmath}
\usepackage{chngpage}
%\usepackage{bigints}

\usepackage{vmargin}
% left top textwidth textheight headheight
% headsep footheight footskip
\setmargins{2.0cm}{2.5cm}{16 cm}{22cm}{0.5cm}{0cm}{1cm}{1cm}
\renewcommand{\baselinestretch}{1.3}

\setcounter{MaxMatrixCols}{10}
\begin{document}


10
The population of elderly people in a prison is observed during the period
1 January 1994 to 31 December 1996. The duration of residence (measured to
the nearest number of months) is recorded for those who die during the
period, for those who are released from the prison during the period and for
those who are still in residence on 31 December 1996.
The recorded data measured in months are
6 † 6 6 6 7 9 †
10 † 10 11 † 13 16 17 †
20 23 †
where † indicates those who were released from the prison during the period
or who were still in residence on 31 December 1996.
(i) State the type(s) of censoring present in these data.
(ii) Calculate the product-limit (Kaplan-Meier) estimate of the survival
function, S(t), where t is the duration of residence in the prison.
%%%%%%%%%%%%%%%%%%%%%%%%%%%%%%%%%%%%%%%
(iii) State the assumptions underlying the estimate in (ii), and explain how
each of these assumptions would apply to these data.

[Total 13]
104—5

\newpage

%%%%%%%%%%%%%%%%%%%%%%%%%%%%%%%%%%%%%%%%%%%%%%%%%%%%%%%%%%%%%%%%%%%%%%%%%%%%%%%%%%%%%%%%%%%%%%%%%%%%%%%%%%%%%%

%%---- Page 9Subject 104 (Survival Models) — %%%%%%%%%%%%%%%%%%%%%%%%%%%%%%%%%%%%%%%5
10
(i)
Those lives who are released from prison during the period of
investigation or who are still alive and resident in the prison at the end of
the period of investigation are right censored.
Duration is recorded to the nearest month so there is interval censoring.
Being released from prison is a form of random censoring.
(ii)
We can summarise the data to obtain the statistics necessary to complete
the estimation.
j t j
months R j
Risk set
= R j-1 - d j-1 - C j-1 C j
# censored
in [t j , t j+1 ] d j
# of deaths
1
2
3
4
5
6 6
7
10
13
16
20 14
10
8
5
4
2 1
1
2
0
1
1 3
1
1
1
1
1
Then estimates of survival probabilities are
j t j
1
2
3
4
5
6
7 6
7
10
13
16
20
23
s j = s j-1  ́
( R j - 1 - d j - 1 )
1
1  ́ (14 - 3) / 14
= .7857
.7857  ́ (10 - 1) / 10 = .7071
.7071  ́ (8 - 1) / 8 = .6188
.6188  ́ (5 - 1) / 5 = .4950
.4950  ́ (4 - 1) / 4 = .3712
.3712  ́ (2 - 1) / 2 = .1856
So estimates of the survival function is:
ì 1
ï
ï .79
ï .71
ï
S(t) = í .62
ï .50
ï
ï .37
ï
î .19
%%---- Page 10
R j - 1
0 £ t £ 6
6 < t £ 7
7 < t £ 10
10 < t £ 13
13 < t £ 16
16 < t £ 20
20 < t £ 23Subject 104 (Survival Models) — %%%%%%%%%%%%%%%%%%%%%%%%%%%%%%%%%%%%%%%5
(iii)
Non-informative censoring: Time to censoring i.e. leaving for reasons
other than death is independent of Time to death.
If we believe that more healthy lives will tend to leave and thus
have lighter mortality, this assumption is violated. [OR could say
less healthy lives may tend to leave — same conclusion.]
However if we believe that being released is unrelated to state of
health, then assumption is OK.
Lives are independent i.e. time to censoring, i.e. being released, or time to
death determined independently for each life.
Other relevant comments received credit.


\end{document}
