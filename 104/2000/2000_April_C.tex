8
X is a random variable which measures the duration from the date of a kidney
transplant until death.
(i) Express the hazard rate and the integrated hazard function at duration
x, in terms of probabilities.

(ii) If the hazard rate at duration x is
h(x) = \alpha\lambda x \alpha−1
derive an expression for the integrated hazard, H(x).
(iii)

The hazard rate h(x), as defined in part (i), varies between transplant
patients in such a way that
\alpha = \alpha 0 + \alpha 1 z 1
\lambda = \lambda 1 z 1 + \lambda 2 z 2
where \alpha 0 , \alpha 1 , \lambda 1 , \lambda 2 are constants; z 1 is the age of the patient at the date
of the transplant and z 2 is the patient’s sex where z 2 = 0 = female,
z 2 = 1 = male.
Show that these hazards are not in general proportional, but that if
\alpha 1 = 0 the hazards are proportional.

[Total 9]
104—49
A special endowment assurance issued to a man aged 40 exact has a term of 25
years. A sum assured of £10,000 is payable immediately on death, and a sum
assured of £15,000 is payable on survival to the end of the term.
The policy is secured by annual premiums of P during the first five years, 2P
during the second five years and 3P during the remainder of the term.
Premiums are payable annually in advance for 25 years or until death, if
earlier.
(i) Show that the initial net annual premium for this policy is £164.23.
[6]
(ii) Determine the prospective net premium policy value at the end of the
twelfth policy year, immediately before the payment of the thirteenth
premium.

Basis: A1967–70 Ultimate Mortality Table. 4% per annum interest.
[Total 10]


%%%%%%%%%%%%%%%%%%%%%%%%%%%%%%%

8
(i)
Hazard Rate, h(x) = Lim
h ® 0 +
P [ x < X £ x + h 1⁄2 X > x ]
h
Alternative expressions are also correct.
u = x
Integrated Hazard H(x) =
ò
h ( u ) . du = - ln(S(x))
u = 0
= - ln . P[X > x]
where S(x) = P[X > x]
u = x
(ii)
From (i)
H(x) =
ò
al u a - 1 . du
u = 0
=
a
l u a
a
u = x
u = 0
= lx =
(iii)
x > 0
h(x) = (a 0 + a 1 z 1 )(l 1 z 1 + l 2 z 2 ) x a 0 +a 1 z 1 - 1
Then
h ( x 1⁄2 z )
h ( x 1⁄2 z *)
=
( a 0 + a 1 z 1 )( l 1 z 1 + l 2 z 2 ) x a 0 +a 1 z - 1
( a 0 + a 1 z 1 * )( l 1 z 1 * + l 2 z 2 * ) x a 0 +a 1 z * - 1
which is not in general independent of x, so hazards are
not proportional.
%%---- Page 7Subject 104 (Survival Models) — %%%%%%%%%%%%%%%%%%%%%%%%%%%%%%%%%%%%%%%5
If a 1 = 0, then
h ( x 1⁄2 z )
h ( x 1⁄2 z *)
=
=
a 0 ( l 1 z 1 + l 2 z 2 ) x a 0 - 1
a 0 ( l 1 z 1 * + l 2 z 2 * ) x a 0 - 1
l 1 z 1 + l 2 z 2
l 1 z 1 * + l 2 z 2 *
which is independent of x, so the hazards are proportional.
Alternatively if a 1 = 0, then
h(x1⁄2z) = a 0 (l 1 z 1 + l 2 z 2 ) x a 0 - 1
which is of the form h 0 (x) = a 0 x a 0 - 1  ́ c ( z ) ; where c(z) = l 1 z 1 + l 2 z 2
9
(i)
Expected Present Value of death benefit 1
= 10000A 40:25
Expected Present Value of survival benefit 1
= 15000A 40:25
Expected Present Value of Premiums to be paid
= Pa 40:25 + P 5 1⁄2 a 40:20 + P 10 1⁄2 a 40:15
Evaluation
1
A 40:25
1
; (1 + 1⁄2 i ) A 40:25
æ 0.40005 0.30690 ö
ç
1 ÷
= 1.02  ́ ç A 40:25 - A 40:25
÷
ç
÷
è
ø
= 0.09501
2144.1713
as
1
A 40:25
=
D 65
D 40
6986.4959
%%---- Page 8
= 0.30690Subject 104 (Survival Models) — %%%%%%%%%%%%%%%%%%%%%%%%%%%%%%%%%%%%%%%5
Also
40:25
a
= 15.599
5689.1776
40:20
5 1⁄2 a
D 45
D 65
=
13.488
45:20 = 10.9834
a
6986.4959
4597.0607
and
40:15
10 1⁄2 a
D 50
D 40
=
10.995
50:15 = 7.2346
a
6986.4959
Equation of Value
1
1
10000 A 40:25
+ 15000 A 40:25
= P ( a 40:25 + 5 1⁄2 a 40:20 + 10 1⁄2 a 40:15 )
950.10
P =
(ii)
33.817
4603.50
5553.60
= £164.2251
33.817
£164.23 H
KKKKKKKKKK
Prospective Policy Value after 12 years
1
10000 A 52:13
+ 15000 A 52:13 1 - 3  ́ 164.23 a 52:13
0.50965
0.11272
9.876
Evaluation
2144.1713
A 52:13 1 =
D 65
D 52
= 0.50965
4207.1417
0.62016
1
A 52:13
Policy Value
0.50965
; 1.02( A 52:13 - A 52:13 1 ) = 0.11272
1127.2 + 7644.75 - 4865.806 = £3906.14
