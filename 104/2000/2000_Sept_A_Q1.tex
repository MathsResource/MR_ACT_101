\documentclass[a4paper,12pt]{article}

%%%%%%%%%%%%%%%%%%%%%%%%%%%%%%%%%%%%%%%%%%%%%%%%%%%%%%%%%%%%%%%%%%%%%%%%%%%%%%%%%%%%%%%%%%%%%%%%%%%%%%%%%%%%%%%%%%%%%%%%%%%%%%%%%%%%%%%%%%%%%%%%%%%%%%%%%%%%%%%%%%%%%%%%%%%%%%%%%%%%%%%%%%%%%%%%%%%%%%%%%%%%%%%%%%%%%%%%%%%%%%%%%%%%%%%%%%%%%%%%%%%%%%%%%%%%

\usepackage{eurosym}
\usepackage{vmargin}
\usepackage{amsmath}
\usepackage{graphics}
\usepackage{epsfig}
\usepackage{enumerate}
\usepackage{multicol}
\usepackage{subfigure}
\usepackage{fancyhdr}
\usepackage{listings}
\usepackage{framed}
\usepackage{graphicx}
\usepackage{amsmath}
\usepackage{chngpage}

%\usepackage{bigints}
\usepackage{vmargin}

% left top textwidth textheight headheight

% headsep footheight footskip

\setmargins{2.0cm}{2.5cm}{16 cm}{22cm}{0.5cm}{0cm}{1cm}{1cm}

\renewcommand{\baselinestretch}{1.3}

\setcounter{MaxMatrixCols}{10}

\begin{document}

ã Faculty of Actuaries
ã Institute of Actuaries1
(i) Explain precisely what is meant by
(ii) Write down an integral expression for
3  5
q 50 .
3  5

q 50 in terms of the hazard rates
over the appropriate age range only.
(iii)
Evaluate
3  5

q 50 using the A1967-70 Ultimate mortality table.


%%%%%%%%%%%%%%%%%%%%%%%%%%%%%%%%%%%%%%%%%%%%%%%%%%%%%%%%%%%%%%%%%%%%%

1
(i) Pr(a life aged 50 dies between exact ages 53 and 58)
(ii) 3|5 q 50
= ò
8
t
3
p 50 . \mu 50 + t dt
5
= 3 p 50 ò t p 53 . \mu 53 + t dt
0
3
5
t
= exp é ê − ò \mu 50 + r dr ù ú ò exp é ê − ò \mu 53 + r dr ù ú \mu 53 + t dt
ë
û
0
ë
0
o
û
Several other alternatives were acceptable, in particular
8
ò 3
t
exp é ê − ò \mu 50 + r dr ù ú \mu 50 + t dt
ë
ì
ï
0
3
û
é
ü
ï ì
ï
ï
þ ê
ë ï
î
5
ù
ü
ï
or exp í − ò \mu 50 + r dr ý ê 1 − exp í − ò \mu 53 + r dr ý ú
ï
î
0
0
ï
þ ú
û
3
8
ì
ü
ì
ü
ï
ï
ï
ï
or exp í − ò \mu 50 + r dr ý − exp. í − ò \mu x 50 + r dr ý
ï 0
ï
ï 0
ï
î
þ
î
þ
3
5
r
ì
ü
ì
ü
ï
ï
ï
ï
or exp í − ò \mu 50 + r . dr ý ò exp. í − ò \mu 53 + s . ds ý \mu 53 + r . dr
ï 0
ï 0
ï 0
ï
î
þ
î
þ
(iii)
2
l 53 − l 58 32,143.546 − 30,795.116
=
= 0.04127
l 50
32,669.855
 
Present value of the loss = λ = 20000 v K + 1 − 520 a
where K = curtate future
K + 1 |
lifetime, of the life aged 50.
(i) EPV loss
  50 = 20 , 000 × 0 . 38450 − 520 × 16 . 003 = − 631 . 56
= Ε ( λ ) = 20 , 000 A 50 − 520 a
(ii) λ = 20000 v K + 1 − 520 ç
æ 1 − v K + 1 ö
÷
è d 0.04 ø
= constant + ( 520 d 0 . 04 + 20 , 000 ) v K + 1
(
)
Var v K + 1 = 2 A 50 − A 50 2 (where 2 A is @ 8.16% and A @ 4%)
= A 50 − 0 . 38450 2 ,
2
Then Var ( λ ) = 33 , 520 2
(
2
A 50 − 0 . 38450 2
)
So b = 33520 2 = 1,123,590,400
Page 3%%%%%%%%%%%%%%%%%%%%%%%%%%%%%%%%%%%%%%%%%%%%%%5for September 2000
C = -(33,520 x 0.38450) 2 = − 166,111,886
An approach using co-variances is possible, but much more complicated.
3
In both cases the retrospective policy value is
D x æ
ö
  − A 1 ÷ .
ç P a
x : t | ø
D x + t è x : t |
P x > P x :n |
V x will exceed t V x :n | if
t
1
1
− d >
− d
 x
a
a  x : n
a  x < a  x : n
  x = a
  + a nonnegative term.
But a
x :n |
Hence it is not possible.
Alternatively,
If
1 −
t V x
> t V x : n then

a
 x + t
a
> 1 − x + t : n − t
 x

a
a
x : n
which means
1 −

a
 x + t
a
< x + t : n − t
 x

a
a
x : n
But a  x = a  x : n + v n n P x a  x + n
a  x + t = a  x + t : n − 1 + v n − t
 x + n
n − t P x + t a
So the inequality will hold if
( a

x + t : n − t
)
which means
 x + n a


 x + n
v n − t n − t P x + t a
< a
v n n P x a
x : n
x + t : n − t


a
< v t t P x a
x : n
x + t : n − t


= a
− a
x : n
x : t
Page 4
(
 x + n a


 x + n
+ v n − t n − t P x + t a
< a
a  x : n + v n n P x a
x : n
x + t : n − t
)%%%%%%%%%%%%%%%%%%%%%%%%%%%%%%%%%%%%%%%%%%%%%%5for September 2000
which is not true, hence the original assumption is untrue.
[Or: Write in terms of commutation functions
N x + t
N − N x + n
< x + t
N x
N x − N x + n
which implies
N x+t > N x
Which is not true, hence the original assumption is untrue.
Credit was also given for labelled sketches on t V x and t V x : n against t, for well
chosen numerical examples and for careful verbal arguments.
It was appreciated that the wording of the question would have been improved if it
specifically asked for a comparison for 0 \leq t \leq n.


\end{document}
