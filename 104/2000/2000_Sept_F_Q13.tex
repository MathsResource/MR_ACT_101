\documentclass[a4paper,12pt]{article}

%%%%%%%%%%%%%%%%%%%%%%%%%%%%%%%%%%%%%%%%%%%%%%%%%%%%%%%%%%%%%%%%%%%%%%%%%%%%%%%%%%%%%%%%%%%%%%%%%%%%%%%%%%%%%%%%%%%%%%%%%%%%%%%%%%%%%%%%%%%%%%%%%%%%%%%%%%%%%%%%%%%%%%%%%%%%%%%%%%%%%%%%%%%%%%%%%%%%%%%%%%%%%%%%%%%%%%%%%%%%%%%%%%%%%%%%%%%%%%%%%%%%%%%%%%%%

\usepackage{eurosym}
\usepackage{vmargin}
\usepackage{amsmath}
\usepackage{graphics}
\usepackage{epsfig}
\usepackage{enumerate}
\usepackage{multicol}
\usepackage{subfigure}
\usepackage{fancyhdr}
\usepackage{listings}
\usepackage{framed}
\usepackage{graphicx}
\usepackage{amsmath}
\usepackage{chngpage}

%\usepackage{bigints}
\usepackage{vmargin}

% left top textwidth textheight headheight

% headsep footheight footskip

\setmargins{2.0cm}{2.5cm}{16 cm}{22cm}{0.5cm}{0cm}{1cm}{1cm}

\renewcommand{\baselinestretch}{1.3}

\setcounter{MaxMatrixCols}{10}

\begin{document}


104(S)—813
A life insurance company issues a 20-year temporary assurance to lives aged 45.
The sum assured, which is payable immediately on death, is £300,000 for the
first 10 years, and £100,000 thereafter. Level annual premiums are payable in
advance for 20 years, or until earlier death.
The premium basis is:
Mortality: A1967-70 Ultimate
Interest: 4% per annum.
Expenses: nil.
(i) Show that the premium payable is approximately £1,339.57 per annum.

(ii) Find the net premium policy value ten years after the commencement of
the policy, immediately before the payment of the eleventh premium,
assuming the reserving basis is the same as the premium basis.

(iii) Give an explanation of your numerical answer to part (ii). Describe the
disadvantages to the insurance company of issuing this policy.

(iv) How could the terms of the policy be altered, so as to remove the
disadvantages described in part (iii)?
104(S)—9

[Total 13]

%%%%%%%%%%%%%%%%%%%%%%%%%%%%%%%%%%%%%%%%%%%%%%%%%%%%%%%%%%%%%%%%%%%%%%%%%%%%%%%

Page 12%%%%%%%%%%%%%%%%%%%%%%%%%%%%%%%%%%%%%%%%%%%%%%5for September 2000
13
(i)
P a   45 : 20 | = 10 5 æ ç A 1
è
i.e.
+ 2 A 1
45 : 20 |
45 : 10 |
ö
÷
ø
13 . 488 P ≈ 10 5 × 1 . 04 1⁄2 ( 0 . 10434833 + 2 × 0 . 03641229 )
P = 1,339.57
1
M x − M x + n
D x
è
æ
Using A x 1 : n | = (1 + i ) 2 ç
ö
÷
ø
Other assumptions produce small numerical differences in the answer. All
received full credit.
%%%%%%%%%%%%%%%%%%%%%%%%%%%%%%%%%%%%%%%%%%%%%%%%
(ii)
V = 10 5 A 1
10
55 : 10 |
− P a   45 : 10 | ≈ 10 5 × 1 . 04 1⁄2 × 0 . 10546950 − 8 . 045 P = − 21 . 02
Again other assumptions which produce different numerical answers
received full credit.
A retrospective calculation is also satisfactory.

%%%%%%%%%%%%%%%%%%%%%%%%%%%%%%%%%%%%%%%%%(iii)
More cover provided in the first 10 years than is paid for by the premiums
in those years.
Hence policyholder “in debt” at time 10, size of debt equals negative policy
value.
If policy is lapsed during first ten years (possibly longer) the company will
suffer a loss
Not possible to recover this loss from policyholder.
(iv)
Collect the premiums more quickly
e.g. shorten premium paying term
make premiums larger in earlier years, smaller in later years
Change the pattern of benefits to reduce benefits in first ten years and
increase them in last ten years
Other sensible suggestions received credit.
\end{document}
