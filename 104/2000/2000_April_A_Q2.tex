\documentclass[a4paper,12pt]{article}
%%%%%%%%%%%%%%%%%%%%%%%%%%%%%%%%%%%%%%%%%%%%%%%%%%%%%%%%%%%%%%%%%%%%%%%%%%%%%%%%%%%%%%%%%%%%%%%%%%%%%%%%%%%%%%%%%%%%%%%%%%%%%%%%%%%%%%%%%%%%%%%%%%%%%%%%%%%%%%%%%%%%%%%%%%%%%%%%%%%%%%%%%%%%%%%%%%%%%%%%%%%%%%%%%%%%%%%%%%%%%%%%%%%%%%%%%%%%%%%%%%%%%%%%%%%%
\usepackage{eurosym}
\usepackage{vmargin}
\usepackage{amsmath}
\usepackage{graphics}
\usepackage{epsfig}
\usepackage{enumerate}
\usepackage{multicol}
\usepackage{subfigure}
\usepackage{fancyhdr}
\usepackage{listings}
\usepackage{framed}
\usepackage{graphicx}
\usepackage{amsmath}
\usepackage{chngpage}
%\usepackage{bigints}

\usepackage{vmargin}
% left top textwidth textheight headheight
% headsep footheight footskip
\setmargins{2.0cm}{2.5cm}{16 cm}{22cm}{0.5cm}{0cm}{1cm}{1cm}
\renewcommand{\baselinestretch}{1.3}

\setcounter{MaxMatrixCols}{10}
\begin{document}

[2]
2 Explain why crude estimates of mortality rates should be smoothed before
they can be used in financial calculations.





%%%%%%%%%%%%%%%%%%%%
2
(i)
The underlying model of mortality is that rates change smoothly with age.
Two aspects of estimation are in conflict with this model
· rates are estimated separately for each age and there is no constraint
in the estimation procedures which ensures that rates at adjacent ages
conform to the model
· superimposed on this is the sampling error of each estimate. This will
be an important effect at ages where there is a small exposed to risk.
If these features of estimation are not corrected before rates are used
then, for example, premium rates may not progress smoothly by age.
This could present opportunities for “lapse and re-entry” options by
healthy policyholders resulting in increased expenses, and an irregular
progression would be hard to justify in practice.
Creit given for other relevant examples.
