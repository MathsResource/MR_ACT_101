\documentclass[a4paper,12pt]{article}
%%%%%%%%%%%%%%%%%%%%%%%%%%%%%%%%%%%%%%%%%%%%%%%%%%%%%%%%%%%%%%%%%%%%%%%%%%%%%%%%%%%%%%%%%%%%%%%%%%%%%%%%%%%%%%%%%%%%%%%%%%%%%%%%%%%%%%%%%%%%%%%%%%%%%%%%%%%%%%%%%%%%%%%%%%%%%%%%%%%%%%%%%%%%%%%%%%%%%%%%%%%%%%%%%%%%%%%%%%%%%%%%%%%%%%%%%%%%%%%%%%%%%%%%%%%%
\usepackage{eurosym}
\usepackage{vmargin}
\usepackage{amsmath}
\usepackage{graphics}
\usepackage{epsfig}
\usepackage{enumerate}
\usepackage{multicol}
\usepackage{subfigure}
\usepackage{fancyhdr}
\usepackage{listings}
\usepackage{framed}
\usepackage{graphicx}
\usepackage{amsmath}
\usepackage{chngpage}
%\usepackage{bigints}

\usepackage{vmargin}
% left top textwidth textheight headheight
% headsep footheight footskip
\setmargins{2.0cm}{2.5cm}{16 cm}{22cm}{0.5cm}{0cm}{1cm}{1cm}
\renewcommand{\baselinestretch}{1.3}

\setcounter{MaxMatrixCols}{10}
\begin{document}




[Total 8]6
In a mortality investigation the actual number of deaths at age x last birthday
is d x . The goodness of fit between the data and the force of mortality, \mu x+1⁄2 ,
over the age range x 1 , x 1 + 1, x 1 + 2 ..., x 1 + m − 1 can be tested using the
statistic
x = x 1 + m − 1
∑
x = x 1
( d
x
− E x c \mu x + 1⁄2
E x c \mu x + 1⁄2
)
2
where E x c is the central exposed to risk which corresponds to d x .
For each of the following cases state the null hypothesis being tested, and give
the sampling distribution of this statistic as precisely as you can from the
given information:
(a) if \mu x+1⁄2 are taken from a standard mortality table;
(b) if \mu x+1⁄2 are the graduated rates obtained from the crude estimates using
the Gompertz-Makeham formula
g(x) = a 0 + exp{b 0 + b 1 x + b 2 x 2 };
(c)
if \mu x+1⁄2 are the graduated rates obtained from the crude estimates using
the formula
g(x) = \mu sx + k
where \mu sx are taken from a standard mortality table and k is a constant;
and
(d)
if \mu x+1⁄2 are the graduated rates obtained from a graphical graduation of
the crude estimates d x / E x c .
104—3
[8]




6
(a)
H 0 : the observed transition rates m ˆ x + 1⁄2 come from a population in which
the standard table rates are the true rates.
c 2 with m degrees of freedom
(b)
H 0 : the observed transition rates m ˆ x + 1⁄2 come from a population in which
the graduated transition rates are the true rates
In the graduation process four parameters have been estimated so
sampling distribution is c 2 with m - 4 degrees of freedom.
%%---- Page 5Subject 104 (Survival Models) — %%%%%%%%%%%%%%%%%%%%%%%%%%%%%%%%%%%%%%%5
(c)
H 0 : same as in (b).
1 degree of freedom is lost for each parameter estimated, plus an
unknown additional number for each constraint imposed by the standard
table; say p degrees of freedom, so:
c 2 with m - 1 - p degrees of freedom.
(d)
H 0 : same as in (b)
The best fitting curve will usually be drawn as a series of curved segments
joined smoothly.
Each segment imposes a constraint of height, slope and curvature; so lose
2 or 3 degrees of freedom for each section of about 10 ages drawn; so in 2
sections for example we have
c 2 with say m - 5 degrees of freedom.
