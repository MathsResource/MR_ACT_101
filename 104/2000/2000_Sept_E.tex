104(S)—7
12
The following data have been taken from a mortality investigation of a life
insurance company’s male whole of life policyholders aged between 25 and 65.
The graduated rates were obtained by fitting a mathematical formula to the
data. The formula has four parameters, which were estimated using the method
of maximum likelihood.
Age
(x)
Initial
exposed
to risk
E x
Observed
deaths
θ x
Graduated
rate
Q x
Expected
deaths
E x q x
E x q x p x
Standar-
dised
deviations
θ x − E x q x
E x q x p x
35
36
37
38
39
40
41
14 211
12 381
11 704
11 038
10 947
13 885
11 507
85 673
17
21
27
24
29
21
30
169
0.001998
0.002061
0.002124
0.002187
0.002250
0.002314
0.002378
28.39
25.52
24.86
24.14
24.63
32.13
27.36
187.03
28.33
25.47
24.81
24.09
24.57
32.06
27.29
186.62
−2.14
−0.89
−0.43
−0.03
+0.88
−1.97
+0.51
(i) Perform the Chi-squared test, and two other appropriate tests for the
adherence of this graduation to the data. Comment on the results of the
tests.
[8]
(ii) Describe the statistical rationale underlying the cumulative deviations
test, a test which is often used to assess adherence to data in a
graduation. State an important limitation of the test when used for this
purpose, and explain how you could overcome this limitation.

[Total 12]
%%%%%%%%%%%%%%%%%%%%%%%%%%%%%%%%%%%%%%%%%%%%%%%%%%%%%%%%%%%%%%%%%%%%%%%%%%
12
(i)
For all tests we assume H 0 : the graduated rates of mortality are the true
underlying mortality rates of the experience.
Chi-squared test: χ 2 = 10.5 with 3 degrees of freedom.
Significant at the 5% level since greater than 7.815.
Candidates who pointed out that the data given was only a sample from
the data used to fit the mathematical formula and thus argued that the
degrees of freedom should be >3 were also given credit.
NOW any two of
Range
Observed
Expected
-∞,-3
0
0
-3,-2
1
0.15
-2,-1
1
0.96
-1,0
3
2.39
0,+1
2
2.39
+1,+3
0
0.96
+2,+3
0
0.15
+3,∞
0
0
Individual standardized deviations: −2.14 and −1.97 (just) outside ±1.96
2 out of 7 lie outside ±1.960, the upper and lower 2.5% points of standard
normal, and both are negative
There are 2 positive and 5 negative deviations, when are equal number of
each and are expected if this is true.
There are 3 absolute deviations < 0.67 and 4 greater than 0.67, when an
equal number of each expected if H 0 is true.
These results cast some doubt on whether the individual standardized
deviations are approximately standard normal, and this whether H 0 is
true.
These results cast some doubt on whether the individual standardized
deviations are approximately standard normal, and thus whether H 0 is
true.
Group signs of deviations (stevens Test)L with n 1 =2 positive and n 2 =5
negative deviations , these are 2 groups of +ve signs.
Page 10%%%%%%%%%%%%%%%%%%%%%%%%%%%%%%%%%%%%%%%%%%%%%%5for September 2000
å
2
Prob (number of groups less than or equal to 2) =
t = 1
=
=
æ 1 ö æ 6 ö
ç ÷ ç ÷
è 0 ø è 1 ø
æ 7 ö
ç ÷
è 2 ø
+
æ n 1 − 1 ö æ n 2 + 1 ö
ç
֍
÷
è t − 1 øè t ø
æ n ö
ç ÷
è n 1 ø
æ 1 ö æ 6 ö
ç ÷ ç ÷
è 1 ø è 2 ø
æ 7 ö
ç ÷
è 2 ø
6 15
+
21 21
=1
Under null hypothesis, the probability of 2 or fewer groups of positive
signs is 1 so no reason to reject null hypothesis.
Serial Correlation Test:
4.07
= − 0.5814
7
Then
z =−
I
z i − z
z i + 1 − z
1
-1.56
-0.31
2
-0.31
0.15
And r 1 =
− 2.2219
6 × 8.1065
7
3
0.15
0.55
4
0.55
1.46
5
1.46
-1.39
6
-1.39
1.09
7
1.09
= − 0.3198
Then standardised r 1 = − 0.3198 7 = − 0.8460
which is standard normal if H 0 is true, so no reason to reject null
hypothesis.
Cumulative deviations:
169 − 187.03
186.62
= − 1.32 , which is standard normal if H 0 is true
Not significant at 5% level, or even at 10%.
Signs of deviations: 3 out of 7 positive.
Page 11%%%%%%%%%%%%%%%%%%%%%%%%%%%%%%%%%%%%%%%%%%%%%%5for September 2000
If H 0 is true
7
7!
1
(
2 )
k !(7 − k !)
giving
K
Probability
0
0.00781
1
0.05469
2
0.16406
3
0.27344
So not significant at 5% level.
Credit was not given for both the grouping of signs and serial correlations
test
Comments: The graduated rates are apparently too high over this age
range (see individual standardized deviations) but otherwise appear
adequate.
There was a misprint in the question. The standardised deviation of –0.43
should be +0.43. The majority of candidates used the standardised
deviations as given. Any candidate who recalculated the standardised
deviation and used +0.43 in subsequent tests received full credit. An upper
case Q in place of the correct lower case q in the column headings caused no
confusion to those who noticed it.
(ii)
H 0 :
the graduated rates = true underlying rates
Under H 0 , approx. ~ N ( E x q x , E x q x p x ) .
If independence can be assumed then
å ( θ
x
− E x q x ) ~ N (0,
x
approx. and
z =
å E q p )
x
x
x
å ( θ − E q ) ~ N (0,1) approx.
å E q p
x
x
x
x
x
x
Reject H 0 at 5% level of significance if z > 1.96
But independence assumption is highly dubious. Negative dependence is
very likely if one considers the whole age range (because graduation
θ x and
E x q x ). Can be
attempts to achieve approximate equality of
å
å
overcome by splitting the range into two or three equal ranges and testing
each separately.
