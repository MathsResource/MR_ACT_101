
\documentclass[a4paper,12pt]{article}

%%%%%%%%%%%%%%%%%%%%%%%%%%%%%%%%%%%%%%%%%%%%%%%%%%%%%%%%%%%%%%%%%%%%%%%%%%%%%%%%%%%%%%%%%%%%%%%%%%%%%%%%%%%%%%%%%%%%%%%%%%%%%%%%%%%%%%%%%%%%%%%%%%%%%%%%%%%%%%%%%%%%%%%%%%%%%%%%%%%%%%%%%%%%%%%%%%%%%%%%%%%%%%%%%%%%%%%%%%%%%%%%%%%%%%%%%%%%%%%%%%%%%%%%%%%%

\usepackage{eurosym}
\usepackage{vmargin}
\usepackage{amsmath}
\usepackage{graphics}
\usepackage{epsfig}
\usepackage{enumerate}
\usepackage{multicol}
\usepackage{subfigure}
\usepackage{fancyhdr}
\usepackage{listings}
\usepackage{framed}
\usepackage{graphicx}
\usepackage{amsmath}
\usepackage{chngpage}

%\usepackage{bigints}
\usepackage{vmargin}

% left top textwidth textheight headheight

% headsep footheight footskip

\setmargins{2.0cm}{2.5cm}{16 cm}{22cm}{0.5cm}{0cm}{1cm}{1cm}

\renewcommand{\baselinestretch}{1.3}

\setcounter{MaxMatrixCols}{10}

\begin{document}
104 S2004
610
A medium-sized UK pension scheme has carried out an investigation of the mortality
of its pensioners over the period 2000 2002.
\item  Explain why the crude rates will require graduation, and suggest with reasons
an appropriate method of graduation in this case.

\item  The data used to produce the crude rates, and the proposed graduated rates, are
shown below.
Age
x Central
Exposed to
Risk
E x c
60 64
65 69
70 74
75 79
80 84
85 89
90 94
95 99
100+ 1,388.9
1,188.8
880.5
841.6
402.8
123.9
27.9
10.0
7.5
Number
of Deaths
d x
10
17
28
34
41
19
7
3
2
Crude
Force of
Mortality
x 1⁄2
0.0072
0.0143
0.0318
0.0404
0.1018
0.1533
0.2509
0.3000
0.2666
Graduated
Force of
Mortality
x 1⁄2 Standardised
Deviation
z x
0.0061
0.0131
0.0262
0.0487
0.0839
0.1338
0.1975
0.2706
0.3455 0.5249
0.3615
1.0266
1.0912
1.2394
0.5949
0.6346
0.1787
0.3673
Test the proposed graduation for:
(a)
(b)
overall goodness of fit; and
bias

\item 
104 S2004
Comment on the use of the graduated rates to value the benefits payable from
the scheme.

[Total 14]
7

%%%%%%%%%%%%%%%%%%%%%%%%%
10
\item 
Reasons why crude rates will require graduation:
low data volumes at older ages
graduation means
data at nearby ages can be used to improve estimates
overall low data volumes mean that crude rates are likely to be subject to
relatively large sampling errors and therefore will not progress smoothly
with age, as we assume that the underlying rates do
Volume of data will be too small to attempt a direct graduation, ruling out use
of parametric formula.
Also, large sampling errors would make graphical graduation imprecise, and
in any case computationally inefficient. So suggest we graduate via some
simple relationship to a standard table, many of which exist based on large
volumes of data relating to similar lives. This approach also allows us to
compare our population with the population underlying the standard table.
\item 
H 0 : The crude rates come from a population in which the true underlying rates
of mortality are the graduated rates.
(a)
Test for overall goodness of fit (Chi-squared)
z i 2 ,
The test statistic is X
i
which under H 0 has a
2
distribution.
The degrees of freedom will be the number of age groups less some
allowance for the method of graduation, so at most 9 in this case.
The observed value of the test statistic
z i 2 is 5.11.
This is a one-sided test, so we wish to compare this to the upper 5%
point of the 2 distribution.
But 5.11 is below the upper 5% point for all degrees of freedom
except 1.
Hence we conclude there is insufficient evidence to reject H 0 .
Page 14 %%%%%%%%%%%%%%%%%%%%%%%%%%%%%%%%%%%%%
(b)
September 2004
%%%%%%%%%%%%%%%%%%%%%%%%%%%%%%%%%%%%%
Test for Bias
ALTERNATIVE 1
Test statistic is P = number of age groups for which the graduated rate
is below the crude rate = number of positive standardised deviations.
Under H 0 P ~ Bin(9, 0.5)
The observed value of P is 7.
Pr(P >= 7) = 0.5 9 (1 + 9 + 9 * 8 / 2) = 0.0898
This is a 2-sided test, so we would reject H 0 if this probability is below
0.025.
Since it is not, we conclude that we do not have significant evidence of
bias in the graduation.
ALTERNATIVE 2 (cumulative deviations)
d x E x c x 1⁄2
The test statistic is Z =
E x c x 1⁄2
Under H 0 , Z ~ N(0, 1)
Observed value of Z is
11 . 7192
= 0.9592
149 . 2808
This is a two-sided test so we would reject H 0 if |Z| > 1.96
Since it is not, we accept H 0 and conclude that we do not have
significant evidence of bias.
\item 
Use of graduated rates to value benefits
Areas for consideration:
Do we expect the experience of 2000 2002 to be typical for the scheme as
a whole?
Is there any reason why rates might have been higher / lower (e.g. harsh
winter?)
These rates would be used to value benefits payable many years into the
future.
Mortality rates have fallen consistently in the past, so we should make
Page 15 %%%%%%%%%%%%%%%%%%%%%%%%%%%%%%%%%%%%%
September 2004
%%%%%%%%%%%%%%%%%%%%%%%%%%%%%%%%%%%%%
some allowance for future changes also. If we do not there is a danger that
we will undervalue the benefits.
\end{document}
