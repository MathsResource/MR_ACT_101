
\documentclass[a4paper,12pt]{article}

%%%%%%%%%%%%%%%%%%%%%%%%%%%%%%%%%%%%%%%%%%%%%%%%%%%%%%%%%%%%%%%%%%%%%%%%%%%%%%%%%%%%%%%%%%%%%%%%%%%%%%%%%%%%%%%%%%%%%%%%%%%%%%%%%%%%%%%%%%%%%%%%%%%%%%%%%%%%%%%%%%%%%%%%%%%%%%%%%%%%%%%%%%%%%%%%%%%%%%%%%%%%%%%%%%%%%%%%%%%%%%%%%%%%%%%%%%%%%%%%%%%%%%%%%%%%

\usepackage{eurosym}
\usepackage{vmargin}
\usepackage{amsmath}
\usepackage{graphics}
\usepackage{epsfig}
\usepackage{enumerate}
\usepackage{multicol}
\usepackage{subfigure}
\usepackage{fancyhdr}
\usepackage{listings}
\usepackage{framed}
\usepackage{graphicx}
\usepackage{amsmath}
\usepackage{chngpage}

%\usepackage{bigints}
\usepackage{vmargin}

% left top textwidth textheight headheight

% headsep footheight footskip

\setmargins{2.0cm}{2.5cm}{16 cm}{22cm}{0.5cm}{0cm}{1cm}{1cm}

\renewcommand{\baselinestretch}{1.3}

\setcounter{MaxMatrixCols}{10}

\begin{document}
%%---  Question 9
In a clinical trial, 100 patients were treated with a new vaccine for a serious but
treatable disease and then observed over a 2 year period. The results were as follows:
Withdrew from study: 13 patients
Contracted disease during study: 8 patients
Still clear of disease at end of study: 79 patients
The times (in months) at which patients withdrew or contracted the disease were as
follows:
Withdrew
Contracted disease
1
2
3
2
3
2
4
6
4
6
5
8
7
8
7 9
17
10 11 13 15
\item  Calculate the Kaplan-Meier estimate of the survival function based on this
data.

\item  A control group of patients who were not treated with the vaccine, was also
observed over 2 years. The Kaplan-Meier estimate of the survival function
and the implied hazard function for this control group are:
Survival
0 t < 5
5 t < 8
8 t < 12
t 12
1.0
0.92
0.895
0.87
t Hazard function
5
8
12 0.08
0.02717
0.02793
The cost of treating a patient who contracts the disease is \$1,000, paid as a
lump sum at the time of diagnosis. The vaccine costs \$20 per patient.
Compare the expected present value of the cost over 2 years of treating a
vaccinated patient with that for a patient not given the vaccine. Hence
calculate the expected saving made by vaccinating a patient.
Basis: Interest: 4% per annum throughout

[Total 11]
%%%%%%%%%%%%%%%%%%
9
September 2004
%%%%%%%%%%%%%%%%%%%%%%%%%%%%%%%%%%%%%
\item 
t j n j d j c j
0
2
6
8
17 100
99
91
87
80 0
3
2
2
1 1
5
2
5
79
S(t) =
t
0<= t < 2
2<= t < 6
6<= t < 8
8<= t < 17
t>= 17
\item 
j
= d j /n j 1 -
0
1/33
2/91
2/87
1/80 1
32/33
89/91
85/87
79/80
j
(1 - j )
1
0.9697
0.9484
0.9266
0.9150
The expected present value of the cost over 2 years of treating a patient who
contracts the disease is given by
2
X = \$1,000 .
S ( t ). h ( t ). v t
0
For a patient who has not been vaccinated, this is
X 1 = \$1,000 x (1.0 x 0.08 x 1.04 -0.41666 + 0.92 x 0.02717 x 1.04 -0.6667
+ 0.895 x 0.02793 x 1.04 -1 )
= \$1,000 x 0.12708
= \$127.08
And, similarly for a patient who has been vaccinated
X 2 = \$1,000 1.0
1
1.04
33
0.9484
0.1666
2
1.04
87
0.9697
0.6667
2
1.04
91
0.9266
0.5
1
1.04
80
1.4166
= \$1,000 x 0.08320
= \$83.20
Page 13 %%%%%%%%%%%%%%%%%%%%%%%%%%%%%%%%%%%%%
September 2004
%%%%%%%%%%%%%%%%%%%%%%%%%%%%%%%%%%%%%
So, the total expected present value of the cost of treating a vaccinated patient
is \$83.20 + \$20(cost of vaccine) = \$103.20.
This is less than the expected cost of treating a patient without the vaccine and
the expected saving is \$23.88.
\end{document}
