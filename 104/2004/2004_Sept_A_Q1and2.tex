
\documentclass[a4paper,12pt]{article}

%%%%%%%%%%%%%%%%%%%%%%%%%%%%%%%%%%%%%%%%%%%%%%%%%%%%%%%%%%%%%%%%%%%%%%%%%%%%%%%%%%%%%%%%%%%%%%%%%%%%%%%%%%%%%%%%%%%%%%%%%%%%%%%%%%%%%%%%%%%%%%%%%%%%%%%%%%%%%%%%%%%%%%%%%%%%%%%%%%%%%%%%%%%%%%%%%%%%%%%%%%%%%%%%%%%%%%%%%%%%%%%%%%%%%%%%%%%%%%%%%%%%%%%%%%%%

\usepackage{eurosym}
\usepackage{vmargin}
\usepackage{amsmath}
\usepackage{graphics}
\usepackage{epsfig}
\usepackage{enumerate}
\usepackage{multicol}
\usepackage{subfigure}
\usepackage{fancyhdr}
\usepackage{listings}
\usepackage{framed}
\usepackage{graphicx}
\usepackage{amsmath}
\usepackage{chngpage}

%\usepackage{bigints}
\usepackage{vmargin}

% left top textwidth textheight headheight

% headsep footheight footskip

\setmargins{2.0cm}{2.5cm}{16 cm}{22cm}{0.5cm}{0cm}{1cm}{1cm}

\renewcommand{\baselinestretch}{1.3}

\setcounter{MaxMatrixCols}{10}

\begin{document}1
Describe the benefit which has present value random variable Z below; here T denotes
the future lifetime of a life aged x.
Z =
v T a n
0
T
if T n
if T n

2
A life office issued a term assurance policy to a life aged exactly 45 years. The term
of the policy is 20 years and the benefit, payable at the end of the year of death, is
\$20,000. The policy is funded by a level annual premium P, payable in advance.
\item  Using standard actuarial notation, write down the equation of value for this
policy.
%%---  Question 
\item  Write down expressions for the prospective and retrospective policy values of
this policy when the life is aged exactly 45 + t years. Show that the policy
values are equal if calculated on the same basis as the premiums.

%%%%%%%%%%%%%%%%%%%%%%%%%%%%%%%%%%%%%%%%%%%%%%%%%%%%%%%%%%%%%%%%%%%%%
11
\item  Sex
Smoker/non-smoker status
Occupation
Known impairments
Geographical location
\item  ALTERNATIVE 1
Using AM92, we find that e 39
39 and e 40
40
so 39 < a < 40.
If
e a
a , then
l x dx
a
a
. (*)
l a
Assuming a uniform distribution of deaths (UDD) between ages 39 and 40,
then
l a l 39 ( a 39) d 39
So that, substituting into (*) we obtain
l x dx
a
a
(**)
l 39 ( a 39) d 39
To evaluate l x dx , note that
a
40
l x dx
a
40
l x dx
a
l x dx
40
l x dx e 40 l 40 ,
a
and that, assuming UDD,
40
l x dx
a
40 a
( l a l 40 ) e 40 l 40
2
w
Therefore, substituting for
l x dx in (**) we obtain
a
Page 16 %%%%%%%%%%%%%%%%%%%%%%%%%%%%%%%%%%%%%
September 2004
%%%%%%%%%%%%%%%%%%%%%%%%%%%%%%%%%%%%%
40 a
( l 39 ( a 39) d 39 l 40 ) e 40 l 40
2
a =
l 39 ( a 39) d 39
From the tables, e 40 = 39.064, so using the approximation given e 40 = 39.064
+ 0.5 = 39.564. Inserting this, the tabulated values for l 39 , l 40 and d 39 and the
value 39.778 for a into the equation, the right hand side becomes:
40 a
(9,864.8688 (8.5824)(0.778) 9,856.2863) (39.564)(9,856.2863)
2
9,864.8688 (0.778)(8.5824)
0.222
(9,858.1917 9,856.2863) 389,954.1112
= 2
9,858.1917
= 39.778 (to 3 decimal places)
ALTERNATIVE 2
Using AM92, we find that e 39
39 and e 40
40
so 39 < a < 40.
To evaluate e x t (0
t
1) assuming UDD between exact ages
x and x+1
consider that e x t is a weighted average of the complete expectations of life of
those who survive to exact age x+1 and those who die between x+t and exact
age x+1.
The proportion of those alive at age x+t who survive to
1 t p x t .
exact age x+1 is
o
These lives will have a complete expectation of life equal to 1 t e x 1 .
The proportion of those alive at age x+t who die before exact age x+1 is
1- 1 t p x t .
1
These lives will live, on average, 1 t more years.
2
Therefore
o
e x
t
1
1 t 1
2
o
1 t
p x
t
1 t e x
1 1 t
p x t . (*)
Page 17 %%%%%%%%%%%%%%%%%%%%%%%%%%%%%%%%%%%%%
September 2004
%%%%%%%%%%%%%%%%%%%%%%%%%%%%%%%%%%%%%
In our case, we have t = 0.778, and x = 39. From the tables, e 40 = 39.064,
so using the approximation given, e 40 = 39.064 + 0.5 = 39.564.
Note also that under UDD,
1 t
p x
1
t
1 t
q x
t
1
(1 t ) q x
1 tq x
1 q x
.
1 tq x
From the tables, q 39 = 0.00087. Therefore
1 0.00087
0.99981
1 t p x t =
1 (0.00087)(0.778)
Substituting into (*) above, therefore, produces
o
e 39.778
=
1
1 0.778 (1 0.99981)
2
1 0.778 39.564 (0.99981)
1
(0.222)(0.00019) (39.786)(0.99981)
2
= 39.778 (to 3 d.p.)
Therefore we have shown that a = 0.778.
\item 
In this case, 2a = 79.556, so the birthday nearest to this age is the 80 th .
An annuity paid weekly can be treated as being paid continuously, so the
premium for the annuity, P a , is given by the equation
P a 10, 000( a 40:40 a 40:25 )
Evaluation proceeds as follows: for the first term we have
a 40:40 a 40:40 0.5(1 40 40 p 40 )
a 40
D 80
a 80 0.5(1
D 40
40 l 80
l 40
)
and, from the AM92 tables this is evaluated as
228.48
5266.4604
20.005
(6.818) 0.5 1 (0.20829)
2052.96
9856.2863
20.005 (0.111293)(6.818) 0.5[1 (0.20829)(0.534325)]
= 20.005 0.75880 0.44435
= 18.80185.
For the second term we have
a 40:25
Page 18
a 40:25
0.5(1
25 l 65
l 40
) . %%%%%%%%%%%%%%%%%%%%%%%%%%%%%%%%%%%%%
September 2004
%%%%%%%%%%%%%%%%%%%%%%%%%%%%%%%%%%%%%
From tables, this is evaluated as
8821.2612
)
9856.2863
15.884 0.5[1 (0.37512)(0.89499)]
15.884 0.5(1 0.37512
15.884 0.33214
15.5519
Therefore
P a
10, 000(18.80185 15.5519)
\$32,500.
For the pure endowment, the premium, P e , is given by
P e
100, 000 v 40 40 p 40
which, using the figures in the previous evaluation, is
5266.4604
100,000 x (0.20829)
= \$11,129.
9856.2863
So the total single premium payable is
\$11,129 + \$32,500 = \$43,629.
\item 
When pricing these annuities and pure endowments, companies will need to
consider the likely risk of future mortality improvements.
If mortality improves, then the expected outgo will increase. Premiums
should be set to take this into account, for otherwise the office might become
insolvent.
In addition, offices might wish to use a mortality basis that takes into account
factors other than those listed in the solution to part \item , such as the policy size,
the level of underwriting and the sales channel.
If an office does not take these factors into account, and if mortality is related
to them, then the premiums charged will be too high for the low risks (who
will take their business elsewhere to other offices who do take these factors
into account) and too low for the high risks (resulting in the possibility of
insolvency).
END OF %%%%%%%%%%%%%%%%%%%%%%%%%%%%%%%%%%%%%
Page 19
3
Page 6
\item  Females who received the existing treatment at the time of diagnosis.
\item  (a)
h i t
h 0 t exp
0.5
6
12
0.01 1
0.05 1 %%%%%%%%%%%%%%%%%%%%%%%%%%%%%%%%%%%%%
September 2004
%%%%%%%%%%%%%%%%%%%%%%%%%%%%%%%%%%%%%
= h 0 t e 0.21
t
(b)
S t
exp
h s
ds
0
t
exp
h 0 s
e
0.21
ds
0
t
exp
0.21
e
h 0 s
ds
0
e
t
exp
h 0 s
0.21
ds
0
\item 
ALTERNATIVE 1
For the female life:
h i t
h 0 t exp 0.5 0
exp
0.05 0
h 0 t e 0.01
e 0.01
5
S 5
0.01 1
h 0 s
ds
0.75
0
5
exp
h 0 s
ds
0.75
e
0.01
0
So, for the male life:
e 0.21
5
S 5
exp
h 0 s
ds
0
0.75
e
0.01
e 0.21
= 0.7037
ALTERNATIVE 2
Page 7 %%%%%%%%%%%%%%%%%%%%%%%%%%%%%%%%%%%%%
September 2004
%%%%%%%%%%%%%%%%%%%%%%%%%%%%%%%%%%%%%
Defining the hazard for the male h 1 and the hazard for the female h 2 , the ratio
of hazards is
h 0 5 e 0.21
h 1
h 2
h 0 5 e 0.01
e 0.20
S 1
S 2
e 0.20
0.75
e 0.20
0.7037

%%%%%%%%%%%%%%%%%%%%%%%%%%%%%%%%%%%%%
1
This is a benefit of 1 p.a. payable continuously between the death of the life and the
date that would have been the life s x + n th birthday.
If the life survives to age x + n, then no benefit is payable.
2
1
20, 000 A 45:20
\item  Pa 45:20
\item  ALTERNATIVE 1
The prospective policy value at duration t (when the life is aged 45 + t) is
1
Pa 45 t :20 t .
t V pro 20, 000 A 45 t :20 t
The retrospective policy value is
P
a 45: t 1
20, 000 A 45:
t
v t t p 45 v t t p 45
.
To show that they are equal, note that
t
a 45:20 a 45: t
t p 45 a 45 t :20 t ,
so that
a 45
a 45:20
t :20 t
t
a 45: t
t p 45
.
Substituting in the above expression for t V pro yields
t V pro
P
1
20, 000 A 45
t :20 t
t
t
p 45
( a 45:20
a 45: t )
But from the solution to \item  above,
1
20, 000A 45:20
a 45:20
,
P
so that
t V pro
1
20, 000 A 45
t :20 t
1
20, 000 A 45:20
P
t
t
p 45
Since
A 1 45:20
then
Page 4
1
A 45:
t
t
t
p 45 A 1 45
t :20 t
,
P
a 45: t . %%%%%%%%%%%%%%%%%%%%%%%%%%%%%%%%%%%%%
t V pro
=
1
20, 000 A 45
t :20 t
t
%%%%%%%%%%%%%%%%%%%%%%%%%%%%%%%%%%%%%
20, 000
t
t
1
p 45 A 45
t :20 t
P
t p 45
a 45: t
1
20, 000 A 45:
t
Pa 45: t
t
20, 000 A 1 45: t
P
t
September 2004
t
p 45
t
p 45
which is the retrospective policy value.
ALTERNATIVE 2
The prospective policy value at duration t (when the life is aged 45 + t) is
1
Pa 45 t :20 t .
t V pro 20, 000 A 45 t :20 t
The retrospective policy value may be written
D 45
Pa 45: t 20, 000 A 1 45: t .
D 45 t
From part \item  we have
1
Pa 45:20 20, 000 A 45:20
0 . (*)
Multiplying both sides of equation (*) by
D 45
Pa 45:20
D 45 t
20, 000 A 1 45:20
D 45
we have
D 45 t
0 .
Adding the left-hand side of this equation to the prospective policy value
produces
D 45
1
Pa 45 t :20 t
Pa 45:20 20, 000 A 1 45:20
t V pro 20, 000 A 45 t :20 t
D 45 t
so that
t V pro
D 45
1
20, 000 A 45
D 45 t
D 45
Pa 45: t
D 45 t
t :20 t
D 45 t
D 45
20, 000 A 1 45:20
Pa 45:20
Pa 45
t :20 t
D 45 t
D 45
20, 000 A 1 45: t
which is the same as the retrospective policy value.
Page 5 %%%%%%%%%%%%%%%%%%%%%%%%%%%%%%%%%%%%%
September 2004
%%%%%%%%%%%%%%%%%%%%%%%%%%%%%%%%%%%%%
ALTERNATIVE 3
Using commutation functions, we have
M 45 t M 65
N
N 65
P 45 t
t V pro 20, 000
D 45 t
D 45 t
and
t V retro
P
D 45 N 45 N 45
D 45 t
D 45
N
N 45
P 45
D 45 t
From part \item  we have
1
Pa 45:20 20, 000 A 45:20
t
20000
D 45 M 45 M 45
D 45 t
D 45
M 45 M 45
20000
D 45 t
t
t
.
t
0 ,
which, using commutation functions, may be written as
P ( N 45 N 65 ) 20, 000( M 45 M 65 ) 0 .
Therefore, dividing this equation by D 45+t produces
N
N 65
M 45 M 65
P 45
20, 000
0 .
D 45 t
D 45 t
Subtracting this equation from the retrospective policy value
produces
N
N 45 t
N
N 65
P 45
P 45
t V retro
D 45 t
D 45 t
20000
M 45 M 45
D 45 t
t
which may be simplified to
M 45 t M 65
t V retro 20000
D 45 t
20000
P
M 45 M 65
D 45 t
N 45 t N 65
D 45 t
which is the same as the prospective policy value.
