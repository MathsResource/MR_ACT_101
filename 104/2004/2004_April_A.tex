\documentclass[a4paper,12pt]{article}

%%%%%%%%%%%%%%%%%%%%%%%%%%%%%%%%%%%%%%%%%%%%%%%%%%%%%%%%%%%%%%%%%%%%%%%%%%%%%%%%%%%%%%%%%%%%%%%%%%%%%%%%%%%%%%%%%%%%%%%%%%%%%%%%%%%%%%%%%%%%%%%%%%%%%%%%%%%%%%%%%%%%%%%%%%%%%%%%%%%%%%%%%%%%%%%%%%%%%%%%%%%%%%%%%%%%%%%%%%%%%%%%%%%%%%%%%%%%%%%%%%%%%%%%%%%%

\usepackage{eurosym}
\usepackage{vmargin}
\usepackage{amsmath}
\usepackage{graphics}
\usepackage{epsfig}
\usepackage{enumerate}
\usepackage{multicol}
\usepackage{subfigure}
\usepackage{fancyhdr}
\usepackage{listings}
\usepackage{framed}
\usepackage{graphicx}
\usepackage{amsmath}
\usepackage{chngpage}

%\usepackage{bigints}
\usepackage{vmargin}

% left top textwidth textheight headheight

% headsep footheight footskip

\setmargins{2.0cm}{2.5cm}{16 cm}{22cm}{0.5cm}{0cm}{1cm}{1cm}

\renewcommand{\baselinestretch}{1.3}

\setcounter{MaxMatrixCols}{10}

\begin{document}

1
A life office has a portfolio of whole of life assurance policies, issued to lives aged
exactly 60. The benefit of \$3,000 under each policy is payable at the end of year of
death. No further premiums are payable in respect of these policies.
The reserve held per policy after 5 years, V 5 , is \$1,200.
Assuming a rate of interest of 6% and mortality rates given below, calculate V 6 , the
reserve held per policy after 6 years.
Age q x
65
66
67 0.020
0.025
0.030

%%%%%%%%%%%%%%%%%%%%%%%%%%%%%%%%%%%%%%55555
1
April 2004
\newpage
%%%%%%%%%%%%%%%%%%%%%%%%%%%%%%%%%%%%%
We can use the recursive formula:
V 5 (1 + i) = p 65 . V 6 + q 65 . 3,000
So that
V 6
= (V 5 (1 + i)
= (1200(1.06)
=
q 65 . 3,000)
0.02
1
p 65
3,000)
1
(1 0.02)
1, 212
0.98
= 1,236.73
= \$1,237 to nearest \$.
\end{document}
