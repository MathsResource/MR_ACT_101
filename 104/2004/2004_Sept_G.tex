
\documentclass[a4paper,12pt]{article}

%%%%%%%%%%%%%%%%%%%%%%%%%%%%%%%%%%%%%%%%%%%%%%%%%%%%%%%%%%%%%%%%%%%%%%%%%%%%%%%%%%%%%%%%%%%%%%%%%%%%%%%%%%%%%%%%%%%%%%%%%%%%%%%%%%%%%%%%%%%%%%%%%%%%%%%%%%%%%%%%%%%%%%%%%%%%%%%%%%%%%%%%%%%%%%%%%%%%%%%%%%%%%%%%%%%%%%%%%%%%%%%%%%%%%%%%%%%%%%%%%%%%%%%%%%%%

\usepackage{eurosym}
\usepackage{vmargin}
\usepackage{amsmath}
\usepackage{graphics}
\usepackage{epsfig}
\usepackage{enumerate}
\usepackage{multicol}
\usepackage{subfigure}
\usepackage{fancyhdr}
\usepackage{listings}
\usepackage{framed}
\usepackage{graphicx}
\usepackage{amsmath}
\usepackage{chngpage}

%\usepackage{bigints}
\usepackage{vmargin}

% left top textwidth textheight headheight

% headsep footheight footskip

\setmargins{2.0cm}{2.5cm}{16 cm}{22cm}{0.5cm}{0cm}{1cm}{1cm}

\renewcommand{\baselinestretch}{1.3}

\setcounter{MaxMatrixCols}{10}

\begin{document}
8
On 1 January 2002 a pension scheme has 100 members aged exactly 75, each in
receipt of an annual pension of \$10,000 payable in arrears. In addition, the scheme
pays a lump sum of \$2,000 on the death of a member, the lump sum being payable at
the end of the year of death. No further premiums are payable in respect of these
benefits.
Of these members, 7 die during 2002. Calculate the mortality profit or loss to the
scheme for 2002.
Basis: Mortality: PMA92C20
Interest: 4% per annum throughout
Expenses are ignored
104 S2004
5
[10]

\newpage
\begin{itemize}
\item We need to calculate the reserve V needed for each life who survives to age 76. This
is just the value of the future benefits.
The value of the survival benefits is
\$10, 000 a 76 = \$10, 000 (9.049 1)
= \$80, 490
The value of the death benefits can be calculated via premium conversion:
\$2, 000 A 76 = \$2, 000(1 da 76 )
= \$2, 000(1 0.04 /1.04 9.049)
= \$1,304
\item The total reserve required is therefore \$80,490 + \$1,304 = \$81,794.
The death strain at risk is equal to
S - (V t + R)
\$2,000 - (\$81,794 + \$10,000) = -\$89,794
The expected number of deaths during 2002 is 100 x q 75
= 100 x 0.028121 = 2.8121
The mortality profit is therefore equal to
Expected death strain - Actual death strain
= (Expected deaths - actual deaths) x DSAR
= (2.8121 - 7) x (-\$89,794)
= \$376,048.
\item An alternative method provided by a number of candidates used
Mortality profit =
reserve at beginning of year plus interest
- actual payments on death and in pensions
- end of year reserves
\end{itemize}
This method was given full credit.
Page 12 %%%%%%%%%%%%%%%%%%%%%%%%%%%%%%%%%%%%%
\end{document}
