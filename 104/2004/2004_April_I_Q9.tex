
\documentclass[a4paper,12pt]{article}

%%%%%%%%%%%%%%%%%%%%%%%%%%%%%%%%%%%%%%%%%%%%%%%%%%%%%%%%%%%%%%%%%%%%%%%%%%%%%%%%%%%%%%%%%%%%%%%%%%%%%%%%%%%%%%%%%%%%%%%%%%%%%%%%%%%%%%%%%%%%%%%%%%%%%%%%%%%%%%%%%%%%%%%%%%%%%%%%%%%%%%%%%%%%%%%%%%%%%%%%%%%%%%%%%%%%%%%%%%%%%%%%%%%%%%%%%%%%%%%%%%%%%%%%%%%%

\usepackage{eurosym}
\usepackage{vmargin}
\usepackage{amsmath}
\usepackage{graphics}
\usepackage{epsfig}
\usepackage{enumerate}
\usepackage{multicol}
\usepackage{subfigure}
\usepackage{fancyhdr}
\usepackage{listings}
\usepackage{framed}
\usepackage{graphicx}
\usepackage{amsmath}
\usepackage{chngpage}

%\usepackage{bigints}
\usepackage{vmargin}

% left top textwidth textheight headheight

% headsep footheight footskip

\setmargins{2.0cm}{2.5cm}{16 cm}{22cm}{0.5cm}{0cm}{1cm}{1cm}

\renewcommand{\baselinestretch}{1.3}

\setcounter{MaxMatrixCols}{10}

\begin{document}
%%---  Question 9
An investigation into the mortality of young adult males in a developed country has
been undertaken. The table below shows an extract from the results.
\item 
Age Exposed-to-risk Observed deaths Standardised deviation
z x
18
19
20
21
22
23
24
25
26
27 34,000
33,000
29,500
30,000
25,500
24,000
17,000
23,500
18,000
14,000 40
35
27
26
22
19
13
20
12
11 1.9159
1.4541
0.4460
0.0394
0.1469
0.5106
0.5067
0.0467
0.8437
0.2609
Someone suggests to you that the underlying mortality of these young men is
the same as that in English Life Table 15 (Males) and that you should test this
using the chi-squared test.
(a)
(b)
(c)
(d)
Define the standardised deviation.
State the null hypothesis to be tested.
State the test statistic and its distribution under the null hypothesis.
Carry out the test.

\item 
(a) Describe two possible differences between the underlying mortality of
the men in the sample and English Life Table 15 (Males) which the
chi-squared test might fail to detect.
(b) For each of the differences you describe, carry out a different test to
see if the observed experience is significantly different from that of
English Life Table 15 (Males). For each test you carry out, state
explicitly what the test is designed to detect, and state your
conclusions.

\item  Comment on your results in \item  and \item  above.
\item  Explain how you would graduate the observed experience by reference to a
standard mortality table.

[Total 24]
END OF PAPER
104 A2004
8
%%%%%%%%%%%%%%%%%%%%%%%%%%%5

Page 13 %%%%%%%%%%%%%%%%%%%%%%%%%%%%%%%%%%%%%
9
(i)
(a)
April 2004
%%%%%%%%%%%%%%%%%%%%%%%%%%%%%%%%%%%%%
If the observed number of deaths at age x is x , the exposed-to-risk at
age x is E x , and the mortality rate in English Life Table 15 (Males) at
age x is q x , then the standardised deviation in age-group x, z x , is given
by the formula
z x =
E x q x
x
,
E x q x
using the approximation E x q x
E x q x (1 q x ) ,
Alternatively, if the observed number of deaths at age x is
x ,
the
central exposed-to-risk at age x is E x c , and the force of mortality in
English Life Table 15 (Males) at age x is x 0.5 , then the standardised
deviation in age-group x, z x , is given by the formula
x
z x =
E x c
x 0.5
c
E x x 0.5
.
(b) The null hypothesis is that the underlying mortality of the lives in the
investigation is that of English Life Table 15 (Males).
(c) The test statistic is
z x 2
2
m ,
x
where m is the number of age groups (m = 10 in our case), because we
are comparing an experience with a standard table.
(d)
The calculations are shown in the table below
Age x E
18
19
20
21
22
23
24
25
26
27
Sums 34,000
33,000
29,500
30,000
25,500
24,000
17,000
23,500
18,000
14,000
x
40
35
27
26
22
19
13
20
12
11
q x E x q x z x z x 2
0.00087
0.00083
0.00084
0.00086
0.00089
0.00089
0.00088
0.00086
0.00085
0.00085 29.58
27.39
24.78
25.80
22.70
21.36
14.96
20.21
15.30
11.90
213.98 1.9159
1.4541
0.4460
0.0394
0.1469
0.5106
0.5067
0.0467
0.8437
0.2609
1.5399 3.6707
2.1144
0.1989
0.0016
0.0216
0.2607
0.2567
0.0022
0.7118
0.0681
7.3067
z x 2 = 7.3067 .
Using the data in the table above,
x
Page 14 %%%%%%%%%%%%%%%%%%%%%%%%%%%%%%%%%%%%%
April 2004
%%%%%%%%%%%%%%%%%%%%%%%%%%%%%%%%%%%%%
2
The critical value of the 10
distribution at the 5% level is 18.31,
which is much greater than the calculated value.
So we accept the null hypothesis.
On the basis of this test, we conclude that the underlying mortality of
the lives in our investigation is represented by English Life Table 15
(Males), and that the suggestion seems to be true.
(ii)
(a)
(1) There could be a few large deviations offset by a lot of very small
deviations. The chi-squared test will be satisfied, but the data do
not satisfy the distributional assumptions which underlie it.
(2) The graduation might be biased above or below the data by a
smallish amount at all ages.
(3) Even if the graduation is not biased as a whole, there could be
significant groups of consecutive ages (runs or clumps) over
which it is biased up or down.
Full marks were available for any two of these.
(b)
The appropriate tests are as follows.
For difference (1) (few large deviations): standardised deviations test.
For difference (2) (overall bias): signs test; cumulative deviations test
over whole age range.
For difference (3) (runs or clumps): grouping of signs (Stevens ) test;
serial correlation test.
Candidates were expected to perform one test for each of the
difference they identified in part (a).
%%%%%%%%%%%%%%%%%%%%%%%%%%%%%%%%%%%%%%%%%%%%%%%%%%%%%%%%%%%%%%%%%%%%%%%%%%%%%%%%%%%%%5
\newpage
STANDARDISED DEVIATIONS TEST
This tests for the possibility that there are a small number of age
groups with large differences between the mortality rates in the
investigation and the standard table.
The z x s comprise m independent samples from a Normal (0,1)
distribution. We can compare the expected and actual number of z x in
the following intervals
Interval
Expected number
Actual number
(- , -3)
0
0
(-3, -2)
0.2
0
(-2, -1)
1.4
0
(-1, 0)
3.4
6
(0, 1)
3.4
2
(1, 2)
1.4
2
(2, 3)
0.2
0
(3, )
0
0
Therefore under the null hypothesis we should expect fewer than 1 in
20 to be > 2 in absolute magnitude.
In this case none of the z x s exceeds 2 in absolute value.
Page 15 %%%%%%%%%%%%%%%%%%%%%%%%%%%%%%%%%%%%%
April 2004
%%%%%%%%%%%%%%%%%%%%%%%%%%%%%%%%%%%%%
So we accept the null hypothesis.
SIGNS TEST
This tests for the possibility of the mortality rates in the investigation
being systematically lower or higher than those in the standard table.
Let P be the number of z x s that are positive.
Then under the null hypothesis, P ~ Binomial(10, 0.5).
We have 4 positive signs. The probability of getting 4 or fewer
positive signs if the null hypothesis is true is
10
10
1
10
1
10
4 2
3 2
10! 1
10! 1
=
10
4!6! 2
3!7! 2 10
= (210 120 45 10
10
1
10
10
1
10
2 2
1 2
10! 1
10! 1
10
2!8! 2
1!9! 2 10
1)0.0009765625
10
1
0 2 10
1
2 10
= 0.37695
[Alternatively, candidates could just evaluate the probability of getting
exactly 4 positive deviations, which is 0.205].
This is greater than 0.025 (2-tailed test)
We accept the null hypothesis.
CUMULATIVE DEVIATIONS TEST
When using the whole age range, this tests for the possibility of the
mortality rates in the investigation being systematically lower or higher
than those in the standard table.
Under the null hypothesis,
m
(
x
E x q x s )
x 1
m
Normal(0,1)
E x q x s
x 1
Using the data in the table, we have
Page 16 %%%%%%%%%%%%%%%%%%%%%%%%%%%%%%%%%%%%%
m
(
x
E x q x s )
x 1
m
April 2004
=
E x q x s
11.025
213.98
%%%%%%%%%%%%%%%%%%%%%%%%%%%%%%%%%%%%%
0.75369
x 1
Since both positive and negative cumulative deviations are of interest
we use a two-tailed test.
Since |0.75369|<1.96,
we accept the null hypothesis.
GROUPING OF SIGNS TEST
This tests for runs of deviations of the same sign, that is for subsections
of the age range for which the mortality rates of lives in the
investigation are systematically lower or higher than the rates in the
standard table.
Let G be the number of groups of positive z x s, n 1 be the number of
positive z x s and n 2 be the number of negative z x s.
In our case G = 1, n 1 = 4 and n 2 = 6.
Then the probability of getting 1 group of positive signs is
3
7
7!
0 1
7
= 1!6! =
= 0.0333
10! 210
10
4!6!
4
Using a one-tailed test, since only small values of G are of interest, we
find that 0.0333 < 0.05.
We reject the null hypothesis.
SERIAL CORRELATIONS TEST
This tests for runs of deviations of the same sign, that is for subsections
of the age range for which the mortality rates of lives in the
investigation are systematically lower or higher than the rates in the
standard table.
The correlation coefficient at lag 1 is
Page 17 %%%%%%%%%%%%%%%%%%%%%%%%%%%%%%%%%%%%%
m 1
( z x
r 1 =
x 1
m 1
April 2004
z * )( z x
z * )
( z x
1
z ** )
.
m 1
x 1
%%%%%%%%%%%%%%%%%%%%%%%%%%%%%%%%%%%%%
( z x
1
z ** )
x 1
The calculations are shown in the table below.
z x
Age
x
18
19
20
21
22
23
24
25
26
27
z x
1.9159
1.4541
0.4460
0.0394
0.1469
0.5106
0.5067
0.0467
0.8437
0.2609
z *
z x
1.7158
1.2540
0.2459
0.1607
0.3470
0.7107
0.7068
0.2468
1.0438
0.4610
z **
1.4959
0.4878
0.0812
-0.1051
-0.4688
-0.4649
-0.0049
-0.8019
-0.2191
z * = 0.2001
z ** =
0.0418
Age
x
18
19
20
21
22
23
24
25
26
Total
Page 18
( z x
z * ) 2
( z x
1
z ** ) 2
( z x
z * )( z x
1
2.9440
1.5725
0.0605
0.0258
0.1204
0.5051
0.4996
0.0609
1.0895 2.2377
0.2379
0.0066
0.0111
0.2198
0.2162
0.0000
0.6431
0.0480 2.5666
0.6117
0.0200
0.0169
0.1627
0.3304
0.0035
0.1979
0.2287
6.8783 3.6203 4.1384
z ** ) %%%%%%%%%%%%%%%%%%%%%%%%%%%%%%%%%%%%%
April 2004
%%%%%%%%%%%%%%%%%%%%%%%%%%%%%%%%%%%%%
Hence
m 1
( z x
r 1 =
x 1
m 1
( z x
x 1
z * )( z x
z * )
z ** )
1
=
m 1
( z x
1
z ** )
4.1383
(6.8783)(3.6203)
= 0.8293.
x 1
Now r 1 m ~ Normal(0, 1).
Since m = 10, we have r 1 m = 3.1623 x 0.8293 = 2.6225.
Using a one-tailed test (since we are only interested in positive serial
correlations), the probability of getting a value as high as 2.6225 is
(1 0.9957) = 0.0043
Therefore we have evidence to reject the null hypothesis at the 5%
level.
(iii)
The results of the tests suggest that the underlying mortality of the lives in the
investigation is, overall, not significantly different from ELT 15 (Males).
There are no individual ages with suspiciously large deviations.
Neither does there appear to be any overall bias.
However, although, overall, the experience fits ELT 15 well, there is a
problem with the shape of the mortality curve. At younger ages the observed
mortality is systematically higher than that of ELT 15 (Males), whereas at all
ages above 21 years it is lower.
The serial correlations and grouping of signs (Stevens ) tests, which are
designed to pick up this kind of difference, thus lead us to reject the null
hypothesis that the underlying mortality is the same as ELT 15 (Males).
It seems that the observed experience has a much more pronounced accident
hump in it than ELT 15 (Males).
Therefore, ELT 15 (Males) is probably not a good standard table to use for
graduating this experience.
In fact, ELT 14 (Males), which is based on 1980-82 data, might be better,
since the accident hump is more obvious in ELT 14 than ELT 15.
Page 19 %%%%%%%%%%%%%%%%%%%%%%%%%%%%%%%%%%%%%
(iv)
April 2004
%%%%%%%%%%%%%%%%%%%%%%%%%%%%%%%%%%%%%
Choose an appropriate standard table.
Seek a simple function relating the observed experience to that of the standard
table.
o
o
If, for example, q x and
s
x
x refer
to the observed mortality at age x, and q x s and
refer to the corresponding mortality in the standard table, we might choose
o
q x = a bq x s ,
or
o
x
=
s
x
k
where a, b and k are parameters.
o
To find a suitable function we can plot q x against q x s to check for a linear
o
log(1 q x ) against log(1 q x s ) to check for a
relationship in the q x s , or
linear relationship in the
x s.
If a simple function cannot be found, then a different standard table should be
chosen and the procedure repeated.
Once a possible relationship has been identified, the best-fitting parameters
may be found using maximum likelihood or weighted least squares methods.
In weighted least squares, natural weights would be the exposed-to-risk at
o
each age, as we wish to give more emphasis to those ages where the q x s or
o
x s
are based on abundant data.
The resulting graduation needs to be tested for goodness-of-fit, but not for
smoothness, since the standard table should already be smooth.
If the graduation fails the goodness-of-fit tests, then either a new function or a
new standard table should be sought and the graduation repeated.
END OF REPORT
Page 20
\end{document}
