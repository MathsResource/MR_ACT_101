
\documentclass[a4paper,12pt]{article}

%%%%%%%%%%%%%%%%%%%%%%%%%%%%%%%%%%%%%%%%%%%%%%%%%%%%%%%%%%%%%%%%%%%%%%%%%%%%%%%%%%%%%%%%%%%%%%%%%%%%%%%%%%%%%%%%%%%%%%%%%%%%%%%%%%%%%%%%%%%%%%%%%%%%%%%%%%%%%%%%%%%%%%%%%%%%%%%%%%%%%%%%%%%%%%%%%%%%%%%%%%%%%%%%%%%%%%%%%%%%%%%%%%%%%%%%%%%%%%%%%%%%%%%%%%%%

\usepackage{eurosym}
\usepackage{vmargin}
\usepackage{amsmath}
\usepackage{graphics}
\usepackage{epsfig}
\usepackage{enumerate}
\usepackage{multicol}
\usepackage{subfigure}
\usepackage{fancyhdr}
\usepackage{listings}
\usepackage{framed}
\usepackage{graphicx}
\usepackage{amsmath}
\usepackage{chngpage}

%\usepackage{bigints}
\usepackage{vmargin}

% left top textwidth textheight headheight

% headsep footheight footskip

\setmargins{2.0cm}{2.5cm}{16 cm}{22cm}{0.5cm}{0cm}{1cm}{1cm}

\renewcommand{\baselinestretch}{1.3}

\setcounter{MaxMatrixCols}{10}

\begin{document}
%%---  Question 11
In a certain country, people finance their retirement by purchasing deferred annuities
and pure endowment policies. According to the law in this country, these policies
must both be purchased on the birthday nearest age a years, where a is the age at
which a person s complete expectation of life is equal to his or her current age: e a a .
The value of a is calculated, for each person, on the basis of a set of life tables
published for different classes of life by the Government Actuary.
Both types of policy are funded by a single premium paid by the life on the birthday
nearest age a. The deferred annuity pays a level weekly amount from the
policyholder s 65th birthday until the birthday nearest the age 2a, or until death (if
earlier). If the policyholder survives to the birthday nearest age 2a the pure
endowment provides a sum assured at that time.
\item  The Government Actuary publishes life tables for different classes of lives.
List some relevant factors the Government Actuary might wish to use to
classify lives for this purpose.

\item  The mortality experience of one class of lives in this country is the same as
that described in AM92 Ultimate. Show that at age 39.778 the complete
expectation of life is 39.778 (to 3 decimal places) and that, therefore, the
birthday nearest age a is the 40 th birthday. You should assume a uniform
distribution of deaths within each year of age and may use the approximation
e x = e x 0.5.

\item  A member of this class of lives who is aged exactly 40 wishes to purchase one
of each of these policies. The weekly annuity payable will be a sum
equivalent to \$10,000 per year from the 65 th birthday to the birthday nearest
exact age 2a. The pure endowment lump sum will be \$100,000 on survival to
the birthday nearest exact age 2a. Calculate the total premium.
Basis: Mortality: AM92 Ultimate
Interest: 4% per annum throughout
Expenses are ignored

\item 
Discuss the factors that companies selling these annuities and pure
endowments might wish to take into account when pricing them.
END OF PAPER
104 S2004
8

[Total 18]

%%%%%%%%%%%%%%%%%%%%%%%%%%%
11
(i) Sex
Smoker/non-smoker status
Occupation
Known impairments
Geographical location
(ii) ALTERNATIVE 1
Using AM92, we find that e 39
39 and e 40
40
so 39 < a < 40.
If
e a
a , then
l x dx
a
a
. (*)
l a
Assuming a uniform distribution of deaths (UDD) between ages 39 and 40,
then
l a l 39 ( a 39) d 39
So that, substituting into (*) we obtain
l x dx
a
a
(**)
l 39 ( a 39) d 39
To evaluate l x dx , note that
a
40
l x dx
a
40
l x dx
a
l x dx
40
l x dx e 40 l 40 ,
a
and that, assuming UDD,
40
l x dx
a
40 a
( l a l 40 ) e 40 l 40
2
w
Therefore, substituting for
l x dx in (**) we obtain
a
Page 16Subject 104 (Survival Models)
September 2004
Examiners Report
40 a
( l 39 ( a 39) d 39 l 40 ) e 40 l 40
2
a =
l 39 ( a 39) d 39
From the tables, e 40 = 39.064, so using the approximation given e 40 = 39.064
+ 0.5 = 39.564. Inserting this, the tabulated values for l 39 , l 40 and d 39 and the
value 39.778 for a into the equation, the right hand side becomes:
40 a
(9,864.8688 (8.5824)(0.778) 9,856.2863) (39.564)(9,856.2863)
2
9,864.8688 (0.778)(8.5824)
0.222
(9,858.1917 9,856.2863) 389,954.1112
= 2
9,858.1917
= 39.778 (to 3 decimal places)
ALTERNATIVE 2
Using AM92, we find that e 39
39 and e 40
40
so 39 < a < 40.
To evaluate e x t (0
t
1) assuming UDD between exact ages
x and x+1
consider that e x t is a weighted average of the complete expectations of life of
those who survive to exact age x+1 and those who die between x+t and exact
age x+1.
The proportion of those alive at age x+t who survive to
1 t p x t .
exact age x+1 is
o
These lives will have a complete expectation of life equal to 1 t e x 1 .
The proportion of those alive at age x+t who die before exact age x+1 is
1- 1 t p x t .
1
These lives will live, on average, 1 t more years.
2
Therefore
o
e x
t
1
1 t 1
2
o
1 t
p x
t
1 t e x
1 1 t
p x t . (*)
Page 17Subject 104 (Survival Models)
September 2004
Examiners Report
In our case, we have t = 0.778, and x = 39. From the tables, e 40 = 39.064,
so using the approximation given, e 40 = 39.064 + 0.5 = 39.564.
Note also that under UDD,
1 t
p x
1
t
1 t
q x
t
1
(1 t ) q x
1 tq x
1 q x
.
1 tq x
From the tables, q 39 = 0.00087. Therefore
1 0.00087
0.99981
1 t p x t =
1 (0.00087)(0.778)
Substituting into (*) above, therefore, produces
o
e 39.778
=
1
1 0.778 (1 0.99981)
2
1 0.778 39.564 (0.99981)
1
(0.222)(0.00019) (39.786)(0.99981)
2
= 39.778 (to 3 d.p.)
Therefore we have shown that a = 0.778.
(iii)
In this case, 2a = 79.556, so the birthday nearest to this age is the 80 th .
An annuity paid weekly can be treated as being paid continuously, so the
premium for the annuity, P a , is given by the equation
P a 10, 000( a 40:40 a 40:25 )
Evaluation proceeds as follows: for the first term we have
a 40:40 a 40:40 0.5(1 40 40 p 40 )
a 40
D 80
a 80 0.5(1
D 40
40 l 80
l 40
)
and, from the AM92 tables this is evaluated as
228.48
5266.4604
20.005
(6.818) 0.5 1 (0.20829)
2052.96
9856.2863
20.005 (0.111293)(6.818) 0.5[1 (0.20829)(0.534325)]
= 20.005 0.75880 0.44435
= 18.80185.
For the second term we have
a 40:25
Page 18
a 40:25
0.5(1
25 l 65
l 40
) .Subject 104 (Survival Models)
September 2004
Examiners Report
From tables, this is evaluated as
8821.2612
)
9856.2863
15.884 0.5[1 (0.37512)(0.89499)]
15.884 0.5(1 0.37512
15.884 0.33214
15.5519
Therefore
P a
10, 000(18.80185 15.5519)
£32,500.
For the pure endowment, the premium, P e , is given by
P e
100, 000 v 40 40 p 40
which, using the figures in the previous evaluation, is
5266.4604
100,000 x (0.20829)
= £11,129.
9856.2863
So the total single premium payable is
£11,129 + £32,500 = £43,629.
(iv)
When pricing these annuities and pure endowments, companies will need to
consider the likely risk of future mortality improvements.
If mortality improves, then the expected outgo will increase. Premiums
should be set to take this into account, for otherwise the office might become
insolvent.
In addition, offices might wish to use a mortality basis that takes into account
factors other than those listed in the solution to part (i), such as the policy size,
the level of underwriting and the sales channel.
If an office does not take these factors into account, and if mortality is related
to them, then the premiums charged will be too high for the low risks (who
will take their business elsewhere to other offices who do take these factors
into account) and too low for the high risks (resulting in the possibility of
insolvency).
END OF EXAMINERS REPORT
Page 19
