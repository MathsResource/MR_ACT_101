
\documentclass[a4paper,12pt]{article}

%%%%%%%%%%%%%%%%%%%%%%%%%%%%%%%%%%%%%%%%%%%%%%%%%%%%%%%%%%%%%%%%%%%%%%%%%%%%%%%%%%%%%%%%%%%%%%%%%%%%%%%%%%%%%%%%%%%%%%%%%%%%%%%%%%%%%%%%%%%%%%%%%%%%%%%%%%%%%%%%%%%%%%%%%%%%%%%%%%%%%%%%%%%%%%%%%%%%%%%%%%%%%%%%%%%%%%%%%%%%%%%%%%%%%%%%%%%%%%%%%%%%%%%%%%%%

\usepackage{eurosym}
\usepackage{vmargin}
\usepackage{amsmath}
\usepackage{graphics}
\usepackage{epsfig}
\usepackage{enumerate}
\usepackage{multicol}
\usepackage{subfigure}
\usepackage{fancyhdr}
\usepackage{listings}
\usepackage{framed}
\usepackage{graphicx}
\usepackage{amsmath}
\usepackage{chngpage}

%\usepackage{bigints}
\usepackage{vmargin}

% left top textwidth textheight headheight

% headsep footheight footskip

\setmargins{2.0cm}{2.5cm}{16 cm}{22cm}{0.5cm}{0cm}{1cm}{1cm}

\renewcommand{\baselinestretch}{1.3}

\setcounter{MaxMatrixCols}{10}

\begin{document}

[Total 8]7
A life insurance company is investigating the mortality of a group of policyholders.
The possible definitions of age by which deaths may be grouped are as follows:
(a)
(b)
(c)
age at birthday in calendar year of death
age last birthday at entry plus curtate duration
age next birthday at policy anniversary in calendar year of death
The force of mortality for lives labelled aged x is to be estimated as
x
=
number of deaths aged x
Central exposed to risk for age x
x estimates
the force of mortality
x f
.
For each of the possible age definitions, state the type of rate interval and determine
the value of f. State any assumptions that you make.
[10]
%%%%%%%%%%%%%%%%%%%%
x + f is the average age of the lives half-way through the rate interval.
(a)
Age changes on 1 January each year, so calendar year rate interval starting on
1 January.
The age range at the start of the interval is (x - 1, x).
Assuming birthdays are uniformly distributed throughout the year,
1
the average age at the start of the interval is x
2
1 1
So, average age half-way through interval is x
x
2 2
So, f = 0
(b)
Age x = [Age last birthday at entry] + [curtate duration]
Age changes on a policy anniversary, so policy year rate interval.
The age range at the start of the interval is (x, x + 1).
Assuming birthdays are uniformly distributed throughout the policy year,
1
the average age at the start of the interval is x
2
1 1
So, average age half-way through interval is x
x 1
2 2
So, f = 1
(c)
Age x = Age next birthday at anniversary in calendar year of death
Age changes at start of calendar year, so calendar year rate interval starting on
1 January.
The age range at the start of the interval is (x - 2, x).
Assuming birthdays and policy anniversaries are uniformly distributed
throughout the calendar year, the average age at the start of the interval is x 1
1
1
So, average age half-way through interval is x 1
x
2
2
Page 11 %%%%%%%%%%%%%%%%%%%%%%%%%%%%%%%%%%%%%
So, f
8
September 2004
%%%%%%%%%%%%%%%%%%%%%%%%%%%%%%%%%%%%%
1
2


\end{document}
