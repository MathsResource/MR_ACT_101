\documentclass[a4paper,12pt]{article}

%%%%%%%%%%%%%%%%%%%%%%%%%%%%%%%%%%%%%%%%%%%%%%%%%%%%%%%%%%%%%%%%%%%%%%%%%%%%%%%%%%%%%%%%%%%%%%%%%%%%%%%%%%%%%%%%%%%%%%%%%%%%%%%%%%%%%%%%%%%%%%%%%%%%%%%%%%%%%%%%%%%%%%%%%%%%%%%%%%%%%%%%%%%%%%%%%%%%%%%%%%%%%%%%%%%%%%%%%%%%%%%%%%%%%%%%%%%%%%%%%%%%%%%%%%%%

\usepackage{eurosym}
\usepackage{vmargin}
\usepackage{amsmath}
\usepackage{graphics}
\usepackage{epsfig}
\usepackage{enumerate}
\usepackage{multicol}
\usepackage{subfigure}
\usepackage{fancyhdr}
\usepackage{listings}
\usepackage{framed}
\usepackage{graphicx}
\usepackage{amsmath}
\usepackage{chngpage}

%\usepackage{bigints}
\usepackage{vmargin}

% left top textwidth textheight headheight

% headsep footheight footskip

\setmargins{2.0cm}{2.5cm}{16 cm}{22cm}{0.5cm}{0cm}{1cm}{1cm}

\renewcommand{\baselinestretch}{1.3}

\setcounter{MaxMatrixCols}{10}

\begin{document}
\begin{enumerate}
7
(i) Derive a formula to estimate the constant force of mortality for lives aged x
nearest birthday. State any assumptions you make in deriving your formula.
[5]
(ii) Use your formula in (i) to calculate numerical estimates of m 41 and m 42 .
[2]
[Total 7]
(i) You have been asked to undertake an investigation of the mortality of male
term assurance policyholders for the period 1 January 1995 to 31 December
1998. List the data that should be recorded for each life so that the
investigation can be undertaken using a life year rate interval with an exact
calculation of the Central Exposed to Risk. The list should include only data
items that are essential to complete the calculations.
[3]
(ii) Using the data items from (i) describe how the force of mortality at particular
ages would be estimated.
[4]
[Total 7]
104 S2002—3
PLEASE TURN OVER8
(i)
In the context of the graduation of a set of estimated mortality rates explain
what is meant by:
(a)
(b)
(ii)
smoothness
adherence to data
[4]
Graduation is said to “resolve the conflicting requirements of smoothness and
adherence to data”.
Explain:
(a)
(b)
9
how this conflict arises, and
how the process of graduation resolves the conflict
[4]
[Total 8]

%%%%%%%%%%%%%%%%%%%%%%%%%%%%%%%%%%
7
(i) Essential data is:
· date of birth (or date of x th birthday or exact age)
· multiple policy indicator
Either
· date of purchase of term assurance policy (*)
· date of policy lapse, date of policy expiry or date of death (*)
· if occurred between 1.1.95 and 31.12.98
Or
· Date of entry into investigation
· Date of exit from investigation
· Reason for exit
(ii) Data for all lives that had died during the Period of Investigation (1.1.95 to
31.12.98) would be tabulated by age last birthday, x at date of death. Counts
of the number of deaths at each age x, q x would be recorded for all x .
For each age x and for each life two dates would be calculated.
START DATE
The latest date of
1 January 1995
date of purchase of policy
date of xth birthday
END DATE
The earliest date of
31 December 1998
date of death or date of leaving (if any)
date of x + 1th birthday
Then calculate END DATE - START DATE (if this is > 0) and total these
values for all lives. Record this answer in years. This is the Central Exposed
to Risk at age, x, E x c .
Tabulate these values for all x.
Then: m ˆ x =
q x
is an estimate of the force of mortality at age x + 1⁄2, assuming
E x c
that the force is constant over the year of age x to x + 1.
ALTERNATIVELY
Tabulations could be produced for age nearest birthday, in which
START DATE must be amended to use “date of attaining x nearest birthday”,
and END DATE to use “date of attaining x + 1 nearest birthday”.
Page 7Subject 104 (Survival Models) — September 2002 — Examiners’ Report
Then: m ˆ x =
q x
is an estimate of the force of mortality at age x, assuming that
E x c
the force is constant over the year of age (x - 1⁄2, x + 1⁄2).
8
(i)
(a)
Mathematical smoothness is usually defined in terms of differentiality — a continuous function which is differentiable everywhere is smooth.
Empirical smoothness is about the curvature and rate of change of curvature of a fitted function.
Smoothness implies no sharp curves.

This is usually checked by using finite differences.
Small first differences, and smaller second differences with a regular progression with age imply low curvature and no rapid change of
curvature with age.
(b)
Observed rates are smoothed (graduated) and replaced by the smoothed or graduated rates.
While we want the rates we use to be smooth they should not deviate too far from the observed rates.
If the observed rates look like a set of estimates that might have been obtained from a population in which the graduated rates are the true
rates they are said to “adhere to the data”.

(ii)
(a)
Maximum smoothness would be achieved by ignoring the plotted estimated rates and drawing a graduation curve which is very smooth, e.g. straight line. The deviations between the rates read from such a
curve and the observed rates are likely to be very large. The graduation curve will be very smooth but have poor “adherence to data”.
On the other hand joining up a plot of the estimated values will give perfect adherence to data but is likely to produce a “curve” with
rapidly changing curvature which would not satisfy the smoothness criteria.
Suitably explained graphs demonstrating the above points were given credit.
(b)
Graduation aims to resolve these conflicts by choosing a half way
house.
Graduated rates can be obtained by many methods, some ensure moothness, e.g. graduation by a mathematical form (the chosen
functional form will ensure smoothness), reference to a standard table
(a simple relationship with an already smooth set of standard table
Page 8Subject 104 (Survival Models) — September 2002 — Examiners’ Report
rates will ensure smoothness). In this case the graduated rates just
need to satisfy tests of adherence to data.

Graphical graduation does not ensure smoothness, so graduated rates must be checked for smoothness and adherence to data. The
graduation process must be repeated until both criteria are satisfied.
