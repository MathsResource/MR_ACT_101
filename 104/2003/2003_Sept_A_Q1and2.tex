\documentclass[a4paper,12pt]{article}

%%%%%%%%%%%%%%%%%%%%%%%%%%%%%%%%%%%%%%%%%%%%%%%%%%%%%%%%%%%%%%%%%%%%%%%%%%%%%%%%%%%%%%%%%%%%%%%%%%%%%%%%%%%%%%%%%%%%%%%%%%%%%%%%%%%%%%%%%%%%%%%%%%%%%%%%%%%%%%%%%%%%%%%%%%%%%%%%%%%%%%%%%%%%%%%%%%%%%%%%%%%%%%%%%%%%%%%%%%%%%%%%%%%%%%%%%%%%%%%%%%%%%%%%%%%%

\usepackage{eurosym}
\usepackage{vmargin}
\usepackage{amsmath}
\usepackage{graphics}
\usepackage{epsfig}
\usepackage{enumerate}
\usepackage{multicol}
\usepackage{subfigure}
\usepackage{fancyhdr}
\usepackage{listings}
\usepackage{framed}
\usepackage{graphicx}
\usepackage{amsmath}
\usepackage{chngpage}

%\usepackage{bigints}
\usepackage{vmargin}

% left top textwidth textheight headheight

% headsep footheight footskip

\setmargins{2.0cm}{2.5cm}{16 cm}{22cm}{0.5cm}{0cm}{1cm}{1cm}

\renewcommand{\baselinestretch}{1.3}

\setcounter{MaxMatrixCols}{10}

\begin{document}

1 State the reasons why crude mortality rates are graduated before they are used in
financial calculations.

2 An investigation into mortality by an insurer follows N independent and identical
lives over the year of age from x to x + 1. Each life holds one or more term assurance
policies. The proportion of the N lives holding i policies is p i for i = 1, 2, ...
Let C denote the total number of claims made on all policies, and let D be the number
of deaths among the N policyholders. Show that
i 2 p i
å
Var( C )
= i
.
Var( D )
å p i

i
%% ---  4%%%%%%%%%%%%%%%%%%%%%%%%%%%%%%%%%%%%%%%%%%%%— September 2003 — %%%%%%%%%%%%%%%%%%%%%%%%%%%%%%%%%%%%%%%%%%%%
1
It is normally assumed that the true mortality rates in the population under
investigation are smooth functions of age.
Crude mortality rates are typically estimated separately at each integer age, and
therefore may not progress smoothly.
Graduation allows information from adjacent ages to be used to improve the estimate
at each age, thus reducing sampling errors.
It is desirable that financial quantities progress smoothly with age, as irregularities are
hard to justify in practice. If the underlying mortality rates are smooth, then financial
quantities calculated using them will also be smooth.
2
Let D i denote the number of deaths among the π i N lives with i policies, and let C i
denote the number of claims among the same group. Then C i = iD i and
D =
\sum  D i
i
C =
\sum  C i = \sum  iD i .
i
i
Now D i ~ B (π i N , q x ) since lives are genuinely independent and
Var( C )
= Var ⎛ ⎜ \sum  iD i ⎞ ⎟
⎝ i
⎠
=
\sum  i 2 Var( D i )
i
=
\sum  i 2 π i Nq x (1 − q x )
i
and
Var(D)
= Var ⎛ ⎜ \sum  D i ⎞ ⎟
⎝ i
⎠
=
\sum  π i Nq x (1 − q x )
i
