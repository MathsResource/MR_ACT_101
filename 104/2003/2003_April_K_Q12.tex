\documentclass[a4paper,12pt]{article}

%%%%%%%%%%%%%%%%%%%%%%%%%%%%%%%%%%%%%%%%%%%%%%%%%%%%%%%%%%%%%%%%%%%%%%%%%%%%%%%%%%%%%%%%%%%%%%%%%%%%%%%%%%%%%%%%%%%%%%%%%%%%%%%%%%%%%%%%%%%%%%%%%%%%%%%%%%%%%%%%%%%%%%%%%%%%%%%%%%%%%%%%%%%%%%%%%%%%%%%%%%%%%%%%%%%%%%%%%%%%%%%%%%%%%%%%%%%%%%%%%%%%%%%%%%%%

\usepackage{eurosym}
\usepackage{vmargin}
\usepackage{amsmath}
\usepackage{graphics}
\usepackage{epsfig}
\usepackage{enumerate}
\usepackage{multicol}
\usepackage{subfigure}
\usepackage{fancyhdr}
\usepackage{listings}
\usepackage{framed}
\usepackage{graphicx}
\usepackage{amsmath}
\usepackage{chngpage}

%\usepackage{bigints}
\usepackage{vmargin}

% left top textwidth textheight headheight

% headsep footheight footskip

\setmargins{2.0cm}{2.5cm}{16 cm}{22cm}{0.5cm}{0cm}{1cm}{1cm}

\renewcommand{\baselinestretch}{1.3}

\setcounter{MaxMatrixCols}{10}

\begin{document}
%%%%%%%%%%%%%%%%%%%%%%%%%%%%%%%%%12
A manufacturer of computer chips is attempting to estimate the useful working
lifetime of its products. To do so, it has been tracking 20 chips sold on 1 January
1997. The purchaser of each chip was contacted at three-monthly intervals up to
1 January 2002 to check whether the chip was still functioning correctly. The results
are summarised below (where B means observation ceased because the chip stopped
functioning and O means observation ceased for some other reason).
Date observation ceased Reason
1 April 1997
1 July 1997
1 October 1997
1 October 1997
1 October 1997
1 January 1998
1 January 1998
1 July 1998
1 July 1998
1 October 1998
1 July 1999
1 July 1999
1 July 1999
1 October 1999
1 April 2000
1 July 2000
1 October 2000
1 January 2001
1 January 2002
1 January 2002 B
B
O
O
O
B
B
O
O
O
O
O
O
B
O
O
B
B
O
O
(i) Explain how the manufacturer’s tracking method introduces censoring into the
study.

(ii) Calculate the Nelson-Aalen estimate of the integrated hazard for these
computer chips. State any assumptions that you make about the exact time
each chip stopped functioning.

(iii) Use this to approximate the Kaplan-Meier estimate of the survival function
and sketch the survival function.

(iv) Suggest two ways in which the breakdown rate for computer chips is similar
to typical patterns for human mortality.

[Total 14]
%%%%%%%%%%%%%%%%%%%%%%%%—6

12
(i)
We only know during which three month period the chip broke down, not the
actual date of breakdown.
The investigation is cut short at 1 January 2002.
We don’t have any information on the chips where observation ceased for
some other reason.
(ii)
Either
Assuming that the chips break down or are censored at the dates given in the
question, and measuring time in months, we have
t j n j c j d j
0
3
6
12
33
45
48 20
20
19
15
7
4
3 0
0
3
6
2
0
2 0
1
1
2
1
1
1
d j
n j
0
1/20
1/19
2/15
1/7
1/4
1/3
L t = å
d j
n j
0
0.05
0.1026
0.2360
0.3788
0.6288
0.9622
Or
Assuming that the chips break down or are censored mid-way between the
three-monthly checks (i.e. that a chip reported as breaking down or being
censored on 1 April 1997 actually broke down or was censored mid-way
between 1 January 1997 – when it was known to be working – and 1 April
1997), we have
d j
d j
n j
c j
d j
L t = å
t j
n j
n j
0
1.5
4.5
10.5
31.5
43.5
46.5
20
20
19
15
7
4
3
0
0
3
6
2
0
2
0
1
1
2
1
1
1
0
1/20
1/19
2/15
1/7
1/4
1/3
0
0.05
0.1026
0.2360
0.3788
0.6288
0.9622
%% ---  24%%%%%%%%%%%%%%%%%%%%%%%%%%%%%%%%%%%%%%%%%%%%— April 2003 — %%%%%%%%%%%%%%%%%%%%%%%%%%%%%%%%%%%%%%%%%%%%
S(t) = exp( -L t ),
so, depending on which assumption is made about the exact dates of
breakdowns and censorings
Either
t S(t)
0 \$ t < 3
3 \$ t < 6
6 \$ t < 12
12 \$ t < 33
33 \$ t < 45
45 \$ t < 48
48 \$ t 1
0.9512
0.9025
0.7898
0.6847
0.5332
0.3821
or
t S(t)
0 \$ t < 1.5
1.5 \$ t < 4.5
4.5 \$ t < 10.5
10.5 \$ t < 31.5
31.5 \$ t < 43.5
43.5 \$ t < 46.5
46.5 \$ t 1
0.9512
0.9025
0.7898
0.6847
0.5332
0.3821
1
0.9
0.8
0.7
0.6
(iii)
0.5
0.4
0.3
0.2
0.1
0
0
12
24
36
48
60
Duration (months)
%% ---  25%%%%%%%%%%%%%%%%%%%%%%%%%%%%%%%%%%%%%%%%%%%%— April 2003 — %%%%%%%%%%%%%%%%%%%%%%%%%%%%%%%%%%%%%%%%%%%%
(iv)
Two similarities:
High initial levels of failure, similar to high infant mortality in humans.
Increasing failure for older chips, similar to rising mortality in old age for
humans.
Note that the diagram shown above in the solution to part (iii) assumes that
breakdowns and censorings take place at the dates given in the question. The
alternative assumption implies that the steps in the step function should all be shifted
1.5 months to the left.
The two alternative assumptions given about when breakdowns and censorings take
place are the only plausible ones. Other assumptions (for example, that breakdowns
and censorings are uniformly distributed within each interval) are wrong and did not
receive credit
%% ---  26%%%%%%%%%%%%%%%%%%%%%%%%%%%%%%%%%%%%%%%%%%%%— April 2003 — %%%%%%%%%%%%%%%%%%%%%%%%%%%%%%%%%%%%%%%%%%%%
