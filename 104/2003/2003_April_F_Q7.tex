\documentclass[a4paper,12pt]{article}

%%%%%%%%%%%%%%%%%%%%%%%%%%%%%%%%%%%%%%%%%%%%%%%%%%%%%%%%%%%%%%%%%%%%%%%%%%%%%%%%%%%%%%%%%%%%%%%%%%%%%%%%%%%%%%%%%%%%%%%%%%%%%%%%%%%%%%%%%%%%%%%%%%%%%%%%%%%%%%%%%%%%%%%%%%%%%%%%%%%%%%%%%%%%%%%%%%%%%%%%%%%%%%%%%%%%%%%%%%%%%%%%%%%%%%%%%%%%%%%%%%%%%%%%%%%%

\usepackage{eurosym}
\usepackage{vmargin}
\usepackage{amsmath}
\usepackage{graphics}
\usepackage{epsfig}
\usepackage{enumerate}
\usepackage{multicol}
\usepackage{subfigure}
\usepackage{fancyhdr}
\usepackage{listings}
\usepackage{framed}
\usepackage{graphicx}
\usepackage{amsmath}
\usepackage{chngpage}

%\usepackage{bigints}
\usepackage{vmargin}

% left top textwidth textheight headheight

% headsep footheight footskip

\setmargins{2.0cm}{2.5cm}{16 cm}{22cm}{0.5cm}{0cm}{1cm}{1cm}

\renewcommand{\baselinestretch}{1.3}

\setcounter{MaxMatrixCols}{10}

\begin{document}


%%%%%%%%%%%%%%%%%%%%%%%%%%%%%%%%%7
8
9
In a mortality investigation spanning N + 1 years, information is available on the
calendar year of birth and the calendar year of death of each life dying in each year K,
K + 1, ..., K + N. In addition, information is available on the number of lives,
classified by age x last birthday, on 1 January in each of years K, K + 1, ..., K + N + 1.
(i) In terms of the available data, derive an approximation for the central exposed
to risk which corresponds to the deaths data.

(ii) The initial rates of mortality for deaths having age label x in this investigation

estimate q x+f . Determine f, stating any assumptions you make.
[Total 7]
A life office wishes to compare the mortality of its male term assurance policyholders
with the standard table AM92. A Chi-squared test has been carried out to test the null
hypothesis that the true underlying mortality rates of the policyholders are those of the
standard table. Based on the results of this test, the null hypothesis has been accepted.
(i) State three possible defects that the Chi-squared test may have failed to detect
and explain why these are not detected.

(ii) Describe how you would carry out a suitable test to check for one of the above
defects. You should state which defect you are checking for.

%%%%%%%%%%%%%%%%%%%%%%%%%%%%%%%%%%%%%
7
(i)
Since the age label will change at the end of the calendar year, we have a
calendar year rate interval.
Given the data we have, a person dying in calendar year t and born in year
t - x will be classified as aged x at death, OR age x next birthday at previous
or coincident 1 January. 1
This person will be aged between exact ages x - 1 and x at the start of the
year.
So, at the end of the year the person would (had he or she not died during the
year) have been aged between exact ages x and x + 1.
P x,t is the number of persons aged x last birthday on 1 January in year t.
We want the central exposed to risk at age x during year t corresponding to the
definition of the deaths data (principle of correspondence).
Assuming P x,t varies linearly over the calendar year, this consists of the
average of the number of persons aged x - 1 last birthday at the start of year t
and the number of persons aged x last birthday at the start of year
t + 1, ie 0.5 (P x - 1,t + P x,t+1 ).
Summing this over the whole period of the investigation produces
K + N
å 0.5( P x - 1, t + P x , t + 1 ) .
K
(ii)
At the start of the rate interval, ages range from x - 1 to x exact.
Thus, assuming birthdays are distributed evenly over the calendar year, the
average age at the start of the rate interval is x - 1⁄2,
so q estimates q x - 1
2
Thus f = - 1⁄2.
%% ---  12%%%%%%%%%%%%%%%%%%%%%%%%%%%%%%%%%%%%%%%%%%%%— April 2003 — %%%%%%%%%%%%%%%%%%%%%%%%%%%%%%%%%%%%%%%%%%%%
