\documentclass[a4paper,12pt]{article}

%%%%%%%%%%%%%%%%%%%%%%%%%%%%%%%%%%%%%%%%%%%%%%%%%%%%%%%%%%%%%%%%%%%%%%%%%%%%%%%%%%%%%%%%%%%%%%%%%%%%%%%%%%%%%%%%%%%%%%%%%%%%%%%%%%%%%%%%%%%%%%%%%%%%%%%%%%%%%%%%%%%%%%%%%%%%%%%%%%%%%%%%%%%%%%%%%%%%%%%%%%%%%%%%%%%%%%%%%%%%%%%%%%%%%%%%%%%%%%%%%%%%%%%%%%%%

\usepackage{eurosym}
\usepackage{vmargin}
\usepackage{amsmath}
\usepackage{graphics}
\usepackage{epsfig}
\usepackage{enumerate}
\usepackage{multicol}
\usepackage{subfigure}
\usepackage{fancyhdr}
\usepackage{listings}
\usepackage{framed}
\usepackage{graphicx}
\usepackage{amsmath}
\usepackage{chngpage}

%\usepackage{bigints}
\usepackage{vmargin}

% left top textwidth textheight headheight

% headsep footheight footskip

\setmargins{2.0cm}{2.5cm}{16 cm}{22cm}{0.5cm}{0cm}{1cm}{1cm}

\renewcommand{\baselinestretch}{1.3}

\setcounter{MaxMatrixCols}{10}

\begin{document}




%%%%%%%%%%%%%%%%%%%%%%%%%%%%%%%%%%%%%%%%%%%%%%%%%%%
8
An insurance company offers different annuity rates to smokers and non-smokers.
The premium rates are calculated using the following forces of mortality:
non-smokers: m NS
x + t = m x + t
smokers: m Sx + t = m x + t + 0.019048
m x + t is from the standard mortality table AM92 ultimate.
An annuity is issued to a life aged exactly 50. The annuity is deferred for 15 years
and is then payable continuously for life. A single premium of \$20,000 is paid at the
date the policy is issued.
(i)
Calculate the level annual annuity that would be paid assuming the
policyholder is:
(a)
(b)
a non-smoker
a smoker

(ii)
A policyholder fails to declare the fact that he is a smoker and is sold a non-
smoker policy. Calculate the expected present value of the profit that the
insurance company will make.

Basis: 4% per annum interest throughout
Expenses are ignored
[Total 10]
104 S2003—5
%%%%%%%%%%%%%%%%%%%%%%%%%%%%%%%%%


8
(i)
(a)
Equation of value: P = X ⋅ 15 a 50 = X ⋅
⇒ 20000 = X ⋅
D 65
D
a 65 = X ⋅ 65 ( a  65 − 0.5 )
D 50
D 50
689.23
⋅ ( 12.276 − 0.5 ) = 5.939055 X
1366.61
⇒ X = \$3,367.54 per annum
(b)
Using the special mortality rates,
S
t p x
⎧ ⎪ t
⎫ ⎪
= exp ⎨ − ∫ ( \mu x + s + 0.019048 ) ds ⎬
⎪ ⎩ 0
⎪ ⎭
t
⎧ ⎪
⎫ ⎪
= exp ⎨ − 0.019048 t − ∫ \mu x + s ds ⎬ = e − 0.019048 t ⋅ t p x
⎪ ⎩
⎪ ⎭
0
So,
D x S + t
D x S
= v t ⋅ t p x s = v t ⋅ e − 0.019048 t ⋅ t p x =
D x + t
calculated at rate of
D x
interest j such that (1 + j) − 1 = v.e − 0.019048 => j = 6%.
Or
and
%% ---  12
D x S + t
D x S
= e − 0.019048 t ⋅
D x + t
calculated at 4%
D x%%%%%%%%%%%%%%%%%%%%%%%%%%%%%%%%%%%%%%%%%%%%— September 2003 — %%%%%%%%%%%%%%%%%%%%%%%%%%%%%%%%%%%%%%%%%%%%
\infty
a x S
=
∫ v
\infty
t
⋅ t p x S dt
=
0
(
∫ v e
t
− 0.19048 t
)
⋅ t p x dt =
0
\infty
∫ (
)
t
v ⋅ e − 0.19048 ⋅ t p x dt
0
= a x calculated at rate of interest 6%
So, 20000 = X ⋅
1.06 − 65 \int 8821.2612
1.06 − 50 \int 9712.0728
⋅ ( 10.569 − 0.5 ) = 3.81608 X
⇒ X = \$5,240.98 per annum
Or 20000 = X ⋅ e − 0.28572 ⋅
689.23
⋅ ( 10.569 − 0.5 ) = 3.81610 X
1366.61
⇒ X = \$5,240.96 per annum
(ii)
EPV Profit = EPV of income – EPV of outgo
= 20000 − 3367.54 \int
S
D 65
S
D 50
\int a 65
= 20000 − 3367.54 \int 3.81610
= \$7,149.20
