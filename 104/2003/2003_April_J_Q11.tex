\documentclass[a4paper,12pt]{article}

%%%%%%%%%%%%%%%%%%%%%%%%%%%%%%%%%%%%%%%%%%%%%%%%%%%%%%%%%%%%%%%%%%%%%%%%%%%%%%%%%%%%%%%%%%%%%%%%%%%%%%%%%%%%%%%%%%%%%%%%%%%%%%%%%%%%%%%%%%%%%%%%%%%%%%%%%%%%%%%%%%%%%%%%%%%%%%%%%%%%%%%%%%%%%%%%%%%%%%%%%%%%%%%%%%%%%%%%%%%%%%%%%%%%%%%%%%%%%%%%%%%%%%%%%%%%

\usepackage{eurosym}
\usepackage{vmargin}
\usepackage{amsmath}
\usepackage{graphics}
\usepackage{epsfig}
\usepackage{enumerate}
\usepackage{multicol}
\usepackage{subfigure}
\usepackage{fancyhdr}
\usepackage{listings}
\usepackage{framed}
\usepackage{graphicx}
\usepackage{amsmath}
\usepackage{chngpage}

%\usepackage{bigints}
\usepackage{vmargin}

% left top textwidth textheight headheight

% headsep footheight footskip

\setmargins{2.0cm}{2.5cm}{16 cm}{22cm}{0.5cm}{0cm}{1cm}{1cm}

\renewcommand{\baselinestretch}{1.3}

\setcounter{MaxMatrixCols}{10}

\begin{document}

[Total 8]
A man aged exactly 45 has decided to invest some money to purchase a deferred
annuity from an insurance company. The man plans to invest \$1,500 per annum for
20 years; the first payment being made now, with subsequent payments made
annually if he is alive. In addition he will have lump sums of \$3,000 to invest in 5
and 10 years’ time, if he is then alive.
The man has to choose between a standard whole life annuity and an annuity where
the benefits are guaranteed to be paid for at least 10 years and for life thereafter. Both
annuities are level, commence at age 65 and are payable annually in arrears.
(i) Write down the equations of value for the standard whole life annuity \$X and
the guaranteed annuity \$X* that can be bought.

(ii) Show that by choosing the guaranteed annuity option, the annuity payable is
\$155 per annum lower than under the standard option.

(iii)
The policyholder chooses to take the standard annuity option. He pays the
premiums and reaches age 65. Calculate the probability that the policyholder
survives long enough that he receives annuity payments whose nominal
amount exceeds the total premiums paid.

[Total 9]
Basis: Mortality before age 65: AM92 ultimate
Mortality from age 65: PMA92C20
Interest: 4% per annum throughout
Expenses are ignored
%%%%%%%%%%%%%%%%%%%%%%%%—5



11
(i)
The equations of value are, for the standard annuity:
Either
(
¢  ́ v 20 20 p 45 = 1500 a && 45:20 + 3000  ́ v 5 5 p 45 + v 10 10 p 45
X  ́ a 65
)
Or
¢  ́
X  ́ a 65
æ D
D 65
( N - N 65 )
D ö
+ 3000 ç 50 + 55 ÷
= 1500  ́ 45
D 45
D 45
è D 45 D 45 ø
and for the guaranteed annuity
Either
(
)
(
¢  ́ a 75
¢  ́ v 20 20 p 45 = 1500 a && 45:20 + 3000  ́ v 5 5 p 45 + v 10 10 p 45
X * a 10 + v 10  ́ 10 p 65
Or
X *  ́ a ¢
65:10
(
 ́ v 20 20 p 45 = 1500 a && 45:20 + 3000  ́ v 5 5 p 45 + v 10 10 p 45
)
where p ¢ and a ¢ denote PMA92C20 mortality.
(ii)
Present value of premiums is given by
æ D
D ö
1500 a && 45:20 + 3000  ́ ç 50 + 55 ÷ @ 4%
è D 45 D 45 ø
æ 1366.61 + 1105.41 ö
= 1500  ́ 13.780 + 3000  ́ ç
÷ (from tables)
1677.97
è
ø
= 25089.66
The present value of the standard annuity is
D 65
689.23
¢ =
 ́ a 65
 ́ (13.666 - 1) (from tables)
1677.97
D 45
=
5.2026
and the standard annual annuity is
%% ---  22
)%%%%%%%%%%%%%%%%%%%%%%%%%%%%%%%%%%%%%%%%%%%%— April 2003 — %%%%%%%%%%%%%%%%%%%%%%%%%%%%%%%%%%%%%%%%%%%%
25089.66
= 4822.52
5.2026
The present value of the guaranteed annuity is
689.23 æ
8405.16
ö
 ́ 1.04 - 10  ́ (9.456 - 1) ÷ (from tables)
ç 8.1109 +
1677.97 è
9647.797
ø
= 5.3758
and the guaranteed annuity is therefore
25089.66
= 4667.15
5.3758
so the difference is
4822.52 - 4667.15 = 155.37, i.e. \$155 to nearest \$
(iii)
We must solve
4822.52  ́ n = 1500  ́ 20 + 3000  ́ 2
The solution is n = 7.46. So the purchaser must survive 8 years. The
probability of doing so is
¢
8 p 65
= ¢
l 73
¢
l 65
= 8803.265
(using PMA92C20 mortality)
9647.797
= 0.9125
%% ---  23%%%%%%%%%%%%%%%%%%%%%%%%%%%%%%%%%%%%%%%%%%%%— April 2003 — %%%%%%%%%%%%%%%%%%%%%%%%%%%%%%%%%%%%%%%%%%%%
