\documentclass[a4paper,12pt]{article}

%%%%%%%%%%%%%%%%%%%%%%%%%%%%%%%%%%%%%%%%%%%%%%%%%%%%%%%%%%%%%%%%%%%%%%%%%%%%%%%%%%%%%%%%%%%%%%%%%%%%%%%%%%%%%%%%%%%%%%%%%%%%%%%%%%%%%%%%%%%%%%%%%%%%%%%%%%%%%%%%%%%%%%%%%%%%%%%%%%%%%%%%%%%%%%%%%%%%%%%%%%%%%%%%%%%%%%%%%%%%%%%%%%%%%%%%%%%%%%%%%%%%%%%%%%%%

\usepackage{eurosym}
\usepackage{vmargin}
\usepackage{amsmath}
\usepackage{graphics}
\usepackage{epsfig}
\usepackage{enumerate}
\usepackage{multicol}
\usepackage{subfigure}
\usepackage{fancyhdr}
\usepackage{listings}
\usepackage{framed}
\usepackage{graphicx}
\usepackage{amsmath}
\usepackage{chngpage}

%\usepackage{bigints}
\usepackage{vmargin}

% left top textwidth textheight headheight

% headsep footheight footskip

\setmargins{2.0cm}{2.5cm}{16 cm}{22cm}{0.5cm}{0cm}{1cm}{1cm}

\renewcommand{\baselinestretch}{1.3}

\setcounter{MaxMatrixCols}{10}

\begin{document}

Let X be a random variable representing the present value of the benefits of a 20 year
term assurance issued to a life aged exactly 45, the benefit being payable at the end of
the year of death.
Let Y be a random variable representing the present value of the benefits of a 20 year
pure endowment policy issued to a (different, independent) life also aged exactly 45.
Finally, let Z denote the present value of the benefits of a 20 year endowment
assurance issued to a third independent life aged exactly 45, the death benefit being
payable at the end of the year of death.
The benefit payable under each policy is \$1.
(i) Calculate the variances of X and Y.
(ii) Without carrying out any further calculations explain why
Var(Z) < Var(X) + Var(Y).
Basis: Mortality: AM92 ultimate
Interest: 4% per annum throughout
Expenses are ignored
%%%%%%%%%%%%%%%%%%%%%%%%—4
8
(i)
A few large deviations can be offset by a lot of small deviations. This is not
detected as the information is summarised into just one figure.
The true mortality may be consistently slightly lighter or heavier than the
standard table. As the test statistic is based on squared deviations, the test will
miss this. (Note that large differences should be detected.)
Even if the true mortality is not biased as a whole, there could be significant
runs or clumps of ages for which it is biased. Again because of the use of
squared deviations and a single figure, this will not be detected.
(ii)
Standardised deviations test
This is used to test for a small number of large deviations.
Calculate the standardised deviations at each age x, z x =
d x - E x q x s
E x q x s (1 - q x s )
where
d x = observed number of deaths at age x
E x = exposed to risk at age x
q x s = mortality rate for age x from standard table
Under the null hypothesis, the z x are independent samples from N(0, 1)
distribution.
We can calculate the expected number of z x in various intervals (for example
we expect fewer than 1 in 20 to be greater than 2 or less than - 2) and compare
with the observed number in each interval.
The comparison can be formalised by using a c 2 -statistic equal to
å
( Actual number
all intervals
- Expected number )
2
Expected number
provided the number of intervals is large (such that the expected number in
each interval is not less that five (as a rule of thumb).
Signs test
This tests for the true mortality being lighter or heavier than the standard table.
Either
%% ---  13%%%%%%%%%%%%%%%%%%%%%%%%%%%%%%%%%%%%%%%%%%%%— April 2003 — %%%%%%%%%%%%%%%%%%%%%%%%%%%%%%%%%%%%%%%%%%%%
Calculate the standardised deviations at each age x, z x =
d x - E x q x s
E x q x s (1 - q x s )
where
d x = observed number of deaths at age x
E x = exposed to risk at age x
q x s = mortality rate for age x from standard table
Or
Calculate the sign of the deviation at each age x, d x - E x q x s .
where
d x = observed number of deaths at age x
E x = exposed to risk at age x
q x s = mortality rate for age x from standard table
Then
The test statistic is P, the number of positive deviations.
Under the null hypothesis P ~ Bin(m, 1⁄2) where m is the number of ages.
This is a two-tailed test as too many positive or negative deviations is a defect.
Then either
Find k*, the smallest k such that
k
m
æ m öæ 1 ö
å ç è j ÷ç
÷ 3 0.025 (available in tables).
2
è
ø
ø
j = 0
At the 5% level, we would accept the null hypothesis if k* < P < m - k*.
Or
Find the p-value for the test statistic P.
If this p-value is greater than 0.025, we would accept the null hypothesis (at
the 5% level.
Or
If the number of age groups is large, use the approximation
%% ---  14%%%%%%%%%%%%%%%%%%%%%%%%%%%%%%%%%%%%%%%%%%%%— April 2003 — %%%%%%%%%%%%%%%%%%%%%%%%%%%%%%%%%%%%%%%%%%%%
P : Normal( m , m ).
2 4
Cumulative deviations
This tests for the true mortality being lighter or heavier than the standard table
either over the whole age range or over subsections.
Calculate the deviations at each age x, d x - E x q x s .
where
d x = observed number of deaths at age x
E x = exposed to risk at age x
q x s = mortality rate for age x from standard table
Sum these over all ages and standardise, to give our test statistic:
å d x - E x q x s
all x
å ( E x q x s ( 1 - q x s ) )
all x
Under the null hypothesis this is distributed N(0, 1).
We use a two-tailed test as both positive and negative cumulative deviations
are of interest.
At the 5% level, we will accept the null hypothesis if the absolute value of the
statistic is less than 1.96.
Grouping of signs
This tests for runs of deviations of the same sign.
Either
Calculate the standardised deviations at each age x, z x =
d x - E x q x s
E x q x s (1 - q x s )
where
%% ---  15%%%%%%%%%%%%%%%%%%%%%%%%%%%%%%%%%%%%%%%%%%%%— April 2003 — %%%%%%%%%%%%%%%%%%%%%%%%%%%%%%%%%%%%%%%%%%%%
d x = observed number of deaths at age x
E x = exposed to risk at age x
q x s = mortality rate for age x from standard table
Or
Calculate the sign of the deviation at each age x, d x - E x q x s .
where
d x = observed number of deaths at age x
E x = exposed to risk at age x
q x s = mortality rate for age x from standard table
Then
Calculate n 1 = number of positive deviations
n 2 = number of negative deviations
G = number of groups of positive deviations
Under the null hypothesis
æ n 1 - 1 ö æ n 2 + 1 ö
ç
֍
÷
t - 1 ø è t ø
è
.
P(G = t) =
æ n 1 + n 2 ö
ç
÷
è n 1 ø
This is a one-tailed test as we are only interested in small values of G.
Then either
Find k*, the smallest k such that
æ n 1 - 1 ö æ n 2 + 1 ö
ç
֍
÷
è t - 1 ø è t ø 3 .05
å æ n 1 + n 2 ö
t = 1
ç
÷
è n 1 ø
k
At the 5% level, we accept the null hypothesis if G > k*.
Or
Calculate
%% ---  16%%%%%%%%%%%%%%%%%%%%%%%%%%%%%%%%%%%%%%%%%%%%— April 2003 — %%%%%%%%%%%%%%%%%%%%%%%%%%%%%%%%%%%%%%%%%%%%
æ n 1 - 1 ö æ n 2 + 1 ö
ç
֍
÷
G - 1 ø è G ø
Pr[exactly G positive groups] = è
æ n 1 + n 2 ö
ç
÷
è n 1 ø
and if this is greater than 0.05, accept the null hypothesis.
Or
If the number of age groups is large (> about 20)
use a normal approximation as follows:
æ n ( n + 1) ( n 1 n 2 ) 2 ö
G : Normal ç 1 2
,
.
3 ÷
è n 1 + n 2 ( n 1 + n 2 ) ø
Serial correlations
This tests for runs of deviations of the same sign.
Calculate the standardised deviations at each age x, z x =
d x - E x q x s
E x q x s (1 - q x s )
where
d x = observed number of deaths at age x
E x = exposed to risk at age x
q x s = mortality rate for age x from standard table
Calculate the correlation coefficient of the jth lagged sequence, r j using the
formula in the Gold Book.
Under the null hypothesis, r j is distributed N(0, 1/m).
This is a one-tailed test as we are only interested in high values of r j (which
indicates a tendency for z x to cluster).
Calculate m  ́ r j
At the 5% level, we will accept the null hypothesis if
m  ́ r j is less than
1.645.
%% ---  17%%%%%%%%%%%%%%%%%%%%%%%%%%%%%%%%%%%%%%%%%%%%— April 2003 — %%%%%%%%%%%%%%%%%%%%%%%%%%%%%%%%%%%%%%%%%%%%
