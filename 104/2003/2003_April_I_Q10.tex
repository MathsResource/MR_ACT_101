\documentclass[a4paper,12pt]{article}

%%%%%%%%%%%%%%%%%%%%%%%%%%%%%%%%%%%%%%%%%%%%%%%%%%%%%%%%%%%%%%%%%%%%%%%%%%%%%%%%%%%%%%%%%%%%%%%%%%%%%%%%%%%%%%%%%%%%%%%%%%%%%%%%%%%%%%%%%%%%%%%%%%%%%%%%%%%%%%%%%%%%%%%%%%%%%%%%%%%%%%%%%%%%%%%%%%%%%%%%%%%%%%%%%%%%%%%%%%%%%%%%%%%%%%%%%%%%%%%%%%%%%%%%%%%%

\usepackage{eurosym}
\usepackage{vmargin}
\usepackage{amsmath}
\usepackage{graphics}
\usepackage{epsfig}
\usepackage{enumerate}
\usepackage{multicol}
\usepackage{subfigure}
\usepackage{fancyhdr}
\usepackage{listings}
\usepackage{framed}
\usepackage{graphicx}
\usepackage{amsmath}
\usepackage{chngpage}

%\usepackage{bigints}
\usepackage{vmargin}

% left top textwidth textheight headheight

% headsep footheight footskip

\setmargins{2.0cm}{2.5cm}{16 cm}{22cm}{0.5cm}{0cm}{1cm}{1cm}

\renewcommand{\baselinestretch}{1.3}

\setcounter{MaxMatrixCols}{10}

\begin{document}

[Total 7]10
An illness-death model has three states:
1 = healthy
2 = sick
3 = dead
(i)
(ii)
11
Draw and label a diagram showing the three states and the transition
intensities between them.
Show, from first principles, that in this illness-death model
¶ 12
11 12
12 21
12 23
t p x = t p x m x + t - t p x m x + t - t p x m x + t .
¶ t

%%%%%%%%%%%%%%%%%%%%%%%%%%%%%%%%%%%%%%%
10
(i)
m 12
x
1 Healthy
2 Sick
m 21
x
m 13
x
(ii)
m 23
x
3 Dead
By the Markov assumption, consider the survival probability
condition on the state occupied at t .
t + dt
p 12
x and
We have
t + dt
11
12
12
22
13
32
p 12
x = t p x dt p x + t + t p x dt p x + t + t p x dt p x + t . (*)
But the last term in this equation is zero, since
dt
p 32
x + t = 0.
By the law of total probability,
p x 22 + t = 1 - dt p x 21 + t - dt p x 23 + t ,
and, substituting in (*), this produces
dt
11
12
12
21
23
p 12
x = t p x dt p x + t + t p x (1 - dt p x + t - dt p x + t ) .
t + dt
We now assume that, for small dt ,
dt 12
p 12
x + t = m x + t dt + o ( dt ) ,
dt p x 21 + t = m 21
x + t dt + o ( dt ) , and
dt p x 23 + t = m 23
x + t dt + o ( dt ) ,
where o ( dt ) is the probability that a life makes two or more transitions
in the time interval dt , and
o ( dt )
= 0 .
dt ® 0 dt
lim
Substituting for
t + dt
dt
21
p 12
x + t , dt p x + t and
dt
p x 23 + t gives
11 12
12
21
23
p 12
x = t p x m x + t dt + t p x ( 1 - m x + t dt - m x + t dt ) + o ( dt )
%% ---  20%%%%%%%%%%%%%%%%%%%%%%%%%%%%%%%%%%%%%%%%%%%%— April 2003 — %%%%%%%%%%%%%%%%%%%%%%%%%%%%%%%%%%%%%%%%%%%%
Thus
t + dt
12
11 12
12 21
12 23
p 12
x - t p x = t p x m x + t dt - t p x m x + t dt - t p x m x + t dt + o ( dt )
and
¶ 12
t p x = lim +
¶ t
dt ® 0
t + dt
12
p 12
12
12 21
12 23
x - t p x
= t p 11
x m x + t - t p x m x + t - t p x m x + t
dt
%% ---  21%%%%%%%%%%%%%%%%%%%%%%%%%%%%%%%%%%%%%%%%%%%%— April 2003 — %%%%%%%%%%%%%%%%%%%%%%%%%%%%%%%%%%%%%%%%%%%%
\end{document}
