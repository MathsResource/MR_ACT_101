
\documentclass[a4paper,12pt]{article}

%%%%%%%%%%%%%%%%%%%%%%%%%%%%%%%%%%%%%%%%%%%%%%%%%%%%%%%%%%%%%%%%%%%%%%%%%%%%%%%%%%%%%%%%%%%%%%%%%%%%%%%%%%%%%%%%%%%%%%%%%%%%%%%%%%%%%%%%%%%%%%%%%%%%%%%%%%%%%%%%%%%%%%%%%%%%%%%%%%%%%%%%%%%%%%%%%%%%%%%%%%%%%%%%%%%%%%%%%%%%%%%%%%%%%%%%%%%%%%%%%%%%%%%%%%%%

\usepackage{eurosym}
\usepackage{vmargin}
\usepackage{amsmath}
\usepackage{graphics}
\usepackage{epsfig}
\usepackage{enumerate}
\usepackage{multicol}
\usepackage{subfigure}
\usepackage{fancyhdr}
\usepackage{listings}
\usepackage{framed}
\usepackage{graphicx}
\usepackage{amsmath}
\usepackage{chngpage}

%\usepackage{bigints}
\usepackage{vmargin}

% left top textwidth textheight headheight

% headsep footheight footskip

\setmargins{2.0cm}{2.5cm}{16 cm}{22cm}{0.5cm}{0cm}{1cm}{1cm}

\renewcommand{\baselinestretch}{1.3}

\setcounter{MaxMatrixCols}{10}

\begin{document}
10

[Total 13]
On 1 September 2003 a man aged exactly 60 takes out a 5 year temporary annuity of
\$1,000 per year payable annually in arrears. The annuity is deferred for 5 years. The
annuity is purchased by a single premium paid on 1 September 2003.
(i) Give the dates of the first and last annuity payments to be made (assuming the
man is still alive).

(ii) Let X represent the present value of the benefits payable under this annuity
and K x the man’s curtate future lifetime. Calculate and plot the value of X for

K x = 0, 1, ........
(iii) P[X = x] denotes the probability that the random variable X has a value of x.
Calculate P[X = x] for all possible values of x and plot a graph of P[X = x]
against x.

(iv)
Calculate the premium.
Basis: Mortality: AM92 ultimate
5% per annum interest throughout
Expenses are ignored
104 S2003—6

%%%%%%%%%%%%%%%%%%%%%%%%%%%%%%%%%%%%%%%%%%%%%%%%%%%%
\newpage

10
(i)
First payment is on 31 August 2009 (1 September 2009 acceptable).
Last payment is on 31 August 2013 (1 September 2013 acceptable).
(ii)
For K x - 5, present value (PV) = 0.
K x
For 6 - K x - 10, PV = 1000 \sum  v k .
k = 6
10
For K x > 10, PV = 1000 \sum  v k .
k = 6
Hence, using v =
1
we have
1.05
K x present value
0 - 5 0
6
7
8
9
. 10 746.22
1,456.90
2,133.74
2,778.35
3,392.26
The plot is shown below.
4000
3500
3000
2500
2000
1500
1000
500
0
0 1 2 3 4 5 6 7 8 9 10 11 12 13 14 15
Kx
Credit was also given to candidates who used the approximation given in the Gold book .
%% ---  16%%%%%%%%%%%%%%%%%%%%%%%%%%%%%%%%%%%%%%%%%%%%— September 2003 — %%%%%%%%%%%%%%%%%%%%%%%%%%%%%%%%%%%%%%%%%%%%
(iii)
The calculations are shown below, using AM92 ultimate mortality.
P[ X = 0] = 6 q 60 = 1 −
l 66
8695.6199
= 0.06370
= 1 −
9287.2164
l 60
P[ X = 746.22] = 6 p 60 . q 66 = l 66 − l 67 d 66 138.6082
=
=
= 0.01492
l 60
l 60 9287.2164
P[ X = 1,456.90] = = l 67 − l 68 d 67 152.5202
=
=
= 0.01642
l 60
l 60 9287.2164
P[ X = 2,133.73] = 8 p 60 . q 68 = l 68 − l 69 d 68 167.3586
=
=
= 0.01802
l 60
l 60 9287.2164
P[ X = 2,778.35] = 9 p 60 . q 69 = l 69 − l 70 d 69 183.0785
=
=
= 0.01971
l 60
l 60 9287.2164
P[ X = 3,392.26] =
7 p 60 . q 67
10 p 60
=
l 70 8054.0544
=
= 0.86722
l 60 9287.2164
%% ---  17%%%%%%%%%%%%%%%%%%%%%%%%%%%%%%%%%%%%%%%%%%%%— September 2003 — %%%%%%%%%%%%%%%%%%%%%%%%%%%%%%%%%%%%%%%%%%%%
The plot is shown below.
1
0.9
0.8
0.7
0.6
0.5
0.4
0.3
0.2
0.1
0
0
1000
2000
3000
4000
X
%--------------------------%
(iv)
Single premium = expected present value of benefits
Using the data from part (iii) we have
EPV benefits =
\sum  x .(P[X = x ])
The calculations are shown in the table below
x Pr[ X = x ] x . Pr[ X = x ]
0
746.22
1,456.90
2,133.74
2,778.35
3,392.26 0.06370
0.01492
0.01642
0.01802
0.01971
0.86722 0
11.1336
23.9223
38.4500
54.7613
2941.8357
sum
Therefore the premium required is \$3,070.10.
%% ---  18
3,070.1029%%%%%%%%%%%%%%%%%%%%%%%%%%%%%%%%%%%%%%%%%%%%— September 2003 — %%%%%%%%%%%%%%%%%%%%%%%%%%%%%%%%%%%%%%%%%%%%
