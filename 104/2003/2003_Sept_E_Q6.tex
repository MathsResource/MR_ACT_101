\documentclass[a4paper,12pt]{article}

%%%%%%%%%%%%%%%%%%%%%%%%%%%%%%%%%%%%%%%%%%%%%%%%%%%%%%%%%%%%%%%%%%%%%%%%%%%%%%%%%%%%%%%%%%%%%%%%%%%%%%%%%%%%%%%%%%%%%%%%%%%%%%%%%%%%%%%%%%%%%%%%%%%%%%%%%%%%%%%%%%%%%%%%%%%%%%%%%%%%%%%%%%%%%%%%%%%%%%%%%%%%%%%%%%%%%%%%%%%%%%%%%%%%%%%%%%%%%%%%%%%%%%%%%%%%

\usepackage{eurosym}
\usepackage{vmargin}
\usepackage{amsmath}
\usepackage{graphics}
\usepackage{epsfig}
\usepackage{enumerate}
\usepackage{multicol}
\usepackage{subfigure}
\usepackage{fancyhdr}
\usepackage{listings}
\usepackage{framed}
\usepackage{graphicx}
\usepackage{amsmath}
\usepackage{chngpage}

%\usepackage{bigints}
\usepackage{vmargin}

% left top textwidth textheight headheight

% headsep footheight footskip

\setmargins{2.0cm}{2.5cm}{16 cm}{22cm}{0.5cm}{0cm}{1cm}{1cm}

\renewcommand{\baselinestretch}{1.3}

\setcounter{MaxMatrixCols}{10}

\begin{document}




[Total 7]
%%%%%%%%%%%%%%%%%%%%%%%%%%%%%%%%%6
(i)
(ii)
(iii)
T x denotes the future lifetime of a life currently aged x. Write down the
probability density function of T x .

Using your answer to (i), show that:
(a) ¶
log s p x = -m x + s , and
¶ s
(b) ì ï t
ü ï
=
-
m
p
exp
ds
í ò x + s
ý .
t x
ï î 0
ï þ

In a certain population, the force of mortality is given by:
\mu x
60 < x ≤ 70
70 < x ≤ 80
x > 80
0.01
0.015
0.025
Calculate the probability that a life aged exactly 65 will die between exact
ages 80 and 83.


%%%%%%%%%%%%%%%%%%%%%%%%%%%%

%% ---  8%%%%%%%%%%%%%%%%%%%%%%%%%%%%%%%%%%%%%%%%%%%%— September 2003 — %%%%%%%%%%%%%%%%%%%%%%%%%%%%%%%%%%%%%%%%%%%%
6
(i)
f x ( t ) = t p x ⋅\mu x + t
s
(ii)
(a)
Note that s q x = ∫ f x ( r ) dr
0
⇒
∂
∂
s p x = −
s q x = − f x ( s ) = − s p x \mu x + s
\frac{\partial}{\partial } s
\frac{\partial}{\partial } s
∂
s p x
∂
s
∂
and
log s p x =
\frac{\partial}{\partial } s
s p x
So,
(b)
∂
log s p x = − \mu x + s
\frac{\partial}{\partial } s
t t
0 0
∂
Hence, ∫ log s p x ds = − ∫ \mu x + s ds
\frac{\partial}{\partial } s
⇒ [ log
]
t
s p x 0
t
= log t p x = − ∫ \mu x + s ds (since 0 p x = 1 )
0
taking exponentials of both sides gives,
⎧ ⎪ t
⎫ ⎪
p
=
exp
ds
−
\mu
⎨ ∫ x + s ⎬ as required
t x
⎪ ⎩ 0
⎪ ⎭
(iii)
Required probability is
15 p 65
15 3 q 65
=
15 p 65 ⋅ 3 q 80
=
15 p 65 ⋅
( 1 − 3 p 80 )
15
⎧ ⎪ 5
⎫ ⎪
= exp ⎨ − ∫ 0.01 ds − ∫ 0.015 ds ⎬
⎪ ⎩ 0
⎪ ⎭
5
= exp { − ( 5 \int 0.01 ) − ( 10 \int 0.015 ) } = e − 0.2
⎧ ⎪ 3
⎫ ⎪
− 0.075
p
=
exp
−
0.025
ds
⎨ ∫
⎬ = exp { − 3 \int 0.025 } = e
3 80
⎩ ⎪ 0
⎭ ⎪
Required probability = e − 0.2 (1 − e − 0.075 ) = 0.059
%% ---  9%%%%%%%%%%%%%%%%%%%%%%%%%%%%%%%%%%%%%%%%%%%%— September 2003 — %%%%%%%%%%%%%%%%%%%%%%%%%%%%%%%%%%%%%%%%%%%%
