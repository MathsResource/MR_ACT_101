\documentclass[a4paper,12pt]{article}

%%%%%%%%%%%%%%%%%%%%%%%%%%%%%%%%%%%%%%%%%%%%%%%%%%%%%%%%%%%%%%%%%%%%%%%%%%%%%%%%%%%%%%%%%%%%%%%%%%%%%%%%%%%%%%%%%%%%%%%%%%%%%%%%%%%%%%%%%%%%%%%%%%%%%%%%%%%%%%%%%%%%%%%%%%%%%%%%%%%%%%%%%%%%%%%%%%%%%%%%%%%%%%%%%%%%%%%%%%%%%%%%%%%%%%%%%%%%%%%%%%%%%%%%%%%%

\usepackage{eurosym}
\usepackage{vmargin}
\usepackage{amsmath}
\usepackage{graphics}
\usepackage{epsfig}
\usepackage{enumerate}
\usepackage{multicol}
\usepackage{subfigure}
\usepackage{fancyhdr}
\usepackage{listings}
\usepackage{framed}
\usepackage{graphicx}
\usepackage{amsmath}
\usepackage{chngpage}

%\usepackage{bigints}
\usepackage{vmargin}

% left top textwidth textheight headheight

% headsep footheight footskip

\setmargins{2.0cm}{2.5cm}{16 cm}{22cm}{0.5cm}{0cm}{1cm}{1cm}

\renewcommand{\baselinestretch}{1.3}

\setcounter{MaxMatrixCols}{10}

\begin{document}



(i) State the age ranges over which Gompertz’ Law is an appropriate model for
human mortality.

(ii) Show that, under Gompertz’ Law, the probability of survival from age x to age
x + t is equal to:
é
æ
B ö ù
ê exp ç -
÷ ú
è log c ø û
ë
(iii)
%%%%%%%%%%%%%%%%%%%%%%%%—3
c x ( c t - 1)
.

Describe a method of estimating the parameters, B and c, when graduating a
set of crude mortality rates using Gompertz’ Law.

[Total 7]

6
(i)
Gompertz’ Law is a suitable model for human mortality for middle to older
ages, say 35 and over.
There is evidence that the Gompertz’ Law breaks down at very advanced ages
and therefore 35 to 90 years is acceptable.
(ii)
t
Since t p x = exp é ê - ò m x + s ds ù ú ,
ë 0
û
putting \mu x = Bc x produces
t
t
p x = exp é ê - ò Bc x + s ds ù ú .
ë 0
û
Evaluating the integral, we obtain
æ é x s ù t ö
ç Bc c ú ÷ ,
t p x = exp - ê
ç ê ë log c ú û ÷
0 ø
è
æ é Bc x c t - Bc x ù ö
= exp ç - ê
ú ÷ ÷
ç ê
log
c
ë
û ú ø
è
æ - B ö
= exp ç
÷
è log c ø
(iii)
c x ( c t - 1)
.
Suppose that a Poisson model is used in the investigation.
Then the likelihood for the age interval x to x + 1 is
K æ ç m x + 1 ö ÷
2 ø
è
dx
exp æ ç -m x + 1 E x c ö ÷ ,
2
è
ø
where K is a constant, m x + 1 is the force of mortality between exact ages x
2
and x + 1, d x is the number of deaths between exact ages x and x + 1 and E x c is
the central exposed to risk at age x.
Gompertz’ Law implies that m x + 1 = Bc
2
x + 1
2 .
Substituting this expression into the likelihood produces
%% ---  10%%%%%%%%%%%%%%%%%%%%%%%%%%%%%%%%%%%%%%%%%%%%— April 2003 — %%%%%%%%%%%%%%%%%%%%%%%%%%%%%%%%%%%%%%%%%%%%
æ x + 1 ö
K ç Bc 2 ÷
è
ø
dx
x + 1
æ
ö
exp ç - Bc 2 E x c ÷ .
è
ø
Over all ages in the investigation the likelihood is then proportional to
æ x + 1 ö
Õ ç è Bc 2 ÷ ø
x
dx
x + 1
æ
ö
exp ç - Bc 2 E x c ÷ ,
è
ø
and B and c can be obtained by numerical maximisation of this expression.
Alternatively,
If, between exact ages x and x + 1, the force of mortality is
m x + 1 , then Gompertz’ Law implies that
2
m x + 1 = Bc
2
x + 1
2 .
Taking logarithms of this equation produces
log m x + 1 = log B + ( x + 1 ) log c ,
2
2
and B and c can be estimated from the crude estimates
of m x + 1 by a linear regression of log m x + 1 against x + 1⁄2.
2
2
Alternative solutions such as these using weighted squares were also given credit.
Calculate the crude mortality rates q̂ x
Calculate
å w ( q ˆ
x
- q x ) = S 2
2
x
where w x = E x or w x \mu
1
Var ( E x )
c x ( c - 1 )
é æ - b ö ù
÷ ÷ ú
and q x = 1 - ê exp ç ç
log
c
ø û
è
ë
Choose B and c to minimise S 2
Or similarly, but using the crude values m̂ x and m x = Bc x
%% ---  11%%%%%%%%%%%%%%%%%%%%%%%%%%%%%%%%%%%%%%%%%%%%— April 2003 — %%%%%%%%%%%%%%%%%%%%%%%%%%%%%%%%%%%%%%%%%%%%
