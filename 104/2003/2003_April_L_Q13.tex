\documentclass[a4paper,12pt]{article}

%%%%%%%%%%%%%%%%%%%%%%%%%%%%%%%%%%%%%%%%%%%%%%%%%%%%%%%%%%%%%%%%%%%%%%%%%%%%%%%%%%%%%%%%%%%%%%%%%%%%%%%%%%%%%%%%%%%%%%%%%%%%%%%%%%%%%%%%%%%%%%%%%%%%%%%%%%%%%%%%%%%%%%%%%%%%%%%%%%%%%%%%%%%%%%%%%%%%%%%%%%%%%%%%%%%%%%%%%%%%%%%%%%%%%%%%%%%%%%%%%%%%%%%%%%%%

\usepackage{eurosym}
\usepackage{vmargin}
\usepackage{amsmath}
\usepackage{graphics}
\usepackage{epsfig}
\usepackage{enumerate}
\usepackage{multicol}
\usepackage{subfigure}
\usepackage{fancyhdr}
\usepackage{listings}
\usepackage{framed}
\usepackage{graphicx}
\usepackage{amsmath}
\usepackage{chngpage}

%\usepackage{bigints}
\usepackage{vmargin}

% left top textwidth textheight headheight

% headsep footheight footskip

\setmargins{2.0cm}{2.5cm}{16 cm}{22cm}{0.5cm}{0cm}{1cm}{1cm}

\renewcommand{\baselinestretch}{1.3}

\setcounter{MaxMatrixCols}{10}

\begin{document}




13
A 20 year endowment assurance policy was issued to a man aged exactly 40 on
1 January 1991. The sum assured of \$10,000 is payable immediately on death or on
survival to the end of the term.
Premiums of \$300 per annum are payable annually in advance for the term of the
policy, ceasing on earlier death.
(i) Explain why the life insurance company will hold reserves for this policy. 
(ii) (a)
(b)
(iii) Calculate the prospective policy value at the end of 2003.
(iv) Calculate the mortality profit for this policy for 2003 if the policyholder dies
during 2003.

[Total 14]
Define the prospective and retrospective policy values for the policy.
Explain why the two policy values may be different.

Basis: Mortality: AM92 ultimate
Interest: 6% per annum throughout
Expenses are ignored
%%%%%%%%%%%%%%%%%%%%%%%%—7

13
(i)
For this policy, the expected cost of paying benefits increases over the term,
but the premiums paid are level.
This means that the premiums paid in the early years will be more than the
expected cost of the benefits, but those in later years will be less.
It is prudent for the insurance company to set aside (or reserve) the premiums
not required in the early years to cover the shortfall in later years. If the
company spent all the premiums received in the early years it may not be able
to find the money required to pay the benefit later in the contract — ultimately
the company could become insolvent.
(ii)
(a)
Prospective policy value
= Expected present value of future outgo
less
Expected present value of future
income
Retrospective policy value = Accumulated value of premiums
received to date, allowing for interest
and survivorship
less
Accumulated value of benefits (and
expenses) paid to date, allowing for
interest and survivorship
Alternatively
At time t:
Prospective policy value
= 10000 A 40 + t :20 - t - 300  ́ a && 40 + t :20 - t
Retrospective policy value =
(b)
300  ́ s && 40: t - 10000  ́
(1.06) t
 ́ A 1
40: t
p
t 40
The two values will be equal if:
1 they are calculated on the same basis and
2 the same basis was used to calculate the premiums in the policy
value calculations
The assumptions used to calculate the retrospective value will be based
on the experienced conditions to date.
For the prospective calculation, the assumptions will be those
considered appropriate for the future remaining term.
%% ---  27%%%%%%%%%%%%%%%%%%%%%%%%%%%%%%%%%%%%%%%%%%%%— April 2003 — %%%%%%%%%%%%%%%%%%%%%%%%%%%%%%%%%%%%%%%%%%%%
These two sets of assumptions will not generally be the same.
Further the assumptions used to calculate the premiums, whilst
appropriate at the time, may not be considered appropriate for either
policy value calculation.
(iii)
The prospective policy value at the end of 2003 is given by the formula:
t V x : t
So,
= 10000  ́ A x + t : n - t - P  ́ a && x + t : n - t
13 V 40:20
= 10000  ́ A 53:7 - P  ́ a && 53:7
= 10000  ́ A 53:7 - 300  ́ 5.847
But,
A 53:7 = A 1
53:7
1
53:7
+ A
= (1.06)
1
53:7
A
=
1
v 60 l 60
v 53 l 53
2 A 1
53:7
=
1
53:7
+ A
= (1.06)
1
2  ́
1.06 - 60  ́ 9287.2164
1.06 - 53  ́ 9630.0522
( A
53:7
1
53:7
- A
) + A
1
53:7
= 0.64138
A 53:7 = 0.66904(from tables)
(iv)
So, A 53:7 = 1.06
So, 13 V 40:20
1
2  ́ (0.66904 - 0.64138) + 0.64138 = 0.66986
= 10, 000  ́ 0.66986 - 300  ́ 5.847 = \$4,944.50
Mortality profit = Expected Death Strain – Actual Death Strain
Expected Death Strain = q x + t (S - t +1 V) = q 52 (10000 – 13 V)
= 0.003152 (10000 – 4944.50)
= \$15.93
Actual Death Strain = 1  ́ (S - t +1 V) = 10000 – 13 V
= 10000 – 4944.50
= \$5,055.50
Mortality Profit = 15.93 – 5055.50 = - \$5,039.57 (i.e. a loss)
This answer assumes the death benefit is paid at the end of year of death. The
alternative, assuming payment on average half-way through the year was
given full credit.
%% ---  28

\end{document}
