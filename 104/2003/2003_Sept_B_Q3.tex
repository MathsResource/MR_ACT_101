\documentclass[a4paper,12pt]{article}

%%%%%%%%%%%%%%%%%%%%%%%%%%%%%%%%%%%%%%%%%%%%%%%%%%%%%%%%%%%%%%%%%%%%%%%%%%%%%%%%%%%%%%%%%%%%%%%%%%%%%%%%%%%%%%%%%%%%%%%%%%%%%%%%%%%%%%%%%%%%%%%%%%%%%%%%%%%%%%%%%%%%%%%%%%%%%%%%%%%%%%%%%%%%%%%%%%%%%%%%%%%%%%%%%%%%%%%%%%%%%%%%%%%%%%%%%%%%%%%%%%%%%%%%%%%%

\usepackage{eurosym}
\usepackage{vmargin}
\usepackage{amsmath}
\usepackage{graphics}
\usepackage{epsfig}
\usepackage{enumerate}
\usepackage{multicol}
\usepackage{subfigure}
\usepackage{fancyhdr}
\usepackage{listings}
\usepackage{framed}
\usepackage{graphicx}
\usepackage{amsmath}
\usepackage{chngpage}

%\usepackage{bigints}
\usepackage{vmargin}

% left top textwidth textheight headheight

% headsep footheight footskip

\setmargins{2.0cm}{2.5cm}{16 cm}{22cm}{0.5cm}{0cm}{1cm}{1cm}

\renewcommand{\baselinestretch}{1.3}

\setcounter{MaxMatrixCols}{10}

\begin{document}

%%%%%%%%%%%%%%%%%%%%%%%%%%%%%%%%%%%%%%%%%%%%%%%%%%%

3
(i) List the data required for the exact calculation of the central exposed to risk of
lives aged x last birthday in a mortality investigation over the two year period
from 1 January 2001 to 1 January 2003.

(ii) In an investigation of mortality during the period 1 January 2001 to 1 January
2003, data are available on the number of lives under observation, aged x last
birthday, on 1 January 2001, 1 July 2001 and 1 January 2003.
Derive an approximation for the central exposed to risk at age x last birthday
over the period in terms of the populations recorded on each of these three
dates.

3
(i)
For each life in the investigation we require:
date of birth (or date of xth birthday);
date of entry into observation;
date of exit from observation.
(ii)
The central exposed to risk, E x c is given by the formula
2
E x c = ∫ P x , t dt ,
0
where P x , t is the number of lives under observation aged x last birthday at time
t, measured as the duration since the start of the investigation.
To estimate this, we use the trapezium rule (assuming P x,t is linear between
census dates).
Let the population aged x last birthday on 1 January 2001 be P x,0 ; and the
corresponding populations on 1 July 2001 and 1 January 2003 be P x,0.5 and
P x,2 .
