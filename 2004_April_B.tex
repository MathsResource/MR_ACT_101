\documentclass[a4paper,12pt]{article}

%%%%%%%%%%%%%%%%%%%%%%%%%%%%%%%%%%%%%%%%%%%%%%%%%%%%%%%%%%%%%%%%%%%%%%%%%%%%%%%%%%%%%%%%%%%%%%%%%%%%%%%%%%%%%%%%%%%%%%%%%%%%%%%%%%%%%%%%%%%%%%%%%%%%%%%%%%%%%%%%%%%%%%%%%%%%%%%%%%%%%%%%%%%%%%%%%%%%%%%%%%%%%%%%%%%%%%%%%%%%%%%%%%%%%%%%%%%%%%%%%%%%%%%%%%%%

\usepackage{eurosym}
\usepackage{vmargin}
\usepackage{amsmath}
\usepackage{graphics}
\usepackage{epsfig}
\usepackage{enumerate}
\usepackage{multicol}
\usepackage{subfigure}
\usepackage{fancyhdr}
\usepackage{listings}
\usepackage{framed}
\usepackage{graphicx}
\usepackage{amsmath}
\usepackage{chngpage}

%\usepackage{bigints}
\usepackage{vmargin}

% left top textwidth textheight headheight

% headsep footheight footskip

\setmargins{2.0cm}{2.5cm}{16 cm}{22cm}{0.5cm}{0cm}{1cm}{1cm}

\renewcommand{\baselinestretch}{1.3}

\setcounter{MaxMatrixCols}{10}

\begin{document}

\begin{enumerate}
\item
5 Suppose that is an unbiased estimator of a parameter and has variance
2
10
.
Derive an expression for the mean square error of k , where k is a constant, and
determine the value of k for which the mean square error is a minimum.
%%%%%%%%%%%%%%%%%%%%%%%%%%%%%%%%%%%%%%%%%%%%%%%%%%%%%%%%%%%%%%%%%%%%%%%%%%%%%%%%%%%%%%%%%%%%%%%%%%%%%%%%%%%
\item Let X and Y be random variables which each takes values 1 and 2 only.
Calculate E[X|Y = 2], given that E[X] = 6/5, E[X|Y = 1] = 7/6, and P(Y = 1) = 3/5.
[2]
%%%%%%%%%%%%%%%%%%%%%%%%%%%%%%%%%%%%%%%%%%%%%%%%%%%%%%%%%%%%%%%%%%%%%%%%%%%%%%%%%%%%%%%%%%%%%%%%%%%%%%%%%%%
\item An insurance company has a portfolio of large industrial insurance risks. Three such
independent risks, labelled A, B, C, are being investigated. The sums insured, in units
of £100 million, are 2.5, 4.8, 7.2 for A, B, C, respectively. The company's expert risk
assessors estimate that the probabilities of a claim arising in the next calendar year are
0.01, 0.02, 0.005 for A, B, C, respectively. If a claim does arise, then the full sum
insured will be paid out and no further claims can then arise for that risk. The above
probability estimates should be used throughout this question.
(i) Determine, for this group of three risks, using suitable indicator variables or
otherwise, the mean and standard deviation of the total claim amount paid out
in the next calendar year.
[3]
(ii) You are told that exactly one claim has arisen. Calculate the mean total claim
amount and comment on your answer in relation to your answer in part (i).
%%%%%%%%%%%%%%%%%%%%%%%%%%%%%%%%%%%%%%%%%%%%%%%%%%%%%%%%%%%%%%%%%%%%%%%%%%%%%%%%%%%%%%%%%%%%%%%%%%%%%%%%%%%
\item Let X1 and X2 be independent Poisson variables with respective parameters 1
and 2 .
Let S = X1 X2
(i) Show that S has a Poisson distribution with mean 1 2 . [2]
(ii) Show that the conditional distribution of X1 given S = s is binomial, and state
its parameters.
\end{enumerate}
%%%%%%%%%%%%%%%%%%%%%%%%%%%%%%%%%%%%%%%%%%%%%%%%%%%%%%%%%%%%%%%%%%%%%%%%%%%%%%%%%%%%%%%%%%%%%%%%%%%%%%%%%%%
\newpage
5 MSE( k ) = E(( k )2) = var( k ) + [E( k )]2 2 E( k ) + 2
=
2 2
2 ( )2 2 2 2 2 2 1
10 10
k
k k k k k
2
1
10
( ) 1 10
2 2 2 0 .
10 1 11
dMSE k k
k k
dk
This is clearly a minimum as the MSE is a quadratic in k and the coefficient of k2 is
positive.
OR: Note explicitly in first line that
2
MSE k var k bias k
[Note: Many candidates stated in error that k was unbiased for . Some attempted
to minimise the function of k by differentiating with respect to rather than k.]
6 E[X] = E[E[X|Y]] = P(Y = 1) E[X|Y = 1] + P(Y = 2) E[X|Y = 2]
So 1.2 = 0.6 (7/6) + 0.4 E[X|Y = 2] and so E[X|Y = 2] = 1.25
7 (i) Let IA, IB, IC be indicator variables such that
P(IA = 1) = 0.01, P(IB = 1) = 0.02, P(IC = 1) = 0.005 and 0 otherwise.
%%%%%%%%%%%%%%%%%%%%%%%%%%%%%%%%%%%%%
Page 4
Let T = total claim amount = 2.5 IA + 4.8 IB + 7.2 IC
E(T) = 2.5(0.01) + 4.8(0.02) + 7.2(0.005) = 0.157 (or £15.7m)
Var(T) = 2.52(0.01)(0.99) + 4.82(0.02)(0.98) + 7.22(0.005)(0.995) = 0.7714
s.d.(T) = 0.878 (or £87.8m)
(ii) P(1 claim) = P(A not BC) + P(B not AC) + P(C not AB)
= (0.01)(0.98)(0.995) + (0.02)(0.99)(0.995) + (0.005)(0.99)(0.98)
= 0.009751 + 0.019701 + 0.004851 = 0.034303
P(A|1 claim) = 0.009751/0.034303 = 0.2843
P(B|1 claim) = 0.019701/0.034303 = 0.5743
P(C|1 claim) = 0.004851/0.034303 = 0.1414
E(T |1 claim) = 2.5(0.2843) + 4.8(0.5743) + 7.2(0.1414) = 4.485
(or £448.5m)
Much larger given that a claim does in fact arise.



8 (i)
1 ( ) exp{ 1( 1)} t
MX t e
2 2 ( ) exp{ ( t 1)}
MX t e
1 2
MS (t) MX (t)MX (t)
exp{ 1( 1)} exp{ 2 ( 1)} et et
exp{( 1 2 )( 1)} et
This is the MGF of a Po( 1 2 ) .
S Po( 1 2 )
(ii) 1 1 1 2
1 1 1 2
1 2
( , )
( | )
( )
P X x X X s
P X x S X X s
P X X s
1 1 1 2
1 1
1 2
1 2
1 1 2 1 1 1 1 2
( )
1 2 1 2 1 1 2
( ) ( ) ! ( )!
( ) ( ) ( )
!
x s x
x s x
s s
e e
P X x P X s x x s x s
P X X s e x
s
1 1
1 1
1 1 2 1 2
1
x s x s
x
%%%%%%%%%%%%%%%%%%%%%%%%%%%%%%%%%%%%%
Page 5
Binomial distribution with parameters: n = s, 1
1 2
p .

\end{document}
