
\documentclass[a4paper,12pt]{article}

%%%%%%%%%%%%%%%%%%%%%%%%%%%%%%%%%%%%%%%%%%%%%%%%%%%%%%%%%%%%%%%%%%%%%%%%%%%%%%%%%%%%%%%%%%%%%%%%%%%%%%%%%%%%%%%%%%%%%%%%%%%%%%%%%%%%%%%%%%%%%%%%%%%%%%%%%%%%%%%%%%%%%%%%%%%%%%%%%%%%%%%%%%%%%%%%%%%%%%%%%%%%%%%%%%%%%%%%%%%%%%%%%%%%%%%%%%%%%%%%%%%%%%%%%%%%

\usepackage{eurosym}
\usepackage{vmargin}
\usepackage{amsmath}
\usepackage{graphics}
\usepackage{epsfig}
\usepackage{enumerate}
\usepackage{multicol}
\usepackage{subfigure}
\usepackage{fancyhdr}
\usepackage{listings}
\usepackage{framed}
\usepackage{graphicx}
\usepackage{amsmath}
\usepackage{chngpage}

%\usepackage{bigints}
\usepackage{vmargin}

% left top textwidth textheight headheight

% headsep footheight footskip

\setmargins{2.0cm}{2.5cm}{16 cm}{22cm}{0.5cm}{0cm}{1cm}{1cm}

\renewcommand{\baselinestretch}{1.3}

\setcounter{MaxMatrixCols}{10}

\begin{document}
\begin{enumerate}
%%%%%%%%%%%%%%%%%%%%%%%%%%%%%%%%%%%%%%%%%%%%%%%%%%%%%%%%%%%%%%%%%%%%%%%%%%%%%%%%%%%%%%%%%%%%%%%%%%%%%%%%%
\item 12 Twenty overweight executives take part in an experiment to compare the
effectiveness of two exercise methods, A (isometric), and B (isotonic). They are
allocated at random to the two methods, ten to isometric, ten to isotonic methods.
After several weeks, the reductions in abdomen measurements are recorded in
centimetres with the following results:
A (isometric method) 3.1 2.1 3.3 2.7 3.4 2.7 2.7 3.0 3.0 1.6
B (isotonic method) 4.5 4.1 2.7 2.2 4.7 2.2 3.6 3.0 3.3 3.4
\begin{enumerate}[(a)]
\item (a) Plot the data for the two exercise methods on a single diagram.
Comment on whether the response values for each exercise method
are well modelled by normal random variables.
(b) Perform a test to investigate whether the assumption of equal
variability for the responses for the two exercise methods is
reasonable.
(c) Perform a t-test to investigate whether these data support the
claim that the isotonic method is more effective than the other
method. [9]
\item (a) Determine a two-sided 95\% confidence interval for the difference in
the means for the two exercise methods.
(b) Assuming that the two sets of 10 measurements are taken from
normal populations with the same variance, determine a 95%
confidence interval for the common standard deviation. [7]
\end{enumerate}

%%%%%%%%%%%%%%%%%%%%%%%%%%%%%%%%%%%%%%%%%%%%%%%%%%%%%%%%%%%%%%%%%%%%%%%%%%%%%%%%%%%%%%%%%%%%%%%%%%%%%%%%%
\item 14 In a medical study on hypertension amongst young male athletes the researchers
were interested in the effects of the use of a particular (legal) stimulant on systolic
blood pressure (bp).
Ten young male athletes from a larger group who had agreed to take part in the
study were selected at random — none of those in the group were currently users of
the stimulant. The initial bp of each of the sample was measured in controlled
conditions. Each sample member was then exposed to the use of the stimulant in a
controlled manner for a fixed period of time. Each sample member was subject to a
similar exercise regime, and at the end of the period the bp of each of the sample
was again measured in the same controlled conditions as initially, giving the followup
bp.
The data obtained were as follows:
Athlete 1 2 3 4 5 6 7 8 9 10
Initial bp 116 107 129 119 116 113 135 121 112 123
Follow-up bp 123 111 140 129 130 118 143 128 110 132
Summaries: Initial Σx = 1191, Σx2 = 142471 Follow-up Σx = 1264, Σx2 = 160872

%%%%%%%%%%%%%%%%%%%%%%%%%%%%%%%%%%%%%%%%%%%%%%%%%%%%%%%%%%%%%%%%%%%%%%%%%%%%%%%%%%%%%%%%%%%%%%%%%%%%%%%%%
The following models are proposed as possible bases for the analysis:
(M1) The initial bp of the ith athlete, Xi , is distributed as Xi ~ N(μ, σ1
2) and the
follow-up bp, Yi , is distributed such that $Yi|Xi = x ~ N(x + \alpha, \sigma^2
2)$.
(M2) The initial bp of the ith athlete, Xi , is distributed as $Xi ~ N(μi , σ1
2)$ and the
follow-up bp, Yi , is distributed such that $Yi|Xi = x ~ N(x + \alpha, \sigma^2
2)$.
(M3) The initial bp of the ith athlete, Xi , is distributed as Xi ~ N(μi , σ1
2) and the
follow-up bp, Yi , is distributed such that $Yi|Xi = x ~ N(x + \alpha_i , \sigma^2
2)$.
\begin{enumerate}
\item (a) Explain briefly the differences between the physical assumptions
underlying the three models proposed above.
(b) Explain briefly why model (M3) above could not be used in the
analysis of the data as given. [7]
\item Adopting model (M1) above
(a) Using the initial bp data, calculate a symmetrical, two-sided, 95%
confidence interval for μ, and
(b) Calculate a point estimate of $\alpha$. [6]
\item Adopting model (M2) above, calculate a one-sided 95% confidence interval
for $\alpha$ which brings out the minimum realistic value for the mean increase
in bp attributable to taking the stimulant.
\end{enumerate}


%%%%%%%%%%%%%%%%%%%%%%%%%%%%%%%%%%%%%%%%%%%%%%%%%%%%%%%%%%%%%%%%%%%%%%%%%%%%%%%%%%%%%%%%%%%%%%%%%%%%%%%%%
\item 15 In a study into employee share ownership plans, data were obtained from ten
large insurance companies on the following two variables:
employee satisfaction with the plan (x);
employee commitment to the company (y).
For each company a random sample (of the same size) of employees completed
questionnaires in which satisfaction and commitment were recorded on a 1–10
scale, with 1 representing low satisfaction/commitment and 10 representing high
satisfaction/commitment. The resulting means provide each company’s
employees’ satisfaction and commitment score. These scores are given in the
following table:
Co. A B C D E F G H I J
x 5.05 4.12 5.38 4.17 3.81 4.47 5.41 4.88 4.64 5.19
y 5.36 4.59 5.42 4.35 4.03 5.34 5.64 4.89 4.52 5.88
% x = 47.12, x2 = 224.8554, y = 50.02, y2 = 253.5796, xy = 238.3676
\begin{enumerate}[(a)]
\item Draw a scatterplot of y against x and comment briefly on any relationship
between employee satisfaction and commitment. 
\item Calculate the fitted linear regression equation of y on x. 
\item Calculate the coefficient of determination R2 and relate its value to your
comment . 
\item Assuming the full normal model, calculate an estimate of the error
variance $\sigma^2$ and obtain a 95\% confidence interval for $\sigma^2$. 
\item Calculate a 95\% confidence interval for the true underlying slope
coefficient. 
\item For companies with an employees’ satisfaction score of 5.0, calculate an
estimate of the expected employees’ commitment score together with 95%
confidence limits. 
\end{enumerate}

%%%%%%%%%%%%%%%%%%%%%%%%%%%%%%%%%%%%%%%%%%%%%%%%%%%%
\end{enumerate}

%%%%%%%%%%%%%%%%%%%%%%%%%%%%%%%%%%%%%%%%%%%%%%%%%%%%%%%%%%%%%%%%%%%%%%%%%%%%%%%%%%%%%%%%%%%%%%%%%%%%%%%%%
13  (a) Plot for Isotonic-Isometric exercise methods:
Dotplot for Isotonic-Isometric
Normality seems OK for each data set.
Let XA , XB be reductions in measurements from the isometric and
isotonic methods, respectively.
A: 2 2 = 27.6, = 78.90 ; % xA xA xA = 2.76, sA = 0.3027, nA =10
B: 2 2 = 33.7, =120.53 ; % = 3.37, = 0.7734, 10 B B B B B x x x s n 
(b) 2 = 0.3027 ; 2 = 0.7734 A B s s
2 2 2 2
0 1 : = ; : A B A B H %   H   
1.6 2.6 3.6 4.6
Isotonic
Isometric

%%%%%%%%%%%%%%%%%%%%%%%%%%%%%%%%%%%%%%%%%%%%%%%%%%%%%%%%%%%%%%%%%%%%%%%%%%%%%%%%%%%%%%%%%%%%%%%%%%%%%%%%%
0.7734
= =2.56
0.3027
F on 9,9 d.f.
Upper 5\% point is 3.179, so p-value > 0.10.
Therefore do not reject H0.
(c) The pooled sample variance
%        2 2 2 = 1 1 / 2 p A A B B A B s n  s  n  s n  n 
% = {(9  0.3027) + (9  0.7734)}/18 = 0.538.
% The test statistic   2 1 1
= / B A
A B
t x x s
n n
%%%%%%%%%%%%%%%%
1 1
= (3.37 2.76) / 0.538
10 10
%%%%%%%%%%%%%%
= 1.86.
% H0 :  = 0; H1 :  > 0  : mean difference in reduction in
% abdomen measurements (B  A).
A one-sided test is appropriate.
There are 18 d.f. The upper 5% point of t18 is 1.734. Thus the
probability value is less than 0.05. There is sufficient evidence, at
the 5% level, to suggest that the isotonic method (B) is more
effective in reducing abdomen measurement.
\item (a) Two-sided 95% confidence interval for B  A :
%   2
18
1 1
(2.5%)
10 10 B A x x t s
2
3.37 2.76 2.101 0.538
10

0.61  0.69 = (0.08,1.3)
[Note that this just includes zero.]
(b)  
2
2
2 2 2 ~ A B
p
A B n n
S
n n     

Here nA + nB  2 = 18
95% confidence interval for 2 (common variance)

%%%%%%%%%%%%%%%%%%%%%%%%%%%%%%%%%%%%%%%%%%%%%%%%%%%%%%%%%%%%%%%%%%%%%%%%%%%%%%%%%%%%%%%%%%%%%%%%%%%%%%%%%
2 2
2 2
18 18
18 18
,
(0.025) (0.975)
%  s s 
%%%%%%%%%%%%%%%%%%%%%%%%%%%%%%%%%%%%%%%%%%%%%%5
18(0.538) 18(0.538)
,
31.53 8.231
%%%%%%%%%%%%%%%%%%%%%%%%%%%%%%%%%%%%%%%%%%%%%%%
Taking square-roots gives the 95% confidence interval for the
common standard deviation %  as  0.31, 1.18  = (0.55 ,1.08)
Some candidates used a % “paired samples” approach in part \item(a). This was quite
inappropriate.

%%%%%%%%%%%%%%%%%%%%%%%%%%%%%%%%%%%%%%%%%%%%%%%%%%%%%%%%%%%%%%%%%%%%%%%%%%%%%%%%%%%%%%%%%%%%%%%%%%%%%%%%%
14 
\begin{itemize}
\item (a) In Model M1 we have a basic model for the initial bp of the whole
population of young male athletes, with mean , and the mean bp
increases by%   after using the stimulant.
(Note: E(follow-up bp) = $E(Y) = E[E(Y|X)] = E[X + ] =  + )
Model M2 extends M1 by allowing for a different initial mean for
each athlete % (i).
Model M3 extends M2 by allowing for a different mean increase in
bp for each athlete % (i).
(Note: In all three models we have a single population variance for
initial bp and a single, but different, variance for follow-up bp.)
(Note: V(follow-up bp) = V(Y) = V[E(Y|X)] + E[V(Y|X)] = 1
2 + $ 2
2)
(b) For 10 athletes, M3 has 22 unknown parameters — but we only
have 20 data points. So estimation of parameters is impossible.
\item (a) Initial bp: % x = 1191, x2 = 142471 so x = 119.1 , s2 = 69.211
t9(0.025) = 2.262
95% CI for  is 119.1  {2.262  (69.211/10)½} i.e. 119.1  5.95
i.e. (113.15 , 125.05)
(b) Follow-up bp : x = 1264 x = 126.4 so ˆ = 126.4  119.1 = 7.3
\item Use the differences (follow-up less initial) for each athlete:
di : 7, 4, 11, 10, 14, 5, 8, 7, 2, 9
\end{itemize}


%%%%%%%%%%%%%%%%%%%%%%%%%%%%%%%%%%%%%%%%%%%%%%%%%%%%%%%%%%%%%%%%%%%%%%%%%%%%%%%%%%%%%%%%%%%%%%%%%%%%%%%%%
% d = 73, d2 = 705 so d = 7.3 , s2 = 19.122
t9(0.05) = 1.833
%  95% CI (one-sided) for  is (7.3  1.833(19.122/10)½, ) i.e. (4.77, )
The early part of this question (on comparing models) looked hard, but,
pleasingly, was generally well-attempted.
15 \item see plot
there seems to be an increasing and linear relationship.
\item
47.122
= 224.8554 = 2.82596
10 xx S 
50.022
= 253.5796 = 3.37956
10 yy S 
(47.12)(50.02)
= 238.3676 = 2.67336
10 xy S 
% ˆ 2.67336 = = 0.946001
2.82596
% 
% ˆ 50.02 47.12 = (0.946001) = 0.544
10 10
%  
y = 0.544 + 0.9460x
4.0 4.5 5.0 5.5
4
5
6
x
y
Commitment v. Satisfaction

%%%%%%%%%%%%%%%%%%%%%%%%%%%%%%%%%%%%%%%%%%%%%%%%%%%%%%%%%%%%%%%%%%%%%%%%%%%%%%%%%%%%%%%%%%%%%%%%%%%%%%%%%
\begin{itemize}
    \item 

2
2 (2.67336)
=
(2.82596)(3.37956)
R = 0.748 or 74.8%
quite high, showing agreement with a linear relationship.
\item
2
2 1 2.67336
ˆ = (3.37956 ) = 0.1063
8 2.82596
 
For confidence interval use
2
2
2 2
( 2) ˆ
~ n
n
%%%%%%%%%%%%%%%%%%%%%%%%%%%%%%%%%%%%%%%
2 2
2 2
2 2
( 2) ˆ ( 2) ˆ
,
(0.025) (0.975) n n
n n
%%%%%%%%%%%%%%%%%%%%%%%%%%%%%%%%%%%%%%%%%%
8(0.1063) 8(0.1063)
= , = (0.0485,0.3902)
17.53 2.180
%%%%%%%%%%%%%%%%%%%%%%%%%%%%%%%%%%%%%%%%
\item 
= 0.9460
its standard error is
ˆ 2 0.1063
= = 0.1939
2.82596 xx S
\end{itemize}
%%%%%%%%%%%%%%%%%%%%%%%%%%%%%%%%%%
95\% confidence interval is 8

% ˆ  t (0.025)  s.e.
% = 0.9460  2.306(0.1939) = 0.946  0.447 or (0.499, 1.393)
% (vi) estimate is 0
% ˆ = ˆ  ˆ(5.0) = 0.544 + 0.9460(5.0) = 5.274
2
2
0
1 (5.0 )
. .( ˆ ) = ˆ ( )
xx
x
s e
n S

  
1 (5.0 4.712)2
= 0.1063( ) = 0.1173
10 2.82596
%%%%%%%%%%%%%%%
95\% confidence limits are  2.306(0.1173) =  0.270 or (5.004, 5.544)
\end{document}
