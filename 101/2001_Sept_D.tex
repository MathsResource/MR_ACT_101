\documentclass[a4paper,12pt]{article}

%%%%%%%%%%%%%%%%%%%%%%%%%%%%%%%%%%%%%%%%%%%%%%%%%%%%%%%%%%%%%%%%%%%%%%%%%%%%%%%%%%%%%%%%%%%%%%%%%%%%%%%%%%%%%%%%%%%%%%%%%%%%%%%%%%%%%%%%%%%%%%%%%%%%%%%%%%%%%%%%%%%%%%%%%%%%%%%%%%%%%%%%%%%%%%%%%%%%%%%%%%%%%%%%%%%%%%%%%%%%%%%%%%%%%%%%%%%%%%%%%%%%%%%%%%%%

\usepackage{eurosym}
\usepackage{vmargin}
\usepackage{amsmath}
\usepackage{graphics}
\usepackage{epsfig}
\usepackage{enumerate}
\usepackage{multicol}
\usepackage{subfigure}
\usepackage{fancyhdr}
\usepackage{listings}
\usepackage{framed}
\usepackage{graphicx}
\usepackage{amsmath}
\usepackage{chngpage}

%\usepackage{bigints}
\usepackage{vmargin}

% left top textwidth textheight headheight

% headsep footheight footskip

\setmargins{2.0cm}{2.5cm}{16 cm}{22cm}{0.5cm}{0cm}{1cm}{1cm}

\renewcommand{\baselinestretch}{1.3}

\setcounter{MaxMatrixCols}{10}

\begin{document}
%%%%%%%%%%%%%%%%%%%%%%%%%%%%%%%%%%%%%%%%%%%%%%%%%%%%%%%%%%%%%%%%%%%%%%%%%%%%%%%%%%%%%%%%%%%%%%%%%%%%%%%%%
\item The probability density function of a random variable $X$ is given by

\[ f(x) = \begin{cases} kx(1-ax^2) & \mbox{ for } 0 \leq x \leq 1 \\
0 & \mbox{otherwise}
\end{cases}
\]

where $k$ and $a$ are positive constants.
\begin{enumerate}
\item Show that $a \leq 1$, and determine the value of $k$ in terms of $a$. 
\item For the case $a = 1$, determine the mean of $X$. 
\end{enumerate}


\begin{itemize}
    \item 

\[f(x) = kx(1 \;-\; ax^2), 0 
 x 
 1,\]
0, otherwise.
\item To be a pdf f(x)  0 for 0 
 x 
 1  (1 \;-\; ax2)  0 since k > 0
 1  ax2 for x 
 1  a 
 1.

%%%%%%%%%%%%%%%%%%%%%%%%%%%%%%%%%%%%%%%%%%%%%%%%%%%%%%%%%%%%%%%%%%%%%%%%%%%%%%%%%%%%%%%%%%%%%%%%%%%%%%%%%
Also \;-\; 
1 1 3
0 0
\item 
Also

\[ \int^{1}_{0} f ( x ) dx \equiv 1\] therefore 

\begin{eqnarray*}
\int^{1}_{0} f ( x ) dx &=&  k \int^{1}_{0} ( x - ax^2) dx \\ 
&=& k \left[ \frac{1}{2}x^2 - \frac{a}{4}x^4\right]^{1}_{0} \\
&=& k \left[ \left( \frac{1}{2} - \frac{a}{4} \right) - \left( 0 - 0 \right) \right] \\
&=& k \left( \frac{2-a}{4} \right) \\
&=& 1 \\
\end{eqnarray*}
 
\[  k \left( \frac{2-a}{4} \right) = 1 \]
Re-arranging \[ k = \frac{4}{2 \;-\; a} \]

\item 
If $a=1$ then necessarily \[ k = \frac{4}{2-(1)} = 4\]

f(x) = 4x(1-x^2)

%%%%%%%%%%%%%%
\begin{eqnarray*}
E(X) &=& \int^{1}{0} 4x^2(1-x^2) dx \\
&=& \left[ \frac{4}{3}x^3 - \frac{4}{5}x^5 \right]^{1}_{0}\\
&=&  \frac{4}{3} -  \frac{4}{5}\\
&=&  \frac{20}{15} -  \frac{12}{15}\\
&=& \frac{8}{15}\\
\end{eqnarray*}
%%%%%%%%%%%%%%%%%%%%
\end{itemize}

%%%%%%%%%%%%%%%%%%%%%%%%%%%%%%%%%%%%%%%%%%%%%%%%%%%%%%%%%%%%%%%%%%%%%%%%%%%%%%%%%%%%%%%%%%%%%%%%%%%%%%%%%
\newpage 

A job takes $X$ minutes to complete, where $X$ is modelled as a $N(28,22)$ random
variable. Another job, independent of the first, takes $Y$ minutes to complete, and begins 5 minutes after the first job begins. $Y$ is modelled as a $N(25,12)$ random
variable.
Calculate the probability that the job that was begun last is first to be completed.

%%%%%%%%%%%%%%%%%%%%%%%%%%%%%%%%%%%%%%%
%% Question 8 
%---------------%
\begin{itemize}
\item $X \sim N(28,2^2)$
\item $Y \sim N(25,1^2)$
\end{itemize}

We require $P(X-Y >5$.

$X-Y \sim N(28-25, (2^2 + 1^2)$, i.e. $X-Y \sim N(3, 5$,

\[P\left( Z > \frac{5-3}{\sqrt{5}} \right) = P(Z > 0.894) = 0.186\]




\end{document}
