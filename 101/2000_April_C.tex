\documentclass[a4paper,12pt]{article}

%%%%%%%%%%%%%%%%%%%%%%%%%%%%%%%%%%%%%%%%%%%%%%%%%%%%%%%%%%%%%%%%%%%%%%%%%%%%%%%%%%%%%%%%%%%%%%%%%%%%%%%%%%%%%%%%%%%%%%%%%%%%%%%%%%%%%%%%%%%%%%%%%%%%%%%%%%%%%%%%%%%%%%%%%%%%%%%%%%%%%%%%%%%%%%%%%%%%%%%%%%%%%%%%%%%%%%%%%%%%%%%%%%%%%%%%%%%%%%%%%%%%%%%%%%%%

\usepackage{eurosym}
\usepackage{vmargin}
\usepackage{amsmath}
\usepackage{graphics}
\usepackage{epsfig}
\usepackage{enumerate}
\usepackage{multicol}
\usepackage{subfigure}
\usepackage{fancyhdr}
\usepackage{listings}
\usepackage{framed}
\usepackage{graphicx}
\usepackage{amsmath}
\usepackage{chngpage}

%\usepackage{bigints}
\usepackage{vmargin}

% left top textwidth textheight headheight

% headsep footheight footskip

\setmargins{2.0cm}{2.5cm}{16 cm}{22cm}{0.5cm}{0cm}{1cm}{1cm}

\renewcommand{\baselinestretch}{1.3}

\setcounter{MaxMatrixCols}{10}

\begin{document}
%%%%%%%%%%%%%%%%%%%%%%%%%%%%%%%%%%%%%%%%%%%%%%%%%%%%%%%
\enumerate}
\item The discrete random variable X has the following probability function:
\[P(X = i) = 0.2 + ai : i = \{-2, -1, 0, 1, 2.\}\]

\begin{enumerate}[(a)]
\item (i) State the possible values that a can take. 
\item (ii) Given a random sample x1 , x2 , ..., xn from this distribution, determine
the method of moments estimate of a and show that this can result in
inadmissible estimates (i.e. estimates outside the range of possible
values of a). 
\end{enumerate}
 %%%%%%%%%%%%%%%%%%%%%%%%%%%%%%%%%%%%%%%%%%%%%%%%%%%%%%%%%%%%%%%%%%%%%
9 

\begin{itemize} 
\item (i) This will be a probability function provided the specified probabilities are
non-negative; i.e. if and only if $-0.1 < a < 0.1$.
\item (ii) The method of moments estimate of a is obtained by equating the sample
mean to the population mean. 

\end{itemize}
To do this note that
 = 2
2
 i(0.2 + ai) = a 2
2 
 i2 = 10a.
Thus, the method of moments estimate is 10
X .

As X can take any value between -2 and +2, the method of moments
estimate can take any value between –0.2 and +0.2. Thus it can be
outside the range (-0.1, 0.1).
\newpage
%%%%%%%%%%%%%%%%%%%%%%%%%%%%%%%%%%%%%%%%%%%%%%%%%%%%%%%
\item %%% 2000 April Q 10

Under a particular model for the evolution of the size of a population over
time, the probability generating function of $X_t$ , the size at time $t$, 
$G_X_t ( s )$ , is
given by

\[ G_X_t ( s ) =  \frac{s + \lambda t (1 − s )}{1 + \lambda t (1 − s )} \]


where $G_X_t ( s ) = E ( s ^{X_t} )$ and $\lambda > 0$.


If the population dies out, it remains in this extinct state for ever.
\begin{enumerate}[(a)]
    \item Show that the expected size of the population at any time t is 1. 
\item  Show that the probability that the population has become extinct by
time t is given by lt / (1 + lt). 
\item Comment briefly on the future prospects for the population. [1]
\end{enumerate}

%%%%%%%%%%%%%%%%%%%%%%%%%%%%%%%%%%%%%%%%%%%%%%%%%%%
10 

\begin{itemize}
    \item (i)
t
\[X G^{\prime} (s) = (1 - \lambda t){1 + \lambda t(1 - s)}1 -{s + \lambda t(1 - s)}{-\lambda t}{1 + \lambda t(1 - s)}2
  =
t\]
\[X G^{\prime} (1) = 1 - \lambda t + \lambda t = 1\]
\item (ii) P(extinct by time t) = P(population size at time t is zero)
= (0)
t
X G = \lambda t / (1 + \lambda t)
\item (iii) P(extinct by time t) 
 1 as t 
 , so eventual extinction is certain.
\end{itemize}
%%%%%%%%%%%%%%%%%%%%%%%%%%%%%%%%%%%%%%%%%%%%%%%%%%%%%%%%%%%%%%

\newpage
\item A charity sent eight hundred of its supporters information packs about its activities and asked for donations to help it continue its work. Two hundred were sent a pack about its work in education, two hundred a pack about its
work in health care, and four hundred a pack about its anti-poverty programmes. Ninety-seven of the eight hundred people responded with donations; the breakdown is shown in the table below.



\begin{center}
\begin{tabular}{|c|c|c|c|c|}\hline
Expected 	&	Education 	&	Health 	&	Poverty 	&	Total	\\ \hline
Donate & 26 & 31 & 40 & 97 \\ \hline
Don’t donate  & 174 & 169& 360&  703\\ \hline
Total	&	200	&	200	&	400	&	800	\\ \hline
\end{tabular}
\end{center}

Perform a $\chi^2$ test on this table to investigate whether the proportions who
send donations are affected by the type of pack received. 

%%%%%%%%%%%%%%%%%%%%%%%%%%%%%%%%%%%%%%%%%%%%%%%%%%%%%%%
\medskip 
Complete the table of Expected values:
\begin{center}
\begin{tabular}{|c|c|c|c|c|}\hline
Expected 	&	Education 	&	Health 	&	Poverty 	&	Total	\\ \hline
Donate 	&	24.25	&	24.25	&	48.5	&	97	\\ \hline
Don’t donate	&	175.75	&	175.75	&	351.5	&	703	\\ \hline
	&	200	&	200	&	400	&	800	\\ \hline
\end{tabular}
\end{center}

Calculate $\chi^2$ = 3.98.
The 5\% point of a $\chi^2$ random variable on 2 degrees of freedom is 5.991,
so the $\chi^2$ test is not significant at the 5\% level.
On the basis of the data collected, it is plausible that the three packs are equally
effective.
\newpage
%%%%%%%%%%%%%%%%%%%%%%%%%%%%%%%%%%%%%%%%%%%%%%%%%%%%%%%
\item A random sample of 200 pairs of observations $(x, y)$ from a discrete bivariate
distribution $(X, Y)$ is as follows:
\begin{itemize}
    \item the observation (-2, 2) occurs 50 times
\item the observation (0, 0) occurs 90 times
\item the observation (2, -1) occurs 60 times.
\end{itemize}

Calculate the sample correlation coefficient for these data.
%%%%%%%%%%%%%%%%%%%%%%%%%%%%%%%%%%%%%%%%%%%%%%%%%%%%%%%


12
\begin{itemize}
    \item $\sum x = 50(-2) + 0 + 60(2) = 20$
    \item $\sum x2 = 50(4) + 0 + 60(4) = 440$
\item $\sum y = 50(2) + 0 + 60(-1) = 40$ 
\item $\sum y2 = 50(4) + 0 + 60(1) = 260$
\item $\sum xy = 50(-4) + 0 + 60(-2) = -320$
\end{itemize}
so\[ r = \frac{[-320 - (20 \times 40)/200]}{ \sqrt{ [(440 - 202/200)(260 - 402/200)]}} = -0.975\]
%%%%%%%%%%%%%%%%%%%%%%%%%%%%%%%%%%%%%%%%%%%%%%%%%%%%%%%
\end{enumerate}
\end{document}
