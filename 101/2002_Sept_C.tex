\documentclass[a4paper,12pt]{article}

%%%%%%%%%%%%%%%%%%%%%%%%%%%%%%%%%%%%%%%%%%%%%%%%%%%%%%%%%%%%%%%%%%%%%%%%%%%%%%%%%%%%%%%%%%%%%%%%%%%%%%%%%%%%%%%%%%%%%%%%%%%%%%%%%%%%%%%%%%%%%%%%%%%%%%%%%%%%%%%%%%%%%%%%%%%%%%%%%%%%%%%%%%%%%%%%%%%%%%%%%%%%%%%%%%%%%%%%%%%%%%%%%%%%%%%%%%%%%%%%%%%%%%%%%%%%

\usepackage{eurosym}
\usepackage{vmargin}
\usepackage{amsmath}
\usepackage{graphics}
\usepackage{epsfig}
\usepackage{enumerate}
\usepackage{multicol}
\usepackage{subfigure}
\usepackage{fancyhdr}
\usepackage{listings}
\usepackage{framed}
\usepackage{graphicx}
\usepackage{amsmath}
\usepackage{chngpage}

%\usepackage{bigints}
\usepackage{vmargin}

% left top textwidth textheight headheight

% headsep footheight footskip

\setmargins{2.0cm}{2.5cm}{16 cm}{22cm}{0.5cm}{0cm}{1cm}{1cm}

\renewcommand{\baselinestretch}{1.3}

\setcounter{MaxMatrixCols}{10}

\begin{document}

\begin{enumerate}
%%%%%%%%%%%%%%%%%%%%%%%%%%%%%%%%%%%%%%%%%%%%%%%%%%%%%%%%%%%%%%%%%%
\item 9 Thirty employees working in the call centre of a bank were chosen at random from
the workforce and agreed to assist in the assessment of two different training
programmes (A and B).
Ten of the selected employees were randomly assigned to each of the two training
programmes, while the other ten were to receive no additional training and act as
controls.
Shortly after the training began, one of the employees on training programme B and
two of the control group left the employment of the bank.
At the end of the training period each of the twenty-seven employees still involved
was observed at work over a period of time by performance assessors. Each
employee was then given a performance score (measuring a range of aspects of their
work) expressed as a mark out of a possible maximum of 100.
The results were as follows:
\begin{verbatim}
    Performance scores Totals
Control 55 74 64 62 37 78 50 44 — — 464
Training programme A 63 79 60 75 89 58 75 72 84 69 724
Training programme B 64 55 57 73 51 60 62 78 68 — 568

\end{verbatim}
xij
2 = 118,128 .
\begin{enumerate}[(a)]
\item Plot the data in a simple and informative way which displays both the “within
treatment” and “between treatment” variation. [2]
\item  Conduct an analysis of variance to investigate whether differences exist
among the three “treatments” and comment briefly on your findings. [7]
\end{enumerate}
\item 10 The number of claims N which arise from a portfolio of business is modelled as a
Poisson variable with mean . The claim amounts Xi : i = 1, 2 , ..., N are modelled as
independent gamma variables each with parameters  and  and are independent of
N. Let S be the total claim amount arising from this portfolio.
\begin{enumerate}[(a)]

\item (i) Obtain expressions for the mean and standard deviation of S in terms of , 
and using general results for the mean and variance of S, 
\item (ii) Consider the case where  = 100 and the individual claim amounts have mean
£100 and standard deviation £50.
(a) Calculate the mean and standard deviation of the total claim amount S.
\item (b) Calculate an approximate value for the probability that the total claim
amount S exceeds £12,500, giving a brief justification of your
approach. 
\end{enumerate}
%%%%%%%%%%%%%%%%%%%%%%%%%%%%%%%%%%%%%%%%%%%%%%%%%%%%%%%%%%%%%%%%%%
\newpage
\item 11 Consider a group of n males each aged 30 years living in a particular society. The
lives may be assumed to be independent.
Suppose that x of the n men die by the end of a subsequent period of duration t0 years
and that n  x survive the period.
Suppose that the lifetime of these men from age 30 is to be modelled as a random
variable with an exponential distribution with mean 1/ hours.
\begin{enumerate}[(a)]
    \item 
(i) State the distribution of the random variable X whose value x, the number of
men who die within t0 years, has been observed. 
\item (ii) (a) Verify that the log-likelihood of the observation x is given by:
\[  = log L( ) = x log(1 e t0 ) (n x) t0 
       + constant\]
(b) Determine ˆ, the maximum likelihood estimator of .
\item (iii) Show that
\[0
0
2 2
0
2 =
1
t
t
E nt e
e


  
        	
 [4]\]
\item (iv) (a) Consider the case n = 1,000, t0 = 20 years, and x = 320.
Evaluate ˆ, and show that the value of its standard error is
approximately 0.00108.
\item (b) Hence, calculate an approximate 95% confidence interval for the mean
lifetime from age 30 of such men. [7]
\end{enumerate}
%%%%%%%%%%%%%%%%%%%%%%%%%%%%%%%%%%%%%%%%%%%%%%%%%%%%%%%%%%%%%%%%%%
\end{enumerate}

%%%%%%%%%%%%%%%%%%%%%%%%%%%%%%%%%%%%%%%%%%%%%%%%%%%%%%%%%%%%%%%%%%
9 (i)
^ indicates treatment means : helpful but not required for the marks
“Within treatment” variation is the spread within each set of points, “between
treatment” variation is the spread among the 3 treatment means.
(ii) SST = 118128 – 17562/27 = 3923.0
SSB = (4642/8 + 7242/10 + 5682/9) – 17562/27 = 971.67
 SSR = 3923.0 – 971.67 = 2951.3
Source of variation df SS MSS
Between treatments 2 971.67 485.84
Residual 24 2951.3 122.97
Total 26 3923.0
Under H0 : no treatment effects F = 485.84/122.97 = 3.95 on 2,24 df
P-value is < 0.05, so we have some evidence against H0.
0 10 20 30 40 50 60 70 80 90 100
^
^
^
score
control
prog A
prog B
%%%%%%%%%%%%%%%%%%%%%%%%%%%%%%%%%%%%%%%
Page 6
We conclude that there is evidence of differences among the treatments. The
data suggest that training programme A gives a higher mean score than the
others.
10 (i) E(S) = E(N)E(X ) = 


2 2
2 2 Var(S) = E(N)Var(X ) Var(N)[E(X )] = ( ) = (1 )   
    
  
2
sd(S) = (1 )  


(ii) (a) 2
2 =100 & = 50 = 4 , = 0.04  
  
 
E(S) = 100(100) = £10,000
2
( ) = 100(4)(1 4) = £1,118
0.04
sd S 
(b) Since N is going to be large, the Central Limit Theorem allows
normality.
( 12500) ( 12500 10000 = 2.236) where ~ (0,1)
1118
=1 0.987 = 0.013
P S P Z Z N 
  

11 (i) P(T < x) = 1 	 exp(	
x) so X ~ bi(n, 1 	 exp(	
t0))
(ii) (a) L(
)  (1 	 exp(	
t0))x [exp(	
t0) ]nx
(
) = xlog(1 	 exp(	
t0)) + (n 	 x)log[exp(	
t0)] + constant
= xlog(1 	 exp(	
t0)) 	 (n 	 x)
t0 + constant
(b) 

 = xt0 exp(	
t0) /(1 	 exp(	
t0)) 	 (n 	 x)t0
= xt0[1 + exp(	
t0) /(1 	 exp(	
t0))] 	 nt0
= xt0/[(1 	 exp(	
t0)] 	 nt0
%%%%%%%%%%%%%%%%%%%%%%%%%%%%%%%%%%%%%%%
Page 7
Setting 0 


  exp(	
t0) = 1 	 x/n 
0
ˆ 1 log(1 x)
t n
   
[OR: direct from MLE of P(survive) = exp(	
t0) is observed
proportion which survive, namely 1 – x/n ; hence MLE of 
.]
(iii) From (ii) (b)
2
2


 = 	xt0(1 	 exp(	
t0))2(t0 exp(	
t0))
= 	 xt0
2 exp(	
t0) (1 	exp(	
t0))2

2
2 E( ) 


 = t0
2 exp(	
t0)(1 	 exp(	
t0))2E(X)
= t0
2exp(	
t0)(1 	 exp(	
t0))2 n(1 	 exp(	
t0))
[using mean of binomial distribution from part (i)]
= n t0
2 exp(	
t0) /[1 	 exp(	
t0)]
(iv) (a) ˆ = 	 (1/20) log(0.68) = 0.019283
Estimate of exp(	
t0) is 0.68,
so
2
2 E( ) 


  1000  202  0.68/0.32 = 850000
 s.e.( ˆ ) 850000 1/ 2 0.0010847  0.00108     
(b) 0.019283 
 (1.96  0.0010847)
i.e. 0.019283 
 0.002126 i.e. 0.01716 to 0.02141
95% CI for mean lifetime 1/
 is (1/0.02141, 1/0.01716)
i.e. 46.7 years to 58.3 years
There was a slight error in line 6 of Question 11: the last word should have appeared as
“years” instead of “hours”. The examiners took this into account, and gave full credit to
any candidates who provided an alternative answer because of this.
\end{document}
