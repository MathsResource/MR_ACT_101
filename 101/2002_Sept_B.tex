\documentclass[a4paper,12pt]{article}

%%%%%%%%%%%%%%%%%%%%%%%%%%%%%%%%%%%%%%%%%%%%%%%%%%%%%%%%%%%%%%%%%%%%%%%%%%%%%%%%%%%%%%%%%%%%%%%%%%%%%%%%%%%%%%%%%%%%%%%%%%%%%%%%%%%%%%%%%%%%%%%%%%%%%%%%%%%%%%%%%%%%%%%%%%%%%%%%%%%%%%%%%%%%%%%%%%%%%%%%%%%%%%%%%%%%%%%%%%%%%%%%%%%%%%%%%%%%%%%%%%%%%%%%%%%%

\usepackage{eurosym}
\usepackage{vmargin}
\usepackage{amsmath}
\usepackage{graphics}
\usepackage{epsfig}
\usepackage{enumerate}
\usepackage{multicol}
\usepackage{subfigure}
\usepackage{fancyhdr}
\usepackage{listings}
\usepackage{framed}
\usepackage{graphicx}
\usepackage{amsmath}
\usepackage{chngpage}

%\usepackage{bigints}
\usepackage{vmargin}

% left top textwidth textheight headheight

% headsep footheight footskip

\setmargins{2.0cm}{2.5cm}{16 cm}{22cm}{0.5cm}{0cm}{1cm}{1cm}

\renewcommand{\baselinestretch}{1.3}

\setcounter{MaxMatrixCols}{10}

\begin{document}

\begin{enumerate}
\item 5 Suppose that a line is fitted by least squares to a set of data $\leq$$\leq$ xi , yi  , i =1, 2,$\leq$, n
which has sample correlation coefficient r. Let the fitted value at xi be denoted yˆi .
Show that the sample correlation coefficient of the data $\leq$$\leq$ yi , yˆi , i =1, 2,$\leq$, n, that
is, of the observed and fitted y values, is also equal to r. [3]
%%%%%%%%%%%%%%%%%%%%%%%%%%%%%%%%%%%%%%%%%%%%%%%%%%%%%%%%%%%%%%%%%%
\item 6 As part of an investigation an insurance company collected data for the year 2000 on
claims sizes for all claims on a certain type of motor insurance policy. The resulting
data are given below in the form of a grouped frequency distribution.
\begin{verbatim}
Claim size (£) Frequency
$\leq$ 100 862
> 100 and $\leq$ 200 608
> 200 and $\leq$ 300 1253
> 300 and $\leq$ 400 1066
> 400 and $\leq$ 500 558
> 500 1290
Total 5637
\end{verbatim}


\begin{enumerate}[(a)]
    \item (i) Calculate the cumulative frequencies and draw a graph of the claim size
distribution function (i.e. the cumulative frequencies against claim size). [3]
    \item (ii) You have been asked to determine the proportion of claim sizes which are less
than £250. Use linear interpolation to approximate this proportion to two
decimal places. 
    \item (iii) Use linear interpolation to approximate (to the nearest £) the value of the
median of the claim size distribution. [2]
\end{enumerate}

%%%%%%%%%%%%%%%%%%%%%%%%%%%%%%%%%%%%%%%%%%%%%%%%%%%%%%%%%%%%%%%%%%
\item 7 Suppose that X is a continuous random variable uniformly distributed on (0, 1).
\begin{enumerate}[(a)]
    \item  Show that the moment generating function of Y =  log X is
MY(t) = 1/(1  t) for $t < 1$. 
    \item Hence state the distribution of Y. 
\end{enumerate}
%%%%%%%%%%%%%%%%%%%%%%%%%%%%%%%%%%%%%%%%%%%%%%%%%%%%%%%%%%%%%%%%%%
\item 8 Assume that, for a group of insurance policies, the number of claims on each policy
can be modelled by a Poisson distribution with the same rate  per year,
independently for each policy. Such a group of 1500 policies gave rise to a total of
183 claims during the last year.
\begin{enumerate}[(a)]
    \item (i) State the value of the maximum likelihood estimate of . 
\item (ii) Hence obtain estimates of the following quantities:
(a) The probability that a subset of 10 of these policies results in no claims
over the next six months.
(b) The probability that a subset of 250 of these policies results in more
than 40 claims over the next year. 
\end{enumerate}
%%%%%%%%%%%%%%%%%%%%%%%%%%%%%%%%%%%%%%%%%%%%%%%%%%%%%%%%%%%%%%%%%%
\end{enumerate}
\newpage

5 yˆi = y $\leq$ˆ(xi  x) so$\leq$yˆi = ny so mean of fitted values is y
 Correlation coefft. of observed and fitted values is
$\leq$ $\leq$ 
$\leq$ $\leq$  $\leq$   $\leq$  2 2 1/ 2 2 1/ 2
ˆ ˆ
= =
ˆ ˆ
i i xy
i i yy xx
y y y y S
r
y y y y S S
  $\leq$
  $\leq$
$\leq$
$\leq$ $\leq$
%%%%%%%%%%%%%%%%%%%%%%%%%%%%%%%%%%%%%%%
Page 3
6 (i) Cumulative frequency distribution is
x F(x)
0 0
100 862
200 1470
300 2723
400 3789
500 4347
max 5637
(ii) For x = 250 the corresponding F(x) is approximately
1470 (2723 1470) (250 200) 2096.5
100
$\leq$
 $\leq$ 
and the corresponding approximate proportion is 2096.5 0.37
5637
$\leq$
(iii) For the median we need x such that F(x) = 0.5(5637) = 2818.5
median 300 2818.5 2723 (100) £309
1066
$\leq$
  
The maximum claim size is not given, and candidates were expected to exercise common
sense and judgement. One sensible approach is to leave the x-scale open-ended and indicate
the presence of a horizontal asymptote at cumulative frequency 5637.
0 250 500
0
1000
2000
3000
4000
5000
6000
x
F(x)
Claim size distribution function
for median
for P(X<250)
%%%%%%%%%%%%%%%%%%%%%%%%%%%%%%%%%%%%%%%
Page 4
7 (i) ( ) = ( tY ) = ( t log X )
MY t E e E e$\leq$
1 1 1
0
0
= ( )= =
1
t
E X t x tdx x
t
$\leq$ 
$\leq$ $\leq$
$\leq$ 
 
  
$\leq$
= 1 , 1.
1
t
t
$\leq$

(ii) This is the MGF of an exponential distribution with parameter 1, and by
uniqueness for the MGF this then implies that Y has this distribution.
%%%%%%%%%%%%%%%%%%%%%%%%%%%%%%%%%%%%%%%%%%%%%%%%%%%%%%%%%%%%%%%%%%
8 (i) ˆ = 183 = 0.122
1500

\begin{itemize}
    \item (ii) (a) Number of claims X for 10 policies in six months is Poisson with
parameter 10(0.5)(0.122) = 0.61
\item P(no claims) = P(X = 0) = exp(	0.61) = 0.543
\item Alternative:
For a single policy P(no claim) = exp(	0.061) = 0.9408
For 10 policies P(no claims) = (0.9408)10 = 0.543
\item (b) Number of claims X for 250 policies in one year is Poisson with
parameter 250(0.122) = 30.5
\item Outwith scope of the Green tables, so we must use a normal
approximation:
X $\leq$ N(30.5,30.5)
\item Applying a continuity correction:
\end{itemize}

( 40) ( 40.5) = ( 40.5 30.5 =1.81) =1 0.96485
30.5
= 0.035
P X P X P Z $\leq$
    $\leq$

\end{document}
