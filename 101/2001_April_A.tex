\documentclass[a4paper,12pt]{article}

%%%%%%%%%%%%%%%%%%%%%%%%%%%%%%%%%%%%%%%%%%%%%%%%%%%%%%%%%%%%%%%%%%%%%%%%%%%%%%%%%%%%%%%%%%%%%%%%%%%%%%%%%%%%%%%%%%%%%%%%%%%%%%%%%%%%%%%%%%%%%%%%%%%%%%%%%%%%%%%%%%%%%%%%%%%%%%%%%%%%%%%%%%%%%%%%%%%%%%%%%%%%%%%%%%%%%%%%%%%%%%%%%%%%%%%%%%%%%%%%%%%%%%%%%%%%

\usepackage{eurosym}
\usepackage{vmargin}
\usepackage{amsmath}
\usepackage{graphics}
\usepackage{epsfig}
\usepackage{enumerate}
\usepackage{multicol}
\usepackage{subfigure}
\usepackage{fancyhdr}
\usepackage{listings}
\usepackage{framed}
\usepackage{graphicx}
\usepackage{amsmath}
\usepackage{chngpage}

%\usepackage{bigints}
\usepackage{vmargin}

% left top textwidth textheight headheight

% headsep footheight footskip

\setmargins{2.0cm}{2.5cm}{16 cm}{22cm}{0.5cm}{0cm}{1cm}{1cm}

\renewcommand{\baselinestretch}{1.3}

\setcounter{MaxMatrixCols}{10}

\begin{document}

\begin{enumerate}
%%%%%%%%%%%%%%%%%%%%%%%%%%%%%%%%%%%%%%%%%%%%%%%%%%%%%%%%%%%%%%%%%%%%%%%%%%%%%%%%%%%%%%%%%%%%%%%%%%%%%%%%%%%%%%%%%%%
\item 1 The following amounts are the sizes of claims (£) on house insurance policies for a
certain type of repair.
\begin{verbatim}
198 221 215 209 224 210 223 215 203 210
220 200 208 212 216
\end{verbatim}

Determine the lower quartile, median, upper quartile and interquartile range of
these claim amounts. 
%%%%%%%%%%%%%%%%%%%%%%%%%%%%%%%%%%%%%%%%%%%%%%%%%%%%%%%%%%%%%%%%%%%%%%%%%%%%%%%%%%%%%%%%%%%%%%%%%%%%%%%%%%%%%%%%%%%
\item 2 A certain medical test either gives a positive or negative result. The positive test
result is intended to indicate that a person has a particular (rare) disease, while
a negative test result is intended to indicate that they do not have the disease.
\begin{itemize}
    \item Suppose, however, that the test sometimes gives an incorrect result: 1 in 100 of
those who do not have the disease have positive test results, and 2 in 100 of those
having the disease have negative test results.
\end{itemize}

If 1 person in 1000 has the disease, calculate the probability that a person with a
positive test result has the disease. 
%%%%%%%%%%%%%%%%%%%%%%%%%%%%%%%%%%%%%%%%%%%%%%%%%%%%%%%%%%%%%%%%%%%%%%%%%%%%%%%%%%%%%%%%%%%%%%%%%%%%%%%%%%%%%%%%%%%
\item 3 Suppose that the occurrence of events which give rise to claims in a portfolio of
motor policies can be modelled as follows: the events occur through time at
random, at rate $\mu$ per hour. Then the number of events which occur in a given
period of time has a Poisson distribution (you are given this).
Show that the time between two consecutive events occurring has an exponential
distribution with mean $1/\mu$ hours. 
%%%%%%%%%%%%%%%%%%%%%%%%%%%%%%%%%%%%%%%%%%%%%%%%%%%%%%%%%%%%%%%%%%%%%%%%%%%%%%%%%%%%%%%%%%%%%%%%%%%%%%%%%%%%%%%%%%%
\item 4 For a certain type of policy the probability that a policyholder will make a claim
in a year is 0.001. If a random sample of 10,000 policyholders is selected,
calculate an approximate value for the probability that not more than 5 will
make a claim next year. 
\end{enumerate}
\newpage

%%%%%%%%%%%%%%%%%%%%%%%%%%%%%%%%%%%%%%%%%%%%%%%%%%%%%%%%%%%%%%%%%%%%%%%%%%%%%%%%%%%%%%%%%%%%%%%%%%%%%%%%%%%%%%%%%%%%
1 Ordered data are:
\begin{verbatim}
 198 200 203 208 209 210 210 212 
 215 215 216 220 221 223 224   
\end{verbatim}

1
1
15 4
4
th
n = Q = observation = £208.25, median = 8th observation = £212
3
3 4 3 1 11 observation £219 £10.75 th Q = = Q −Q =
[OR: Using the alternative definitions of quartiles as 16/4th and 48/4th
observations gives Q1 = 208, Q3 = 220, Q3 − Q1 = 12.]
%%%%%%%%%%%%%%%%%%%%%%%%%%%%%%%%%%%%%%%%%%%%%%%%%%%5
\medskip 
2 Apply Bayes' theorem:
1
1000
p = (Probability of having disease)
P (has disease|positive result) =
98
100 98 0.089
98 1 1097 (1 )
100 100
p
p p
= =
+ −
%%%%%%%%%%%%%%%%%%%%%%%%%%%%%%%%%%%%%%%%%%%%%555
3 Let the number of events in a period of time of length t hours be Xt .
\begin{itemize}
\item Then $X_1 \sim Poisson(\mu)$ and $X_t \sim Poisson(\mu t)$.
\item Let the time between two consecutive events be T.
\item Then $P(T > t) = P(\mbox{no events in period of length t}) = P(Xt = 0) = exp(−\mu t)$.
\item So $P(T < t) = 1 − exp(−\mut)$, and so $f(t) = \mu exp(−\mut) , t >0$
\item Hence T ~ exponential with mean $1/\mu$ hours.
\end{itemize}

%%%%%%%%%%%%%%%%%%%%%%%%%%%%%%%%%%%%%%%%%%%%%%%%%%%%%
4 Binomial (10,000, 0.001) approximated by Poisson with mean of 10.
\begin{itemize}
    \item Approximate probability of no more than 5 claims:
\item Binomial (10,000, 0.001) approximated by Poisson with mean of 10 ($\lambda = np$).
\item Approximate probability of no more than 5 claims:

\end{itemize}


\[P(X\leq 5) = e^{-10} \left( 1 + 10 + \frac{10^2}{2!} + \frac{10^3}{3!} + \frac{10^4}{4!} + \frac{10^5}{5!} \right)\]


\[P(X\leq 5) =  0.0671.\]

[OR: 0.06709 using Green Book]
[OR: Use normal approximation with continuity correction]
%%%%%%%%%%%%%%%%%%%%%%%%%%%%%%%%%%%%%%%%%%%%%%%%%%%%%%%%%%%%%%%%%%%%%%%%%%%%%%%%%%%%%%%%%%%%%%
\end{document}
