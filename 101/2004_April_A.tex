\documentclass[a4paper,12pt]{article}

%%%%%%%%%%%%%%%%%%%%%%%%%%%%%%%%%%%%%%%%%%%%%%%%%%%%%%%%%%%%%%%%%%%%%%%%%%%%%%%%%%%%%%%%%%%%%%%%%%%%%%%%%%%%%%%%%%%%%%%%%%%%%%%%%%%%%%%%%%%%%%%%%%%%%%%%%%%%%%%%%%%%%%%%%%%%%%%%%%%%%%%%%%%%%%%%%%%%%%%%%%%%%%%%%%%%%%%%%%%%%%%%%%%%%%%%%%%%%%%%%%%%%%%%%%%%

\usepackage{eurosym}
\usepackage{vmargin}
\usepackage{amsmath}
\usepackage{graphics}
\usepackage{epsfig}
\usepackage{enumerate}
\usepackage{multicol}
\usepackage{subfigure}
\usepackage{fancyhdr}
\usepackage{listings}
\usepackage{framed}
\usepackage{graphicx}
\usepackage{amsmath}
\usepackage{chngpage}

%\usepackage{bigints}
\usepackage{vmargin}

% left top textwidth textheight headheight

% headsep footheight footskip

\setmargins{2.0cm}{2.5cm}{16 cm}{22cm}{0.5cm}{0cm}{1cm}{1cm}

\renewcommand{\baselinestretch}{1.3}

\setcounter{MaxMatrixCols}{10}

\begin{document}

\begin{enumerate}
\item
%%%%%%%%%%%%%%%%%%%%%%%%%%%%%%%%%%%%%%%%%%%%%%%%%%%%%%%%%%%%%%%%%%%%%%%%%%%%%%%%%%%%%%%%%%%%%%%%%%%%%%%%%%%
1 The probability that a claim is made on a certain type of policy in a particular year is
0.04. Five hundred policies are selected at random.
Use a suitable normal approximation to calculate the probability that no more than 30
of these will result in a claim during the year. 
\item 2 The interquartile range of a continuous distribution with distribution function $F(x)$ is
defined as IQR = x2 x1 where $F(x1) = 0.25$ and $F(x2) = 0.75$.
Show that for a normal distribution with variance 2, the interquartile range is
IQR = 1.349 . 
\item 3 The cumulant generating function of a random variable X is given by:
 \[CX (t) = logMX (t) = 2 1 t 1\]
where $M_X(t)$ is the moment generating function.
Determine the mean and variance of the distribution of $X$. 
\item 4 Claim sizes in a certain insurance situation are modelled by an exponential
distribution with mean \$20,000. The insurer defines a claim to be a large claim if the
claim size exceeds \$35,000.
State, with a reason, the expected size of a large claim. 
\end{enumerate}
\newpage
%%%%%%%%%%%%%%%%%%%%%%%%%%%%%%%%%%%%%%%%%%%%%%%%%%%%%%%%%%%%%%%%%%%%%%%%%%%%%%%%%%%%%%%%%%%%%%%%%%%%%%%%%%%

1 X = number of policies where a claim is made.
X bi (500, 0.04)
\[X N((500)(0.04), 500(0.04)(0.96))\] \[N(20, 19.2)\]
P(X 30) P(X 30.5) using continuity correction
30.5 20 10.5
(2.40)
19.2 19.2
= 0.9918
%%%%%%%%%%%%%%%%%%%%%%%%%%%%%%%%%%%%%%%%%%%%
2 
\begin{itemize}
    \item The required values of x1 and x2, the lower and upper quartiles, are such that:
$F(x1) = 0.25$ and $F(x2) = 0.75$,
i.e. 1 0.25
x
and 2 0.75,
x
where is the standard normal distribution function.
\item From the statistical table of the percentage points of the standard normal distribution,
we have
x1
= 0.6745 x1 = 0.6745
x2
= 0.6745 x2 = + 0.6745
where is the population mean and is the population standard deviation.
\item IQR = x2 - x1
= 0.6745 + 0.6745
= 1.349 .
\end{itemize}
%%%%%%%%%%%%%%%%%%%%%%%%%%%%%%%%%%%%%
\newpage
3 C´(t) = 20(1 t) 11 , C´´(t) = 220(1 t) 12
\[E[X] = C´(0) = 20\]
\[V[X] = C´´(0) = 220\]
[OR as coefficients of t and of t2/2! in expansion of C(t)]
%%%%%%%%%%%%%%%%%%%%%%%%%%%%%%%%%
4 Answer: $35,000 + 20,000 = $55,000
\begin{itemize}
    \item Reason: the memoryless property of the exponential distribution (the excess above
35,000 itself has an exponential distribution with mean 20,000).
\item Note: relatively few candidates were able to exploit the memoryless property of the
exponential distribution to advantage.
\end{itemize}


\end{document}
