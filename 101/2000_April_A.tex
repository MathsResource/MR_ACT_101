\documentclass[a4paper,12pt]{article}

%%%%%%%%%%%%%%%%%%%%%%%%%%%%%%%%%%%%%%%%%%%%%%%%%%%%%%%%%%%%%%%%%%%%%%%%%%%%%%%%%%%%%%%%%%%%%%%%%%%%%%%%%%%%%%%%%%%%%%%%%%%%%%%%%%%%%%%%%%%%%%%%%%%%%%%%%%%%%%%%%%%%%%%%%%%%%%%%%%%%%%%%%%%%%%%%%%%%%%%%%%%%%%%%%%%%%%%%%%%%%%%%%%%%%%%%%%%%%%%%%%%%%%%%%%%%

\usepackage{eurosym}
\usepackage{vmargin}
\usepackage{amsmath}
\usepackage{graphics}
\usepackage{epsfig}
\usepackage{enumerate}
\usepackage{multicol}
\usepackage{subfigure}
\usepackage{fancyhdr}
\usepackage{listings}
\usepackage{framed}
\usepackage{graphicx}
\usepackage{amsmath}
\usepackage{chngpage}

%\usepackage{bigints}
\usepackage{vmargin}

% left top textwidth textheight headheight

% headsep footheight footskip

\setmargins{2.0cm}{2.5cm}{16 cm}{22cm}{0.5cm}{0cm}{1cm}{1cm}

\renewcommand{\baselinestretch}{1.3}

\setcounter{MaxMatrixCols}{10}

\begin{document}
\begin{enumerate}
%%%%%%%%%%%%%%%%%%%%%%%%%%%%%%%%%%%%%%%%%%%%%%%%%%%%%%%
\item  Fourteen economists were asked to provide forecasts for the percentage rate
of inflation for the third quarter of 2002. They produced the forecasts given
below.
\begin{verbatim}
1.2 1.4 1.5 1.5 1.7 1.8 1.8
1.9 1.9 2.1 2.7 3.2 3.9 5.0    
\end{verbatim}

Calculate the median and the upper and lower quartiles of these forecasts. 

1 As n = 14 the median is half way between the 7th and 8th value
i.e. m = (1.8 + 1.9)/2=1.85.
The quartiles are the 4th and 11th values, so Q1 = 1.5 and Q3 = 2.7 .
OR: Using the definition of the quartiles as the 15/4th and 45/4th value gives
Q1 = 1.5 and Q3 = 2.8.
\newpage
%%%%%%%%%%%%%%%%%%%%%%%%%%%%%%%%%%%%%%%%%%%%%%%%%%%%%%%
\item Insurance policies providing car insurance are such that the sizes of claims are
normally distributed with mean £1,870 and standard deviation £610. In one
month 50 claims are made. Assuming that claims are independent, calculate
the probability that the total of the claim sizes is more than £100,000. 
%%%%%%%%%%%%%%%%%%%%%%%%%%%%%%%%%%%%%%%%%%%%%%%%%%%%%%%%%%%%%%%%%%%%%
2 The total claim, T, will be normally distributed with mean 50  1870 = 93500
and variance 50  6102 = 18,605,000 = 43132.
(Alternatively, we can work with the mean claim.)
Thus, the probability that the total claim is greater than £100,000 is
100,000 93,500
1
4313
  
 q 
 
= 1  (1.507) = 0.066.
\newpage
%%%%%%%%%%%%%%%%%%%%%%%%%%%%%%%%%%%%%%%%%%%%%%%%%%%%%%%
\item  In an investigation into the proportion ($\theta$) of lapses in the first year of a
certain type of policy, the uncertainty about q is modelled by taking $\theta$ to have a
beta distribution with parameters a = 1 and b = 9, that is, with density
\[f(\theta) = 9(1 - \theta)^8 : 0 < \theta < 1.\]
Using this distribution, calculate the probability that $\theta$ exceeds 0.2. 


3

\begin{eqnarray*}
P(\theta > 0.2) &=& \int^{1}_{0.2} 9(1- \theta)^8) d \theta \\
& & \\
&=& \left[ -(1-\theta)^9  \right]^{1}_{0.2}\\
& & \\
&=& 0 + (1 - 0.2)^9 \\ 
&=& 0.8^9 \\
&=& 0.13\\
\end{eqnarray*}


%%%%%%%%%%%%%%%%%%%%%%%%%%%%%%%%%%%%%%%%%%%%%%%%%%%%%%%
\newpage
\item Consider the following three probability statements concerning an F variable
with 6 and 12 degrees of freedom.
\begin{enumerate}
\item $P(F6,12 > 0.250) = 0.95$
\item $P(F6,12 < 4.821) = 0.99$
\item $P(F6,12 < 0.130) = 0.01$
\end{enumerate}

State, with reasons, whether each of these statements is true. 


4 
\begin{itemize}
\item For (a) to be true, 0.250 must be lower 5\% pt of F6,12 i.e. reciprocal of upper 5\% pt
of F12,6 which is
1
4.000
= 0.250  true.
    \item For (b) to be true, 4.821 must be upper 1\% pt of F6,12 which is 4.821 true.
\item For (c) to be true, 0.130 must be lower 1\% pt of F6,12 i.e. reciprocal of upper 1\% pt
of F12,6 which is
1
7.718
= 0.130 true.

\end{itemize}

\newpage
 
\end{enumerate}
%%%%%%%%%%%%%%%%%%%%%%%%%%%%%%%%%%%%%%%%%%%%%%%%%%%%%%%

%%  101 April 2000
%%  Subject 101 — Statistical Modelling

%%%%%%%%%%%%%%%%%%%%%%%%%%%%%%%%%%%%%%%%%%%%%%%%%%%%%%%%%%%%%%%%%%%%%



%%%%%%%%%%%%%%%%%%%%%%%%%%%%%%%%%%%%%%%%%%%%%%%%%%%%%%%%%%%%%%%%%%%%%
\end{document}
