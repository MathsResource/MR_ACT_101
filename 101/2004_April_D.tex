\documentclass[a4paper,12pt]{article}

%%%%%%%%%%%%%%%%%%%%%%%%%%%%%%%%%%%%%%%%%%%%%%%%%%%%%%%%%%%%%%%%%%%%%%%%%%%%%%%%%%%%%%%%%%%%%%%%%%%%%%%%%%%%%%%%%%%%%%%%%%%%%%%%%%%%%%%%%%%%%%%%%%%%%%%%%%%%%%%%%%%%%%%%%%%%%%%%%%%%%%%%%%%%%%%%%%%%%%%%%%%%%%%%%%%%%%%%%%%%%%%%%%%%%%%%%%%%%%%%%%%%%%%%%%%%

\usepackage{eurosym}
\usepackage{vmargin}
\usepackage{amsmath}
\usepackage{graphics}
\usepackage{epsfig}
\usepackage{enumerate}
\usepackage{multicol}
\usepackage{subfigure}
\usepackage{fancyhdr}
\usepackage{listings}
\usepackage{framed}
\usepackage{graphicx}
\usepackage{amsmath}
\usepackage{chngpage}

%\usepackage{bigints}
\usepackage{vmargin}

% left top textwidth textheight headheight

% headsep footheight footskip

\setmargins{2.0cm}{2.5cm}{16 cm}{22cm}{0.5cm}{0cm}{1cm}{1cm}

\renewcommand{\baselinestretch}{1.3}

\setcounter{MaxMatrixCols}{10}

\begin{document}

\begin{enumerate}
\item
%%%%%%%%%%%%%%%%%%%%%%%%%%%%%%%%%%%%%%%%%%%%%%%%%%%%%%%%%%%%%%%%%%%%%%%%%%%%%%%%%%%%%%%%%%%%%%%%%%%%%%%%%%%
13 A researcher studying the claims experience of a company (in a particular class of business) records the payments on 100 recent claims. The payments (in units of £1000, and sorted) are given below.
Payments
\begin{verbatim}
0.30 0.89 0.96 1.16 1.67 1.77 1.93 1.98 2.07 2.09 2.30 2.48
2.58 2.78 3.00 3.19 3.21 3.21 3.25 3.31 3.34 3.37 3.66 3.95
4.16 4.18 4.60 4.72 4.73 4.76 5.01 5.17 5.21 5.63 5.72 6.00
6.13 6.17 6.24 6.37 6.47 6.48 6.87 7.05 7.16 7.21 7.51 7.72
7.74 8.00 8.00 8.03 8.04 8.54 9.11 9.18 9.49 9.59 10.00 10.36
10.85 11.08 11.22 11.27 11.38 11.45 11.69 11.78 12.27 12.30 12.50 13.04
13.28 13.43 13.48 13.85 14.27 14.31 14.49 14.55 14.62 14.68 14.70 14.83
15.67 15.70 15.77 16.28 16.44 17.17 17.89 18.03 18.12 20.72 22.00 24.33
25.41 28.30 31.00 32.80
\end{verbatim}

For these 100 observations: $\sum x = 952.75$ , $\sum x^2 = 13584.5217$.
The researcher wants to examine whether or not the exponential distribution provides
a good description of the distribution of the payments.
\begin{enumerate}[(a)]
\item Calculate the sample mean and standard deviation for the 100 payments, and comment briefly on whether or not you think the exponential distribution will provide an acceptable fit to the data.
\item Specify the fitted exponential distribution, giving a brief justification of your approach.
Note: you are not required to give any mathematical derivation in your justification.
\item The researcher decides to conduct a formal chi-squared goodness-of-fit test of the exponential distribution to the data, using five equi-probable intervals (i.e. intervals each with associated probability 0.2).
(a) Show that the value x which is exceeded with probability p by an exponential variable with mean satisfies $x = \log p$.
(b) Calculate the values which divide the positive real numbers into five equi-probable intervals for your fitted exponential distribution.
\item  Conduct the formal goodness-of-fit test outlined in part (ii) above and
comment on the result. 
\end{enumerate}
%%%%%%%%%%%%%%%%%%%%%%%%%%%%%%%%%%%%%%%%%%%%%%%%%%%%%%%%%%%%%%%%%%%%%%%%%%%%%%%%%%%%%%%%%%%%%%%%%%%%%%%%%%%
\item Forensic scientists use various methods for determining the likely time of death from
post-mortem examination of human bodies.

\begin{itemize}
\item 
A recently suggested objective method uses the concentration of a compound (3-methoxytyramine or 3-MT) in a particular part of the brain.
\item In a study of the relationship between post-mortem interval and the concentration of 3-MT, samples of the appropriate part of the brain were taken from coroners cases for which the time of death had been determined from eye-witness accounts. 
\item The
intervals (x; in hours) and concentrations (y; in parts per million) for 18 individuals who were found to have died from organic heart disease are given in the following table. 
\item For the last two individuals (numbered 17 and 18 in the table), there was no eye-witness testimony directly available, and the time of death was established on the
basis of other evidence including knowledge of the individuals activities.
\end{itemize}

\begin{verbatim}
  Observation
number
Interval
(x)
Concentration
(y)
1 5.5 3.26
2 6.0 2.67
3 6.5 2.82
4 7.0 2.80
5 8.0 3.29
6 12.0 2.28
7 12.0 2.34
8 14.0 2.18
9 15.0 1.97
10 15.5 2.56
11 17.5 2.09
12 17.5 2.69
13 20.0 2.56
14 21.0 3.17
15 25.5 2.18
16 26.0 1.94
17 48.0 1.57
18 60.0 0.61
x = 337 x2 = 9854.5 y = 42.98 y2 = 109.7936 xy = 672.8  
\end{verbatim}

In this investigation you are required to explore the relationship between
concentration (regarded as the response/dependent variable) and interval (regarded as the explanatory/independent variable).
\begin{enumerate}[(a)]
    \item  Construct a scatterplot of the data. Comment on any interesting features of the data and discuss briefly whether linear regression is appropriate to model the relationship between concentration of 3-MT and the interval from death.

    \item Calculate the correlation coefficient for the data, and use it to test the null
hypothesis that the population correlation coefficient is equal to zero.

    \item  Calculate the equation of the least-squares fitted regression line, and use it to
estimate the concentrations of 3-MT:
(a) after 1 day and
(b) after 2 days
Comment briefly on the reliability of these estimates. [5]
    \item Calculate a 99\% confidence interval for the slope of the regression line. Using this confidence interval, test the hypothesis that the slope of the regression line is equal to zero. Comment on your answer in relation to the answer given in
part (ii) above.
\end{enumerate}

\end{enumerate}
\newpage
%%%%%%%%%%%%%%%%%%%%%%%%%%%%%%%%%%%%%%%%%%%%%%%%%%%%%%%%%%%%%%%%%%%%%%%%%%%%%%%%%%%%%%%%%%%%%%%%%%%%%%%%%%%

13 (i) (a) x 952.75 /100 9.53
0.5
1 952.752
13584.5217 6.75
99 100
s
An exponential distribution has mean = standard deviation so there must be some doubt here as to whether such a distribution will be a good description of these data.
(b) The fitted model is an exponential distribution with mean 9.53 
(i.e. with parameter 1/9.5275 = 0.105).
Reason: the maximum likelihood estimate (and the method of moments estimate) of the mean of an exponential distribution is the sample mean
%%%%%%%%%%%%%%%%%%%%%%%%%%%%%%%%%%%%%
Page 9
(ii) (a) For X ~ exp(mean = ) we have P(X > x) = exp( x/ )
exp( x/ ) = p x = logp
(b) Using p = 0.8, 0.6, 0.4, and 0.2 and fitted mean 9.5275
we get x = 2.13, 4.87, 8.73, 15.33
(iii) The numbers of observations in (0,2.13), (2.13,4.87), (4.87,8.73), (8.73,15.33)
and (15.33, ) are, by inspection from the data, respectively 10, 20, 24, 30,
and 16.
\begin{verbatim}
Interval 0 2.13 2.13 4.87 4.87 8.73 8.73 15.33 15.33
Observed frequency 10 20 24 30 16
Expected frequency 20 20 20 20 20
\end{verbatim}
\[\chi^2 = \frac{(102 + 0 + 42 + 102 + 42)}{20} = 232/20 = 11.6\]
df = 3
P-value = P( 2 > 11.6) = 0.009 (from Yellow Tables p164), so we reject the exponential distribution as a description it provides a very poor fit to the data.
%%%%%%%%%%%%%%%%%%%%%%%%%%%%%%%%%%%%%%%%%%%%%%%%%%%%%%%%%%%%%%%%%%%%%%%%%%%%%%%%%%%%%%%%%%%%%%%%%%%%%%%%%%%
14 (i)
\begin{itemize}
    \item The concentration of 3-MT in the brain decreases as the post-mortem interval increases from 5.5 hours to 60 hours.
    \item There are two points with a much higher post-mortem interval than the other observations.
\item The data seem to be appropriate for linear regression, but care should be taken when evaluating the effect for the higher interval values as there are only 2 points in the higher x-range.
\end{itemize}

(ii) n = 18
0 10 20 30 40 50 60
3
2
1
Interval (x)
Concentration (y)
Plot of Concentration against Interval
%%%%%%%%%%%%%%%%%%%%%%%%%%%%%%%%%%%%%
Page 10
\begin{itemize}
    \item Sxx = x2 ( x)2/n
= 9854.5 (337)2/18
= 3545.1111
    \item Syy = y2 ( y)2/n
= 109.7936 (42.98)2/18
= 7.1669111
    \item Sxy = xy ( x)( y)/n
= 672.8 (337)(42.98)/18
= 131.88111
\end{itemize}

Correlation coefficient:
131.88111
(3545.1111)(7.1669111)
xy
xx yy
S
r
S S
= 0.827
Test H0: = 0 vs H1: 0
2
2
1
n
t r
r
~ tn 2
= 2
16
( 0.827)
1 ( 0.827)
= 5.89
\begin{itemize}
\item $t_{16}(2.5\%) = 2.120$
\item $t_{16}(0.5\%) = 2.921$
\end{itemize}

t = 5.89 is larger than the critical values at the 5\% and 1\% significance levels. Therefore reject the null hypothesis that the population correlation coefficient is equal to zero, and conclude that there is a linear relationship
between interval and concentration.
(iii) Model: y = + x
Slope:
131.88111
3545.1111
xy
xx
S
S
= 0.0372008 or 0.0372 (to 4 d.p.)
%%%%%%%%%%%%%%%%%%%%%%%%%%%%%%%%%%%%%%%%%%%%%%%%%%%%%%%%%%%%%%%%%%%%%%%%%%%%%%%%%%%%%%%%%%%%%%%%%
Intercept:
y x
= 42.98/18 ( 0.0372008)(337/18)
= 3.084259 or 3.0843 (to 4 d.p.)
y = x
= 3.0843 0.0372 x
i.e. \[\mbox{Concentration} = 3.0843 0.0372 \mbox{Interval}\]


(a) At 1 day = 24 hours:
\begin{eqnarray*}
y &=& 3.0843 -  0.0372 x\\
&=& 3.0843 - 0.0372 (24)\\
&=& 2.19\\
\end{eqnarray*}
(b) At 2 days = 48 hours:


\begin{eqnarray*}
y &=& 3.0843 -  0.0372 x\\
&=& 3.0843 - 0.0372 (48)\\
&=& 1.30\\
\end{eqnarray*}
This data set contains accurate data up to 26 hours, as for observations 17 and 18 (at 48 hours and 60 hours respectively) there was no eye-witness testimony directly available. Predicting 3-MT concentration after 26 hours may not be
advisable, even though x = 48 is within the range of the x-values (5.5 hours to 60 hours).
(iv)
2
2 1
2
xy
yy
xx
S
S
n S
1 ( 131.88111)2
7.1669111
16 3545.1111
2.2608231
16
= 0.1413014
0.3759
s.e.( ) =
2 0.3759
Sxx 3545.111
= 0.00631331
%%%%%%%%%%%%%%%%%%%%%%%%%%%%%%%%%%%%%
\begin{itemize}
\item 99\% confidence interval for slope is tn 2 s.e.( ) df = n 2 = 16
= 0.0372008 (2.921) (0.00631331)
= 0.0372 0.0184
or ( 0.0556, 0.0188)
\item  0 is not within this 99\% confidence interval, therefore we would reject the
null hypothesis = 0, at the 1\% significance level.
\item Therefore there does appear to be a (negative) linear relationship between
Interval and Concentration.
\item This confirms the result in (ii) where the correlation coefficient was shown to
not equal zero at the 1\% significance level.
\item Note: Plots of data need not be works of graphic art, but should at least be neat and clear
enough (with adequate title and labels) to convey the information.
\end{itemize}
%%%%%%%%%%%%%%%%%%%%%%%%%%%%%%%%%%%%%%%%%%%%%%%%%%%%%%%%%%%%%%%%%%%%%%%%%%%%%%%%%%%%%%%%%%%%%%%%%%%%%%%%%%%
\end{document}
