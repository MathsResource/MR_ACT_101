\documentclass[a4paper,12pt]{article}

%%%%%%%%%%%%%%%%%%%%%%%%%%%%%%%%%%%%%%%%%%%%%%%%%%%%%%%%%%%%%%%%%%%%%%%%%%%%%%%%%%%%%%%%%%%%%%%%%%%%%%%%%%%%%%%%%%%%%%%%%%%%%%%%%%%%%%%%%%%%%%%%%%%%%%%%%%%%%%%%%%%%%%%%%%%%%%%%%%%%%%%%%%%%%%%%%%%%%%%%%%%%%%%%%%%%%%%%%%%%%%%%%%%%%%%%%%%%%%%%%%%%%%%%%%%%

\usepackage{eurosym}
\usepackage{vmargin}
\usepackage{amsmath}
\usepackage{graphics}
\usepackage{epsfig}
\usepackage{enumerate}
\usepackage{multicol}
\usepackage{subfigure}
\usepackage{fancyhdr}
\usepackage{listings}
\usepackage{framed}
\usepackage{graphicx}
\usepackage{amsmath}
\usepackage{chngpage}

%\usepackage{bigints}
\usepackage{vmargin}

% left top textwidth textheight headheight

% headsep footheight footskip

\setmargins{2.0cm}{2.5cm}{16 cm}{22cm}{0.5cm}{0cm}{1cm}{1cm}

\renewcommand{\baselinestretch}{1.3}

\setcounter{MaxMatrixCols}{10}

\begin{document}

\begin{enumerate}
\item
%%%%%%%%%%%%%%%%%%%%%%%%%%%%%%%%%%%%%%%%%%%%%%%%%%%%%%%%%%%%%%%%%%%%%%%%%%%%%%%%%%%%%%%%%%%%%%%%%%%%%%%%%%%
9 In order to simulate an observation of a normal random variable it is suggested that
\[S = \sum^{n}{i=1} X_i\]
is used, where $X_1, \ldots , X_n$ is a random sample from a continuous uniform distribution
on the interval $\displaystyle \left( -\frac{1}{2}, \frac{1}{2} \right)$ .
\begin{enumerate}[(a)]
    \item (i) Determine the approximate distribution of S. 
    \item (ii) Determine the value of n which should be used if S is required to represent a
standard normal random variable. 
    \item (iii) Explain why S has the same coefficient of skewness as a standard normal
random variable. 
\end{enumerate}

%%%%%%%%%%%%%%%%%%%%%%%%%%%%%%%%%%%%%%%%%%%%%%%%%%%%%%%%%%%%%%%%%%%%%%%%%%%%%%%%%%%%%%%%%%%%%%%%%%%%%%%%%%%
\item 10 The number of claims which arise in a year under a policy of a certain type follows a
Poisson distribution with mean . It is required to test
H0 : = 0.2 v H1 :
and it is decided to reject H0 in favour of H1 if 15 or more claims arise in a year under
a group of 50 independent such policies.
By using tables of Poisson distribution probabilities, calculate the power of this test in
each of the cases:
(a) = 0.3, and
(b) = 0.4
and comment briefly on the results. [5]
\item 11 The following table gives the sums insured (in units of £1,000) for a random sample
of insurance policies on the contents of private houses for each of four insurance
companies.
Company
\begin{verbatim}
1 2 3 4
(y1) (y2) (y3) (y4)
39 24 21 33
29 28 30 26
33 33 30 28
36 22 51 30
27 29 23 27
36 20 23 37
27 23 35 37
y1 = 227 y2 = 179 y3 = 213 y4 = 218
y1
2 = 7,501 y2
2 = 4,703 y3
2 = 7,125 y4
2 = 6,916
\end{verbatim}

\begin{enumerate}[(a)]
\item (i) Test whether there are any differences between the population means of the
four companies, using one-way analysis of variance.
\item 
(ii) A plot of the residuals for the fitted one-way analysis of variance model is
given below:
Comment on the adequacy of the model. 
\end{enumerate}

1 2 3 4
-10
0
10
20
Company
Residuals
%%%%%%%%%%%%%%%%%%%%%%%%%%%%%%%%%%%%%%%%%%%%%%%%%%%%%%%%%%%%%%%%%%%%%%%%%%%%%%%%%%%%%%%%%%%%%%%%%%%%%%%%%%%
\item 12 For the estimation of a binomial probability p = P(success), a series of n independent
trials are performed and X represents the number of successes observed.
\begin{enumerate}[(a)]
    \item (i) Write down the likelihood function $L(p)$ and show that the maximum
likelihood estimator (MLE) of p is =
X
p
n
.

%%%%%%%%%%%%%%%%%%%%%%%%%%5
    \item (ii) (a) Determine the Cramer-Rao lower bound for the estimation of p.
(b) Show that the variance of the MLE is equal to the Cramer-Rao lower
bound.
    \item (c) Write down an approximate sampling distribution for p valid for
large n.
%%%%%%%%%%%%%%%%%%%%%%%%%%5
    \item (iii) In order to develop an approximate 95\% confidence interval for $p$ for large $n$,
the following pivotal quantity is to be used
(0,1)
(1 )
p p
N
p p
n
.
Assuming that this pivotal quantity is monotonic in p, show that rearrangement
of the inequality
1.96 1.96
(1 )
p p
p p
n
leads to a quadratic inequality in p, and hence determine an approximate 95%
confidence interval for p of the form pL ( p) p pU ( p) .

%%%%%%%%%%%%%%%%%%%%%%%%%%5
    \item (iv) A simpler and more widely used approximate confidence interval is obtained
by using the following pivotal quantity
(0,1)
(1 )
p p
N
p p
n
.
Determine the resulting approximate 95\% confidence interval using this. [2]
\end{enumerate}


(v) In two separate applications the following data were observed:
(a) 4 successes out of 10 trials
(b) 80 successes out of 200 trials
In each case calculate the two approximate confidence intervals from parts
(iii) and (iv) and comment briefly on your answers.
\end{enumerate}
%%%%%%%%%%%%%%%%%%%%%%%%%%%%%%%%%
\newpage
%%%%%%%%%%%%%%%%%%%%%%%%%%%%%%%%%%%%%%%%%%%%%%%%%%%%%%%%%%%%%%%%%%%%%%%%%%%%%%%%%%%%%%%%%%%%%%%%%%%%%%%%%%%
9 
\begin{itemize}
    \item (i) S is approximately normal for large n by Central Limit Theorem.
Using results from the Yellow Book for Xi
1 1
U( 2 , 2 ) :
E[S] = nE[Xi] = n 0 = 0 , [ ] [ ] [ ]
12 i i
n
Var S Var X nVar X
So S ~ N(0, n/12)
\item (ii) S approx N(0, 1) if n = 12.
\item (iii) The distribution of each $X_i$ is symmetric and so the distribution of the sum
S = Xi is also symmetric. So skewness is zero, as for any normal distribution.
\end{itemize}
%%%%%%%%%%%%%%%%%%%%%%%%%%%%%%%%%%%%%%%%%%%%%%%%%%%%%%%%%%%%%%%%%%%%%%%%%%%%%%%%%%%%%%%%%%%%%%%%%%%%%%%%%%%

%%- April 2004 Question 10

Total number of claims $N \sim \mbox{Poisson}(50 \lambda)$

In the case $\lambda = 0.3$

\begin{eqnarray*}
\mbox{Power} 
&=& P[N \geq 15| N \sim \mbox{Poisson}(15)]\\
&=& 1 -  P[N \leq 14|  N \sim \mbox{Poisson}(15)]\\
&=& 1 - 0.466\\
&=& 0.534 \\
\end{eqnarray*}

In the case $\lambda = 0.4$

\begin{eqnarray*}
\mbox{Power} 
&=& P[N \geq 15| N \sim \mbox{Poisson}(20)]\\
&=& 1 -  P[N \leq 14|  N \sim \mbox{Poisson}(20)]\\
&=& 1 - 0.105  \\
&=& 0.895 \\
\end{eqnarray*}

Comment: power higher for
$\lambda = 0.4$, which is further away from $H_0$ value 0.2
%%%%%%%%%%%%%%%%%%%%%%%%%%%%%%%%%%%%%%%%%%%%%%%%%%%%%%%%%%%%%%%%%%%%%%%%%%%%%%%%%%%%%%%%%%%%%%%%%%%%%%%%%%%

11 (i) n1 = n2 = n3 = n4 = 7 n = 28
y1 = 227 y2 = 179 y3 = 213 y4 = 218 yij = 837

\begin{itemize}
    \item $\sum y_1^2 = 7501$ 
\item $\sum y^2_2 = 4703 $
\item $\sum y_3^2 = 7125 $
\item $\sum y_4^2 = 6916 $
\item $\sum yij^2 = 26245 $
\end{itemize}
SST = 26245
8372
28
= 26245 25020.321 = 1224.7
SSB =
1
7
(2272 + 1792 + 2132 + 2182) 25020.321
= 25209 25020.321 = 188.7
%%%%%%%%%%%%%%%%%%%%%%%%%%%%%%%%%%%%%
Page 6
SSR = SST SSB = 1224.7 188.7 = 1036.0
Source of variation d.f. SS MSS
Companies 3 188.7 62.9
Residual 24 1036.0 43.2
Total 27 1224.7
F =
62.9
43.2
= 1.46 on (3,24) degrees of freedom.
The 10% point of F3,24 distribution is 2.327. Therefore, there is insufficient
evidence to reject the null hypothesis that the population means for the four
companies are equal, i.e., the distributions of the sums insured are the same for
the four companies.
(ii) 
\begin{itemize}
    \item The model used in (i) assumes that the sums insured for each company follow
a normal distribution, and the population variances are equal.
\item The plot of residuals shows:
normality seems appropriate, but observation £51,000 seems to be an
outlier
companies have similar sample variances. \item However one could argue that there is
a suggestion that the variance for Company 3 is higher than the variances
for the other companies.
\item Therefore, overall the one-way analysis of variance model seems adequate and
the conclusions in (i) are valid.
\end{itemize}
%%%%%%%%%%%%%%%%%%%%%%%%%%%%%%%%%%%%%%%%%%%%%%%%%%%%%%%%%%%%%%%%%%%%%%%%%%%%%%%%%%%%%%%%%%%%%%%%%%%%%%%%%%%

12 (i) %%%%%%%%%%%%%%%%%%%%%%%%%%%%%%%%%%%%%%%%%%%%%%%%%%%%%%%%%%%%%%%%5

%%- April 2004 Question 12

\[L ( p ) = {n \choose x} p^x (1-p)^{n-x}\]

\[ \log L ( p )  = \log {n \choose x} + x \log p + (n-x) \log(1-p)\]

\[ \frac{\partial}{\partial p} \left( \log L ( p ) \right)  =  0 +  \frac{x}{p} +  \frac{n-x}{1-p} \]


equate to zero. \[ \frac{x}{p} +  \frac{n-x}{1-p} = 0\]

i.e. 
\[\frac{x}{p} =  \frac{n-x}{1-p}

\[ x(1-p) = (n-x)p\]

\[p = \frac{x}{n}\]
clearly maximises $L(p)$
%%%%%%%%%%%%%%%%%%%%%%%%%%%
\begin{framed}

\[ \log( A \times B \times C) = \log(A) + \log(B) + \log(C) \]

\[\frac{d}{dx} \left( \log x \right) = \frac{1}{x} \]

\end{framed}
%%%%%%%%%%%%%%%%%%%%%%%%%%

%%%%%%%%%%%%%%%%%%%%%%%%%%%%%%%%%%%%%
Page 7
MLE is
X
p
n
(ii) 

\[ \frac{\partial^2}{\partial p^2} \left( \log L ( p ) \right)  =  -\frac{x}{p^2} -  \frac{n-x}{(1-p)^2} \]

\begin{eqnarray} 
E\left[\frac{\partial^2}{\partial p^2} \left( \log L ( p ) \right) \right] 
&=& -\frac{np}{p^2} -  \frac{n-np}{(1-p)^2} \\
&=& -\frac{n}{p} -  \frac{n}{1-p} \\
&=& - \frac{n}{p(1-p)}\\
\end{eqnarray*}


CRlb
n n
p p
(b) Var( p ) = 2
np(1 p) p(1 p)
CRlb
n n
(c)
(1 )
( , ) ( , )
p p
p N p CRlb N p
n
(iii) 1.96 1.96
(1 )
p p
p p
n
2
( ) 2
1.96
(1 )
p p
p p
n
2
2 2 1.96 2
p 2 pp p ( p p )
n
2 2
1.96 2 1.96 2
(1 ) p (2 p ) p p 0
n n
\begin{itemize}
    \item This is a quadratic and will be negative between its two roots.
So, pL , pU will be the two roots:
2 2 2
2 2
2
1.96 1.96 1.96
(2 ) (2 ) 4 (1 )
1.96
2(1 )
p p p
n n n
n
with pL from the " " sign, and pU from the "+" sign.
%%%%%%%%%%%%%%%%%%%%%%%%%%%%%%%%%%%%%
\item (iv) 1.96 1.96
(1 )
p p
p p
n
(1 ) (1 )
1.96 1.96
p p p p
p p p
n n
giving pL and pU .
\item (v) (a) x = 4, n = 10
quadratic from (iii) is 1.38416 p2 1.18416 p 0.16
roots give pL = 0.168 and pU = 0.687.
from (iv) pL = 0.096 and pU = 0.704.
quite a difference, especially in pL , but n = 10 is small.
\end{itemize}
(b) x = 80, n = 200
quadratic from (iii) is 1.019208 p2 0.819208 p 0.16
roots give pL = 0.335 and pU = 0.469.
from (iv) pL = 0.332 and pU = 0.468.
very similar ( and much narrower than (a)) with n = 200 being large.
In (i) many candidates wanted to write the likelihood as a product this is OK using
Bernoulli probabilities as the individual components, as in
1
1
( ) 1 i i
n
x x
i
L p p p
where xi = 1 or 0, but not as a product of binomial(n,p) components.

\end{document}
