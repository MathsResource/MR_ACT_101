\documentclass[a4paper,12pt]{article}

%%%%%%%%%%%%%%%%%%%%%%%%%%%%%%%%%%%%%%%%%%%%%%%%%%%%%%%%%%%%%%%%%%%%%%%%%%%%%%%%%%%%%%%%%%%%%%%%%%%%%%%%%%%%%%%%%%%%%%%%%%%%%%%%%%%%%%%%%%%%%%%%%%%%%%%%%%%%%%%%%%%%%%%%%%%%%%%%%%%%%%%%%%%%%%%%%%%%%%%%%%%%%%%%%%%%%%%%%%%%%%%%%%%%%%%%%%%%%%%%%%%%%%%%%%%%

\usepackage{eurosym}
\usepackage{vmargin}
\usepackage{amsmath}
\usepackage{graphics}
\usepackage{epsfig}
\usepackage{enumerate}
\usepackage{multicol}
\usepackage{subfigure}
\usepackage{fancyhdr}
\usepackage{listings}
\usepackage{framed}
\usepackage{graphicx}
\usepackage{amsmath}
\usepackage{chngpage}

%\usepackage{bigints}
\usepackage{vmargin}

% left top textwidth textheight headheight

% headsep footheight footskip

\setmargins{2.0cm}{2.5cm}{16 cm}{22cm}{0.5cm}{0cm}{1cm}{1cm}

\renewcommand{\baselinestretch}{1.3}

\setcounter{MaxMatrixCols}{10}

\begin{document}


%%%%%%%%%%%%%%%%%%%%%%%%%%%%%%%%%%%%%%%%%%%%%%%%%%%%%%%%%%%%%%%%%%%%%%%%%%%%%%%%%%%%%%%%%%%%%%%%%%%%%%%%%%%
9 

In order to simulate an observation of a normal random variable it is suggested that
\[S = \sum^{n}{i=1} X_i\]
is used, where $X_1, \ldots , X_n$ is a random sample from a continuous uniform distribution
on the interval $\displaystyle \left( -\frac{1}{2}, \frac{1}{2} \right)$ .
\begin{enumerate}[(a)]
    \item  Determine the approximate distribution of $S$. 
    \item Determine the value of n which should be used if S is required to represent a
standard normal random variable. 
    \item Explain why $S$ has the same coefficient of skewness as a standard normal
random variable. 
\end{enumerate}


\newpage
%%%%%%%%%%%%%%%%%%%%%%%%%%%%%%%%%%%%%%%%%%%%%%%%%%%%%%%%%%%%%%%%%%%%%%%%%%%%%%%%%%%%%%%%%%%%%%%%%%%%%%%%%%%
9 
\begin{itemize}
    \item (i) S is approximately normal for large n by Central Limit Theorem.
Using results from the Yellow Book for Xi
1 1
U( 2 , 2 ) :
E[S] = nE[Xi] = n 0 = 0 , [ ] [ ] [ ]
12 i i
n
Var S Var X nVar X
So S ~ N(0, n/12)
\item (ii) $S \approx N(0, 1)$ if n = 12.
\item (iii) The distribution of each $X_i$ is symmetric and so the distribution of the sum
S = Xi is also symmetric. So skewness is zero, as for any normal distribution.
\end{itemize}
%%%%%%%%%%%%%%%%%%%%%%%%%%%%%%%%%%%%%%%%%%%%%%%%%%%%%%%%%%%%%%%%%%%%%%%%%%%%%%%

\end{document}
