\documentclass[a4paper,12pt]{article}

%%%%%%%%%%%%%%%%%%%%%%%%%%%%%%%%%%%%%%%%%%%%%%%%%%%%%%%%%%%%%%%%%%%%%%%%%%%%%%%%%%%%%%%%%%%%%%%%%%%%%%%%%%%%%%%%%%%%%%%%%%%%%%%%%%%%%%%%%%%%%%%%%%%%%%%%%%%%%%%%%%%%%%%%%%%%%%%%%%%%%%%%%%%%%%%%%%%%%%%%%%%%%%%%%%%%%%%%%%%%%%%%%%%%%%%%%%%%%%%%%%%%%%%%%%%%

\usepackage{eurosym}
\usepackage{vmargin}
\usepackage{amsmath}
\usepackage{graphics}
\usepackage{epsfig}
\usepackage{enumerate}
\usepackage{multicol}
\usepackage{subfigure}
\usepackage{fancyhdr}
\usepackage{listings}
\usepackage{framed}
\usepackage{graphicx}
\usepackage{amsmath}
\usepackage{chngpage}

%\usepackage{bigints}
\usepackage{vmargin}

% left top textwidth textheight headheight

% headsep footheight footskip

\setmargins{2.0cm}{2.5cm}{16 cm}{22cm}{0.5cm}{0cm}{1cm}{1cm}

\renewcommand{\baselinestretch}{1.3}

\setcounter{MaxMatrixCols}{10}

\begin{document}

\begin{enumerate}
\item

9 A statistician suggests that, since a t variable with k degrees of freedom is
symmetrical with mean 0 and variance
2
k
k
for $k > 2$, one can approximate the
distribution using the normal variable 0,
2
k
N
k
.
\begin{enumerate}[(a)]
\item (i) Use this to obtain an approximation for the upper 5\% percentage points for a
t variable with:
(a) 4 degrees of freedom, and
(b) 40 degrees of freedom
\item 
(ii) Compare your answers with the exact values from tables and comment briefly
on the result. 
\end{enumerate}
%%%%%%%%%%%%%%%%%%%%%%%%%%%%%%%%%%%%%%%%%%%%%%%%%%%%%%%%%%%%%%%%%%%
\newpage


%%- Question 9 

\begin{itemize}
    \item (i) (a) k = 4 using N(0, 4/2) = N(0, 2)
    \item 5% point = 0 + 1.6449 2 = 2.326
    \item (b) k = 40 using N(0, 40/38) = N(0, 1.0526)
    \item 5% point = 0 + 1.6449 1.0526 = 1.688
    \item (ii) Exact values are: (a) 2.132 and (b) 1.684
for small df approximation is poor, but for large df it is quite good.
\end{itemize}

\end{document}
