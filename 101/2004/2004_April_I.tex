\documentclass[a4paper,12pt]{article}

%%%%%%%%%%%%%%%%%%%%%%%%%%%%%%%%%%%%%%%%%%%%%%%%%%%%%%%%%%%%%%%%%%%%%%%%%%%%%%%%%%%%%%%%%%%%%%%%%%%%%%%%%%%%%%%%%%%%%%%%%%%%%%%%%%%%%%%%%%%%%%%%%%%%%%%%%%%%%%%%%%%%%%%%%%%%%%%%%%%%%%%%%%%%%%%%%%%%%%%%%%%%%%%%%%%%%%%%%%%%%%%%%%%%%%%%%%%%%%%%%%%%%%%%%%%%

\usepackage{eurosym}
\usepackage{vmargin}
\usepackage{amsmath}
\usepackage{graphics}
\usepackage{epsfig}
\usepackage{enumerate}
\usepackage{multicol}
\usepackage{subfigure}
\usepackage{fancyhdr}
\usepackage{listings}
\usepackage{framed}
\usepackage{graphicx}
\usepackage{amsmath}
\usepackage{chngpage}

%\usepackage{bigints}
\usepackage{vmargin}

% left top textwidth textheight headheight

% headsep footheight footskip

\setmargins{2.0cm}{2.5cm}{16 cm}{22cm}{0.5cm}{0cm}{1cm}{1cm}

\renewcommand{\baselinestretch}{1.3}

\setcounter{MaxMatrixCols}{10}

\begin{document}

\begin{enumerate}
\item
%%%%%%%%%%%%%%%%%%%%%%%%%%%%%%%%%%%%%%%%%%%%%%%%%%%%%%%%%%%%%%%%%%%%%%%%%%%%%%%%%%%%%%%%%%%%%%%%%%%%%%%%%%%
13 A researcher studying the claims experience of a company (in a particular class of business) records the payments on 100 recent claims. The payments (in units of £1000, and sorted) are given below.
Payments
\begin{verbatim}
0.30 0.89 0.96 1.16 1.67 1.77 1.93 1.98 2.07 2.09 2.30 2.48
2.58 2.78 3.00 3.19 3.21 3.21 3.25 3.31 3.34 3.37 3.66 3.95
4.16 4.18 4.60 4.72 4.73 4.76 5.01 5.17 5.21 5.63 5.72 6.00
6.13 6.17 6.24 6.37 6.47 6.48 6.87 7.05 7.16 7.21 7.51 7.72
7.74 8.00 8.00 8.03 8.04 8.54 9.11 9.18 9.49 9.59 10.00 10.36
10.85 11.08 11.22 11.27 11.38 11.45 11.69 11.78 12.27 12.30 12.50 13.04
13.28 13.43 13.48 13.85 14.27 14.31 14.49 14.55 14.62 14.68 14.70 14.83
15.67 15.70 15.77 16.28 16.44 17.17 17.89 18.03 18.12 20.72 22.00 24.33
25.41 28.30 31.00 32.80
\end{verbatim}

For these 100 observations: $\sum x = 952.75$ , $\sum x^2 = 13584.5217$.
The researcher wants to examine whether or not the exponential distribution provides
a good description of the distribution of the payments.
\begin{enumerate}[(a)]
\item Calculate the sample mean and standard deviation for the 100 payments, and comment briefly on whether or not you think the exponential distribution will provide an acceptable fit to the data.
\item Specify the fitted exponential distribution, giving a brief justification of your approach.
Note: you are not required to give any mathematical derivation in your justification.
\item The researcher decides to conduct a formal chi-squared goodness-of-fit test of the exponential distribution to the data, using five equi-probable intervals (i.e. intervals each with associated probability 0.2).
(a) Show that the value x which is exceeded with probability p by an exponential variable with mean satisfies $x = \log p$.
(b) Calculate the values which divide the positive real numbers into five equi-probable intervals for your fitted exponential distribution.
\item  Conduct the formal goodness-of-fit test outlined in part (ii) above and
comment on the result. 
\end{enumerate}

%%%%%%%%%%%%%%%%%%%%%%%%%%%%%%%%%%%%%%%%%%%%%%%%%%%%%%%%%%%%%%%%%%%%%%%%%%%%%%%%%%%%%%%%%%%%%%%%%%%%%%%%%%%

13 (i) (a) x 952.75 /100 9.53
0.5
1 952.752
13584.5217 6.75
99 100
s
\begin{itemize}
    \item An exponential distribution has mean = standard deviation so there must be some doubt here as to whether such a distribution will be a good description of these data.
   \item(b) The fitted model is an exponential distribution with mean 9.53 
(i.e. with parameter 1/9.5275 = 0.105).
   \item Reason: the maximum likelihood estimate (and the method of moments estimate) of the mean of an exponential distribution is the sample mean
%%%%%%%%%%%%%%%%%%%%%%%%%%%%%%%%%%%%%
Page 9
   \item (ii) (a) For X ~ exp(mean = ) we have P(X > x) = exp( x/ )
exp( x/ ) = p x = logp
   \item (b) Using p = 0.8, 0.6, 0.4, and 0.2 and fitted mean 9.5275
we get $x = \{2.13, 4.87, 8.73, 15.33\}$
   \item(iii) The numbers of observations in (0,2.13), (2.13,4.87), (4.87,8.73), (8.73,15.33)
and (15.33, ) are, by inspection from the data, respectively 10, 20, 24, 30,
and 16.
\end{itemize}

\begin{verbatim}
Interval (0, 2.13) & (2.13, 4.87) & (4.87, 8.73) & (8.73, 15.33) & (15.33
Observed frequency 10 20 24 30 16
Expected frequency 20 20 20 20 20
\end{verbatim}
\[\chi^2 = \frac{(102 + 0 + 42 + 102 + 42)}{20} = 232/20 = 11.6\]
df = 3
P-value = P( 2 > 11.6) = 0.009 (from Yellow Tables p164), so we reject the exponential distribution as a description it provides a very poor fit to the data.

\end{document}
