%-ACT 101 2000 April Q15

\documentclass[a4paper,12pt]{article}

%%%%%%%%%%%%%%%%%%%%%%%%%%%%%%%%%%%%%%%%%%%%%%%%%%%%%%%%%%%%%%%%%%%%%%%%%%%%%%%%%%%%%%%%%%%%%%%%%%%%%%%%%%%%%%%%%%%%%%%%%%%%%%%%%%%%%%%%%%%%%%%%%%%%%%%%%%%%%%%%%%%%%%%%%%%%%%%%%%%%%%%%%%%%%%%%%%%%%%%%%%%%%%%%%%%%%%%%%%%%%%%%%%%%%%%%%%%%%%%%%%%%%%%%%%%%

\usepackage{eurosym}
\usepackage{vmargin}
\usepackage{amsmath}
\usepackage{graphics}
\usepackage{epsfig}
\usepackage{enumerate}
\usepackage{multicol}
\usepackage{subfigure}
\usepackage{fancyhdr}
\usepackage{listings}
\usepackage{framed}
\usepackage{graphicx}
\usepackage{amsmath}
\usepackage{chngpage}

%\usepackage{bigints}
\usepackage{vmargin}

% left top textwidth textheight headheight

% headsep footheight footskip

\setmargins{2.0cm}{2.5cm}{16 cm}{22cm}{0.5cm}{0cm}{1cm}{1cm}

\renewcommand{\baselinestretch}{1.3}

\setcounter{MaxMatrixCols}{10}

\begin{document}
%===================================%
\item 14
Consider the following data, which comprise four groups of claim sizes (y), each
comprising four observations. In scenario I, information is also given on the sum
assured under the policy concerned the sum assured is the same for all four
policies in a group. In scenario II, we regard the policies in the different groups as
having been issued by four different companies
the policies in a group are all
issued by the same company.
All monetary amounts are in units of £10,000. Summaries of the claim sizes in each
group are given in a second table.
Group
Claim sizes y
I: Sum assured x
II: Company
1
0.11 0.46
0.71 1.45
1
A
2
0.52 1.43
1.84 2.47
2
B
3
1.48 2.05
2.38 3.31
3
C
4
1.52 2.36
2.95 4.08
4
D
Summaries of claim sizes:
Group
y
y 2
(i)
1
2.73
2.8303
2
6.26
11.8018
3
9.22
23.0134
4
10.91
33.2289
In scenario I, suppose we adopt the linear regression model
Y i =
+ x i + e i
where Y i is the i th claim size and x i is the corresponding sum assured,
i = 1, , 16.
\begin{enumerate}[(i)]
\item calculate the total sum of squares and its partition into the regression
(model) sum of squares and the residual (error) sum of squares.
\item (b) Fit the model and calculate the fitted values for the first claim size of
group 1 (namely 0.11) and the last claim size of group 4 (namely 4.08).
\item (c) Consider a test of the hypothesis H 0 : = 0 against a two-sided
alternative. By performing appropriate calculations, assess the strength
of the evidence against this no linear relationship hypothesis.
\end{enumerate}
(ii)
In scenario II, suppose we adopt the analysis of variance model
Y ij =
+
i
+ e ij
where Y ij is the j th claim size for company i and
j = 1, 2, 3, 4 and i = A, B, C, D.
is the i th company effect,
(a) Calculate the partition of the total sum of squares into the between
companies (model) sum of squares and the within companies
(residual/error) sum of squares.
(b) Fit the model.
(c) Calculate the fitted values for the first claim size of group 1 and the
last claim size of group 4.
(d) Consider a test of the hypothesis H 0 : i 0 , i = A, B, C, D against a
general alternative. By performing appropriate calculations, assess the
strength of the evidence against this no company effects hypothesis.
\end{enumerate}
\newpage


%=============================================================%

14
(i)
(a)
(b)
y = 29.12, y 2 = 70.8744
SS TOT = 70.8744 29.12 2 /16 = 17.8760
x = 4 10 = 40, x 2 = 4 30 = 120
S xx = 120 40 2 /16 = 20
xy = 1 2.73 + 2 6.26 + 3 9.22 + 4 10.91 = 86.55
S xy = 86.55 40 29.12/16 = 13.75
Regression sum of squares SS REG = 13.75 2 /20 = 9.4531
Residual sum of squares SS RES = 17.8760 9.4531 = 8.4229
13.75 / 20 0.6875
29.12 /16 0.6875 (40 /16) 0.1012
Fitted model is y 0.1012 0.6875 x
y = 0.11, x = 1 fitted value = 0.7887
Page 9Subject 101 (Statistical Modelling)
y = 4.08, x = 4
(c)
s . e .
%------------------------------------------------------%
fitted value = 2.8512
8.4229 /14
20
Under H 0 , P
September 2004
0.5
0.6875
0.1734
P t 14
0.6875 / 0.1734
P t 14
3.965
which is very much lower than 0.005, so P-value of test statistic is very
much lower than 0.01.
We have strong evidence against the no linear relationship
hypothesis (p << 0.01)
(ii)
\begin{itemize}
    \item (a)
SS TOT = 17.8760
Between companies sum of squares
SS B = (2.73 2 + 6.26 2 + 9.22 2 + 10.91 2 )/4
29.12 2 /16 = 9.6709
Residual sum of squares SS RES = 17.8760
    \item (b)
1
3
9.6709 = 8.205
29.12 /16 1.82
2.73 / 4 1.82
1.1375 , 2 6.26 / 4 1.82
0.255
9.22 / 4 1.82 0.485 , 4 10.91/ 4 1.82 0.9075
    \item (c) y = 0.11, company A
y = 4.08, company D
fitted value = 2.73/4 = 0.6825
fitted value = 10.91/4 = 2.7275
    \item (d) Observed F statistic is (9.6709/3) / (8.2051/12) = 4.715 on 3,12 df
P-value of test statistic is lower than 0.05 (but higher than 0.01)
\end{itemize}

We have some evidence against the no company effects hypothesis
(0.01 < p < 0.05)
\end{document}
