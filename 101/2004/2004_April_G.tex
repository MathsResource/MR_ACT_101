\documentclass[a4paper,12pt]{article}

%%%%%%%%%%%%%%%%%%%%%%%%%%%%%%%%%%%%%%%%%%%%%%%%%%%%%%%%%%%%%%%%%%%%%%%%%%%%%%%%%%%%%%%%%%%%%%%%%%%%%%%%%%%%%%%%%%%%%%%%%%%%%%%%%%%%%%%%%%%%%%%%%%%%%%%%%%%%%%%%%%%%%%%%%%%%%%%%%%%%%%%%%%%%%%%%%%%%%%%%%%%%%%%%%%%%%%%%%%%%%%%%%%%%%%%%%%%%%%%%%%%%%%%%%%%%

\usepackage{eurosym}
\usepackage{vmargin}
\usepackage{amsmath}
\usepackage{graphics}
\usepackage{epsfig}
\usepackage{enumerate}
\usepackage{multicol}
\usepackage{subfigure}
\usepackage{fancyhdr}
\usepackage{listings}
\usepackage{framed}
\usepackage{graphicx}
\usepackage{amsmath}
\usepackage{chngpage}

%\usepackage{bigints}
\usepackage{vmargin}

% left top textwidth textheight headheight

% headsep footheight footskip

\setmargins{2.0cm}{2.5cm}{16 cm}{22cm}{0.5cm}{0cm}{1cm}{1cm}

\renewcommand{\baselinestretch}{1.3}

\setcounter{MaxMatrixCols}{10}

\begin{document}
11 
The following table gives the sums insured (in units of £1,000) for a random sample
of insurance policies on the contents of private houses for each of four insurance
companies.
Company
\begin{verbatim}
1 2 3 4
(y1) (y2) (y3) (y4)
39 24 21 33
29 28 30 26
33 33 30 28
36 22 51 30
27 29 23 27
36 20 23 37
27 23 35 37
y1 = 227 y2 = 179 y3 = 213 y4 = 218
y1
2 = 7,501 y2
2 = 4,703 y3
2 = 7,125 y4
2 = 6,916
\end{verbatim}

\begin{enumerate}[(a)]
\item (i) Test whether there are any differences between the population means of the
four companies, using one-way analysis of variance.
\item 
(ii) A plot of the residuals for the fitted one-way analysis of variance model is
given below:
Comment on the adequacy of the model. 
\end{enumerate}

1 2 3 4
-10
0
10
20
Company
Residuals
%%%%%%%%%%%%%%%%%%%%%%%%%%%%%%%%%%%%%%%%%%%%%%%%%%%%%%%%%%%%%%%%%%%%%%%%%%%%%%%%%%%%%%%%%%%%%%%%%%%%%%%%%%%
\newpage

11 (i) n1 = n2 = n3 = n4 = 7 n = 28
y1 = 227 y2 = 179 y3 = 213 y4 = 218 yij = 837

\begin{itemize}
    \item $\sum y_1^2 = 7501$ 
\item $\sum y^2_2 = 4703 $
\item $\sum y_3^2 = 7125 $
\item $\sum y_4^2 = 6916 $
\item $\sum yij^2 = 26245 $
\end{itemize}
SST = 26245
8372
28
= 26245 25020.321 = 1224.7
SSB =
1
7
(2272 + 1792 + 2132 + 2182) 25020.321
= 25209 25020.321 = 188.7
%%%%%%%%%%%%%%%%%%%%%%%%%%%%%%%%%%%%%
Page 6
\begin{verbatim}
 SSR = SST SSB = 1224.7 188.7 = 1036.0
Source of variation d.f. SS MSS
Companies 3 188.7 62.9
Residual 24 1036.0 43.2
Total 27 1224.7
F =
62.9
43.2
= 1.46 on (3,24) degrees of freedom.   
\end{verbatim}

The 10\% point of F3,24 distribution is 2.327. Therefore, there is insufficient
evidence to reject the null hypothesis that the population means for the four
companies are equal, i.e., the distributions of the sums insured are the same for
the four companies.
(ii) 
\begin{itemize}
    \item The model used in (i) assumes that the sums insured for each company follow
a normal distribution, and the population variances are equal.
\item The plot of residuals shows:
normality seems appropriate, but observation £51,000 seems to be an
outlier
companies have similar sample variances. \item However one could argue that there is
a suggestion that the variance for Company 3 is higher than the variances
for the other companies.
\item Therefore, overall the one-way analysis of variance model seems adequate and
the conclusions in (i) are valid.
\end{itemize}
%%%%%%%%%%%%%%%%%%%%%%%%%%%%%%%%%%%%%%%%%%%%%%%%%%%%%%%%%%%%%%%%%%%%%%%%%%%%%%%%%%%%%%%%%%%%%%%%%%%%%%%%%%%
\end{document}
