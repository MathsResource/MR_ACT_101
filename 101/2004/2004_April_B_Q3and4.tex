\documentclass[a4paper,12pt]{article}

%%%%%%%%%%%%%%%%%%%%%%%%%%%%%%%%%%%%%%%%%%%%%%%%%%%%%%%%%%%%%%%%%%%%%%%%%%%%%%%%%%%%%%%%%%%%%%%%%%%%%%%%%%%%%%%%%%%%%%%%%%%%%%%%%%%%%%%%%%%%%%%%%%%%%%%%%%%%%%%%%%%%%%%%%%%%%%%%%%%%%%%%%%%%%%%%%%%%%%%%%%%%%%%%%%%%%%%%%%%%%%%%%%%%%%%%%%%%%%%%%%%%%%%%%%%%

\usepackage{eurosym}
\usepackage{vmargin}
\usepackage{amsmath}
\usepackage{graphics}
\usepackage{epsfig}
\usepackage{enumerate}
\usepackage{multicol}
\usepackage{subfigure}
\usepackage{fancyhdr}
\usepackage{listings}
\usepackage{framed}
\usepackage{graphicx}
\usepackage{amsmath}
\usepackage{chngpage}

%\usepackage{bigints}
\usepackage{vmargin}

% left top textwidth textheight headheight

% headsep footheight footskip

\setmargins{2.0cm}{2.5cm}{16 cm}{22cm}{0.5cm}{0cm}{1cm}{1cm}

\renewcommand{\baselinestretch}{1.3}

\setcounter{MaxMatrixCols}{10}

\begin{document}
3 The cumulant generating function of a random variable X is given by:
 \[CX (t) = logMX (t) = 2 1 t 1\]
where $M_X(t)$ is the moment generating function.
Determine the mean and variance of the distribution of $X$. 

\newpage
3 C´(t) = 20(1 t) 11 , C´´(t) = 220(1 t) 12
\[E[X] = C´(0) = 20\]
\[V[X] = C´´(0) = 220\]
[OR as coefficients of t and of t2/2! in expansion of C(t)]
%%%%%%%%%%%%%%%%%%%%%%%%%%%%%%%%%
\newpage 4 Claim sizes in a certain insurance situation are modelled by an exponential
distribution with mean \$20,000. The insurer defines a claim to be a large claim if the
claim size exceeds \$35,000.
State, with a reason, the expected size of a large claim. %%%%%%%%%%%%%%%%%%%%%%%%%%%%%%%%%%%%%
\newpage
4 Answer: $35,000 + 20,000 = $55,000
\begin{itemize}
    \item Reason: the memoryless property of the exponential distribution (the excess above
35,000 itself has an exponential distribution with mean 20,000).
\item Note: relatively few candidates were able to exploit the memoryless property of the
exponential distribution to advantage.
\end{itemize}
\end{document}
