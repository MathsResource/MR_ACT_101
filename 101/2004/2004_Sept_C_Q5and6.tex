\documentclass[a4paper,12pt]{article}

%%%%%%%%%%%%%%%%%%%%%%%%%%%%%%%%%%%%%%%%%%%%%%%%%%%%%%%%%%%%%%%%%%%%%%%%%%%%%%%%%%%%%%%%%%%%%%%%%%%%%%%%%%%%%%%%%%%%%%%%%%%%%%%%%%%%%%%%%%%%%%%%%%%%%%%%%%%%%%%%%%%%%%%%%%%%%%%%%%%%%%%%%%%%%%%%%%%%%%%%%%%%%%%%%%%%%%%%%%%%%%%%%%%%%%%%%%%%%%%%%%%%%%%%%%%%

\usepackage{eurosym}
\usepackage{vmargin}
\usepackage{amsmath}
\usepackage{graphics}
\usepackage{epsfig}
\usepackage{enumerate}
\usepackage{multicol}
\usepackage{subfigure}
\usepackage{fancyhdr}
\usepackage{listings}
\usepackage{framed}
\usepackage{graphicx}
\usepackage{amsmath}
\usepackage{chngpage}

%\usepackage{bigints}
\usepackage{vmargin}

% left top textwidth textheight headheight

% headsep footheight footskip

\setmargins{2.0cm}{2.5cm}{16 cm}{22cm}{0.5cm}{0cm}{1cm}{1cm}

\renewcommand{\baselinestretch}{1.3}

\setcounter{MaxMatrixCols}{10}

\begin{document}

\begin{enumerate}
\item

5 When comparing the mean premiums for policies issued by two companies, a twosample
t test is performed assuming equal population variances. The sample sizes and
sample variances are given by
2
n1 25 , s1 139.7
2
n2 30 , s2 76.6
Perform an appropriate F test at the 5\% level to investigate the validity of the equal
variance assumption.
%%%%%%%%%%%%%%%%%%%%%%%%%%%%%%%%%%%%%%%%%%%%%%%%%%%%%%%%%%%%%%%%%%%%%%%%%%%%%%%%%%%%%%%%%%%%%%%%%%%%%%
\item A $2 \times 2$ contingency table was set up to investigate whether or not two classification
criteria are independent and resulted in the following data:
I II
A 22 28 50
B 28 22 50
50 50 100
Calculate the observed 2 test statistic and state an appropriate conclusion concerning
the independence of the two criteria. 

%%%%%%%%%%%%%%%%%%%%%%%%%%%%%%%%%%%%%%%%%%%%%%%%%%%%%%%%%%%%%%%%%%%%%%%%%%%%%%%%%%%%%%%%%%%%%%%%%%%%%%
\end{enumerate}
5
2
1
2
2
139.7
1.82
76.6
s
s
F24,29 critical value at 5\% is 2.154 (two-sided test)
accept H0 : equal variances at the 5\% level.

% Examiners Comments: A test for equality of variances as asked for here is a twosided test, but since we always use the observed ratio with the larger sample variance on the numerator, we look at the upper tail of the reference F distribution. For a 5% test we look at the upper 2.5% tail, not the upper 5% tail, as some candidates did.
%%%%%%%%%%%%%%%%%%%%%%%%%%%%%%%%%%%%%
\newpage 
%%--Page 3
6 Expected frequencies are all 25
2
2 3
4 1.44
25
P-value = 2
P( 1 1.44) 1 0.77 0.23
there is no evidence to reject the independence of the two criteria.

% Examiners Comments: Although not covered in the Core Reading, the use of Yates5 correction in this situation (in which the 2 statistic has only 1 degree of freedom) is acceptable. Using it gives 2 = 1, a P-value of 0.32, and the same conclusion.

\end{document}
