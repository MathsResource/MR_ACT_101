\documentclass[a4paper,12pt]{article}

%%%%%%%%%%%%%%%%%%%%%%%%%%%%%%%%%%%%%%%%%%%%%%%%%%%%%%%%%%%%%%%%%%%%%%%%%%%%%%%%%%%%%%%%%%%%%%%%%%%%%%%%%%%%%%%%%%%%%%%%%%%%%%%%%%%%%%%%%%%%%%%%%%%%%%%%%%%%%%%%%%%%%%%%%%%%%%%%%%%%%%%%%%%%%%%%%%%%%%%%%%%%%%%%%%%%%%%%%%%%%%%%%%%%%%%%%%%%%%%%%%%%%%%%%%%%

\usepackage{eurosym}
\usepackage{vmargin}
\usepackage{amsmath}
\usepackage{graphics}
\usepackage{epsfig}
\usepackage{enumerate}
\usepackage{multicol}
\usepackage{subfigure}
\usepackage{fancyhdr}
\usepackage{listings}
\usepackage{framed}
\usepackage{graphicx}
\usepackage{amsmath}
\usepackage{chngpage}

%\usepackage{bigints}
\usepackage{vmargin}

% left top textwidth textheight headheight

% headsep footheight footskip

\setmargins{2.0cm}{2.5cm}{16 cm}{22cm}{0.5cm}{0cm}{1cm}{1cm}

\renewcommand{\baselinestretch}{1.3}

\setcounter{MaxMatrixCols}{10}

\begin{document}

\begin{enumerate}
\item

9 A statistician suggests that, since a t variable with k degrees of freedom is
symmetrical with mean 0 and variance
2
k
k
for k > 2, one can approximate the
distribution using the normal variable 0,
2
k
N
k
.
(i) Use this to obtain an approximation for the upper 5% percentage points for a
t variable with:
(a) 4 degrees of freedom, and
(b) 40 degrees of freedom
[2]
(ii) Compare your answers with the exact values from tables and comment briefly
on the result. [2]
%%%%%%%%%%%%%%%%%%%%%%%%%%%%%%%%%%%%%%%%%%%%%%%%%%%%%%%%%%%%%%%%%%%%%%%%%%%%%%%%%%%%%%%%%%%%%%%%%%%%%%
\item 10 In order to estimate a certain probability of success a single observation is taken from
the binomial random variable X ~ binomial(20, p).
\begin{enumerate}
\item (i) Write down an expression for the mean square error of the maximum
likelihood estimator,
20
X
p , and evaluate this mean square error at p = 0.5.
[2]
\item (ii) Determine an expression for the mean square error of the estimator,
1
21
X
p , and evaluate this mean square error at p = 0.5. [4]
\item (iii) Comment briefly on the comparison of p and p as estimators of p in the case
p = 0.5. [1]
\end{enumerate}
%%%%%%%%%%%%%%%%%%%%%%%%%%%%%%%%%%%%%%%%%%%%%%%%%%%%%%%%%%%%%%%%%%%%%%%%%%%%%%%%%%%%%%%%%%%%%%%%%%%%%%
\item 11 The continuous random variables X and Y have a bivariate probability density
function
\[f(x, y) = 2 for 0 < x + y < 1, x > 0, y > 0.\]
The conditional distribution of X given Y = y is a uniform distribution with probability
density function
f(x|y) =
1
1 y
0 < x < 1 y
and the marginal distribution of Y is a beta distribution with probability density
function
f(y) = 2(1 y) 0 < y < 1.
(i) Show that the conditional expectation of X given Y = y is
E(X|Y = y) =
1
,
2
y
and obtain the conditional variance of X given Y = y. [3]
(ii) Verify in this case that var(X) = var(E(X|Y)) + E(var(X|Y)). [3]
\end{enumerate}
%%%%%%%%%%%%%%%%%%%%%%%%%%%%%%%%%%%%%%%%%%%%%%%%%%%%%%%%%%%%%%%%%%%
\newpage


9 (i) (a) k = 4 using N(0, 4/2) = N(0, 2)
5% point = 0 + 1.6449 2 = 2.326
(b) k = 40 using N(0, 40/38) = N(0, 1.0526)
5% point = 0 + 1.6449 1.0526 = 1.688
(ii) Exact values are: (a) 2.132 and (b) 1.684
for small df approximation is poor, but for large df it is quite good.
10 (i) p is unbiased with variance
(1 )
20
p p
MSE =
(1 )
20
p p
.
Evaluation gives
0.5(1 0.5)
0.0125
20
(ii) p has bias =
20 1 1
21 21
p p
p and variance 2
20 (1 )
21
p p
MSE =
2
2 2
20 (1 ) (1 )
21 21
p p p
Evaluation gives
2
2 2
20(0.5)(1 0.5) (1 0.5)
0.0113 0.0006 0.0119
21 21
(iii) Even though p is the MLE and is unbiased, p is a more efficient estimate
(for p = 0.5) having a smaller mean square error.
%%%%%%%%%%%%%%%%%%%%%%%%%%%%%%%%%%%%%
Page 5
11 (i) Using results for a uniform distribution given in the Yellow Book we obtain
the following
1
( | )
2
y
E X Y y
and
var(X|Y = y) =
(1 )2
12
y
since the conditional distribution of X given Y = y is a continuous uniform
distribution with parameters a = 0 and b = 1 y.
(ii)
1 1 1 1 1
var( ( | )) var var( )
2 4 4 18 72
Y
E X Y Y
since Y has a beta distribution with parameters = 1, = 2, and the Yellow
Book gives
1
var( )
18
Y .
(1 )2
(var( | ))
12
Y
E X Y E
1 2 4 1
0 0
(1 ) 1 (1 ) 1
2(1 ) .
12 6 4 24
y y
y dy
Therefore,
1 1 1
var( ( | )) (var( | )) .
72 24 18
E X Y E X Y
By symmetry with the random variable Y, X has a beta distribution with
parameters = 1, = 2, and
1
var( )
18
X .
Examiners Comment: The question helpfully states that the conditional distribution
is a uniform distribution and the marginal distribution is a beta distribution. Despite
this many candidates failed to quote standard results given in the Yellow Book
relating to uniform and beta distributions and instead performed time-consuming
integrations.
%%%%%%%%%%%%%%%%%%%%%%%%%%%%%%%%%%%%%
\end{document}
