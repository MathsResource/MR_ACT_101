\documentclass[a4paper,12pt]{article}

%%%%%%%%%%%%%%%%%%%%%%%%%%%%%%%%%%%%%%%%%%%%%%%%%%%%%%%%%%%%%%%%%%%%%%%%%%%%%%%%%%%%%%%%%%%%%%%%%%%%%%%%%%%%%%%%%%%%%%%%%%%%%%%%%%%%%%%%%%%%%%%%%%%%%%%%%%%%%%%%%%%%%%%%%%%%%%%%%%%%%%%%%%%%%%%%%%%%%%%%%%%%%%%%%%%%%%%%%%%%%%%%%%%%%%%%%%%%%%%%%%%%%%%%%%%%

\usepackage{eurosym}
\usepackage{vmargin}
\usepackage{amsmath}
\usepackage{graphics}
\usepackage{epsfig}
\usepackage{enumerate}
\usepackage{multicol}
\usepackage{subfigure}
\usepackage{fancyhdr}
\usepackage{listings}
\usepackage{framed}
\usepackage{graphicx}
\usepackage{amsmath}
\usepackage{chngpage}

%\usepackage{bigints}
\usepackage{vmargin}

% left top textwidth textheight headheight

% headsep footheight footskip

\setmargins{2.0cm}{2.5cm}{16 cm}{22cm}{0.5cm}{0cm}{1cm}{1cm}

\renewcommand{\baselinestretch}{1.3}

\setcounter{MaxMatrixCols}{10}

\begin{document}
 Let S(0) denote the price of a certain security, and let S(n) denote the price of the
security at the end of n successive weeks for n = 1, 2, 3, .... A model for the changes
in these prices is such that the price ratios ( )
( 1)
S n
S n\mu
for n \mu1 are independent
identically distributed random variables with a lognormal distribution.
[Note: The random variable Y is lognormal with parameters \mu and \sigma 2 if log(Y) is
normal N(\mu,\sigma 2), that is, Y is lognormal if it can be expressed as Y \mu eX where
X ~ N(\mu,\sigma 2). The mean and variance of Y are given by E(Y) =
2
e 2
\mu
\sigma  and
V(Y) = e2 2 (e 2 1) \mu\sigma  
\mu .]

(i) Using the above lognormal model with parameters $\mu = 0.0125$ and $\sigma  = 0.055$,
determine the probability that:
\begin{enumerate}[(i)]
    \item (a) the price of the security decreases over the next week
\item (b) the price of the security decreases over each of the next two weeks
\item (c) the price at the end of two weeks is greater than it is at present
\item (d) the price at the end of 20 weeks is less than it is at present
\end{enumerate}

[11]
(ii) At the start of a period the price is £1,245 and it is then observed for 10 weeks.
The resulting 10 prices (£) and ratios are given in the following table:
\begin{verbatim}
Week Price Ratio(y)
0 1,245 -
1 1,230 0.988
2 1,280 1.041
3 1,392 1.088
4 1,431 1.028
5 1,428 0.998
6 1,439 1.008
7 1,346 0.935
8 1,265 0.940
9 1,513 1.196
10 1,468 0.970
y = 10.192 y2 = 10.441562    
\end{verbatim}

(a) Based on the specified model, explain why the 10 ratios constitute a
random sample from a lognormal distribution.
(b) Calculate the mean and the standard deviation for the random sample
of 10 ratios.
(c) Hence determine the method of moments estimates of the parameters \mu
and \sigma  for the lognormal model. 
\end{enumerate}
\newpage
15 (i) (a) Let Y = ratio for one week, and $Y = e^X$
\begin{eqnarray*}
P(decrease in one week) &=& P(Y < 1)\\
&=& P(eX < 1) = P(X < 0)\\
&=& P(Z < \frac{0-0.0125}{0.055})\\
&=& − 0.227) \\ 
&=& 1 − 0.5898 \\ &=& 0.41
\end{eqnarray*}

(b) P(decrease in next two weeks) = (0.41)2 = 0.17
(c) We require ( (2) 1) ( (2) . (1) 1)
(0) (1) (0)
P S P S S
S S S
> = >
= P(Y2 .Y1 > 1) = P(X2 + X1 > 0)
where X2, X1 are independent N(\mu,\sum2)
2
∴X2 + X1 ~ N(2\mu,2\sum )
( 0 2(0.0125) 0.321) 0.63
2(0.055)
P PZ −
∴ = > =− = from tables.
(d) Extending the method of part (c):
20
2
1
i ~ (20 , 20 )
i
X N
=
\sum \mu \sum
( 0 20(0.0125) 1.016) 0.155
20(0.055)
P PZ −
∴ = < =− =
\begin{itemize}
    \item (ii) (a) The ratios are independent and identically distributed lognormal r.v.’s.
\item This defines a random sample from a lognormal distribution.
\item (b) For the 10 observed ratios y1, . . . , y10:
\sum y =10.192 ⇒ y =1.0192
\sum y2 =10.441562 ⇒ s2 = 0.005986⇒ s = 0.0774
\item (c) For the method of moments:
solve the following equations for \mu and \sum2
\end{itemize}

%%%%%%%%%%%%%%%%%%%%%%%%%%%%%%%%%%%%%%%%%%%%%%%%%%%%%%%%%%%%%%%%%%%%%%%%%%

1 2
e 2 1.0192
\mu+ \sum
= (1)
2 2 2 e \mu+\sum (e\sum −1) = 0.005986 (2)
2 2 (2) ÷ (1) ⇒e\sum −1 = 0.0057625
∴\sum2 = 0.005746 ∴\sum = 0.0758
(1) log(1.0192) 1 2 0.0161
2
⇒ \mu = − \sum =
[Note: in MME candidates could use \sumˆ 2 =0.005388 with divisor n not
(n−1) to obtain \sum = 0.0719 and \mu = 0.0164 ]

\end{document}
