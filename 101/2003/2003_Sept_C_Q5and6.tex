\documentclass[a4paper,12pt]{article}

%%%%%%%%%%%%%%%%%%%%%%%%%%%%%%%%%%%%%%%%%%%%%%%%%%%%%%%%%%%%%%%%%%%%%%%%%%%%%%%%%%%%%%%%%%%%%%%%%%%%%%%%%%%%%%%%%%%%%%%%%%%%%%%%%%%%%%%%%%%%%%%%%%%%%%%%%%%%%%%%%%%%%%%%%%%%%%%%%%%%%%%%%%%%%%%%%%%%%%%%%%%%%%%%%%%%%%%%%%%%%%%%%%%%%%%%%%%%%%%%%%%%%%%%%%%%

\usepackage{eurosym}
\usepackage{vmargin}
\usepackage{amsmath}
\usepackage{graphics}
\usepackage{epsfig}
\usepackage{enumerate}
\usepackage{multicol}
\usepackage{subfigure}
\usepackage{fancyhdr}
\usepackage{listings}
\usepackage{framed}
\usepackage{graphicx}
\usepackage{amsmath}
\usepackage{chngpage}

%\usepackage{bigints}
\usepackage{vmargin}

% left top textwidth textheight headheight

% headsep footheight footskip

\setmargins{2.0cm}{2.5cm}{16 cm}{22cm}{0.5cm}{0cm}{1cm}{1cm}

\renewcommand{\baselinestretch}{1.3}

\setcounter{MaxMatrixCols}{10}

\begin{document}

\begin{enumerate}
\item
5 Ten independent hypothesis tests are to be conducted, each at the 5\% significance
level.
Calculate the probability that at least one of the tests produces a significant result,
assuming that the null hypothesis for each of the 10 tests is true, and comment briefly
on the value. 
\item 

\end{enumerate}
%%%%%%%%%%%%%%%%%%%%%%%%%%%%%%%%%%%%%%%%%%%%%%%%%%%%%%%%%%%%%%%%%%%%%%%%%%%%%%%%%%%%%%%%%%%%%%%%%%
\newpage

5 P(at least one test is significant | each null hypothesis is true)\\
= 1 − P(no test is significant | each null hypothesis is true)\\
= 1 − (1 − 0.05)10 as the 10 tests are independent\\
= 0.4\\

Comment: a false-positive is very likely with the 10 multiple tests.

%%%%%%%%%%%%%%%%%%%%%%%%%%%%%%%%%%%%%%%%%%%%%%%%%%%%%%%%%%%%%%%%%%%%%%%%%%%%%%%%%%%%%%%%%%%%%%%%%%
6 Show that the slope of the regression line fitted by least squares to the three points
$(0,0) , (1,y) , (2,2)$
is 1 for all values of $y$. 

6 Sxx = 5 – 32/3 = 2 , Sxy = y + 4 – 3(y+2)/3 = 2
So fitted slope = 2/2 = 1
%%%%%%%%%%%%%%%%%%%%%%%%%%%%%%%%%%%%%%%%%%%%%%%%%%%%%%%%%%%%%%%%%%%%%%%%%%%%%%%%%%%%%%%%%%%%%%%%%%

%%%%%%%%%%%%%%%%%%%%%%%%%%%%%%%%%%%%%%%%%%%%%%%%%%%%%%%%%%%%%%%%%%%%%%%%%%%%%%%%%%%%%%%%%%%%%%%%%%
% Page 5
OR: the distribution of X can be modelled approximately using a normal distribution
with mean 0.02n and variance 0.0196n; we require P(X ≤ n − 500) to be at least 0.95;
the analysis is more awkward, but solving a quadratic in n gives n ≥ 515

\end{document}
