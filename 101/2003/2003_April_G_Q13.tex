\documentclass[a4paper,12pt]{article}

%%%%%%%%%%%%%%%%%%%%%%%%%%%%%%%%%%%%%%%%%%%%%%%%%%%%%%%%%%%%%%%%%%%%%%%%%%%%%%%%%%%%%%%%%%%%%%%%%%%%%%%%%%%%%%%%%%%%%%%%%%%%%%%%%%%%%%%%%%%%%%%%%%%%%%%%%%%%%%%%%%%%%%%%%%%%%%%%%%%%%%%%%%%%%%%%%%%%%%%%%%%%%%%%%%%%%%%%%%%%%%%%%%%%%%%%%%%%%%%%%%%%%%%%%%%%

\usepackage{eurosym}
\usepackage{vmargin}
\usepackage{amsmath}
\usepackage{graphics}
\usepackage{epsfig}
\usepackage{enumerate}
\usepackage{multicol}
\usepackage{subfigure}
\usepackage{fancyhdr}
\usepackage{listings}
\usepackage{framed}
\usepackage{graphicx}
\usepackage{amsmath}
\usepackage{chngpage}

%\usepackage{bigints}
\usepackage{vmargin}

% left top textwidth textheight headheight

% headsep footheight footskip

\setmargins{2.0cm}{2.5cm}{16 cm}{22cm}{0.5cm}{0cm}{1cm}{1cm}

\renewcommand{\baselinestretch}{1.3}

\setcounter{MaxMatrixCols}{10}

\begin{document}


%%%%%%%%%%%%%%%%%%%%%%%%%%%%%%%%%%%%%%%%%%%%%%%%%%%%%%%%%%%%%%%%%%%%%%%%%%%%%%%%%%%%%%%%%%%%%%%%
13 The random variable, X, has a gamma distribution with probability density function
given by:
 
 
 
 
1 exp /
0 ,
m
m
x x
f x x
m
  
 \beta
 
where $m$ and $\beta$ are positive constants. This distribution has mean $m\beta$ and variance
m2. Let $x_1, \ldots , x_n$ denote a random sample of n observations on X.
\begin{enumerate}[(a)]
    \item (i) Suppose that m is known.
    \begin{enumerate}[(i)]
 \item (a) Show that the maximum likelihood estimate of $\beta$ is given by
1
ˆ / . n
i i x mn

 
 \item  (b) Show that $\hat{\beta}$ is an unbiased estimator of $\beta$.
 \item  (c) Obtain the Cramer-Rao lower bound for estimators of $\beta$.
 \item (d) Show that the maximum likelihood estimator of $\beta$ has variance equal to the lower bound given in (i)(c). 
 \\ \medskip
Suppose now that m is unknown, and is also to be estimated by maximum likelihood. It is assumed that m is large enough so that $\Gamma(m)$ is well approximated by
\[g(m) exp( m)m(m 0.5) (2 )0.5 \]
   .
\end{enumerate}
    \item (ii) Determine the approximate maximum likelihood estimates of $\beta$ and $m$, substituting g(m) for $\Gamma(m)$ in the likelihood function. 
    \item (iii) Suppose that the sample values are:
\[32 48 51 43 82 155\]
Obtain the approximate maximum likelihood estimates for $\beta$ and $m$ given in (ii). 
\end{enumerate}


%%%%%%%%%%%%%%%%%%%%%%%%%%%%%%%%%%%%%%%%%%%%%%%%%%%%%%%%%%%%%%%%%%%%%%%%%%%%%%%%%%%%%%%%%%%%%%%%
\newpage
13
1 ( ) exp( / ) ( 0)
( )
m
m
f x x x x
m
  
 \beta
 
1 1
1
1
exp
( )
( )
n n m i i
i i n
i i mn n
x x
L fx
m
 


 \beta 
 \Gamma
  	 
 

 


(i) m is known case.
(a) 1
1
log ( 1) log log log ( )
n
n i
i
i
i
x
l L m x mn n m 

      \beta



1 1
2
Then 0 ˆ
n n
i i
i i
x x
l mn l
mn
   

  
 


   
 
(MLE)
(b) E(ˆ) nE(Xi ) nm ˆ
mn mn

    is unbiased
(c)
2
1
2 3 2 2
n
i
i
x
l mn  
  \beta
  

2
2 3 2 2
E l 2 nE(X ) mn mn
  
  \beta  \Gamma \beta  
 	  	 	 	
 CRlb =
2 1
2
E l
    \beta
\Gamma 		 

    

=
2
mn

(d) Variance of ˆ :
2 2
2 2 2 2 2 2
1
( ˆ) 1 ( ) ( )
n
i
i
V V X n V X nm
m n m n m n mn 
 
     
Cramer-Rao lb is attained.
%%%%%%%%%%%%%%%%%%%%%%%%%%%%%%%%%%%%%%%%%%%%%%%%%%%%%%%%%%%%%%%%%%%%%%%%%%%%%%%%%%%%%%%%%%%%%%%%
(ii) m is unknown case.
If m has also to be estimated, ML equations are (approx):
l l 0
m
 
 
 
  1
1 log ( 1) log log
n
n i i
i i
x
l L m x mn 

     



1 1
2 2 n \Gammam\beta (m ) logm\beta log 2
1
2 1
ˆ ˆ ˆ ˆ ˆ 0 /
n
i i n
i i
l x mn x nm 


   \beta 
  

 (MLE of )
 
1
2
1
ˆ ˆ log log log ˆ 0
ˆ
n
i i
l m x n n n n m
m  m
\beta   
  \Gamma   		   \beta 
 

Substituting for \hat{\beta}
gives
 
1
1 log log log ˆ 0
ˆ 2 ˆ
n n
i i
n x n x n n m
 m m
 
 \Gamma  \beta  
 	

2mˆ 1/ log(x / x) where  
1
1 1 , / n n n
i i i i x x x x n
 
   
 mˆ \beta1/ log  x / x 2   \Gamma
 (MLE of m)
There are alternative versions, for example,
 

i x
n
x
m
log 1 log
ˆ 0.5
Most candidates found Question 13(ii) difficult to complete successfully (the Examiners
recognise that this part was on the “hard side”).
(iii) x  68.5, x  59.147 mˆ  3.406, ˆ  20.114H

\end{document}
