\documentclass[a4paper,12pt]{article}

%%%%%%%%%%%%%%%%%%%%%%%%%%%%%%%%%%%%%%%%%%%%%%%%%%%%%%%%%%%%%%%%%%%%%%%%%%%%%%%%%%%%%%%%%%%%%%%%%%%%%%%%%%%%%%%%%%%%%%%%%%%%%%%%%%%%%%%%%%%%%%%%%%%%%%%%%%%%%%%%%%%%%%%%%%%%%%%%%%%%%%%%%%%%%%%%%%%%%%%%%%%%%%%%%%%%%%%%%%%%%%%%%%%%%%%%%%%%%%%%%%%%%%%%%%%%

\usepackage{eurosym}
\usepackage{vmargin}
\usepackage{amsmath}
\usepackage{graphics}
\usepackage{epsfig}
\usepackage{enumerate}
\usepackage{multicol}
\usepackage{subfigure}
\usepackage{fancyhdr}
\usepackage{listings}
\usepackage{framed}
\usepackage{graphicx}
\usepackage{amsmath}
\usepackage{chngpage}

%\usepackage{bigints}
\usepackage{vmargin}

% left top textwidth textheight headheight

% headsep footheight footskip

\setmargins{2.0cm}{2.5cm}{16 cm}{22cm}{0.5cm}{0cm}{1cm}{1cm}

\renewcommand{\baselinestretch}{1.3}

\setcounter{MaxMatrixCols}{10}

\begin{document}
Let $X$ denote the number of accidents a manual worker in a particular factory has in a
year. For a given worker the distribution of $X$ is modelled as a Poisson distribution
with unknown parameter u that varies across the workforce. 

$U$ is regarded as a
random variable which has a gamma distribution with parameters $\alpha$ and $\lambda$, i.e.
\[X | (U = u) \sim Poisson(u),\]
\[U \sim gamma( \alpha , \lambda).\]

\begin{enumerate}
\item (i) Show that the marginal distribution of X has mean $ {\displaystyle \frac{\alpha}{\lambda} }$ and variance
$ {\displaystyle \frac{\alpha}{\lambda} + \frac{\alpha}{\lambda^2} }$ 

\item (ii) A dataset has a sample mean of $\bar{x}$ and a sample variance of $s^2$. 

Show that $\alpha$ and $\lambda$ may be estimated by $ {\displaystyle \hat{\alpha} = \frac{\bar{x}^2}{s^2 - \bar{x}} }$
and
$ {\displaystyle \hat{\lambda} = \frac{\bar{x}}{s^2 - \bar{x}} }$
using the method of moments.

\item 
(iii) State the circumstances under which the method of moments produce
inadmissible estimates of $\alpha$ and $\lambda$. 
\end{enumerate}
%%%%%%%%%%%%%%%%%%%%%%%%%%%%%%%%%%%%%%%%%%%%%%%%%%%%%%%%%%%%%%%%%%%%%%%%%%
%% Page 8
%%----------------- Question 12 
%%%%%%%%%%%%%%%%%%%%%%%%%%%%%%%%%%%%%%%%%%%%%%%%%%%%%%%%%%%%%%%%%%%%%%%%
% 2003 September Q12
\begin{eqnarray*}
E [ X ] &=& E [ E ( X | U )] \\ &=& E [ U ] \\ &=& \frac{\alpha }{\lambda}\\
\end{eqnarray*}

\begin{eqnarray*}
V [ X ] &=& E [ V ( X | U )] + V [ E ( X | U )] \\ &=& E [ U ] + V [ U ]  \\ &=& \frac{\alpha }{\lambda} + \frac{\alpha }{\lambda^2}\\
\end{eqnarray*}
%%%%%%%%%%%%%%%%%%
(ii)
Using the method of moments, $\alpha$ and $\lambda$ may be estimated by solving the
equations

\[ \bar{x} = \frac{\alpha}{\lambda} + \frac{\alpha}{\lambda^2}\]
\[ s^2 = \frac{\alpha}{\lambda} + \frac{\alpha}{\lambda^2}\]
% \alpha \alpha
and s 2 = + 2

% \lambda \lambda
which gives

\[ \hat{\alpha} = \frac{ \bar{x} }{s^2 − \bar{x} } \]
\[ \hat{\lambda} = \frac{ \bar{x} }{s^2 − \bar{x} } \]
.
%%%%%%%%%%%%%%%%%%
(iii) If $s^2 \leq \bar{x}$ , then the method of moments produces inadmissible estimates as the
parameters $\alpha$ and $\lambda$ must be positive and finite.
%%%%%%%%%%%%%%%%%%%%%%%%%%%%%%%%%%%%%%%%%%%%%%%%%%%%%%%%%%%%%%%%%%%%%%%%%%%%%%%%%%%%%%%%%%%%%%%%%%





\end{document}
