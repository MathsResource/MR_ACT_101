\documentclass[a4paper,12pt]{article}

%%%%%%%%%%%%%%%%%%%%%%%%%%%%%%%%%%%%%%%%%%%%%%%%%%%%%%%%%%%%%%%%%%%%%%%%%%%%%%%%%%%%%%%%%%%%%%%%%%%%%%%%%%%%%%%%%%%%%%%%%%%%%%%%%%%%%%%%%%%%%%%%%%%%%%%%%%%%%%%%%%%%%%%%%%%%%%%%%%%%%%%%%%%%%%%%%%%%%%%%%%%%%%%%%%%%%%%%%%%%%%%%%%%%%%%%%%%%%%%%%%%%%%%%%%%%

\usepackage{eurosym}
\usepackage{vmargin}
\usepackage{amsmath}
\usepackage{graphics}
\usepackage{epsfig}
\usepackage{enumerate}
\usepackage{multicol}
\usepackage{subfigure}
\usepackage{fancyhdr}
\usepackage{listings}
\usepackage{framed}
\usepackage{graphicx}
\usepackage{amsmath}
\usepackage{chngpage}

%\usepackage{bigints}
\usepackage{vmargin}

% left top textwidth textheight headheight

% headsep footheight footskip

\setmargins{2.0cm}{2.5cm}{16 cm}{22cm}{0.5cm}{0cm}{1cm}{1cm}

\renewcommand{\baselinestretch}{1.3}

\setcounter{MaxMatrixCols}{10}

\begin{document}

 A social researcher is interested in the gender distribution among children in families, and has collected data for her investigation as follows.
Three hundred families were selected at random. The table below shows frequency
distributions of the numbers of girls in families of size 1, 2, 3, and 4, that is with 1, 2, 3, and 4 children.
Number of girls in family
Size of
family 0 1 2 3 4
Number of
families
1 23 27 - - - 50
2 30 46 24 - - 100
3 9 36 43 12 - 100
4 4 17 15 11 3 50
\begin{enumerate}
    \item (i) The researcher wants to investigate whether the proportion of girls within
families is independent of family size.
\item (a) Construct a suitable $2 \times 4$ contingency table, and calculate the overall
proportion of girls.
(b) State appropriate hypotheses to use in the researcher’s investigation.
(c) Calculate the value of an appropriate test statistic and state whether or not its probability value exceeds 0.05.
(d) State your conclusion. 
\item (ii) (a) Suggest a model (with all parameter values stated or estimated) for the number of girls in a family, for each family size (1, 2, 3, and 4), using your conclusion from part (i).
\item (b) Suppose the researcher was to test the goodness-of-fit of the models you have suggested in part (ii)(a) for family sizes 2, 3, and 4 and that the models were rejected as being unsuitable. Discuss briefly how you would interpret this lack of fit. 
\end{enumerate}

\end{enumerate}
\newpage
%%%%%%%%%%%%%%%%%%%%%%%%%%%%%%%%%%%%%%%%%%%%%%%%%%%%%%%%%%%%%%%%%%%%%%%%%%%%%%%%%%%%%%%%%%%%%%%%
\newpage
14 (i) 
\begin{itemize}
    \item (a) Suitable table is gender  family size.
Family size 3: no. of girls = 36 + 43(2) + 12(3) = 158
so no. of boys = 300 – 158 = 142, etc.

\begin{center}
    \begin{tabular}{|c|c|c|c|c|c}
family size & 1 & 2 & 3 & 4 & total \\ \hline
no. of girls & 27 &  94 & 158 & 92 & 371 \\ \hline
no. of boys & 23 & 106 & 142 & 108 & 379 \\ \hline
total & 50 & 200 & 300 & 200 & 750 \\ \hline
\end{tabular}
\end{center}
\item Overall proportion of girls = 371/750 = 0.495
(b) H0: gender and family size are independent
H1: gender and family size are not independent
\item (c) Expected frequency (under H0) in brackets
27 94 158 92
(24.73) (98.93) (148.40) (98.93)
23 106 142 108
(25.27) (101.07) (151.60) (101.07)

\begin{eqnarray*}\chi^2 &=& (27-24.73)2/24.73 + ….. \\
&=& 0.208 + 0.246 + 0.621 + 0.486 +\\
& & 0.203 + 0.241 + 0.608 + 0.476 \\
&=& 3.088\\
\end{eqnarray*}

df = 3, upper 5\% point is 7.815 so P-value exceeds 0.05.
\item (d) We have no evidence against H0, which can therefore stand, and so we
conclude that the proportion of girls is independent of family size.
(ii) (a) Models: no. of girls ~ binomial(n, 0.495) for n = 1, 2, 3, 4
\item (b) The model assumes that the “trials are independent” i.e. that the gender of each child is independent of that of all other children in the family.
\item We would interpret the lack of fit as evidence that the genders of children in a family are not independently determined.
\item In addition, the gender of the first child (or the genders of the first and second children) may have an influence on – or even decide - the family size.
Another possible reason is variation in $P(girl)$ across families.
\end{itemize}
%%%%%%%%%%%%%%%%%%%%%%%%%%%%%%%%%%%%%%%%%%%%%%%%%%%%%%%%%%%%%%%%%%%%%%%%%%%%%%%%%%%%%%%%%%%%%%%%
% The Examiners did not anticipate that so many candidates would be unable to construct the
% basic 2  4 contingency table appropriate for investigating the matter in question.

\end{document}
