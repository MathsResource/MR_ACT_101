\documentclass[a4paper,12pt]{article}

%%%%%%%%%%%%%%%%%%%%%%%%%%%%%%%%%%%%%%%%%%%%%%%%%%%%%%%%%%%%%%%%%%%%%%%%%%%%%%%%%%%%%%%%%%%%%%%%%%%%%%%%%%%%%%%%%%%%%%%%%%%%%%%%%%%%%%%%%%%%%%%%%%%%%%%%%%%%%%%%%%%%%%%%%%%%%%%%%%%%%%%%%%%%%%%%%%%%%%%%%%%%%%%%%%%%%%%%%%%%%%%%%%%%%%%%%%%%%%%%%%%%%%%%%%%%

\usepackage{eurosym}
\usepackage{vmargin}
\usepackage{amsmath}
\usepackage{graphics}
\usepackage{epsfig}
\usepackage{enumerate}
\usepackage{multicol}
\usepackage{subfigure}
\usepackage{fancyhdr}
\usepackage{listings}
\usepackage{framed}
\usepackage{graphicx}
\usepackage{amsmath}
\usepackage{chngpage}

%\usepackage{bigints}
\usepackage{vmargin}

% left top textwidth textheight headheight

% headsep footheight footskip

\setmargins{2.0cm}{2.5cm}{16 cm}{22cm}{0.5cm}{0cm}{1cm}{1cm}

\renewcommand{\baselinestretch}{1.3}

\setcounter{MaxMatrixCols}{10}

\begin{document}

\begin{enumerate}
\item 12 The following data give the invoiced amounts for work carried out on 12 jobs
performed by a plumber in private customers’ houses. The durations of the jobs are
also given.
\begin{verbatim}
duration x (hrs) 1 1 2 3 4 4 5 6 7 8 9 10
amount y (£) 45 65 80 95 100 125 145 180 180 210 330 240
60, 2 402, 1795, 2 343,725, 11,570 xi  xi  yi  yi  xi yi     
\end{verbatim}

The plumber claims to calculate his total charge for each job on the basis of a single call-out charge plus an hourly rate for the time spent working on the job.
\begin{enumerate}
    \item (i) (a) Draw a scatterplot of the data on graph paper and comment briefly on your plot.
\item (b) The equation of the fitted regression line of y on x is $y = 22.4 + 25.4x$ and the coefficient of determination is $R^2 = 87.8\%$ (you are not asked
to verify these results).
Draw the fitted line on your scatterplot. [5]
\item (ii) (a) Calculate the fitted regression line of invoiced amount on duration of job using only the 11 pairs of values remaining after excluding the
invoice for which x = 9 and y = 330.
(b) Calculate the coefficient of determination of the fit in (ii)(a) above.
(c) Add the second fitted line to your scatterplot, distinguishing it clearly from the first line you added (in part (i)(b) above).
\item (d) Comment on the effect of omitting the invoice for which x = 9 and y = 330.
(e) Carry out a test to establish whether or not the slope in the model fitted in (ii)(a) above is consistent with a rate of £25 per hour for work
carried out.
\end{enumerate}



%%%%%%%%%%%%%%%%%%%%%%%%%%%%%%%%%%%%%%%%%%%%%%%%%%%%%%%%%%%%%%%%%%%%%%%%%%%%%%%%%%%%%%%%%%%%%%%%
12 (i) (a)
The point (9,330) is an “outlier” from the general pattern, which is
strongly linear.
(b)
(ii) (a) Now Σx = 51, Σx2 = 321, Σy = 1,465, Σy2 = 234,825, Σxy = 8,600
Sxx = 84.545, Syy = 39,713.636, Sxy = 1807.727
 ˆ 1807.727 /84.545  21.382, ˆ 1465 /11\betaˆ(51/11)  34.048
Fitted line is y = 34.0 + 21.4x
(b) Coefficient of determination R2 = 1807.7272/(84.545  39713.636)
= 0.973 i.e. 97.3%
(or find these by first calculating the three sums of squares SSTOT
= 39,713.636 as above, SSREG = 1807.7272/84.545 = 38652.515, and
so SSRES = 1061.121)
0 1 2 3 4 5 6 7 8 9 10
400
300
200
100
0
duration
amount
0 1 2 3 4 5 6 7 8 9 10
0
100
200
300
400
duration
amount
%%%%%%%%%%%%%%%%%%%%%%%%%%%%%%%%%%%%%%%%%%%%%%%%%%%%%%%%%%%%%%%%%%%%%%%%%%%%%%%%%%%%%%%%%%%%%%%%%%%%%
(c)

(d) Removing the influence of the single point (9,330) results in a fitted line with a lower slope and a much better fit for the remaining data ($R^2$ has increased from 87.8\% to 97.3\%).


(e) The hourly rate corresponds to the slope in the model
H0: slope = 25 v H1: slope ≠ 25
Estimate of error variance = 1061.121/9
 standard error of slope estimate = [(1061.121/9)/84.545]1/2 = 1.181
t = (21.382 – 25)/1.181 = 	3.06 on 9 df
Upper tail probability is between 0.005 and 0.01 so P-value is between
0.01 and 0.02, so we have quite strong evidence against H0. We
conclude that the data are not consistent with an hourly rate of £25.
0 1 2 3 4 5 6 7 8 9 10
0
100
200
300
400
duration
amount
line (outlier omitted)
original line


\end{document}
