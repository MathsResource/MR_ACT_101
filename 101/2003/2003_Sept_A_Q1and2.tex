\documentclass[a4paper,12pt]{article}

%%%%%%%%%%%%%%%%%%%%%%%%%%%%%%%%%%%%%%%%%%%%%%%%%%%%%%%%%%%%%%%%%%%%%%%%%%%%%%%%%%%%%%%%%%%%%%%%%%%%%%%%%%%%%%%%%%%%%%%%%%%%%%%%%%%%%%%%%%%%%%%%%%%%%%%%%%%%%%%%%%%%%%%%%%%%%%%%%%%%%%%%%%%%%%%%%%%%%%%%%%%%%%%%%%%%%%%%%%%%%%%%%%%%%%%%%%%%%%%%%%%%%%%%%%%%

\usepackage{eurosym}
\usepackage{vmargin}
\usepackage{amsmath}
\usepackage{graphics}
\usepackage{epsfig}
\usepackage{enumerate}
\usepackage{multicol}
\usepackage{subfigure}
\usepackage{fancyhdr}
\usepackage{listings}
\usepackage{framed}
\usepackage{graphicx}
\usepackage{amsmath}
\usepackage{chngpage}

%\usepackage{bigints}
\usepackage{vmargin}

% left top textwidth textheight headheight

% headsep footheight footskip

\setmargins{2.0cm}{2.5cm}{16 cm}{22cm}{0.5cm}{0cm}{1cm}{1cm}

\renewcommand{\baselinestretch}{1.3}

\setcounter{MaxMatrixCols}{10}

\begin{document}



1 In a large corporation 50 new employees joined the company’s pension scheme
during the last year. It is assumed that each new employee has a probability of 0.40
of remaining in the scheme for at least 10 years, independently for each new
employee.
Calculate an approximate value for the probability that more than half of last year’s
50 new employees remain in the pension scheme for at least 10 years. 
\newpage
%%
1 

\begin{itemize}
    \item Let X be the number remaining in the scheme for at least 10 years.
X ~ binomial(50, 0.4)
\item So, approximately $X \sim N(20, 12)$
\item We require $P(X > 25)$.
\item Using a continuity correction,
\end{itemize}

\begin{eqnarray*}
P(X > 25.5) &=& P ( Z > \frac{25.5 − 20}{\sqrt{12}}\\
&=& P ( Z > 1.59) \\
&=& 1 − 0.944 \\
&=& 0.056\\
\end{eqnarray*}
%%%%%%%%%%%%%%%%%%%%%%%%%%%%%%%%%%%%%%%%%%%%%%%%%%%%%%%%%%%%%%%%%%%%%%%%%%%%%%%%%%%%%%%%%%%%%%%%%%
\newpage 2 Let $(X_1, X_2, … , X_9)$ be a random sample from a N(0,2) distribution. Let X and
S2 denote the sample mean and variance respectively.
Find the approximate value of P X  S  by referring to an appropriate statistical
table. 




%%%%%%%%%%%%%%%%%%%%%%%%%%%%%%%%%%%%%%%%%%%%%%%%%%%%%%%%%%%%%%%%%%%%%%%%%%
Page 3
%%%%%%%%%%%%%%%%%%%%%%%%%%%%%%%%%%%%%%%%%%%%%%%%%%%%%%%%%%%%%%%%%%%%%%%%%%%%%%%%
%% Q2

\[ P(\bar{X} > S) = P( \frac{\bar{3X}}{S} > 3) = P(t_8 > 3)\] 
%%%%%%%%%%%%%
S
⎛ ⎞
> = ⎜ > ⎟ = >
⎝ ⎠
which is between 0.005 and 0.01.

\end{document}
