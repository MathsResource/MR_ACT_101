\documentclass[a4paper,12pt]{article}

%%%%%%%%%%%%%%%%%%%%%%%%%%%%%%%%%%%%%%%%%%%%%%%%%%%%%%%%%%%%%%%%%%%%%%%%%%%%%%%%%%%%%%%%%%%%%%%%%%%%%%%%%%%%%%%%%%%%%%%%%%%%%%%%%%%%%%%%%%%%%%%%%%%%%%%%%%%%%%%%%%%%%%%%%%%%%%%%%%%%%%%%%%%%%%%%%%%%%%%%%%%%%%%%%%%%%%%%%%%%%%%%%%%%%%%%%%%%%%%%%%%%%%%%%%%%

\usepackage{eurosym}
\usepackage{vmargin}
\usepackage{amsmath}
\usepackage{graphics}
\usepackage{epsfig}
\usepackage{enumerate}
\usepackage{multicol}
\usepackage{subfigure}
\usepackage{fancyhdr}
\usepackage{listings}
\usepackage{framed}
\usepackage{graphicx}
\usepackage{amsmath}
\usepackage{chngpage}

%\usepackage{bigints}
\usepackage{vmargin}

% left top textwidth textheight headheight

% headsep footheight footskip

\setmargins{2.0cm}{2.5cm}{16 cm}{22cm}{0.5cm}{0cm}{1cm}{1cm}

\renewcommand{\baselinestretch}{1.3}

\setcounter{MaxMatrixCols}{10}

\begin{document}
%%%%%%%%%%%%%%%%%%%%%%%%%%%%%%%%%%%%%%%%%%%%%%%%%%%%%%%
\begin{enumerate}
\item  An engineer is interested in estimating the probability that a particular
electrical component will last at least 12 hours before failing. In order to do
this, a random sample of n components is tested to destruction and their
failure times, , ,..., , 1 2 n x x x are recorded. The engineer models failure times by
assuming that they come from a distribution with distribution function, F, and
probability density function, f, given below.
\[F(x) = 1 − \frac{1}{(1+x)^{\alpha − 1}}, \qquad \mbox{and} \qquad  f(x) = \frac{\alpha − 1}{( 1 + x )^\alpha}\]
$\alpha > 1, x > 0$.
(i) Determine aˆ , the maximum likelihood estimator of a, and, assuming n
is large, use asymptotic theory to show that an approximate 95%
confidence interval for a is given by
\[ˆ 1 ˆ 1.96
n
a -
a ± . \]
(ii) A sample of size n = 80 leads to a maximum likelihood estimate of a of
1.56. Use this figure to
%------------------------%
\begin{enumerate}[(a)]
\item estimate the probability a component will fail before 12 hours,
\item determine an approximate upper 95\% one-sided confidence
interval for a, and
\item hence determine an approximate 95\% one-sided confidence
interval which provides an upper bound for the probability in
part (ii) above. 
\end{enumerate}
%------------------------%
(iii) Sixty-one of the eighty components tested in part (ii) failed before 12
hours, so a second engineer estimates the failure probability by
61/80 = 0.7625, and constructs an upper 95\% confidence interval based
on the binomial distribution.
\begin{enumerate}[(a)]
\item Construct this interval, and
\item comment on the advantages and disadvantages of this method
when compared to the method of part (ii). 
\end{enumerate}
\end{enumerate}
\newpage
%%%%%%%%%%%%%%%%%%%%%%%%%%%%%%%%
\begin{itemize}
\item (i) Start by writing down the likelihood function
\[L(\lambda) =
( 1)
(1 )
n
i x\]

 
 
.
\begin{itemize}
    \item The log-likelihood function is
\[l(\lambda) = log L(\lambda) = n log(\lambda -1)  \lambda log(1 + xi).\]
\item Differentiating gives

\[l=1n \log(1 + xi).\]
\item It is easy to see that the log-likelihood has only one turning point, so this
can be found by equating the derivative to zero. This gives that the
maximum likelihood estimate is
% \[ˆ
% = 1 + .
% log(1 ) i
% n
%   x\]
\item The second derivative is
2
2
 l

= 2 .
( 1)
n
 
\item So an approximate 95% confidence interval for \lambda is
ˆ 1
ˆ 1.96 .
n
 
\end{itemize}

 
%%%%%%%%%%%%%%%%%%%%%%%%%%%%%%%%%%%%%%%%%%%%%%%%%%%%%%%%%%%%%%%%%%%%%

\item (ii) (a) The probability a component will last less than 12 hours before
failing can be estimated by the point estimate
\[1  ˆ 1
1
13
= 1  0.56
1
13
= 0.762.\]
(b) An approximate 95\% confidence upper bound for  is
\[ ˆ 1 ˆ 1.645 n  = 1.56 + 1.645 0.56 80 = 1.663.\]
\item (c) This gives an upper bound for the failure probability of
0.663
1
1
13
 = 0.817.
\item (iii) (a) The endpoint of the binomial confidence interval is
0.7625 + 1.645 
0.7625 (1 0.7625)
80
 
= 0.841.
\item (b) There is no single answer to this part. The main points are:
The second engineer’s (binomial) method will be valid and doesn’t
need any parametric assumptions about the data.
\item The first engineer’s method needs the data to follow the
distribution specified, but in that case it will be more powerful
than the binomial method, which does not use the data efficiently.
\item This is illustrated in this data as the two point estimates are very
close, but the first engineer’s confidence interval is narrower.


\end{itemize}
\end{document}
