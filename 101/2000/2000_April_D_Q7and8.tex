\documentclass[a4paper,12pt]{article}

%%%%%%%%%%%%%%%%%%%%%%%%%%%%%%%%%%%%%%%%%%%%%%%%%%%%%%%%%%%%%%%%%%%%%%%%%%%%%%%%%%%%%%%%%%%%%%%%%%%%%%%%%%%%%%%%%%%%%%%%%%%%%%%%%%%%%%%%%%%%%%%%%%%%%%%%%%%%%%%%%%%%%%%%%%%%%%%%%%%%%%%%%%%%%%%%%%%%%%%%%%%%%%%%%%%%%%%%%%%%%%%%%%%%%%%%%%%%%%%%%%%%%%%%%%%%

\usepackage{eurosym}
\usepackage{vmargin}
\usepackage{amsmath}
\usepackage{graphics}
\usepackage{epsfig}
\usepackage{enumerate}
\usepackage{multicol}
\usepackage{subfigure}
\usepackage{fancyhdr}
\usepackage{listings}
\usepackage{framed}
\usepackage{graphicx}
\usepackage{amsmath}
\usepackage{chngpage}

%\usepackage{bigints}
\usepackage{vmargin}

% left top textwidth textheight headheight

% headsep footheight footskip

\setmargins{2.0cm}{2.5cm}{16 cm}{22cm}{0.5cm}{0cm}{1cm}{1cm}

\renewcommand{\baselinestretch}{1.3}

\setcounter{MaxMatrixCols}{10}

\begin{document}

% \item 7 ( ) ( , ) ( | ) ( ) ( | ) ( ) Y Y E X xf x y dxdy xf x y dx f y dy E X Y y f y dy

Suppose that $X$ and $Y$ are continuous random variables.
Prove that $E(X )  =E(X|Y y)fY( y)dy$ .
%%%%%
Solution: 
\begin{eqnarray*}
E(X)&=&  \int^{infty}_{-\infty} \int^{infty}_{-\infty} x f(x,y) dx dy\\
&=&  \int^{infty}_{-\infty} \left( \int^{infty}_{-\infty} x f(x|y) dx \right) f_Y(y) dy\\
&=&  \int^{infty}_{-\infty} E(X|Y=y) f(x|y) dx dy\\
\end{eqnarray*}
%%%%%%%%%%%%%%%%%%%%%%%%%%%%%%%%%%%%%%%%%%%%%%%%%%%%%%%

%%% Question 7

\begin{eqnarray*} 
E(X) &=& \int^{infty}_{-infty} \int^{infty}_{-infty}x f(x,y) dx dy \\
&=& \int^{infty}_{-infty} \left[ \int^{infty}_{-infty}x f(x|y) dx \right] f_Y{y} dy \\
&=& \int^{infty}_{-infty} E(X|Y=y) f_Y{y} dy \\
\end{eqnarray*}

%%%%%%%%%%%%%%%%%%%%%%%%%%%%%%%%%%%%%%%%%%%%%%%%%%%%%%%
\newpage 
%% 8 

\noindent A device contains an electronic component which has a lifetime modelled by a distribution with mean 3.6 hours and standard deviation 2.6 hours. On failure a new component is automatically and instantaneously inserted as a replacement.


Consider the operation of the device with 100 such components used one after
the other. Determine the approximate probability that the resulting total
lifetime of the device will be greater than 400 hours. 
%%%%%%%%%%%%%%%%%%%%%%%%%%%%%%%%%%%%%%%%%%%

%%% Question 8 

\begin{eqnarray*}
P(T > 400) &=& P(Z \frac{400-360}{26} \\
&=& P(Z > 1.54)\\
&=& 1 - \Phi(1.54) \\
&=& 1 - 0.93822  \mbox{ from tables }\\
&=& 0.062 \\
\end{eqnarray*}



%%%%%%%%%%%%%%%%%%%%%%%%%%%%

$T = \sum^{100}_{i=1} X_i$ has mean 100(3.6) = 360 hours
and s.d. 100(2.6) = 26 hours.
Central limit theorem 
 T is approximately normal as n is large.
Therefore
\begin{eqnarray*}
P(T > 400) &=& P \left(Z > \frac{400-360}{26} \right) \\
&=& P \left(Z >1.54 \right) \\
&=& 1 - 0.93822 \\
&=& 0.062
\end{eqnarray*}
