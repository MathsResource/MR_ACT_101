\documentclass[a4paper,12pt]{article}

%%%%%%%%%%%%%%%%%%%%%%%%%%%%%%%%%%%%%%%%%%%%%%%%%%%%%%%%%%%%%%%%%%%%%%%%%%%%%%%%%%%%%%%%%%%%%%%%%%%%%%%%%%%%%%%%%%%%%%%%%%%%%%%%%%%%%%%%%%%%%%%%%%%%%%%%%%%%%%%%%%%%%%%%%%%%%%%%%%%%%%%%%%%%%%%%%%%%%%%%%%%%%%%%%%%%%%%%%%%%%%%%%%%%%%%%%%%%%%%%%%%%%%%%%%%%

\usepackage{eurosym}
\usepackage{vmargin}
\usepackage{amsmath}
\usepackage{graphics}
\usepackage{epsfig}
\usepackage{enumerate}
\usepackage{multicol}
\usepackage{subfigure}
\usepackage{fancyhdr}
\usepackage{listings}
\usepackage{framed}
\usepackage{graphicx}
\usepackage{amsmath}
\usepackage{chngpage}

%\usepackage{bigints}
\usepackage{vmargin}

% left top textwidth textheight headheight

% headsep footheight footskip

\setmargins{2.0cm}{2.5cm}{16 cm}{22cm}{0.5cm}{0cm}{1cm}{1cm}

\renewcommand{\baselinestretch}{1.3}

\setcounter{MaxMatrixCols}{10}

\begin{document}
%%%%%%%%%%%%%%%%%%%%%%%%%%%%%%%%%%%%%%%%%%%%%%%%%%%%%%%
\begin{enumerate}
\item The discrete random variable X has the following probability function:
\[P(X = i) = 0.2 + ai : i = \{-2, -1, 0, 1, 2.\}\]

\begin{enumerate}[(a)]
\item (i) State the possible values that a can take. 
\item (ii) Given a random sample $x_1 , x_2 , \ldots , x_n$ from this distribution, determine
the method of moments estimate of a and show that this can result in
inadmissible estimates (i.e. estimates outside the range of possible
values of a). 
\end{enumerate}
 %%%%%%%%%%%%%%%%%%%%%%%%%%%%%%%%%%%%%%%%%%%%%%%%%%%%%%%%%%%%%%%%%%%%%
%%- Solution to Q 9 


\begin{center}
\begin{tabular}{|c|r|}
i & P(X = i) = 0.2 + ai  \\ \hline
−2 &   0.2 -2a\\ \hline
−1 &   0.2 -a\\ \hline
0  &  0.2 \\ \hline
1  &  0.2  + a\\ \hline
2  &  0.2 + 2a\\ \hline \hline
$\sum$ & 1 \\ \hline
\end{tabular}
\end{center}
\begin{itemize} 
\item (i) This will be a probability function provided the specified probabilities are
non-negative; i.e. if and only if $-0.1 < a < 0.1$.
\item (ii) The method of moments estimate of a is obtained by equating the sample
mean to the population mean. 

\end{itemize}

%%%%%%%%%%%%%%%%%%%%%%%%%%%%%%%%%%%%%%%%%%%

%%% Question 9

\begin{eqnarray*} 
\mu &=& \sum^{2}_{-2} i(0.2 + ai) \\
&=& a \sum^{2}_{-2} i^2 \\
&=& 10a \\
\end{eqnarray*}


%%%%%%%%%%%%%%%%%%%%%%%%%%%%%%%%%%%%%%%%%%%%%%%%%%%%%%%%%%%%%%%%%%%
To do this note that
 = 2
2
 i(0.2 + ai) = a 2
2 
 i2 = 10a.
Thus, the method of moments estimate is 10
X .

As X can take any value between -2 and +2, the method of moments
estimate can take any value between –0.2 and +0.2. Thus it can be
outside the range (-0.1, 0.1).
\newpage
%%%%%%%%%%%%%%%%%%%%%%%%%%%%%%%%%%%%%%%%%%%%%%%%%%%%%%%
\item %%% 2000 April Q 10

Under a particular model for the evolution of the size of a population over
time, the probability generating function of $X_t$ , the size at time $t$, 
$G_X_t ( s )$ , is
given by

\[ G_X_t ( s ) =  \frac{s + \lambda t (1 − s )}{1 + \lambda t (1 − s )} \]


where $G_X_t ( s ) = E ( s ^{X_t} )$ and $\lambda > 0$.


If the population dies out, it remains in this extinct state for ever.
\begin{enumerate}[(a)]
    \item Show that the expected size of the population at any time t is 1. 
\item  Show that the probability that the population has become extinct by
time t is given by lt / (1 + lt). 
\item Comment briefly on the future prospects for the population. [1]
\end{enumerate}

%%%%%%%%%%%%%%%%%%%%%%%%%%%%%%%%%%%%%%%%%%%%%%%%%%%
10 

\begin{itemize}
    \item (i)
t
\[X G^{\prime} (s) = (1 - \lambda t){1 + \lambda t(1 - s)}1 -{s + \lambda t(1 - s)}{-\lambda t}{1 + \lambda t(1 - s)}2
  =
t\]
\[X G^{\prime} (1) = 1 - \lambda t + \lambda t = 1\]
\item (ii) P(extinct by time t) = P(population size at time t is zero)
= (0)
t
X G = \lambda t / (1 + \lambda t)
\item (iii) P(extinct by time t) 
 1 as t 
 , so eventual extinction is certain.
\end{itemize}
%%%%%%%%%%%%%%%%%%%%%%%%%%%%%%%%%%%%%%%%%%%%%%%%%%%%%%%%%%%%%%


\end{document}
