%-ACT 101 2000 April Q15

\documentclass[a4paper,12pt]{article}

%%%%%%%%%%%%%%%%%%%%%%%%%%%%%%%%%%%%%%%%%%%%%%%%%%%%%%%%%%%%%%%%%%%%%%%%%%%%%%%%%%%%%%%%%%%%%%%%%%%%%%%%%%%%%%%%%%%%%%%%%%%%%%%%%%%%%%%%%%%%%%%%%%%%%%%%%%%%%%%%%%%%%%%%%%%%%%%%%%%%%%%%%%%%%%%%%%%%%%%%%%%%%%%%%%%%%%%%%%%%%%%%%%%%%%%%%%%%%%%%%%%%%%%%%%%%

\usepackage{eurosym}
\usepackage{vmargin}
\usepackage{amsmath}
\usepackage{graphics}
\usepackage{epsfig}
\usepackage{enumerate}
\usepackage{multicol}
\usepackage{subfigure}
\usepackage{fancyhdr}
\usepackage{listings}
\usepackage{framed}
\usepackage{graphicx}
\usepackage{amsmath}
\usepackage{chngpage}

%\usepackage{bigints}
\usepackage{vmargin}

% left top textwidth textheight headheight

% headsep footheight footskip

\setmargins{2.0cm}{2.5cm}{16 cm}{22cm}{0.5cm}{0cm}{1cm}{1cm}

\renewcommand{\baselinestretch}{1.3}

\setcounter{MaxMatrixCols}{10}

\begin{document}
\noindent The table below contains measurements on the strengths of beams. The width
and height of each beam was fixed but the lengths varied. Data are available
on the length (cm) and strength (Newtons) of each beam.
Length l x = log l Strength
p

\begin{center}
\begin{tabular}{c|c|c|c|c|c|}
y    &  log  &  p  &  Fitted  &  value  &  Residual\\ \hline
7  &  1.946  &  11775  &  9.374  &  9.379  &  -0.005\\ \hline
7  &  1.946  &  11275  &  9.330  &  9.379  &  -0.049\\ \hline
9  &  2.197  &  8400  &  9.036  &  9.055  &  -0.019 \\ \hline
9  &  2.197  &  8200  &  9.012  &  9.055  &  -0.043 \\ \hline
12  &  2.485  &  6100  &  8.716  &  8.684  &  0.032 \\ \hline
12  &  2.485  &  6050  &  8.708  &  8.684  &  0.024 \\ \hline
14  &  2.639  &  5200  &  8.556  &  8.486  &  0.070 \\ \hline
18  &  2.890  &  3750  &  8.230  &  8.162  &  0.068 \\ \hline
18  &  2.890  &  3650  &  8.202  &  8.162  &  0.040 \\ \hline
20  &  2.996  &  3275  &  8.094  &  8.026  &  0.068 \\ \hline
20  &  2.996  &  3175  &  8.063  &  8.026  &  0.037 \\ \hline
24  &  3.178  &  2200  &  7.696  &  7.791  &  -0.095 \\ \hline
24  &  3.178  &  2125  &  7.662  &  7.791  &  -0.129 \\ \hline
\end{tabular}
\end{center}
%%%%%%%%%%%%%%%%%%%%%%%%%%%%%%%%%
Sx = 34.023, Sx2 = 91.3978, Sy = 110.679, Sxy = 286.6299
It is thought that P and L satisfy the law P = k/L where k is a constant, so
\[\log P = \log k - \log L, \mbox{ i.e.} Y = \log k - X.\]
A graph of log P against log L is included.
The simple linear regression model $y = a + bx$ has been fitted to the data, and
the fitted values and residuals are recorded in the table above.
\begin{enumerate}[(a)]
\item Use the data summaries above to calculate the least squares estimates
% $ a of a and ˆb of b . 
\item Assuming the usual normal linear regression model

\begin{enumerate}[(i)]
\item estimate the error variance $s^2$,
\item calculate a 95\% confidence interval for b, and
\item discuss briefly whether the data are consistent with the
relationship P = k L . 
\end{enumerate}

\item Plot the residuals of the model against X and comment on the
information contained in the plot. 
[Total 13]
7.5
7.7
7.9
8.1
8.3
8.5
8.7
8.9
9.1
9.3
9.5
1.9 2.1 2.3 2.5 2.7 2.9 3.1 3.3
x = log l
y = log p
\end{enumerate}
\newpage

16 (i) Sxx = 91.3978 
2 34.023
13
= 2.354, and
Sxy = 286.6299 
110.679 34.023
13

= 3.0341.
The least squares estimates are \[ˆ =
\frac{3.0341}{2.354}= 1.289\] and
ˆ
= ˆ
y  x = 11.887.
\begin{itemize}
    \item (ii) (a) The sum of squares of the residuals is 0.049019, so 2 is estimated
by the residual mean square 0.049019
11 = 0.004456.
\item (b) The estimated variance of ˆ is 0.004456
2.3544 = 0.00189.
%%%%%%%%%%%%%%%%%%%%%%%%%%%%%%%%%%%%%%%%%%%%%%%%%%%%%%%%%%%%%%%%%%%%%
Page 9
This leads to the following 95\% confidence interval for :
1.289  2.201 0.00189 = (1.384, 1.193).
\item (c) If the relationship P = k/L is correct the slope parameter of the
regression line should be –1. As the upper end of the interval is
less than –1, the data do not support the suggested relationship.
\item (iii)
The residuals show that the line underfits in the centre. A straight line
doesn’t fit the data very well.
\end{itemize}
-0.15
-0.10
-0.05
0.00
0.05
0.10
1.9 2.1 2.3 2.5 2.7 2.9 3.1 3.3
log L
\end{document}
