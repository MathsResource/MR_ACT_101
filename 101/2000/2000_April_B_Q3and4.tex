%-ACT 101 2000 April Q3 and 3


\documentclass[a4paper,12pt]{article}

%%%%%%%%%%%%%%%%%%%%%%%%%%%%%%%%%%%%%%%%%%%%%%%%%%%%%%%%%%%%%%%%%%%%%%%%%%%%%%%%%%%%%%%%%%%%%%%%%%%%%%%%%%%%%%%%%%%%%%%%%%%%%%%%%%%%%%%%%%%%%%%%%%%%%%%%%%%%%%%%%%%%%%%%%%%%%%%%%%%%%%%%%%%%%%%%%%%%%%%%%%%%%%%%%%%%%%%%%%%%%%%%%%%%%%%%%%%%%%%%%%%%%%%%%%%%

\usepackage{eurosym}
\usepackage{vmargin}
\usepackage{amsmath}
\usepackage{graphics}
\usepackage{epsfig}
\usepackage{enumerate}
\usepackage{multicol}
\usepackage{subfigure}
\usepackage{fancyhdr}
\usepackage{listings}
\usepackage{framed}
\usepackage{graphicx}
\usepackage{amsmath}
\usepackage{chngpage}

%\usepackage{bigints}
\usepackage{vmargin}

% left top textwidth textheight headheight

% headsep footheight footskip

\setmargins{2.0cm}{2.5cm}{16 cm}{22cm}{0.5cm}{0cm}{1cm}{1cm}

\renewcommand{\baselinestretch}{1.3}

\setcounter{MaxMatrixCols}{10}

\begin{document}
\large
\noindent In an investigation into the proportion ($\theta$) of lapses in the first year of a
certain type of policy, the uncertainty about q is modelled by taking $\theta$ to have a
beta distribution with parameters a = 1 and b = 9, that is, with density
\[f(\theta) = 9(1 - \theta)^8 : 0 < \theta < 1.\]
Using this distribution, calculate the probability that $\theta$ exceeds 0.2. 


{
\large
\begin{eqnarray*}
P(\theta > 0.2) &=& \int^{1}_{0.2} 9(1- \theta)^8) d \theta \\
& & \\
&=& \left[ -(1-\theta)^9  \right]^{1}_{0.2}\\
& & \\
&=& 0 + (1 - 0.2)^9 \\ 
&=& 0.8^9 \\
&=& 0.13\\
\end{eqnarray*}
}

%%%%%%%%%%%%%%%%%%%%%%%%%%%%%%%%%%%%%%%%%%%%%%%%%%%%%%%
\newpage
Consider the following three probability statements concerning an F variable
with 6 and 12 degrees of freedom.
\begin{enumerate}
\item $P(F6,12 > 0.250) = 0.95$
\item $P(F6,12 < 4.821) = 0.99$
\item $P(F6,12 < 0.130) = 0.01$
\end{enumerate}

State, with reasons, whether each of these statements is true. 


4 
\begin{itemize}
\item For (a) to be true, 0.250 must be lower 5\% pt of F6,12 i.e. reciprocal of upper 5\% pt
of F12,6 which is
1
4.000
= 0.250  true.
    \item For (b) to be true, 4.821 must be upper 1\% pt of F6,12 which is 4.821 true.
\item For (c) to be true, 0.130 must be lower 1\% pt of F6,12 i.e. reciprocal of upper 1\% pt
of F12,6 which is
1
7.718
= 0.130 true.

\end{itemize}


\end{document}
