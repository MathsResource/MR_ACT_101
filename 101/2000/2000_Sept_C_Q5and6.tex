\documentclass[a4paper,12pt]{article}

%%%%%%%%%%%%%%%%%%%%%%%%%%%%%%%%%%%%%%%%%%%%%%%%%%%%%%%%%%%%%%%%%%%%%%%%%%%%%%%%%%%%%%%%%%%%%%%%%%%%%%%%%%%%%%%%%%%%%%%%%%%%%%%%%%%%%%%%%%%%%%%%%%%%%%%%%%%%%%%%%%%%%%%%%%%%%%%%%%%%%%%%%%%%%%%%%%%%%%%%%%%%%%%%%%%%%%%%%%%%%%%%%%%%%%%%%%%%%%%%%%%%%%%%%%%%

\usepackage{eurosym}
\usepackage{vmargin}
\usepackage{amsmath}
\usepackage{graphics}
\usepackage{epsfig}
\usepackage{enumerate}
\usepackage{multicol}
\usepackage{subfigure}
\usepackage{fancyhdr}
\usepackage{listings}
\usepackage{framed}
\usepackage{graphicx}
\usepackage{amsmath}
\usepackage{chngpage}

%\usepackage{bigints}
\usepackage{vmargin}

% left top textwidth textheight headheight

% headsep footheight footskip

\setmargins{2.0cm}{2.5cm}{16 cm}{22cm}{0.5cm}{0cm}{1cm}{1cm}

\renewcommand{\baselinestretch}{1.3}

\setcounter{MaxMatrixCols}{10}

\begin{document}
% 18 September 2000 (pm)
% Subject 101 — Statistical Modelling
% Faculty of Actuaries Institute of Actuaries
% EXAMINATIONS
%%%%%%%%%%%%%%%%%%%%%%%%%%%%%%%%%%%%%%%%%%%%%%%%%%%%%%%%%%%%%%%5 The number of claims which arise under a policy of a particular type in a year is
to be modelled as a Poisson($\lambda$) random variable. A random sample of 500 such policies gave rise to a total of 84 claims in 1999.
Calculate a 95\% confidence interval for $\lambda$. 
%%%%%%%%%%%%%%%%%%%%%%%%%%%%%%%
\newpage
%%- Question 6

\noindent Suppose that the linear regression model
\[Y = \alpha + \beta x + e\]
is fitted to data $\{(y_i , x_i) : i = 1, 2, \ldots , n\}$, where y is the salary (£) of a company
manager and x (years) is the number of years of relevant experience of that
manager.
State the units of measurement (if any) of
\begin{enumerate}
    \item  $\hat{\alpha}$ , the estimate of $\alpha$ ,
\item $\hat{\beta}$ , the estimate of $\beta$ ,
\item $R^2$ , the coefficient of determination of the fit. 
\end{enumerate}


%%%%%%%%%%%%%%%%%%%%%%%%%%%%%%%%%%%%%%%%%%%%%%%%%%%%%%%%%%%%%%%%%%%%%%%%%%%%%%%%%%55
6 (a) £
(b) £/year i.e. £ × year−1
(c) no units


\newpage


%%%%%%%%%%%%%%%%%%%%%%%%%%%%%%%%%%%%%%%%%%%%%%%%%%%%%%%%%%%%%%%%%%%%%%%%%%%%%%%%%%%%



\end{document}
