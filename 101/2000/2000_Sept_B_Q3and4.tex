\documentclass[a4paper,12pt]{article}

%%%%%%%%%%%%%%%%%%%%%%%%%%%%%%%%%%%%%%%%%%%%%%%%%%%%%%%%%%%%%%%%%%%%%%%%%%%%%%%%%%%%%%%%%%%%%%%%%%%%%%%%%%%%%%%%%%%%%%%%%%%%%%%%%%%%%%%%%%%%%%%%%%%%%%%%%%%%%%%%%%%%%%%%%%%%%%%%%%%%%%%%%%%%%%%%%%%%%%%%%%%%%%%%%%%%%%%%%%%%%%%%%%%%%%%%%%%%%%%%%%%%%%%%%%%%

\usepackage{eurosym}
\usepackage{vmargin}
\usepackage{amsmath}
\usepackage{graphics}
\usepackage{epsfig}
\usepackage{enumerate}
\usepackage{multicol}
\usepackage{subfigure}
\usepackage{fancyhdr}
\usepackage{listings}
\usepackage{framed}
\usepackage{graphicx}
\usepackage{amsmath}
\usepackage{chngpage}

%\usepackage{bigints}
\usepackage{vmargin}

% left top textwidth textheight headheight

% headsep footheight footskip

\setmargins{2.0cm}{2.5cm}{16 cm}{22cm}{0.5cm}{0cm}{1cm}{1cm}

\renewcommand{\baselinestretch}{1.3}

\setcounter{MaxMatrixCols}{10}

\begin{document}
\begin{enumerate}

\item 3 The number of claims arising in a period of one month from a group of policies
can be modelled by a Poisson distribution with mean 24.
Determine the probability that fewer than 20 claims arise in a particular month.



%%%%%%%%%%%%%%%%%%%%%%%%%%%%%%%%%%%%%%%%%%%%%%%%%%%%%%%%%
\newpage

3 Can get answer directly from Green tables:
\[P(X < 20) = P(X \leq 19) = 0.18026 = 0.18\]
OR: use normal approximation with continuity correction.
$X \sim  N(24, \sqrt{24}^2 )$

$P(X < 20) \rightarrow P(X < 19.5)$

\begin{eqnarray*}
P(X < 19.5) &=& P \left( Z < \frac{19.5-24}{\sqrt{24}} \right) \\ 
&=& P\left(Z \leq -0.92 \right) \\
&=&  1 − 0.82 \\
&=&  0.18.\\
\end{eqnarray*}
(Note: without continuity correction answer is 0.15)



%%%%%%%%%%%%%%%%%%%%%%%%%%%%%%%%%%%%%%%%%%%%%%%%%%%%%%%
\newpage


4 Suppose that a random sample of nine observations is taken from a normal
distribution with mean $\mu = 0$. \begin{itemize}
    \item Let $X$ and $S^2$ denote the sample mean and
variance respectively.
\item Determine (to 2 decimal places) the probability that the value of X exceeds that of $S$, i.e. determine $P( X > S)$ . 
\end{itemize}

4 $\mu = 0$, n = 9 so $X / (S / 3) \sim t_8$ so $P( X > S) = P(t_8 > 3) $,
which, from tables, is just less than 0.01
(actually 0.0085, but not needed for the marks)
%%%%%%%%%%%%%%%%%%%%%%%%%%%%%%%%%%%%%%%%%%%%%%%%%%%%%%%%%
Page 3
5 x
x
\pm 196
500
. i.e. $0.168 \pm (1.96 \times 0.01833) \mbox{ i.e. } 0.168 \pm 0.036$

\end{document}
