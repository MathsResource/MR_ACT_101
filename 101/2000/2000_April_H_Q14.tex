\documentclass[a4paper,12pt]{article}

%%%%%%%%%%%%%%%%%%%%%%%%%%%%%%%%%%%%%%%%%%%%%%%%%%%%%%%%%%%%%%%%%%%%%%%%%%%%%%%%%%%%%%%%%%%%%%%%%%%%%%%%%%%%%%%%%%%%%%%%%%%%%%%%%%%%%%%%%%%%%%%%%%%%%%%%%%%%%%%%%%%%%%%%%%%%%%%%%%%%%%%%%%%%%%%%%%%%%%%%%%%%%%%%%%%%%%%%%%%%%%%%%%%%%%%%%%%%%%%%%%%%%%%%%%%%

\usepackage{eurosym}
\usepackage{vmargin}
\usepackage{amsmath}
\usepackage{graphics}
\usepackage{epsfig}
\usepackage{enumerate}
\usepackage{multicol}
\usepackage{subfigure}
\usepackage{fancyhdr}
\usepackage{listings}
\usepackage{framed}
\usepackage{graphicx}
\usepackage{amsmath}
\usepackage{chngpage}

%\usepackage{bigints}
\usepackage{vmargin}

% left top textwidth textheight headheight

% headsep footheight footskip

\setmargins{2.0cm}{2.5cm}{16 cm}{22cm}{0.5cm}{0cm}{1cm}{1cm}

\renewcommand{\baselinestretch}{1.3}

\setcounter{MaxMatrixCols}{10}

\begin{document}
An insurance company issues house buildings policies for houses of similar
size in four different post-code regions A, B, C and D.
\begin{itemize}
\item  An insurance agent takes independent random samples of 10 house
buildings policies for houses of similar size in regions A and B.
\item The
annual premiums (\$) were as follows:

\begin{center}
 \begin{tabular}{c|c|c}\hline
Region A: & 229 241 270 256 241 247 261 243 272 219
& (Sx = 2,479 ; Sx2 = 617,163)\\  \hline
Region B: & 261 269 284 268 249 255 237 270 269 257
& (Sx = 2,619 ; Sx2 = 687,467)\\  \hline
\end{tabular}   
\end{center}

\end{itemize}
\begin{enumerate}[(a)]
\item Perform a two-sample t-test at the 5\% level to compare the
premiums for these two regions.
\item Present the data in a simple diagram and hence comment briefly
on the validity of the assumptions required for the above t-test.
\item Calculate a 95\% confidence interval for the underlying common
standard deviation s of such premiums. 
\end{enumerate}
%------------------------%
(ii) The agent takes further independent random samples of 10 such
policies from the other two regions C and D. The annual premiums
were as follows:
\begin{center}
 \begin{tabular}{c|c|c}\hline
Region C: & 253 247 244 245 221 229 245 256 232 269
& (Sx = 2,441 ; Sx2 = 597,607)\\  \hline
Region D:&  279 268 290 245 281 262 287 257 262 246
& (Sx = 2,677 ; Sx2 = 718,973)\\  \hline
\end{tabular}   
\end{center}


\begin{enumerate}[(a)]
\item Perform a one-way analysis of variance at the 5\% level to
compare the premiums for all four regions.
\item Present the new data in a simple diagram and hence comment
briefly on the validity of the assumptions required for the
analysis of variance.
\item Calculate a 95\% confidence interval for the underlying common
standard deviation $s$ of such premiums in the four regions. 
(iii) Comment briefly on your two confidence intervals in (i) and (ii)
above. 
\end{enumerate}




%%%%%%%%%%%%%%%%%%%%%%%%%%%%%%%%%%%%%%%%%%%%%%%%%%%%%%%
\newpage

\end{enumerate}
%%%%%%%%%%%%%%%%%%%%%%%%%%%%%%%%%%%%%%%%%%%%%%%%%%%%%%%
\newpage


14 (i) (a) A
x = 247.9, 2
A
s =
2 1 2479
617163
9 10
%%%%%%%%%%
= 290.99
B
x = 261.9, 2
B
s =
2 1 2619
687467
9 10
%%%%%%%%%%%%%%%%%%%%%%%%%%%%%%%%%%%%
= 172.32
2
p
s =
290.99 172.32
2
%%%%%%%%%%%%%%%%%%%%%%%%%%%%
= 231.66
Obs t =
247.9 261.9
1 1
231.66
10 10
%%%%%%%%%%%%%%%%%%%%%%%%%%%%%%%%%%%%
= -2.06
For two-sided test, t18(2.5%) = 2.101.
As 2.06 < 2.101, there is no evidence at the 5% level of a difference
between regions A and B.
(b)
Normality — OK in both cases.
Equal variances — OK.
A
B
210 220 230 240 250 260 270 280 290

%%%%%%%%%%%%%%%%%%%%%%%%%%%%%%%%%%%%%%%%%%%%%%%%%%%%%%%%%%%%%%%%%%%%%
Page 6
(c)
2
2
2 18
18
~ p
S

\begin{itemize}
    \item 
 95\% CI for 2 is
2 2
2 2
0.975,18 0.025,18
18 18
, p p
 S S 
 
=
18(231.66) 18(231.66)
,
31.53 8.231


= (132.25, 506.6)
\item  95% CI for  is (11.5, 22.5)
\item (ii) (a) x = 10216, x2 = 2621210
SST = 2621210 
2 10216
40
= 12043.6
SSB =
2 2 2 2 2 2479 2619 2441 2677 10216
10 40
  
 = 3774.8
 
 
 SSR = 8268.8 by subtraction.
%%%%%%%%%%%%%%%%%%%%%%%%%%%%%%%%%%%%%%%%%%%
%%- Question 14
\begin{itemize}
\item $SS_T = 2621210 - \frac{10216^2}{40} =  12043.6$
\item $SS_B = \frac{2479^2 + 2619^2 + 2441^2 + 2677^2}{10} - \frac{10216^2}{40} =  3774.8$
\item $SS_R = SS_T - SS_B = 12043.6 - 3774.8 = 8268.8$
\end{itemize}


\begin{center}
\begin{tabular}{|c|c|c|c|c|}
Source of Variation & df & SS & MSS & F \\ \hline
Between Treatments & 3  & 3774.8 & 1258.3 & 5.48 \\ \hline
Residuals & 36 & 8268.8 & 229.7 & \\\hline
Total & 39 & 12043.6 && \\ \hline
\end{tabular}
\end{center} 

F =
1258.3
229.7
= 5.48 on (3,36) df
F3,36(5%) 

%%%%%%%%%





\item 2.9 by interpolation
Clearly reject H0 : $\mu_A = \mu_B = \mu_C = \mu_D$ at the 5\% level
 Strong evidence of a difference between regions A–D.
(b)
\item [same scale!]
Normality — OK.
Equal variances for A, B, C, D — OK.
 C
D
210 220 230 240 250 260 270 280 290
\end{itemize}
%%%%%%%%%%%%%%%%%%%%%%%%%%%%%%%%%%%%%%%%%%%%%%%%%%%%%%%%%%%%%%%%%%%%%
(c)
2
2
2 36
36ˆ
~



where 2 36ˆ = SSR
 95% CI for 2 is
2 2
0.975,36 0.025,36
,
R R  SS SS 

=
\[\frac{36 \hat{\sigma}^2}}{\sigma^2} \sim \chi^2_{36}\]

where $36 \hat{\sigma}^2 = SS_{R}$



\[  \left( \frac{8268.8}{54.4}  ,\frac{8268.8}{21.37}  \right) \;=\; (152.0,386.9) \]
 

= (152.0, 386.9) [interpolate in tables]
 95\% CI for  is (12.3, 19.7)
\item (iii) Second CI is narrower as it is based on more data.





\end{enumerate}
\end{document}
