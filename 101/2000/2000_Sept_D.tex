\documentclass[a4paper,12pt]{article}

%%%%%%%%%%%%%%%%%%%%%%%%%%%%%%%%%%%%%%%%%%%%%%%%%%%%%%%%%%%%%%%%%%%%%%%%%%%%%%%%%%%%%%%%%%%%%%%%%%%%%%%%%%%%%%%%%%%%%%%%%%%%%%%%%%%%%%%%%%%%%%%%%%%%%%%%%%%%%%%%%%%%%%%%%%%%%%%%%%%%%%%%%%%%%%%%%%%%%%%%%%%%%%%%%%%%%%%%%%%%%%%%%%%%%%%%%%%%%%%%%%%%%%%%%%%%

\usepackage{eurosym}
\usepackage{vmargin}
\usepackage{amsmath}
\usepackage{graphics}
\usepackage{epsfig}
\usepackage{enumerate}
\usepackage{multicol}
\usepackage{subfigure}
\usepackage{fancyhdr}
\usepackage{listings}
\usepackage{framed}
\usepackage{graphicx}
\usepackage{amsmath}
\usepackage{chngpage}

%\usepackage{bigints}
\usepackage{vmargin}

% left top textwidth textheight headheight

% headsep footheight footskip

\setmargins{2.0cm}{2.5cm}{16 cm}{22cm}{0.5cm}{0cm}{1cm}{1cm}

\renewcommand{\baselinestretch}{1.3}

\setcounter{MaxMatrixCols}{10}

\begin{document}
\begin{enumerate}
% Faculty of Actuaries Institute of Actuaries
% EXAMINATIONS
% 18 September 2000 (pm)
% Subject 101 — Statistical Modelling

%%%%%%%%%%%%%%%%%%%%%%%%%%%%%%%%%%%%%%%%%%%%%%%%%%%%%%

\item 13 Consider a one-way analysis of variance for comparing k treatments using the
same number ni = r responses for each treatment. The model is
\[Yij = \mu + τi + e_{ij} : i = 1, 2, ..., k; j = 1,2, ..., r\]
where the errors eij are independent $N(0, \sigma 2)$ random variables and where $\sum τi = 0$.
Show that the parameter estimates $\hat{\mu}$ = Y⋅⋅ and ˆi τ = i Y⋅ −Y⋅⋅ are unbiased and
that their variances are given by

%%%%%%%%%%%%%%%%%%%%%%%%%%%%%%%%%%%%%%%%%%%%%%%%%%%%%%
\newpge 
\item 14 Let X be a random variable with cumulative distribution function
\[FX(x) = P(X < x) = 1 − exp(−x2/\theta) , x > 0 , FX(x) = 0 , x ≤ 0\]
and let $(X1, X2, … , Xn)$ be a random sample from $X$.
\begin{enumerate}[(i)]
\item (i) By considering $P(Y < y)$, show that $Y = X2$ has an exponential distribution.

\item (ii) (a) Show that the maximum likelihood estimator of $\theta$ is given by
\[\hat{\theta}
= 1 2
Xi
n
\sum \]
(b) Show that $\hat{\theta}$ is an unbiased estimator of $\theta$ which attains the
Cramer-Rao lower bound on variance.
(c) Using moment generating functions, show that 22
2 ˆ ~ n
n \theta χ
\theta
. [11]
\item (iii) The above distribution of X is to be used as a model for claim amounts in
a particular situation. A random sample of 50 such claim amounts (in
appropriate units) gives the following summary:
$\sum x2 = 485.7518$
(a) Calculate the maximum likelihood estimate of $\theta$.
(b) Using the result of (ii)(c) above, calculate an exact 95\% confidence interval for $\theta$.

\end{enumerate}
%%%%%%%%%%%%%%%%%%%%%%%%%%%%%%%%%%%%%%%%%%%%%%%%%%%%%%%%%%%%%%%%%%%%%%%%%%%%%%%%%%%%%%%%%%%%%%%%%%%%%%%%%%%%


\end{enumerate}
%%%%%%%%%%%%%%%%%%%%%%%%%%%%%%%%%%%%%%%%%%%%%%%%%%%%%%
\newpage


$V(\hat{\mu} ) = V(\bar{Y}_{\ldot \ldot} ) =

\frac{\sigma^2}{kr}$


as $\bar{Y}_{\ldot \ldot}$ is mean of kr r.v.’s each with $var(\sigma2)$.

\begin{eqnarray*}
V( \hat{\tau}}_i) 
&=& V(\bar{Y}_{i \ldot} −\bar{Y}_{\ldot \ldot}) \\
&=& V(\bar{Y}_{i \ldot}) + V (\bar{Y}_{\ldot \ldot}) − Cov (\bar{Y}_{\ldot}, \bar{Y}_{\ldot \ldot}) \\
&=& \frac{\sigma^2}{r} + \frac{\sigma^2}{kr} - 2 \frac{1}{r} \frac{1}{kr} r \sigma^2 \\
&=& \frac{\sigma^2}{r} + \frac{\sigma^2}{kr} - 2 \left(\frac{\sigma^2}{kr} \right) \\
&=& \frac{\sigma^2}{r} - \frac{\sigma^2}{kr} \\
&=& \frac{(k-1)\sigma^2}{kr} \\
\end{eqnarray}

%%%%%%%%%%%%%%

Alternative method:
( ) i V \bar{Y}_{\ldot} −\bar{Y}_{\ldot \ldot} =


V\left(  \left( 1 - \frac{1}{k}  \right) \bar{Y}_{i}  - \frac{1}{k} \sum_{j \neq i} \bar{Y}_{j} \right)
&=& \left(1 - \frac{1}{k} \right) \frac{\sigma^2}{r}  + \left(- \frac{1}{k} \right)^2 \frac{(k-1)\sigma^2}{r} \\

&=& \frac{\sigma^2}{k^2r} \left[  (k-1)^2 + (k-1) \right] \\
&=& \frac{(k-1)\sigma^2}{kr} \\
\end{eqnarray}





\newpage

14 

\begin{itemize}
    \item (i) \[FY(y) = P(Y < y) = P(X2 < y) = P(X < √y) = 1 − exp(−y/\theta)\]
which is the cdf of an exponential r.v. with mean $\theta$.
\item (ii) (a) fX(x) = 2(x / \theta)exp(−x2 / \theta)
\item So \[L(\theta) = k1 \theta−n exp[−(\sum x2) / \theta \] so \[logL = k2 − nlog \theta − (\sum x2) / \theta\]
\[dlog L / d\theta = −n/\theta + (\sum x2)/\theta2 = 0  \hat{\theta} = 2 1
i X
n
\sum \]
\item (b) From (i) E(X2) = E(Y) = \theta and V(X2) = V(Y) = \theta2
So 
\begin{eqnarray*}E(\hat{\theta} ) &=& (1 / n) ( 2 ) i \sum E X \\ &=& (1 / n) n\theta \\ &=& \theta\\ 
\end{eqnarray*}, and

\end{itemize}
\begin{eqnarray*}
V(\hat{\theta} ) &=& (1 / n2) ( 2 ) i \sum V X \\ &=& (1 / n2) n\theta^2 \\ &=& \theta^2 / n
\end{eqnarray*}
%%%%%%%%%%%%%%%%%%%%%%%%%%%%%%%%%%%%%%%%%%%%%%%%%%%%%%%%%
Page 8
\begin{itemize}
    \item 

Now,\[ d2log L / d\theta2 = n\theta−2 − 2(\sum x2) \theta−3\]
so \[ −E(d2log L / d\theta2) = −n\theta−2 + 2\theta−3(n\theta) = n / \theta2 = 1/V\]
\item So $\hat{\theta}$ is unbiased and attains the C–R bound.
(c) Y = X2 has mgf $(1 − \theta t)−1 so n\hat{\theta} = \sum Yi has mgf (1 − \thetat)−n$
\item So $2n\hat{\theta} / \theta$ has mgf $[1 − \theta(2t / \theta)]−n = (1 − 2t)−n$
which is mgf of χ2 with 2n d.f.
\item (iii) (a) $\hat{\theta} = 485.9028/50 = 9.718$
(b) $P(74.22 < 100\hat{\theta} / \theta < 129.6) = 0.95$ from tables of χ2 with 100 d.f.

\item 95\% CI for \theta is $(100\hat{\theta} / 129.6 , 100\hat{\theta} / 74.22)$

i.e. (971.8056 / 129.6 , 971.8056 / 74.22) i.e. (7.50,13.1)
\end{itemize}
%%%%%%%%%%%%%%%%%%%%%%%%%%%%%%%%%%%%%%%%%%%%%%%%%%%%%%%%%



%%%%%%%%%%%%%%%%%%%%%%%%%%%%%%%%%%%%%%%%%%%%%%%%%%%%%%%%%%%%%%%%%%%%%%%%%%%%%%%%%%%%%%%%%%%%%%%%

\end{document}
