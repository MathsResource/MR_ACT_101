%-ACT 101 2000 April Q1 and 2


\documentclass[a4paper,12pt]{article}

%%%%%%%%%%%%%%%%%%%%%%%%%%%%%%%%%%%%%%%%%%%%%%%%%%%%%%%%%%%%%%%%%%%%%%%%%%%%%%%%%%%%%%%%%%%%%%%%%%%%%%%%%%%%%%%%%%%%%%%%%%%%%%%%%%%%%%%%%%%%%%%%%%%%%%%%%%%%%%%%%%%%%%%%%%%%%%%%%%%%%%%%%%%%%%%%%%%%%%%%%%%%%%%%%%%%%%%%%%%%%%%%%%%%%%%%%%%%%%%%%%%%%%%%%%%%

\usepackage{eurosym}
\usepackage{vmargin}
\usepackage{amsmath}
\usepackage{graphics}
\usepackage{epsfig}
\usepackage{enumerate}
\usepackage{multicol}
\usepackage{subfigure}
\usepackage{fancyhdr}
\usepackage{listings}
\usepackage{framed}
\usepackage{graphicx}
\usepackage{amsmath}
\usepackage{chngpage}

%\usepackage{bigints}
\usepackage{vmargin}

% left top textwidth textheight headheight

% headsep footheight footskip

\setmargins{2.0cm}{2.5cm}{16 cm}{22cm}{0.5cm}{0cm}{1cm}{1cm}

\renewcommand{\baselinestretch}{1.3}

\setcounter{MaxMatrixCols}{10}

\begin{document}
\begin{enumerate}
%%%%%%%%%%%%%%%%%%%%%%%%%%%%%%%%%%%%%%%%%%%%%%%%%%%%%%%
\item  Fourteen economists were asked to provide forecasts for the percentage rate
of inflation for the third quarter of 2002. They produced the forecasts given
below.
\begin{verbatim}
1.2 1.4 1.5 1.5 1.7 1.8 1.8
1.9 1.9 2.1 2.7 3.2 3.9 5.0    
\end{verbatim}


\begin{itemize}
    \item Calculate the median and the upper and lower quartiles of these forecasts. 

\item As n = 14 the median is half way between the 7th and 8th value
i.e. m = (1.8 + 1.9)/2=1.85.
\item The quartiles are the 4th and 11th values, so Q1 = 1.5 and Q3 = 2.7 .
\item OR: Using the definition of the quartiles as the 15/4th and 45/4th value gives
Q1 = 1.5 and Q3 = 2.8.
\end{itemize}

\newpage
%%%%%%%%%%%%%%%%%%%%%%%%%%%%%%%%%%%%%%%%%%%%%%%%%%%%%%%
\item Insurance policies providing car insurance are such that the sizes of claims are
normally distributed with mean 1,870 and standard deviation £610. In one
month 50 claims are made. Assuming that claims are independent, calculate
the probability that the total of the claim sizes is more than £100,000. \\

\medskip
%%%%%%%%%%%%%%%%%%%%%%%%%%%%%%%%%%%%%%%%%%%%%%%%%%%%%%%%%%%%%%%%%%%%%
\noindent \textbf{Solution to Part 2}\\

\begin{itemize}
    \item The total claim, T, will be normally distributed with mean 50  1870 = 93500
and variance 50  6102 = 18,605,000 = 43132.
\item (Alternatively, we can work with the mean claim.)
\item Thus, the probability that the total claim is greater than £100,000 is
100,000 93,500
1
4313
\end{itemize}
  
 q 
 
= 1  (1.507) = 0.066.
\newpage
%%%%%%%%%%%%%%%%%%%%%%%%%%%%%%%%%%%%%%%%%%%%%%%%%%%%%%%

 
\end{enumerate}
%%%%%%%%%%%%%%%%%%%%%%%%%%%%%%%%%%%%%%%%%%%%%%%%%%%%%%%

%%  101 April 2000
%%  Subject 101 — Statistical Modelling

%%%%%%%%%%%%%%%%%%%%%%%%%%%%%%%%%%%%%%%%%%%%%%%%%%%%%%%%%%%%%%%%%%%%%



%%%%%%%%%%%%%%%%%%%%%%%%%%%%%%%%%%%%%%%%%%%%%%%%%%%%%%%%%%%%%%%%%%%%%
\end{document}
