\documentclass[a4paper,12pt]{article}

%%%%%%%%%%%%%%%%%%%%%%%%%%%%%%%%%%%%%%%%%%%%%%%%%%%%%%%%%%%%%%%%%%%%%%%%%%%%%%%%%%%%%%%%%%%%%%%%%%%%%%%%%%%%%%%%%%%%%%%%%%%%%%%%%%%%%%%%%%%%%%%%%%%%%%%%%%%%%%%%%%%%%%%%%%%%%%%%%%%%%%%%%%%%%%%%%%%%%%%%%%%%%%%%%%%%%%%%%%%%%%%%%%%%%%%%%%%%%%%%%%%%%%%%%%%%

\usepackage{eurosym}
\usepackage{vmargin}
\usepackage{amsmath}
\usepackage{graphics}
\usepackage{epsfig}
\usepackage{enumerate}
\usepackage{multicol}
\usepackage{subfigure}
\usepackage{fancyhdr}
\usepackage{listings}
\usepackage{framed}
\usepackage{graphicx}
\usepackage{amsmath}
\usepackage{chngpage}

%\usepackage{bigints}
\usepackage{vmargin}

% left top textwidth textheight headheight

% headsep footheight footskip

\setmargins{2.0cm}{2.5cm}{16 cm}{22cm}{0.5cm}{0cm}{1cm}{1cm}

\renewcommand{\baselinestretch}{1.3}

\setcounter{MaxMatrixCols}{10}

\begin{document}
\begin{enumerate}
\item The size of a claim, X, which arises under a certain type of insurance contract, is to be modelled using a gamma random variable with parameters $\alpha$ and $\theta$ (both > 0) such that the moment generating function of X is given by
\[M(t) = (1 - \theta t)-$\alpha$ , t < 1/\theta .\]
By using the cumulant generating function of X, or otherwise, show that the coefficient of skewness of the distribution of X is given by %2/√$\alpha$. [5]


%%%%%%%%%%%%%%%%%%%%%%%%%%%%%%%

\end{enumerate}
%%%%%%% Question 9

$C(t) = \log M(t) = −\alpha \log(1− \theta t)$
$C^{\prime}(t) = \alpha\theta (1 − \theta t)^{−1}$ , 
C^{\prime}^{\prime}(t) = \alpha\theta 2(1 − \theta t)^{−2} , C^{\prime}^{\prime}^{\prime}(t) = 2\alpha\theta $3(1 − \theta t)^{−3}$
% Therefore 

\begin{itemize}
\item $\kappa_2 = C^{\prime}^{\prime}(0) = \alpha\theta 2$ , 
\item $\kappa_3 = C^{\prime}^{\prime}^{\prime}(0) = 2\alpha\theta^3$ 
\end{itemize}
%-------------%

so coefficient is
\begin{eqnarray*}
\frac{\kappa_3 }{ (\kappa_2)^{3/2} } &=& \frac{2\alpha\theta 3 }{ (\alpha\theta^2)^{3/2} }\\ 
&=& \frac{2}{\sqrt{\alpha}}
\end{eqnarray*}

%%%%%%%%%%%%%%%%%%%%%%%%%%%%%%%%%%%%%%%%%%%%%%
Q9 Comment: Many candidates used the MGF directly (and encountered some
algebraic difficulties) rather than using the more convenient cumulant generating
function.
%%%%%%%%%%%%%%%%%%%%%%%%%%%%%%%%%%%%%%%%%%%%%%%%%%%%%%%%%
%%%%%%%%%%%%%%%%%%%%%%%%%%%%%%%
\item  Let $Z$ be a random variable with mean 0 and variance 1, and let X be a random variable independent of $Z$ with mean 5 and variance 4. Let Y = X - Z.
Calculate the correlation coefficient between X and Y. %[5]



\newpage %Page 5
%10

\begin{eqnarray*}
Cov(X , Y) &=& Cov(X , X - Z) \\ &=& Cov(X , X) - Cov(X , Z) \\ &=& V(X) - 0 \\&=& 4\\
\end{eqnarray*}
V(X) = 4, V(Y) = V(X) + V(Z) = 5
so Corr(X , Y) = 4/(4 × 5)1/2 = 0.894
OR:
\begin{itemize}
    \item By independence, it is immediate that Y has mean and variance both equal
to 5.
\item Corr(X , Y) =
( ) [ ( )][ ( )]
( ) ( )
E XY E X E Y
Var X Var Y
- =
[ ( )] 25.
4 5
E X X - Z -
×
\item The independence of X and Z implies that E(XZ) = 0, and as E(X2) = Var(X) + 52 = 29, we obtain that the correlation is
29 0 25
20
- - =
2
5
= 0.894.
\end{itemize}
%%%%%%%%%%%%%%%%%%%%%%%%%%%%%%%%%%%%%%%%%%%%%%%%%%%%%%%%%%%%%%\end{document}
