\documentclass[a4paper,12pt]{article}

%%%%%%%%%%%%%%%%%%%%%%%%%%%%%%%%%%%%%%%%%%%%%%%%%%%%%%%%%%%%%%%%%%%%%%%%%%%%%%%%%%%%%%%%%%%%%%%%%%%%%%%%%%%%%%%%%%%%%%%%%%%%%%%%%%%%%%%%%%%%%%%%%%%%%%%%%%%%%%%%%%%%%%%%%%%%%%%%%%%%%%%%%%%%%%%%%%%%%%%%%%%%%%%%%%%%%%%%%%%%%%%%%%%%%%%%%%%%%%%%%%%%%%%%%%%%

\usepackage{eurosym}
\usepackage{vmargin}
\usepackage{amsmath}
\usepackage{graphics}
\usepackage{epsfig}
\usepackage{enumerate}
\usepackage{multicol}
\usepackage{subfigure}
\usepackage{fancyhdr}
\usepackage{listings}
\usepackage{framed}
\usepackage{graphicx}
\usepackage{amsmath}
\usepackage{chngpage}

%\usepackage{bigints}
\usepackage{vmargin}

% left top textwidth textheight headheight

% headsep footheight footskip

\setmargins{2.0cm}{2.5cm}{16 cm}{22cm}{0.5cm}{0cm}{1cm}{1cm}

\renewcommand{\baselinestretch}{1.3}

\setcounter{MaxMatrixCols}{10}

\begin{document}
\begin{enumerate}

\item 9 The movement of a stock price is modelled as follows:
In each time period, the stock either goes up 1 with probability 0.35 , stays the
same with probability 0.35, or goes down 1 with probability 0.30.
The change in the stock price after 500 time periods is being considered.
\begin{enumerate}[(a)]
\item Assuming that changes in successive time periods are independent,
explain why the normal distribution can be used as an approximate
model. 
\item  Calculate an approximate value for the probability that, after 500 time periods, the stock will be up by more than 20 from where it started. 
\end{enumerate}
\item 10 Claim amounts of a certain type are modelled using a normal distribution with an unknown mean and a known standard deviation σ = £20.
For a random sample of 20 claim amounts all that is known is that 5 of them are
greater than £200.
\begin{enumerate}[(a)]
\item  Let θ be the probability that a claim amount is greater than £200. Write
down the maximum likelihood estimate of θ. [1]
\item Determine θ in terms of μ and hence calculate the maximum likelihood
estimate of μ. 
\end{enumerate}
%%%%%%%%%%%%%%%%%%%%%%%%%%%%%%%%%%%%%%%%%%%%%%%%%%%%%%%%%%%%%%%%%%%%%%%%%%%%%%%%%%%%%%%%%%%%%%%%%%%%%%%%%%%%%%%%%%%
\item 11 Previous years’ records give that the standard deviation of claim size for a certain
class of policy is £75. The distribution of claim size is to be modelled as a normal
random variable.
\begin{enumerate}[(a)]
\item Determine the minimum sample size required to estimate the mean claim
size such that a 95% confidence interval is of width ±£10. [2]
\item Calculate the corresponding sample size such that a 99% confidence
interval is of width ±£10. [2]
\end{enumerate}
%%%%%%%%%%%%%%%%%%%%%%%%%%%%%%%%%%%%%%%%%%%%%%%%%%%%%%%%%%%%%%%%%%%%%%%%%%%%%%%%%%%%%%%%%%%%%%%%%%%%%%%%%%%%%%%%%%%
\item 12 In 1998 a market research organisation conducted a survey in a large city. A
random sample of 400 households showed that 68 were such that at least one
person in the household was a member of a health/fitness club.
\begin{enumerate}[(a)]
\item  An estimate of the true proportion of households in the city with at least
one person being a member of a health/fitness club is therefore
68
400
= 0.17. Calculate 95\% confidence limits for the true proportion.
\item A similar survey was conducted in 1999 and a random sample of 400
households showed that 80 were such that at least one person in the
household was a member of a health/fitness club.
Perform a one-sided test to investigate whether the true proportion
increased from 1998 to 1999. In particular determine the p-value of the
test and state your conclusion. 
\end{enumerate}
%%%%%%%%%%%%%%%%%%%%%%%%%%%%%%%%%%%%%%%%%%%%%%%%%%%%%%%%%%%%%%%%%%%%%%%%%%%%%%%%%%
%%%%%%%%%%%%%%%%%%%%%%%%%%%%%%%%%%%%%%%%%%%%%%%%%%%%%%%%%%%%%%%%%%%%%%%%%%%%%%%%%%%%%%%%%%%%%%%%%%%%%%%%%%%%%%%%%%%
9 (i) The change after a large number of time periods equals the sum of i.i.d.
r.v.’s.

\begin{itemize}
    \item The Central Limit theorem leads to an approximate normal distribution.
(ii)
x : +1 0 −1
p : 0.35 0.35 0.30
\item E(X) = 0.35 − 0.30 = 0.05
\item E(X2) = 0.35 + 0.30 = 0.65
\item ∴Var(X) = 0.65 − (0.05)2 = 0.6475 ∴s.d.(X) = 0.8047
\item Let Y = price change after 500 periods
\item ∴Y has mean 500(0.05) = 25 and s.d. 500(0.8047) =17.99
\item P(Y > 20)
20 25
( = 0.28) 0.61
17.99
≈ P Z > − − =
\item [Continuity correction can be used if desired]
\end{itemize}


\newpage


%%%% 2001 April Question 10 

(i) Using the basic binomial result


\[  \hat{\theta} = \frac{x}{n} = \frac{5}{20} = 0.25\]


(ii) 
\[X \sim N(\mu , 202)\]

\begin{eqnarray*} 
\theta 
&=& P(X > 200) \\
&=& P \left( Z > \frac{200 -\mu}{20} \right) \\
&=& 1 - \Phi\left(  \frac{200 -\mu}{20} \right)\\
\end{eqnarray*}

where $\Phi(.)$ is the cdf of N(0,1).

Using the invariance property of MLE’s, we get $ˆ\mu$ from the equation

\[ \hat{\theta} = 1 - \Phi\left(  \frac{200 -\mu}{20} \right)\]
So with $\hat{\theta} = 0.25$, we get $\hat{ˆ\mu}$ from

\[\Phi\left(  \frac{200 -\hat{\mu}{20} \right) = 0.75\]

\[\left(  \frac{200 -\hat{\mu}{20} \right)ˆ
= 0.674\] from tables
∴
\[\hat{\mu}ˆ = 200 − 20(0.674) = \$186.52\]

%%%%%%%%%%%%%%%%%%%%%%%%%%%%%%%%%%%%
\newpage
11 
%%%%%%%%%%%%%%%%%%%%%%%%
(i) 95% confidence interval:



\[ \bar{x} \pm 1.96 \frac{75}{\sqrt{n}}
\begin{itemize}

\item Require

\item $1.96 \frac{75}{\sqrt{n}} \leq 10 $

\item $\sqrt{n} \geq \frac{1.96 \times 75}{10}$ 

\item $\sqrt{n} \geq 14.7$

\item $n \geq  216.09 $

\item Therefore take $n = 217$.

\end{itemize}

%%%%%%%%%%%%%

(ii) 99% confidence interval:

\[ \bar{x} \pm 2.576 \frac{75}{\sqrt{n}}

\begin{itemize}

\item  Require

\item  $2.576 \frac{75}{\sqrt{n}} \leq 10 $

\item  $\sqrt{n} \geq \frac{2.576 \times 75}{10}$ 

\item $\sqrt{n} \geq 19.32$

\item $n \geq  373.3 $

\item Therefore take $n = 374$.
\end{itemize}

%%%%%%%%%%%%%
12 (i) pˆ = 0.17
95\% confidence interval for the true proportion p is
ˆ (1 ˆ ) ˆ 1.96
400
p p
p ± −
(0.17)(0.83)
0.17 1.96 = 0.17 0.037 or (0.133,0.207)
400
 ± ±
%%%%%%%%%%%%%%%%%%%%%%%%%%%%%%%%%%%%%%%%%%%%%%%%%%%%
(ii) pˆ1 = 0.17 , 2 pˆ = 0.2
common
ˆ 68 80 = =0.185
800
p +
H0 : p2 = p1 v. H1 : p2 > p1
0.2 0.17 0.03
= = =1.09
1 1 0.0275
(0.185)(0.815)( )
400 400
z −
+
P-value = P(Z > 1.09) = 1 − 0.862 = 0.14
There is not sufficient evidence to conclude that there is an increase in the
true proportion. The observed difference could have occurred by chance.
%%%%%%%%%%%%%%%%%%%%%%%%%%%%%%%%%%%%%%%%%%%%%%%%%%%%
\end{document}
