\documentclass[a4paper,12pt]{article}

%%%%%%%%%%%%%%%%%%%%%%%%%%%%%%%%%%%%%%%%%%%%%%%%%%%%%%%%%%%%%%%%%%%%%%%%%%%%%%%%%%%%%%%%%%%%%%%%%%%%%%%%%%%%%%%%%%%%%%%%%%%%%%%%%%%%%%%%%%%%%%%%%%%%%%%%%%%%%%%%%%%%%%%%%%%%%%%%%%%%%%%%%%%%%%%%%%%%%%%%%%%%%%%%%%%%%%%%%%%%%%%%%%%%%%%%%%%%%%%%%%%%%%%%%%%%

\usepackage{eurosym}
\usepackage{vmargin}
\usepackage{amsmath}
\usepackage{graphics}
\usepackage{epsfig}
\usepackage{enumerate}
\usepackage{multicol}
\usepackage{subfigure}
\usepackage{fancyhdr}
\usepackage{listings}
\usepackage{framed}
\usepackage{graphicx}
\usepackage{amsmath}
\usepackage{chngpage}

%\usepackage{bigints}
\usepackage{vmargin}

% left top textwidth textheight headheight

% headsep footheight footskip

\setmargins{2.0cm}{2.5cm}{16 cm}{22cm}{0.5cm}{0cm}{1cm}{1cm}

\renewcommand{\baselinestretch}{1.3}

\setcounter{MaxMatrixCols}{10}

\begin{document}

\begin{enumerate}
%%%%%%%%%%%%%%%%%%%%%%%%%%%%%%%%%%%%%%%%%%%%%%%%%%%%%%%%%%%%%%%%%%
\item A very crude model for the distribution of claim size, X, in a particular situation
represents X as a discrete random variable which takes the values £5,000, £10,000,
and £20,000 with probabilities 0.4, 0.5, and 0.1 respectively.
Calculate the probability that of five randomly selected claims, three are for £5,000
each and the other two are for larger amounts. 
%%%%%%%%%%%%%%%%%%%%%%%%%%%%%%%%%%%%%%%%%%%%%%%%%%%%%%
\item 2 The probability that a component in a rocket motor will fail when the motor is fired is
0.02. To achieve a greater reliability several similar components are to be fitted in
parallel; the motor will then fail only if all the individual components fail
simultaneously.
Determine the minimum number of components required to ensure that the
probability the motor fails is less than one in a billion (i.e. less than 109), assuming
that components fail independently. 
%%%%%%%%%%%%%%%%%%%%%%%%%%%%%%%%%%%%%%%%%%%%%%%%%%%5
\item A random variable X which can be used in certain circumstances as a model for claim
sizes has cumulative distribution function
\[F(x) = \begin{cases}
0 & \mbox{ where }  x < 0 \\ 
1 - \left(\frac{2}{(2+x)}\right)^2 & \mbox{ where }  x > 0 \\ 
\end{cases}
\]

Calculate the value of the conditional probability $P(X > 3|X > 1)$. 
\item Suppose that the sums assured under policies of a certain type are modelled by a
distribution with mean £8,000 and standard deviation £3,000. Consider a group of
100 independent policies of this type.
Calculate the approximate probability that the total sum assured under this group of
policies exceeds £845,000. 
\end{enumerate}
\newpage

%%%%%%%%%%%%%%%%%%%%%%%%%%%%%%%%%%%%%%%%%%%%%%%%%%%%%%%%%%%%%%%%%%%%%%%%
%% Page 2
1 5 3 2
0.4 0.6 = 10 0.02304 = 0.2304
3
 
  
 


2 With n components the probability that the motor fails is the probability that all of the
components fail simultaneously.
P(motor fails) = P(n independent components fail) = 0.02n
This is less than 109 if n log(0.02)  log(109)  n  log(109)/log(0.02) = 5.30
Therefore the minimum number of components is 6.
[OR: by trial and error]

%%%%%%%%%%%%%%%%%%%%%%%%%%%%%%%%%%%%%%%%%%%%%%%%%%%%%%%%%%%%%%%%%%%%%%%%
%% 2002 September Q 3
3 
\begin{eqnarray*}
P(X > 3| X > 1) &=& \frac{P(X > 3 \mbox{and }X > 1)}{P(X > 1)}\\
&=& \frac{P(X > 3 )}{P(X > 1)}\\
&=& \frac{(2/5)^5 }{(2/3)^3 }\\
&=& 0.216\\
\end{eqnarray*}

%%%%%%%%%%%%%%%%%%%%%%%%%%%%%%%%%%%%%%%%%%%%%%%%%%%%%%%%%%%%%%%%%%%%%
4 Using units of £1000:
Total sum assured S  N(100  8, 100  9) i.e. S  N(800, 900) approximately, by
Central Limit Theorem.
P(S > 845)  P[Z > (845 – 800)/30] = P(Z > 1.5) where Z  N(0, 1)
= 0.067

\end{document}
