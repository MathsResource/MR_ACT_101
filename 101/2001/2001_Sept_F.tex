\documentclass[a4paper,12pt]{article}

%%%%%%%%%%%%%%%%%%%%%%%%%%%%%%%%%%%%%%%%%%%%%%%%%%%%%%%%%%%%%%%%%%%%%%%%%%%%%%%%%%%%%%%%%%%%%%%%%%%%%%%%%%%%%%%%%%%%%%%%%%%%%%%%%%%%%%%%%%%%%%%%%%%%%%%%%%%%%%%%%%%%%%%%%%%%%%%%%%%%%%%%%%%%%%%%%%%%%%%%%%%%%%%%%%%%%%%%%%%%%%%%%%%%%%%%%%%%%%%%%%%%%%%%%%%%

\usepackage{eurosym}
\usepackage{vmargin}
\usepackage{amsmath}
\usepackage{graphics}
\usepackage{epsfig}
\usepackage{enumerate}
\usepackage{multicol}
\usepackage{subfigure}
\usepackage{fancyhdr}
\usepackage{listings}
\usepackage{framed}
\usepackage{graphicx}
\usepackage{amsmath}
\usepackage{chngpage}

%\usepackage{bigints}
\usepackage{vmargin}

% left top textwidth textheight headheight

% headsep footheight footskip

\setmargins{2.0cm}{2.5cm}{16 cm}{22cm}{0.5cm}{0cm}{1cm}{1cm}

\renewcommand{\baselinestretch}{1.3}

\setcounter{MaxMatrixCols}{10}

\begin{document}


%%%%%%%%%%%%%%%%%%%%%%%%%%%%%%%%%%%%%%%%%%%%%%%%%%%%%%%%%%%%%%%%%%%%%%%%%%%%%%%%%%%%%%%%%%%%%%%%%%%%%%%%%
\item 11 A random sample of 11 policies on the contents of private houses was examined
for each of three insurance companies and the sum insured under each policy
noted. The observations were rounded to the nearest $$\$100$ and expressed in units
of $\$1,000$.
The sums and sums of squares of the observations are as follows:
\begin{center}
\begin{tabular}{ccc}
& Sum & Sum of squares\\
Company 1 & 129.1 & 1,534.37\\
Company 2 & 109.8 & 1,109.88\\
Company 3 & 123.5 & 1,401.73\\
\end{tabular}
\end{center}
The data are to be analysed under the one-way analysis of variance model to
examine whether company effects are present.
The ANOVA table is given below, with three entries deleted.
\begin{verbatim}
Source of variation d.f. SS MSS
Between companies 2 *** ***
Residual 30 48.24 1.61
32 ***
\end{verbatim}

\begin{enumerate}[(i)]
\item Copy the table into your answer book, filling in the three values which
have been deleted. 
\item Test the null hypothesis of no company effects and state your conclusion.

\end{enumerate}

%%%%%%%%%%%%%%%%%%%%%%%%%%%%%%%%%%%%%%%%%%%%
11 

129.1 + 109.8 + 123.5 = 362.4 , 1,534.37 + 1,109.88 + 1,401.73 = 4,045.98
% SST = 4,045.98 – 362.42/33 = 66.17
% So SSB = 66.17 – 48.24 = 17.93 **
Table is:
\begin{verbatim}
    Source of variation d.f. SS MSS
Between companies 2 17.93 8.97
Residual 30 48.24 1.61
32 66.17
\end{verbatim}



F = 8.97/1.61 = 5.57 on 2,30 d.f.
P-value is less than 0.01, so reject null hypothesis.
There is strong evidence that there are differences among the (population)
means of the sums insured for the three companies.

%%%%%%%%%%%%%%%%%%%%%%%%%%%%%%%%%%%%%%%%%%%%%%%%%%%%%%%%%%%%%%%%%%%%%%%%%%%%%%%%%%%%%%%%%%%%%%%%%%%%%%%%%
** OR: Calculate SSB directly as
SSB = (129.12 + 109.82 + 123.52) / 11 – 362.42/33 = 17.93
12 H0 : this year’s pattern is the same as last year’s v. H1 : not the same
\begin{itemize}
    \item Under H0 , the expected frequencies are:
% 120  0.184 ; 0.703 ; 0.113 = 22.08 ; 84.36 ; 13.56
% oi ei (o 
e)2/e
15 22.08 2.270
87 84.36 0.083
18 13.56 1.454
3.807 on 2 df
5% point from 22
 is 5.991. So cannot reject H0 at 5% level.
\item These data provide no evidence to suggest that this year’s pattern differs from
that of last year.
\item A few candidates worked with percentages of claims instead of numbers of claims
(when using the chi-squared goodness-of-fit statistic, one must work with observed
and expected frequencies). Such work received few if any marks.
\end{itemize}

\end{document}
