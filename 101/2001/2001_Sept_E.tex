\documentclass[a4paper,12pt]{article}

%%%%%%%%%%%%%%%%%%%%%%%%%%%%%%%%%%%%%%%%%%%%%%%%%%%%%%%%%%%%%%%%%%%%%%%%%%%%%%%%%%%%%%%%%%%%%%%%%%%%%%%%%%%%%%%%%%%%%%%%%%%%%%%%%%%%%%%%%%%%%%%%%%%%%%%%%%%%%%%%%%%%%%%%%%%%%%%%%%%%%%%%%%%%%%%%%%%%%%%%%%%%%%%%%%%%%%%%%%%%%%%%%%%%%%%%%%%%%%%%%%%%%%%%%%%%

\usepackage{eurosym}
\usepackage{vmargin}
\usepackage{amsmath}
\usepackage{graphics}
\usepackage{epsfig}
\usepackage{enumerate}
\usepackage{multicol}
\usepackage{subfigure}
\usepackage{fancyhdr}
\usepackage{listings}
\usepackage{framed}
\usepackage{graphicx}
\usepackage{amsmath}
\usepackage{chngpage}

%\usepackage{bigints}
\usepackage{vmargin}

% left top textwidth textheight headheight

% headsep footheight footskip

\setmargins{2.0cm}{2.5cm}{16 cm}{22cm}{0.5cm}{0cm}{1cm}{1cm}

\renewcommand{\baselinestretch}{1.3}

\setcounter{MaxMatrixCols}{10}

\begin{document}
\begin{enumerate}
%%%%%%%%%%%%%%%%%%%%%%%%%%%%%%%%%%%%%%%%%%%%%%%%%%%%%%%%%%%%%%%%%%%%%%%%%%%%%%%%%%%%%%%%%%%%%%%%%%%%%%%%%
\item Two independent random samples of sizes n1 and n2 are selected from a normal
population with variance $\sigma^2$. The sample variances are denoted by $S^2_1$ and $S^2_2$
respectively. Let $S^2_W$ denote a weighted average of the sample variances given by
\[ S^2_W = \alpha S^2_1 + (1 - \alpha)S^2_2\]
where $\alpha$ is a constant such that $0 \leq \alpha \leq 1$.
\begin{enumerate}[(a)]
\item Show that $S^2_W$
 is an unbiased estimator of $\sigma^2$, and obtain an expression for
the mean square error of $S^2_W$
 .
(You may use
${ \displaystyle \operatorname{Var}(S^2_i)  = \frac{2 \sigma^4}{n_i - 1} }$ 

, i = 1, 2.) 
\item Show that $S^2_W$
 has minimum mean square error if ${ \displaystyle  \alpha = \frac{n_1 - 1}{ n_1 + n_2 - 2} }$
\end{enumerate}

%%%%%%%%%%%%%%%%%%%%%%%%%%%%%%%%%%%%%%%%%%%%%%%%%%%%%%%%%%%%%%%%%%%%%%%%%%%%%%%%%%%%%%%%%%%%%%%%%%%%%%%%%
\item 10 The number of incomplete insurance proposals Y, in a batch of x proposals, is to
be modelled as a Poisson random variable with mean $\lambda x$, where $\lambda$ is unknown.
Data are available from n independent batches of proposals as follows: batch
number i contains xi proposals of which yi are incomplete, $i = 1,2, \ldots, n$.
The least squares estimator of $\lambda$ is that value of $\lambda$for which
\[( )2
1
( )
n
i i
i
Y EY
=
 −\]
is minimised.
\begin{enumerate}[(a)]
\item Show that the least squares estimator of $\lambda$ is given by:
\[2 = i i
i
x Y
x
\lambda
 .\] 
\item Determine $\hat{lambda}$ , the maximum likelihood estimator of $\lambda$. 
\item Determine whether neither, one, or both of $\lambda$ and $\hat{lambda}$ provide unbiased
estimators of $\lambda$. 
\end{enumerate}

\newpage

%%%%%%%%%%%%%%%%%%%%%%%%%%%%%%%%%%%%%%%%%%%%%%%%%%%%%%
\newpage
% 9 2 2 2
% 1 2 ( )= ( ) (1 ) ( ) W E S E S    E S
% = \sigma^2 + (1  )\sigma^2 = \sigma^2 
(using unbiasedness of sample variance)
Therefore 2
W S is unbiased for $\sigma^2$. MSE = ( 2 ) W Var S since unbiased.
MSE = ( 2 ) W Var S 2 2 2 2
% 1 2 =  Var(S )  (1  ) Var(S )
%  2 2 4
4 2
1 2
1 2
= 2 using ( ) = ; 1,2
1 1 1 i
i
Var S i
n n n

% \item 4
1 2
2 2(1 )
= 2
1 1
dMSE
d n n


Setting equal to zero gives
1 2
1
n 1 n 1

% = 0  (n2  1)   (n1  1)(1  ) = 0

${ \displaystyle \operatorname{Var}(S^2_i)  = \frac{2 \sigma^4}{n_i - 1} }$ 
%%%%%%%%%%%%%%%%%%%%%%%%%%%%%%%%%%%%%%%%%%%%%%%%%%%%%%%%%%%%%%%%%%%%%%%%%%%%%%%%%%%%%%%%%%%%%%%%%%%%%%%%%
Thus giving 1
1 2
1
=
2
n
n n
%
which clearly minimises MSE.


\newpage

10 
\begin{itemize}
\item 2 2
1 1
= ( ) =
n n
i i i i
i i
S Y EY Y x
%%%%%%%%%%%%%%%%%%%%%%%%%%%5
= 2 ( ) i i i
dS
x Y x
d
%%%%%%%%%%%%%%%%%%%%%%%%%%%%%%%
 Setting to 0  2 = i i
i
xY
x
%%%%%%%%%%%%%%%%%%%%%%%%%%%5
\item %( ) = xi  Yi const. so log =   log const.
i i i L e x L x Y %
%%%%%%%%%%%%%%%%%%%%%%%%%%%%%%%%%%%%%%
log 1
i i
d L
x Y
d
%%%%%%%%%%%%%%%%%%%%%%%%%%%%%
% Setting to 0  ˆ i
i
Y
x%%%%%%%%%%%%%%%%%%%%%%%%%%%
\item %  2 2
1 1
( )= = = i i i i
i i
E E x Y x x
x x
%     
%  
%  hence unbiased
%  
% ˆ 1 1 ( )= = = i i
i i
E E Y x
x x
    \item 
\end{itemize}
hence unbiased
Some candidates did not appear to understand that the two methods of deriving
estimators could produce different estimators.
%%%%%%%%%%%%%%%%%%%%%%%%%%%%%%%%%%%%%%%%%%%%
\end{document}
