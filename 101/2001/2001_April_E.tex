\documentclass[a4paper,12pt]{article}

%%%%%%%%%%%%%%%%%%%%%%%%%%%%%%%%%%%%%%%%%%%%%%%%%%%%%%%%%%%%%%%%%%%%%%%%%%%%%%%%%%%%%%%%%%%%%%%%%%%%%%%%%%%%%%%%%%%%%%%%%%%%%%%%%%%%%%%%%%%%%%%%%%%%%%%%%%%%%%%%%%%%%%%%%%%%%%%%%%%%%%%%%%%%%%%%%%%%%%%%%%%%%%%%%%%%%%%%%%%%%%%%%%%%%%%%%%%%%%%%%%%%%%%%%%%%

\usepackage{eurosym}
\usepackage{vmargin}
\usepackage{amsmath}
\usepackage{graphics}
\usepackage{epsfig}
\usepackage{enumerate}
\usepackage{multicol}
\usepackage{subfigure}
\usepackage{fancyhdr}
\usepackage{listings}
\usepackage{framed}
\usepackage{graphicx}
\usepackage{amsmath}
\usepackage{chngpage}

%\usepackage{bigints}
\usepackage{vmargin}

% left top textwidth textheight headheight

% headsep footheight footskip

\setmargins{2.0cm}{2.5cm}{16 cm}{22cm}{0.5cm}{0cm}{1cm}{1cm}

\renewcommand{\baselinestretch}{1.3}

\setcounter{MaxMatrixCols}{10}

\begin{document}
\begin{enumerate}

\item 9 The movement of a stock price is modelled as follows:
In each time period, the stock either goes up 1 with probability 0.35 , stays the
same with probability 0.35, or goes down 1 with probability 0.30.
The change in the stock price after 500 time periods is being considered.
\begin{enumerate}[(a)]
\item Assuming that changes in successive time periods are independent,
explain why the normal distribution can be used as an approximate
model. 
\item  Calculate an approximate value for the probability that, after 500 time periods, the stock will be up by more than 20 from where it started. 
\end{enumerate}
\item 10 Claim amounts of a certain type are modelled using a normal distribution with an unknown mean and a known standard deviation σ = £20.
For a random sample of 20 claim amounts all that is known is that 5 of them are
greater than £200.
\begin{enumerate}[(a)]
\item  Let θ be the probability that a claim amount is greater than £200. Write
down the maximum likelihood estimate of θ. [1]
\item Determine θ in terms of μ and hence calculate the maximum likelihood
estimate of μ. 
\end{enumerate}

\end{enumerate}
%%%%%%%%%%%%%%%%%%%%%%%%%%%%%%%%%%%%%%%%%%%%%%%%%%%%%%%%%%%%%%%%%%%%%%%%%%%%%%%%%%
%%%%%%%%%%%%%%%%%%%%%%%%%%%%%%%%%%%%%%%%%%%%%%%%%%%%%%%%%%%%%%%%%%%%%%%%%%%%%%%%%%%%%%%%%%%%%%%%%%%%%%%%%%%%%%%%%%%
9 (i) The change after a large number of time periods equals the sum of i.i.d.
r.v.’s.

\begin{itemize}
    \item The Central Limit theorem leads to an approximate normal distribution.
(ii)
x : +1 0 −1
p : 0.35 0.35 0.30
\item E(X) = 0.35 − 0.30 = 0.05
\item E(X2) = 0.35 + 0.30 = 0.65
\item ∴Var(X) = 0.65 − (0.05)2 = 0.6475 ∴s.d.(X) = 0.8047
\item Let Y = price change after 500 periods
\item ∴Y has mean 500(0.05) = 25 and s.d. 500(0.8047) =17.99
\item P(Y > 20)
20 25
( = 0.28) 0.61
17.99
≈ P Z > − − =
\item [Continuity correction can be used if desired]
\end{itemize}


\newpage


%%%% 2001 April Question 10 

(i) Using the basic binomial result


\[  \hat{\theta} = \frac{x}{n} = \frac{5}{20} = 0.25\]


(ii) 
\[X \sim N(\mu , 20^2)\]

\begin{eqnarray*} 
\theta 
&=& P(X > 200) \\
&=& P \left( Z > \frac{200 -\mu}{20} \right) \\
&=& 1 - \Phi\left(  \frac{200 -\mu}{20} \right)\\
\end{eqnarray*}

where $\Phi(.)$ is the cdf of N(0,1).

Using the invariance property of MLE’s, we get $ˆ\mu$ from the equation

\[ \hat{\theta} = 1 - \Phi\left(  \frac{200 -\mu}{20} \right)\]
So with $\hat{\theta} = 0.25$, we get $\hat{ˆ\mu}$ from

\[\Phi\left(  \frac{200 -\hat{\mu}{20} \right) = 0.75\]

\[\left(  \frac{200 -\hat{\mu}{20} \right)ˆ
= 0.674\] from tables
∴
\[\hat{\mu}ˆ = 200 − 20(0.674) = \$186.52\]


\end{document}
