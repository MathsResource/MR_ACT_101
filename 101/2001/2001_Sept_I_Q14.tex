
\documentclass[a4paper,12pt]{article}

%%%%%%%%%%%%%%%%%%%%%%%%%%%%%%%%%%%%%%%%%%%%%%%%%%%%%%%%%%%%%%%%%%%%%%%%%%%%%%%%%%%%%%%%%%%%%%%%%%%%%%%%%%%%%%%%%%%%%%%%%%%%%%%%%%%%%%%%%%%%%%%%%%%%%%%%%%%%%%%%%%%%%%%%%%%%%%%%%%%%%%%%%%%%%%%%%%%%%%%%%%%%%%%%%%%%%%%%%%%%%%%%%%%%%%%%%%%%%%%%%%%%%%%%%%%%

\usepackage{eurosym}
\usepackage{vmargin}
\usepackage{amsmath}
\usepackage{graphics}
\usepackage{epsfig}
\usepackage{enumerate}
\usepackage{multicol}
\usepackage{subfigure}
\usepackage{fancyhdr}
\usepackage{listings}
\usepackage{framed}
\usepackage{graphicx}
\usepackage{amsmath}
\usepackage{chngpage}

%\usepackage{bigints}
\usepackage{vmargin}

% left top textwidth textheight headheight

% headsep footheight footskip

\setmargins{2.0cm}{2.5cm}{16 cm}{22cm}{0.5cm}{0cm}{1cm}{1cm}

\renewcommand{\baselinestretch}{1.3}

\setcounter{MaxMatrixCols}{10}

\begin{document}
\begin{enumerate}
%%%%%%%%%%%%%%%%%%%%%%%%%%%%%%%%%%%%%%%%%%%%%%%%%%%%%%%%%%%%%%%%%%%%%%%%%%%%%%%%%%%%%%%%%%%%%%%%%%%%%%%%%
\item 12 Twenty overweight executives take part in an experiment to compare the
effectiveness of two exercise methods, A (isometric), and B (isotonic). They are
allocated at random to the two methods, ten to isometric, ten to isotonic methods.
After several weeks, the reductions in abdomen measurements are recorded in
centimetres with the following results:
A (isometric method) 3.1 2.1 3.3 2.7 3.4 2.7 2.7 3.0 3.0 1.6
B (isotonic method) 4.5 4.1 2.7 2.2 4.7 2.2 3.6 3.0 3.3 3.4
\begin{enumerate}[(a)]
\item (a) Plot the data for the two exercise methods on a single diagram.
Comment on whether the response values for each exercise method
are well modelled by normal random variables.
(b) Perform a test to investigate whether the assumption of equal
variability for the responses for the two exercise methods is
reasonable.
(c) Perform a t-test to investigate whether these data support the
claim that the isotonic method is more effective than the other
method. [9]
\item (a) Determine a two-sided 95\% confidence interval for the difference in
the means for the two exercise methods.
(b) Assuming that the two sets of 10 measurements are taken from
normal populations with the same variance, determine a 95%
confidence interval for the common standard deviation. [7]
\end{enumerate}

%%%%%%%%%%%%%%%%%%%%%%%%%%%%%%%%%%%%%%%%%%%%%%%%%%%%%%%%%%%%%%%%%%%%%%%%%%%%%%%%%%%%%%%%%%%%%%%%%%%%%%%%%
\item 14 In a medical study on hypertension amongst young male athletes the researchers
were interested in the effects of the use of a particular (legal) stimulant on systolic
blood pressure (bp).
Ten young male athletes from a larger group who had agreed to take part in the
study were selected at random — none of those in the group were currently users of
the stimulant. The initial bp of each of the sample was measured in controlled
conditions. Each sample member was then exposed to the use of the stimulant in a
controlled manner for a fixed period of time. Each sample member was subject to a
similar exercise regime, and at the end of the period the bp of each of the sample
was again measured in the same controlled conditions as initially, giving the followup
bp.
The data obtained were as follows:
Athlete 1 2 3 4 5 6 7 8 9 10
Initial bp 116 107 129 119 116 113 135 121 112 123
Follow-up bp 123 111 140 129 130 118 143 128 110 132
Summaries: Initial Σx = 1191, Σx2 = 142471 Follow-up Σx = 1264, Σx2 = 160872

%%%%%%%%%%%%%%%%%%%%%%%%%%%%%%%%%%%%%%%%%%%%%%%%%%%%%%%%%%%%%%%%%%%%%%%%%%%%%%%%%%%%%%%%%%%%%%%%%%%%%%%%%
The following models are proposed as possible bases for the analysis:
(M1) The initial bp of the ith athlete, Xi , is distributed as Xi ~ N(μ, σ1
2) and the
follow-up bp, Yi , is distributed such that $Yi|Xi = x ~ N(x + \alpha, \sigma^2
2)$.
(M2) The initial bp of the ith athlete, Xi , is distributed as $Xi ~ N(μi , σ1
2)$ and the
follow-up bp, Yi , is distributed such that $Yi|Xi = x ~ N(x + \alpha, \sigma^2
2)$.
(M3) The initial bp of the ith athlete, Xi , is distributed as Xi ~ N(μi , σ1
2) and the
follow-up bp, Yi , is distributed such that $Yi|Xi = x ~ N(x + \alpha_i , \sigma^2
2)$.
\begin{enumerate}
\item (a) Explain briefly the differences between the physical assumptions
underlying the three models proposed above.
(b) Explain briefly why model (M3) above could not be used in the
analysis of the data as given. [7]
\item Adopting model (M1) above
(a) Using the initial bp data, calculate a symmetrical, two-sided, 95%
confidence interval for μ, and
(b) Calculate a point estimate of $\alpha$. [6]
\item Adopting model (M2) above, calculate a one-sided 95% confidence interval
for $\alpha$ which brings out the minimum realistic value for the mean increase
in bp attributable to taking the stimulant.
\end{enumerate}


%%%%%%%%%%%%%%%%%%%%%%%%%%%%%%%%%%%%%%%%%%%%%%%%%%%%
\end{enumerate}

%%%%%%%%%%%%%%%%%%%%%%%%%%%%%%%%%%%%%%%%%%%%%%%%%%%%%%%%%%%%%%%%%%%%%%%%%%%%%%%%%%%%%%%%%%%%%%%%%%%%%%%%%
14 
\begin{itemize}
\item (a) In Model M1 we have a basic model for the initial bp of the whole population of young male athletes, with mean , and the mean bp
increases by%   after using the stimulant.
(Note: E(follow-up bp) = $E(Y) = E[E(Y|X)] = E[X + ] =  + )
Model M2 extends M1 by allowing for a different initial mean for
each athlete % (i).
Model M3 extends M2 by allowing for a different mean increase in bp for each athlete % (i).
(Note: In all three models we have a single population variance for
initial bp and a single, but different, variance for follow-up bp.)
(Note: V(follow-up bp) = V(Y) = V[E(Y|X)] + E[V(Y|X)] = 1
2 + $ 2
2)
(b) For 10 athletes, M3 has 22 unknown parameters — but we only have 20 data points. So estimation of parameters is impossible.
\item (a) Initial bp: % x = 1191, x2 = 142471 so x = 119.1 , s2 = 69.211
t9(0.025) = 2.262
95% CI for  is 119.1  {2.262  (69.211/10)½} i.e. 119.1  5.95
i.e. (113.15 , 125.05)
(b) Follow-up bp : x = 1264 x = 126.4 so ˆ = 126.4  119.1 = 7.3
\item Use the differences (follow-up less initial) for each athlete:
di : 7, 4, 11, 10, 14, 5, 8, 7, 2, 9
\end{itemize}


%%%%%%%%%%%%%%%%%%%%%%%%%%%%%%%%%%%%%%%%%%%%%%%%%%%%%%%%%%%%%%%%%%%%%%%%%%%%%%%%%%%%%%%%%%%%%%%%%%%%%%%%%
% d = 73, d2 = 705 so d = 7.3 , s2 = 19.122
t9(0.05) = 1.833
%  95% CI (one-sided) for  is (7.3  1.833(19.122/10)½, ) i.e. (4.77, )
The early part of this question (on comparing models) looked hard, but,
pleasingly, was generally well-attempted.

\end{document}
