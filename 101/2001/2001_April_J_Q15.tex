\documentclass[a4paper,12pt]{article}

%%%%%%%%%%%%%%%%%%%%%%%%%%%%%%%%%%%%%%%%%%%%%%%%%%%%%%%%%%%%%%%%%%%%%%%%%%%%%%%%%%%%%%%%%%%%%%%%%%%%%%%%%%%%%%%%%%%%%%%%%%%%%%%%%%%%%%%%%%%%%%%%%%%%%%%%%%%%%%%%%%%%%%%%%%%%%%%%%%%%%%%%%%%%%%%%%%%%%%%%%%%%%%%%%%%%%%%%%%%%%%%%%%%%%%%%%%%%%%%%%%%%%%%%%%%%

\usepackage{eurosym}
\usepackage{vmargin}
\usepackage{amsmath}
\usepackage{graphics}
\usepackage{epsfig}
\usepackage{enumerate}
\usepackage{multicol}
\usepackage{subfigure}
\usepackage{fancyhdr}
\usepackage{listings}
\usepackage{framed}
\usepackage{graphicx}
\usepackage{amsmath}
\usepackage{chngpage}

%\usepackage{bigints}
\usepackage{vmargin}

% left top textwidth textheight headheight

% headsep footheight footskip

\setmargins{2.0cm}{2.5cm}{16 cm}{22cm}{0.5cm}{0cm}{1cm}{1cm}

\renewcommand{\baselinestretch}{1.3}

\setcounter{MaxMatrixCols}{10}

\begin{document}
%%%%%%%%%%%%%%%%%%%%%%%%%%%%%%%%%%%%%%%%%%%%%%%%%%%%%%%%%%%%%%%%%%%%%%%

15 It has been decided to model a claim amount distribution using a gamma
distribution with parameters $\alpha = 4$ and $\lambda$ (unknown), that is, with density
\[f(x; \lambda ) =
1
6
λ4 x3 e−λx: 0 < x < ∞.\]
\begin{enumerate}
    \item (i) A random sample of n claim amounts, X1, X2,\ldots , Xn, is selected and it is
required to estimate the parameter λ.
(a) (1) Determine the method of moments estimator of $\lambda$.
\item (2) Show that the maximum likelihood estimator of $\lambda$ is the same as the method of moments estimator.
\item (b) (1) Write down the moment generating function of Xi and
derive the moment generating function of Y = 2nλX , where X is the sample mean. Explain why the distribution of Y
is 28
χ n .
\item (2) Using Y = 2nλX as a pivotal quantity, derive a 95\% confidence interval for λ. 
(ii) A random sample of 10 claim amounts yields a sample mean of x = £242
and a standard deviation of s = £112.
(a) Using part (i), calculate a 95% confidence interval for λ.
\item (b) (1) Use the Central Limit theorem to obtain an approximate
95% confidence interval for the population mean.
\item (2) Calculate this confidence interval and convert it into an
approximate confidence interval for $\lambda$.
\item (3) Comment briefly on the comparison of this interval with
that obtained in part (ii)(a). 
\end{enumerate}

%%%%%%%%%%%%%%%%%%%%%%%%%%%%%%%%%%%%%%%%%%%%%%%%%%%%%%%%%%%%%%%%%%%%%%%%%%%%%%%%%%%%%%%%%%%%%%%%%%%%%%%%%%%%%%%%%%%
\item  The table below gives data on the lean body mass (the weight without fat) and
resting metabolic rate for twelve women who were the subjects in a study of
obesity. The researchers suspected that metabolic rate is related to lean body
mass.
Lean body mass (kg)
x
Resting metabolic rate
y
36.1 995
54.6 1425
48.5 1396
42.0 1418
50.6 1502
42.0 1256
40.3 1189
33.1 913
42.4 1124
34.5 1052
51.1 1347
41.2 1204
% x = 516.4, x2 = 22741.34
% y = 14821, y2 = 18695125
% xy = 650264.8
\begin{enumerate}[(a)]
    \item (i) Draw a scatter plot of the resting metabolic rate against lean body mass and comment briefly on any relationship.
   \item (ii) Calculate the least squares fit regression line in which resting metabolic rate is modelled as the response and the lean body mass as the explanatory variable. 
   \item (iii) Determine a 95\% confidence interval for the slope coefficient of the model.
State any assumptions made. 
   \item (iv) Use the fitted model to construct 95\% confidence intervals for the mean resting metabolic rate when:
(a) the lean body mass is 50kg
(b) the lean body mass is 75kg 
   \item (v) Comment on the appropriateness of each of the confidence intervals given
in (iv). 
\end{enumerate}

\end{enumerate}

%%%%%%%%%%%%%%%%%%%%%%%%%%%%%%%%%%%%%%%%%%%%%%%%%%%%%%%%%%%%%%%%%%%%%%%%%%%%%%%%%%%%%%%%%%%%%%%%%%%%%%%%%%
15 

%%%%%%%%%%%%%%%%%%%%%%%%%%%%%%%%%%%%%%%%%%%%%%




15 mean $\mu = \frac{4}{\lambda}$

$\bar{X} = \frac{4}{\lambda}$

$\tilde{\lambda} = \frac{4}{\bar{X}}$



%%%%%%%%%%%%%%%%%%%%%%%%%%%%%%%%%%%%%%%%%%%%%%%%%%%%%%%%%

(2) 

\[ L (\lambda) = const.\lambda^{4n} (\product x_i)^3 e{−\lambda \sum x_i} \]

\[ \log L = const. + 4n \log (\lambda) + 3 \log(\product x_i) - \lambda \sum x_i \]


\[ \frac{\partial (\log L)}{\partial \lambda} = \frac{4n}{\lambda} - \sum x_i = 0\]

\[ \hat{\lambda} = \frac{4n}{\sum X_i} = \frac{4}{\bar{X}} \]


%%%%%%%%%%%%%%%%%%%%%%%%%%%%%%%%%%%%%%%%%%%%%%%%%%%%%%%%%
(b)
(1) 

\[M_{X(t)} = E(e^{Xt} = \left( 1- \frac{t}{\lambda}\right)^{-4}\]


from Green book
%%%%%%%%%%%%%%%%%%%%%%%%%%%%%%%%%%%%%%%%%%%%%%%%%%%%%%%%%
\[M_{\sum X_i(t)} = E(e^{\sum X_i(t)} =  \product^n_{i=1}E(e^{\sum X_i(t)} = \left(1 - \frac{t}{\lambda})^{-4n}


\[Y = 2n\lambda \bar{X} = 2 \lambda \sum X_i\]

( )2 2 4 4
( ) = ( Xi t ) = (1 ) n = (1 2 ) n
Y
t
M t E e Σ λ − λ − − t −
λ
28
2 ~ n ∴ nλX χ being gamma(4n, 1/2)
(2) 2 2
0.975,8 0.025,8 ( 2 ) = 0.95 n n P χ < nλX < χ
95\% confidence interval for λ is
2 2
( 0.975,8 , 0.025,8 )
2 2
n n
nX nX
χ χ

\begin{itemize}
    \item (ii) (a) n =10,x = 242 2nx = 4840
2 2
0.975,80 0.025,80 χ = 57.15 ,χ =106.6
\item 95\% confidence interval for λ is
57.15 106.6
( , )
4840 4840
= (0.0118, 0.0220)
(b)
\end{itemize}

(1) Approx. 95% confidence interval for μ is 1.96
s
x
n
±
%--------------------------%
(2)
112
242 1.96 = 242 69.4 = (172.6,311.4)
10
± ±
but
4
μ =
λ
Conversion gives
4 4
( , )
311.4 172.6
= (0.0128, 0.0232)
\begin{itemize}
    \item (3) This is quite close to the exact CI.
    \item So the CLT approximation is quite good even though n=10 is not
very large.
    \item [Note: allow use of t in (ii)(b)
\end{itemize}

112
242 2.262 = (161.9,322.1) (0.0124,0.0247)
10
% ±  ]
16 

\begin{itemize}
    \item (i) x = lean body mass
y = resting metabolic rate
35 45 55
1500
1400
1300
1200
1100
1000
900
Lean body mass (kg)
Resting metabolic rate
Rate against mass
%--------------------------%
Linear relationship between x and y.
% x = 516.4 , y =14821
( )
( )
( )( )
2 2
2 2
= / = 518.927
= / = 389954.92
= / =12467.767
xx
yy
xy
S x x n
S y y n
S xy x y n
−
−
−

%--------------------------%
    \item (ii) Model: y = α + βx
The least squares estimates of α and β:
ˆ 12467.8 = = = 24.03
518.927
xy
xx
S
S
β
ˆ ˆ 14821 516.4 = = (24.03) = 201.2
12 12
α y − βx −
y = 201.2 + 24.03x.
%--------------------------%
    \item (iii) 95\% confidence interval for β: ( ) 10
∠± t (2.5%)s.e. βˆ
2 2
s.e.( ˆ ) = ˆ , ˆ 2 = xy /( 2) = 9040.4
yy
xx xx
S
S n
S S
% σ   β σ  −  −  
%  
∴s.e.(∠) = 4.17
24.03 ± 2.228(4.17) = 24.03 ± 9.299 = (14.7, 33.3)
Assuming normal errors with a constant variance.
%---------------------------%
    \item (iv) 95\% confidence intervals:
( )2
2
10
ˆ 1 (2.5%) ˆ
12 xx
x x
y t
S
%  −  ± σ  + 
%  
%  
(a) For x = 50 kg:
( )2 1 50 43.03
1402.5 2.228 9040.4
12 518.927
%%%%%%%%%%%%%%%%%%%%%%%%%%%%%%%%%%%%%%%%%%%%%%%%%%%%%%%%%%%%%%%%%%%%%%%%%%%%%%%%%%%%%%%%%%%%%%%%%%%%%%%%%%%%%%%%%%%
1402.5 ± 2.228 × 40.0
= (1313, 1492)
(b) For x = 75 kg:
1 (75 43.03)2
2003.1 2.228 9040.4
12 518.927

= 2003.1 ± 2.228 × 136.2
= (1700, 2307)
    \item (v) The confidence interval for x = 50 kg seems OK.
However, the confidence interval at x = 75 kg involves extrapolation. Care
needed!
\end{itemize}

%%%%%%%%%%%%%%%%%%%%%%%%%%%%%%%%%%%%%%%%%%%%%%%%%%%%%%%%%%%%%%%%%%%%%%%%%%%%%%%%%%%%%%%%%%%%%%%%%%%%%%%%%%%
\end{document}
