\documentclass[a4paper,12pt]{article}

%%%%%%%%%%%%%%%%%%%%%%%%%%%%%%%%%%%%%%%%%%%%%%%%%%%%%%%%%%%%%%%%%%%%%%%%%%%%%%%%%%%%%%%%%%%%%%%%%%%%%%%%%%%%%%%%%%%%%%%%%%%%%%%%%%%%%%%%%%%%%%%%%%%%%%%%%%%%%%%%%%%%%%%%%%%%%%%%%%%%%%%%%%%%%%%%%%%%%%%%%%%%%%%%%%%%%%%%%%%%%%%%%%%%%%%%%%%%%%%%%%%%%%%%%%%%

\usepackage{eurosym}
\usepackage{vmargin}
\usepackage{amsmath}
\usepackage{graphics}
\usepackage{epsfig}
\usepackage{enumerate}
\usepackage{multicol}
\usepackage{subfigure}
\usepackage{fancyhdr}
\usepackage{listings}
\usepackage{framed}
\usepackage{graphicx}
\usepackage{amsmath}
\usepackage{chngpage}

%\usepackage{bigints}
\usepackage{vmargin}

% left top textwidth textheight headheight

% headsep footheight footskip

\setmargins{2.0cm}{2.5cm}{16 cm}{22cm}{0.5cm}{0cm}{1cm}{1cm}

\renewcommand{\baselinestretch}{1.3}

\setcounter{MaxMatrixCols}{10}

\begin{document}

\begin{enumerate}
\item

5 When comparing the mean premiums for policies issued by two companies, a twosample
t test is performed assuming equal population variances. The sample sizes and
sample variances are given by
2
n1 25 , s1 139.7
2
n2 30 , s2 76.6
Perform an appropriate F test at the 5% level to investigate the validity of the equal
variance assumption. [3]
%%%%%%%%%%%%%%%%%%%%%%%%%%%%%%%%%%%%%%%%%%%%%%%%%%%%%%%%%%%%%%%%%%%%%%%%%%%%%%%%%%%%%%%%%%%%%%%%%%%%%%
\item A 2 2 contingency table was set up to investigate whether or not two classification
criteria are independent and resulted in the following data:
I II
A 22 28 50
B 28 22 50
50 50 100
Calculate the observed 2 test statistic and state an appropriate conclusion concerning
the independence of the two criteria. 
%%%%%%%%%%%%%%%%%%%%%%%%%%%%%%%%%%%%%%%%%%%%%%%%%%%%%%%%%%%%%%%%%%%%%%%%%%%%%%%%%%%%%%%%%%%%%%%%%%%%%%
\item A claim size distribution is modelled using a simple distribution with density of the
form
(100 ) , 0 100
( )
0 , otherwise
k x x
f x
(i) Verify that k = 0.0002. [1]
(ii) Determine the mean of this claim size distribution. [2]
(iii) Calculate the probability that an individual claim size is greater than 50. [1]
(iv) Calculate the probability that an individual claim size is less than 60 given that
it is greater than 50. [3]
%%%%%%%%%%%%%%%%%%%%%%%%%%%%%%%%%%%%%%%%%%%%%%%%%%%%%%%%%%%%%%%%%%%%%%%%%%%%%%%%%%%%%%%%%%%%%%%%%%%%%%
\item Let X1, X2 , , X100 be independent random variables, each having a gamma(4,1)
distribution (and hence with mean 4 and variance 4).
Calculate an approximate value for the probability that the sum of the variables
assumes a value which exceeds 425. [3]
%%%%%%%%%%%%%%%%%%%%%%%%%%%%%%%%%%%%%%%%%%%%%%%%%%%%%%%%%%%%%%%%%%%%%%%%%%%%%%%%%%%%%%%%%%%%%%%%%%%%%%
\end{enumerate}
5
2
1
2
2
139.7
1.82
76.6
s
s
F24,29 critical value at 5% is 2.154 (two-sided test)
accept H0 : equal variances at the 5% level.
Examiners Comments: A test for equality of variances as asked for here is a twosided
test, but since we always use the observed ratio with the larger sample variance
on the numerator, we look at the upper tail of the reference F distribution. For a 5%
test we look at the upper 2.5% tail, not the upper 5% tail, as some candidates did.
%%%%%%%%%%%%%%%%%%%%%%%%%%%%%%%%%%%%%
Page 3
6 Expected frequencies are all 25
2
2 3
4 1.44
25
P-value = 2
P( 1 1.44) 1 0.77 0.23
there is no evidence to reject the independence of the two criteria.
Examiners Comments: Although not covered in the Core Reading, the use of Yates
correction in this situation (in which the 2 statistic has only 1 degree of freedom) is
acceptable. Using it gives 2 = 1, a P-value of 0.32, and the same conclusion.
7 (i) We require k such that
100
0
k(100 x)dx 1
100 2 100
0 0
10000
(100 ) 100 10000 5000
2 2
x
k x dx k x k k
1
0.0002
5000
k
[or could be argued geometrically]
(ii) Mean =
100 3 100
2
0 0
(0.0002)(100 ) 0.0002 50
3
x
x x dx x
=
1000000
0.0002{500000 }
3
= 33.33
(iii) P(X > 50) =
100 2 100
50 50
0.0002(100 ) 0.0002 100
2
x
x dx x
1002 502
0.0002{100(50) } 0.25
2
[or could be argued geometrically]
(iv)
(50 60)
( 60 | 50)
( 50)
P X
P X X
P X
2 60
50
(50 60) 0.0002 100
2
x
P X x
602 502
0.0002{100(10) } 0.09
2
%%%%%%%%%%%%%%%%%%%%%%%%%%%%%%%%%%%%%
Page 4
0.09
( 60 | 50) 0.36
0.25
P X X
[or could be argued geometrically]
8
100
1
i
i
S X has mean 400 and variance 400
By CLT, S ~ N(400,400) approximately
P(S > 425) P[Z > (425 400)/20] = P(Z > 1.25) = 0.106
\end{document}
