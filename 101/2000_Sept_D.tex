\documentclass[a4paper,12pt]{article}

%%%%%%%%%%%%%%%%%%%%%%%%%%%%%%%%%%%%%%%%%%%%%%%%%%%%%%%%%%%%%%%%%%%%%%%%%%%%%%%%%%%%%%%%%%%%%%%%%%%%%%%%%%%%%%%%%%%%%%%%%%%%%%%%%%%%%%%%%%%%%%%%%%%%%%%%%%%%%%%%%%%%%%%%%%%%%%%%%%%%%%%%%%%%%%%%%%%%%%%%%%%%%%%%%%%%%%%%%%%%%%%%%%%%%%%%%%%%%%%%%%%%%%%%%%%%

\usepackage{eurosym}
\usepackage{vmargin}
\usepackage{amsmath}
\usepackage{graphics}
\usepackage{epsfig}
\usepackage{enumerate}
\usepackage{multicol}
\usepackage{subfigure}
\usepackage{fancyhdr}
\usepackage{listings}
\usepackage{framed}
\usepackage{graphicx}
\usepackage{amsmath}
\usepackage{chngpage}

%\usepackage{bigints}
\usepackage{vmargin}

% left top textwidth textheight headheight

% headsep footheight footskip

\setmargins{2.0cm}{2.5cm}{16 cm}{22cm}{0.5cm}{0cm}{1cm}{1cm}

\renewcommand{\baselinestretch}{1.3}

\setcounter{MaxMatrixCols}{10}

\begin{document}
\begin{enumerate}
% Faculty of Actuaries Institute of Actuaries
% EXAMINATIONS
% 18 September 2000 (pm)
% Subject 101 — Statistical Modelling

%%%%%%%%%%%%%%%%%%%%%%%%%%%%%%%%%%%%%%%%%%%%%%%%%%%%%%

\item 13 Consider a one-way analysis of variance for comparing k treatments using the
same number ni = r responses for each treatment. The model is
\[Yij = μ + τi + eij : i = 1, 2, ..., k; j = 1,2, ..., r\]
where the errors eij are independent $N(0, \sigma 2)$ random variables and where $\sum τi = 0$.
Show that the parameter estimates ˆμ = Y⋅⋅ and ˆi τ = i Y⋅ −Y⋅⋅ are unbiased and
that their variances are given by
V(μˆ ) =
2
kr
\sum  and (ˆ ) i V τ =
( 1) 2
.
k
kr
− \sum  
%%%%%%%%%%%%%%%%%%%%%%%%%%%%%%%%%%%%%%%%%%%%%%%%%%%%%%
\item 14 Let X be a random variable with cumulative distribution function
\[FX(x) = P(X < x) = 1 − exp(−x2/\theta) , x > 0 , FX(x) = 0 , x ≤ 0\]
and let (X1, X2, … , Xn) be a random sample from X.
(i) By considering $P(Y < y)$, show that Y = X2 has an exponential distribution.

(ii) (a) Show that the maximum likelihood estimator of $\theta$ is given by
\[\hat{\theta}
= 1 2
Xi
n
\sum \]
(b) Show that $\hat{\theta}$ is an unbiased estimator of $\theta$ which attains the
Cramer-Rao lower bound on variance.
(c) Using moment generating functions, show that 22
2 ˆ ~ n
n \theta χ
\theta
. [11]
(iii) The above distribution of X is to be used as a model for claim amounts in
a particular situation. A random sample of 50 such claim amounts (in
appropriate units) gives the following summary:
$\sum x2 = 485.7518$
(a) Calculate the maximum likelihood estimate of $\theta$.
(b) Using the result of (ii)(c) above, calculate an exact 95\% confidence interval for $\theta$. 
%%%%%%%%%%%%%%%%%%%%%%%%%%%%%%%%%%%%%%%%%%%%%%%%%%%%%%%%%%%%%%%%%%%%%%%%%%%%%%%%%%%%%%%%%%%%%%%%%%%%%%%%%%%%

\item  A manufacturer uses a certain type of electrical component from supplier A in high quality computing equipment and uses similar components from supplier B in inexpensive playstations. Previous investigations have shown that supplier A
produces components whose resistances are normally distributed about a mean of 100 units and with a standard deviation of 0.1 units. Similarly, the resistances of supplier B’s components are normally distributed about a mean of 100.5 units
and with a standard deviation of 0.5. Components are supplied in large batches and are externally identical.
A batch known to come entirely from one supplier has no labels. The value of X , the mean resistance from a random sample of components, is to be used to decide whether the batch has come from supplier A.
The hypotheses to be tested are:
H0 : components come from supplier A i.e. X ~ N(100, 0.12)
v. H1 : components come from supplier B i.e. X ~ N(100.5, 0.52) .

\begin{enumerate}
    \item  (a) If the manufacturer insists that the probabilities of the type I and
type II errors are each to be restricted to at most 5\%, show that at
least 4 components from the batch have to be examined.
(b) Using a sample size of 4, calculate the power of the test if the
probability of the type I error is reduced to 1\%. 
\item  Suppose that a sample of 10 components from a batch has mean
resistance x = 100.1.
Calculate the probability-value of this observed mean. 
\end{enumerate}


\end{enumerate}
%%%%%%%%%%%%%%%%%%%%%%%%%%%%%%%%%%%%%%%%%%%%%%%%%%%%%%
\newpage


$V(\hat{\mu} ) = V(\bar{Y}_{\ldot \ldot} ) =

\frac{\sigma^2}{kr}$


as $\bar{Y}_{\ldot \ldot}$ is mean of kr r.v.’s each with $var(\sigma2)$.

\begin{eqnarray*}
V( \hat{\tau}}_i) 
&=& V(\bar{Y}_{i \ldot} −\bar{Y}_{\ldot \ldot}) \\
&=& V(\bar{Y}_{i \ldot}) + V (\bar{Y}_{\ldot \ldot}) − Cov (\bar{Y}_{\ldot}, \bar{Y}_{\ldot \ldot}) \\
&=& \frac{\sigma^2}{r} + \frac{\sigma^2}{kr} - 2 \frac{1}{r} \frac{1}{kr} r \sigma^2 \\
&=& \frac{\sigma^2}{r} + \frac{\sigma^2}{kr} - 2 \left(\frac{\sigma^2}{kr} \right) \\
&=& \frac{\sigma^2}{r} - \frac{\sigma^2}{kr} \\
&=& \frac{(k-1)\sigma^2}{kr} \\
\end{eqnarray}

%%%%%%%%%%%%%%

Alternative method:
( ) i V \bar{Y}_{\ldot} −\bar{Y}_{\ldot \ldot} =


V\left(  \left( 1 - \frac{1}{k}  \right) \bar{Y}_{i}  - \frac{1}{k} \sum_{j \neq i} \bar{Y}_{j} \right)
&=& \left(1 - \frac{1}{k} \right) \frac{\sigma^2}{r}  + \left(- \frac{1}{k} \right)^2 \frac{(k-1)\sigma^2}{r} \\

&=& \frac{\sigma^2}{k^2r} \left[  (k-1)^2 + (k-1) \right] \\
&=& \frac{(k-1)\sigma^2}{kr} \\
\end{eqnarray}





\newpage

14 (i) \[FY(y) = P(Y < y) = P(X2 < y) = P(X < √y) = 1 − exp(−y/\theta)\]
which is the cdf of an exponential r.v. with mean $\theta$.
(ii) (a) fX(x) = 2(x / \theta)exp(−x2 / \theta)
So \[L(\theta) = k1 \theta−n exp[−(\sum x2) / \theta \] so logL = k2 − nlog \theta − (\sum x2) / \theta
dlog L / d\theta = −n/\theta + (\sum x2)/\theta2 = 0  \hat{\theta} = 2 1
i X
n
\sum 
(b) From (i) E(X2) = E(Y) = \theta and V(X2) = V(Y) = \theta2
So 
\begin{eqnarray*}E(\hat{\theta} ) &=& (1 / n) ( 2 ) i \sum E X \\ &=& (1 / n) n\theta \\ &=& \theta\\ 
\end{eqnarray*}, and

\begin{eqnarray*}
V(\hat{\theta} ) &=& (1 / n2) ( 2 ) i \sum V X \\ &=& (1 / n2) n\theta^2 \\ &=& \theta^2 / n
\end{eqnarray*}
%%%%%%%%%%%%%%%%%%%%%%%%%%%%%%%%%%%%%%%%%%%%%%%%%%%%%%%%%
Page 8
Now,\[ d2log L / d\theta2 = n\theta−2 − 2(\sum x2) \theta−3\]
so \[ −E(d2log L / d\theta2) = −n\theta−2 + 2\theta−3(n\theta) = n / \theta2 = 1/V\]
So $\hat{\theta}$ is unbiased and attains the C–R bound.
(c) Y = X2 has mgf $(1 − \theta t)−1 so n\hat{\theta} = \sum Yi has mgf (1 − \thetat)−n$
So $2n\hat{\theta} / \theta$ has mgf $[1 − \theta(2t / \theta)]−n = (1 − 2t)−n$
which is mgf of χ2 with 2n d.f.
(iii) (a) $\hat{\theta} = 485.9028/50 = 9.718$
(b) $P(74.22 < 100\hat{\theta} / \theta < 129.6) = 0.95$ from tables of χ2 with 100 d.f.

95\% CI for \theta is $(100\hat{\theta} / 129.6 , 100\hat{\theta} / 74.22)$

i.e. (971.8056 / 129.6 , 971.8056 / 74.22) i.e. (7.50,13.1)
%%%%%%%%%%%%%%%%%%%%%%%%%%%%%%%%%%%%%%%%%%%%%%%%%%%%%%%%%
15 (i) (a) Let critical value be c, such that we reject H0 for X > c for a
sample of size n.
\begin{itemize}
    \item P(type I error) = P( X > c ) where X ~ N(100, 0.12 / n)
∴ $(c − 100) / (0.1 / \sqrt{n}) = 1.645$ …… (*)
    \item P(type II error) = P( X < c ) where X ~ N(100.5, 0.52 / n)
∴ $(c − 100.5) / (0.5 / \sqrt{n}) = −1.645$ …… (**)
\end{itemize}

Solving (*) and (**) for n gives $100 + 1.645(0.1/\sqrt{n})$ = $100.5 − 1.645(0.5 / \sqrt{n})$
 $\sqrt{n} = 1.645 × 0.6/0.5 = 1.974$ $n = 3.9$
∴ sample of size n = 4 should be examined
\begin{itemize}
    \item (b) Let critical value be c
\item P(type I error) = 0.01  (c − 100) / (0.1 / √4) = 2.326
∴ c = 100 + 2.326(0.1 / √4) = 100.1163
\item Power = 1 − P(type II error) = 1 − P[Z < (100.1163 − 100.5) / (0.5 / √4)]
= 1 − P(Z < −1.535) = P(Z < 1.535) = 0.9376 i.e. Power = 93.8%
\item (ii) P-value = P( X > 1001. ) where X ~ N(100, 0.12 / 10)
= P[ Z > (100.1 − 100) / (0.1 / √10)] = P(Z > 3.162) = 1 − 0.9992
= 0.0008 i.e. 0.08%
\end{itemize}

%%%%%%%%%%%%%%%%%%%%%%%%%%%%%%%%%%%%%%%%%%%%%%%%%%%%%%%%%
Page 9
Q15 Comment: Most candidates understood the concepts of testing errors, but
were unable to apply that knowledge.

%%%%%%%%%%%%%%%%%%%%%%%%%%%%%%%%%%%%%%%%%%%%%%%%%%%%%%%%%%%%%%%%%%%%%%%%%%%%%%%%%%%%%%%%%%%%%
16 At the end of the skiing season the tourist board in a mountain region examines
the records of ten ski resorts. For each one it obtains the total number (y,
thousands) of visitor-days during the season as a measure of the resort’s
popularity, and the ski-lift capacity (x, thousands), being the maximum number
of skiers that can be transported per hour. The resulting data are given in the
following table:
\begin{verbatim}
    Resort: A B C D E F G H I J
Lift capacity x: 1.9 3.3 1.2 4.2 1.5 2.2 1.0 5.6 1.9 3.8
Visitor-days y: 15.1 22.6 9.2 37.5 8.9 21.1 5.8 41.0 9.2 32.4
\sum x = 26.6, \sum x2 = 91.08, \sum y = 202.8, \sum y2 = 5603.12, \sum xy = 707.58
\end{verbatim}

\begin{enumerate}[(a)]
\item Draw a scatterplot of y against x and comment briefly on any relationship
between a resort’s popularity and its ski-lift capacity. 
\item Calculate the correlation coefficient between x and y and comment briefly
in the light of your comment in part (i). 
\item Calculate the fitted linear regression equation of y on x. 
\item (a) Calculate the “total sum of squares” together with its partition into
the “regression sum of squares” and the “residual sum of squares”.
(b) Use the values in part (iv)(a) above to calculate the coefficient of
determination R2 and comment briefly on its relationship with the
correlation coefficient calculated in part (ii).
(c) Use the values in part (iv)(a) above to calculate an estimate of the
error variance $\sigma^2$ in the usual linear regression model. 
\item Suppose that a resort can increase its ski-lift capacity by 500 skiers per
hour.
\end{enumerate}
Estimate the increase in the number of visitor-days it can expect in a
season, and specify a standard error for this estimate. 
\newpage
16 (i)
Seems to be a strong increasing linear relationship.

%%%%%%%%%%%%%%%%%%%%%%%%%%%%%%%%%%%%%%%%%%%%%%%%%%%%%%%%%%%%%%%%%%%%%%%%%%%%%%%%%%%%%%%%%%%%%%%%

%%%%%%% Question 14

% ii 

\begin{itemize}
\item
${ \displaystyle Sxx = 91.08 − \frac{26.6^2}{10} = 20.324  }$
\item ${ \displaystyle Syy = 5603.12 − \frac{202.8^2}{10} = 1490.336  }$
\item ${ \displaystyle Sxy = 707.58 −  − \frac{(26.6)(202.8)}{10} = 168.132  }$
\item ${ \displaystyle r = \frac{168.132}{\sqrt{(20.324)(1490.336)}}= 0.966}$
\end{itemize}
Large and positive agreeing with comment above.
Large and positive agreeing with comment above.

%%%%%%%%%%%%%%%%%%%%%%%%%%%%%%%%%%%%%%%%%%%%%%%%%%%%%%%%%
Page 10
\begin{itemize}
    \item 
(iii) y = $\alpha$ˆ + βˆ x where
ˆβ
=
168.132
20.324
= 8.2726
−1.73 + 8.273x
ˆ$\alpha$ =
202.8 26.6
8.2726
10 10
  −  
 
= −1.73
\item (iv) (a) SSTOT = Syy = 1490.336
SSRES = Syy −
2
xy
xx
S
S
(formula from green book)
= 1490.336 −
(168.132)2
20.324
= 99.450
∴ SSREG = 1390.886\\
(b) R2 =
SSREG
SSTOT
= 0.933 = r2 from (ii)\\
(c) \sum ˆ 2 =
1
8
(SSRES) = 12.43
\item (v) If x increases by 0.5 (500 in ‘000’s), then expected increase in y is 0.5βˆ
= 0.5(8.2726) = 4.136 or 4136 visitor days
2
(0.5ˆ ) = (0.5)2 ( ˆ ) = 0.25 ˆ
xx
Var Var
S
\sum 
β β
=
12.43
0.25 = 0.153
20.324
∴ s.d. (0.5βˆ ) = 0.391
∴ standard error of estimate is 391
\item Q16 Comment: Nearly all candidates scored well on parts (i) – (iv), but most used
an inappropriate expression for the standard error of 0.5βˆ .
\end{itemize}
\end{document}
