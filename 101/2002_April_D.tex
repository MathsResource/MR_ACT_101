\documentclass[a4paper,12pt]{article}

%%%%%%%%%%%%%%%%%%%%%%%%%%%%%%%%%%%%%%%%%%%%%%%%%%%%%%%%%%%%%%%%%%%%%%%%%%%%%%%%%%%%%%%%%%%%%%%%%%%%%%%%%%%%%%%%%%%%%%%%%%%%%%%%%%%%%%%%%%%%%%%%%%%%%%%%%%%%%%%%%%%%%%%%%%%%%%%%%%%%%%%%%%%%%%%%%%%%%%%%%%%%%%%%%%%%%%%%%%%%%%%%%%%%%%%%%%%%%%%%%%%%%%%%%%%%

\usepackage{eurosym}
\usepackage{vmargin}
\usepackage{amsmath}
\usepackage{graphics}
\usepackage{epsfig}
\usepackage{enumerate}
\usepackage{multicol}
\usepackage{subfigure}
\usepackage{fancyhdr}
\usepackage{listings}
\usepackage{framed}
\usepackage{graphicx}
\usepackage{amsmath}
\usepackage{chngpage}

%\usepackage{bigints}
\usepackage{vmargin}

% left top textwidth textheight headheight

% headsep footheight footskip

\setmargins{2.0cm}{2.5cm}{16 cm}{22cm}{0.5cm}{0cm}{1cm}{1cm}

\renewcommand{\baselinestretch}{1.3}

\setcounter{MaxMatrixCols}{10}

\begin{document}

\begin{enumerate}
%%%%%%%%%%%%%%%%%%%%%%%%%%%%%%%%%%%%%%%%%%%%%%%%%%%%%%%%%%%%%%%%%%%%%%%%%%%%%%%%%%%%%%%%%%%%%%%%%%%%%%%%%%%%%%%%%%%%%%%%%%%%%%%%%%%%%%%%%%
\item 12 A simple model for the movement of a stock price is such that, independently in each
time period, the stock either:
goes up with probability (1 )
4
% ;
stays the same with probability (5 2 )
8
%   ;
goes down with probability (1 )
8
% .
\begin{enumerate}[(a)]

\item Determine the range of admissible values of the parameter $\theta$ . 
\item (a) Calculate the probability that the stock goes down in one time period, in the case  = 0.1.
(b) Calculate the probability that the stock stays the same for two consecutive time periods, in the case  = 0 .
(c) Calculate the probability that, in four time periods, the stock goes up twice and stays the same twice, in the case  % =  0.2 . [4]
\item Data are collected for 80 consecutive time periods and yield the following
observed frequencies:

\begin{center}
\begin{tabular}{|c|c|c|c|}
change in stock & up & same &  down\\
no. of time periods & 24 & 35 & 21 \\
\end{tabular}
\end{center}

(a) (1) Write down an expression for L() , the likelihood of these
data, and show that
log L( ) 0 % 
%  
% 
reduces to the quadratic equation
% 51202  468 95  0
(2) Explain why one of the roots of this quadratic yields the
maximum likelihood estimate of  and hence determine this
estimate. 
    \item 
\end{enumerate}
(b) (1) Calculate the expected frequencies using the model with the
maximum likelihood estimate of .
(2) Hence perform a $\chi^2$ goodness of fit test of the model and state
your conclusion clearly. 
(c) Comment briefly on what additional information would be needed for
these data in order to investigate the validity of the assumption of
independence used in this model, and comment briefly on how the
validity might be checked. 
%%%%%%%%%%%%%%%%%%%%%%%%%%%%%%%%%%%%%%%%%%%%%%%%%%%%%%%%%%%%%%%%%%%%%%%%%%%%%%%%%%%%%%%%%%%%%%%%%%%%%%%%%%%%%%%%%%%%%%%%%%%%%%%%%%%%%%%%%%
\item  Six insurance companies were being compared with regard to premiums being
charged for house contents insurance for houses in a particular postcode region.
Independent random samples of five policies from each company are examined and
the premiums (in £) were recorded.

\begin{verbatim}
Company A B C D E F
151 152 175 149 123 145
168 141 155 148 132 131
128 129 162 137 142 155
167 120 186 138 161 172
134 115 148 169 152 141
Totals 748 657 826 741 710 744
% yij = 4, 426 ; yi2j = 661,796
\end{verbatim}


\begin{enumerate}[(a)]
\item Compute an ANOVA table for these data, and show that there are no
significant differences, at the 5\% level, between mean premiums being
charged by each company. 
\item State the assumptions required for the above analysis of variance, and, by
drawing a suitable diagram of these data, comment briefly on the validity of
these assumptions. 
\item Calculate a 95\% confidence interval for the underlying common standard
deviation % , using 2
SSR

as a pivotal quantity with a %  2  distribution. 
\item A colleague points out that company C has the largest mean premium of
£165.20 and that Company B has the smallest mean premium of £131.40 and
suggests performing a t-test to compare these two companies.
(a) Perform this t-test, using the estimate of variance from the ANOVA
table, and in particular show that there is a significant difference at the
1\% level.
(b) Your colleague states that there is therefore a significant difference
between the six companies.
Discuss the apparent contradiction with your conclusion and
explain the flaw in your colleague’s argument. 
\end{enumerate}
%%%%%%%%%%%%%%%%%%%%%%%%%%%%%%%%%%%%%%%%%%%%%%%%%%%%%%%%%%%%%%%%%%%%%%%%%%%%%%%%%%%%%%%%%%%%%%%%%%%%%%%%%%%%%%%%%%%%%%%%%%%%%%%%%%%%%%%%%%
\item The table below gives the numbers of deaths nx in a year in groups of women aged x
years. The exposures of the groups, denoted Ex , are also given (the exposure is
essentially the number of women alive for the year in question). The values of the
death rates yx , where yx = nx/Ex , and the log(death rates), denoted wx , are also given.
age x number of deaths nx exposure Ex yx = nx/Ex wx = logyx
% 70 30 426 0.07042 2.6532
% 71 38 471 0.08068 2.5173
% 72 38 454 0.08370 2.4805
% 73 53 482 0.10996 2.2077
% 74 59 445 0.13258 2.0205
% 75 61 423 0.14421 1.9365
% 76 82 468 0.17521 1.7417
% 77 96 430 0.22326 1.4994
% x = 588 x2 = 43260 w = 17.0568 w2 = 37.5173 xw = 1246.7879
\begin{enumerate}[(a)]
\item A scatter plot of yx against x is shown below.
Draw a scatter plot of wx against x and comment briefly on the two scatter
plots and the relationships displayed. 
\item (a) Calculate the least squares fit regression line in which wx is modelled
as the response and x as the explanatory variable.
(b) Draw the fitted line on your scatter plot of wx against x.
(c) Calculate a 95\% confidence interval for the slope coefficient of the
regression model of wx on x, adopting the assumptions of the usual
“normal regression model”.
(d) Calculate the fitted values for the number of deaths for the group aged
71 years and the group aged 76 years. 
\item Explain briefly the relationship between the fitting procedure used in part \item
and a model which states that the number of deaths Nx is a random variable
with mean Exbcx for some constants b and c. 
\end{enumerate}
0.00
0.05
0.10
0.15
0.20
0.25
69 70 71 72 73 74 75 76 77 78
x
y x

%%%%%%%%%%%%%%%%%%%%%%%%%%%%%%%%%%%%%%%%
12 
\begin{itemize}
\item 0 1 1 1 , 3

\begin{itemize}
\item ${\displaystyle 0 \leq \frac{1}{4} -\theta \leq 1 }$  ${\displaystyle  \theta \leq \frac{1}{4}, \theta \geq -\frac{3}{4} }$ -

\item ${\displaystyle 0 \leq \frac{5}{8} +2\theta \leq 1 }$  ${\displaystyle  \theta \leq \frac{3}{16}, \theta \geq -\frac{5}{16} }$ -

\item ${\displaystyle 0 \leq \frac{1}{8} -\theta \leq 1 }$  ${\displaystyle  \theta \leq \frac{1}{8}, \theta \geq -\frac{7}{8} }$ -

\end{itemize}
%%%%%%%%%%%%%%%%%%%%%%%%%%
combining these

\[ - \frac{5}{16} \leq \theta \leq \frac{1}{8}\]
\item
(a) 	 = 0.1 : P(down in one period) = 1 0.1= 0.025
8 
(b) 	 = 0 : P(same in two periods) = [P(same)]2 = (5)2 = 0.391
8
(c) % 	 = 0.2 : P(up) = 0.45, P(same) = 0.225
= 4! (0.45)2 (0.225)2 = 0.062
2!2!
p
\item
(a) (1) ( ) = (1 )24 (5 2 )35 (1 )21
4 8 8
% L     
log = 24log(1 ) 35log(5 2 ) 21log(1 )
4 8 8
% L     
log = 24 2(35) 21 1 5 2 1
4 8 8
 L

equate to zero % 
24(5 2 )(1 ) 70(1 )(1 ) 21(1 )(5 2 ) = 0
8 8 4 8 4 8
    \item 
\end{itemize}
%%%%%%%%%%%%%%%%%%%%%%%%%%%%%%%%%%%%%%%%%%%%%%%%%%%%%%%%%%%%%%%%%%%%%%%%%%%%%%%%%%%%%%%%%%%%%%%%%%%%%%%%%%%%%%%%%%%%%%%%%%%%%%%%%%%%%%%%%%
%  102402  936 190 = 0  51202  468 95 = 0
(2)
468 4682 4(5120)(95) 468 1471.2661 = =
2(5120) 10240


% = + 0.189 or 0.0980
+ 0.189 is inadmissible, 0.0980 is admissible
% MLE ˆ =  0.0980
(b) (1) with ˆ =  0.0980 , estimated probabilities are
P(up) = 0.3480, P(same) = 0.4290, P(down) = 0.2230
Multiply by 80 for expected frequencies:
up : 27.84, same : 34.32, down : 17.84
(2)
% O e (o e)2/e
24 27.84 0.530
35 34.32 0.013
21 17.84 0.560

2 = 1.103

% 2 = 1.103 on (3  1  1) = 1 d.f.
this is well below the 5% point for 2
% 1 from tables (3.841)
%  large P-value, and so no evidence against the model.
the model fits these data well.
(c) We need information on the order of the up/same/down’s;
and some method of investigating whether the up/same/down’s are distributed
randomly.
%%%%%%%%%%%%%%%%%%%%%%%%%%%%%%%%%%%%%%%%%%%%%%%%%%%%%%%%%%%%%%%%%%%%%%%%%%%%%%%%%%%%%%%%%%%%%%%%%%%%%%%%%%%%%%%%%%%%%%%%%%%%%%%%%%%%%%%%%%
13 
\begin{itemize}

\item SST =
44262 661796 = 8813.47
30 
SSB =
2
1 (7482 6572 8262 7412 7102 7442 ) 4426 = 3046.67
5 30

%  SSR = 8813.47  3046.67 = 5766.80
Source df SS MS F
Treatments 5 3046.7 609.3 2.54
Residual 24 5766.8 240.3
Total 29 8813.5
From tables F5,24(5%) = 2.621
Observed F < 2.621. Therefore not significant at 5% level.
\item Assumptions are that, for each company, such premiums are normally
distributed with the same variance.
              .   .           .         ..
  -------+---------+---------+---------+---------+---------
A
      .  .     .       .      .
  -------+---------+---------+---------+---------+---------
B
                            .   .    .        .      .
  -------+---------+---------+---------+---------+---------
C
                    ..      :             .
  -------+---------+---------+---------+---------+---------
D
           .     .      .     .     .
  -------+---------+---------+---------+---------+---------
E
                .      .  .     .           .
  -------+---------+---------+---------+---------+---------
F
       120       135       150       165       180       195
normality and equality of variance both seem reasonable.
%%%%%%%%%%%%%%%%%%%%%%%%%%%%%%%%%%%%%%%%%%%%%%%%%%%%%%%%%%%%%%%%%%%%%%%%%%%%
\item Here 2
2 30 6 24 SSR ~


% 2 P(12.40  SSR  39.36) = 0.95
% 
%  95% CI for 2 is 2
% 39.36 12.40
% SSR SSR
%   
Data gives SSR = 5766.8
%  146.51 < 2 < 465.06
and so 95% CI for  is 12.1 <  < 21.6
\item (a) Using ˆ 2 = 240.3 from the ANOVA
For comparing B and C: = 165.2 131.4 = 33.8 = 3.45
1 1 9.80 240.3( )
5 5
t 
% 
t24(0.5%) = 2.797 from tables.
Observed t > 2.797. Therefore significant at 1% level (two-sided).
(b) There is no contradiction.
It is wrong to pick out the largest and the smallest of a set of treatment
means, test for significance, and then draw conclusions about the set.
% Even if H0 : “i’s all equal” is true, the largest and smallest sample
means would, of course, differ.
\end{itemize}
%%%%%%%%%%%%%%%%%%%%%%%%%%%%%%%%%%%%%%%%%%%%%%%%%%%%%%%%%%%%%%%%%%%%%%%%%%%%%%%%%%%%%%%%%%%%%%%%%%%%%%%%%%%%%%%%%%%%%%%%%%%%%%%%%%%%%%%%%%
14 

\begin{itemize}
\item Plot
Comments: yx v x relationship is not linear
wx v x relationship appears to be linear, strong
\item (a) Sxx = 43260 – 5882/8 = 42
% Sxw = 1246.7879 – (588  17.0568)/8 = 6.8869
%  slope = 6.8869/42 = 0.1640
% so intercept = 17.0568/8  (6.8869/42)  (588/8) = 14.1842
% Fitted line is wx = 14.184 + 0.1640x
(b) Line on plot
-3.0
-2.5
-2.0
-1.5
-1.0
70 71 72 73 74 75 76 77
x
w x
-3.0
-2.5
-2.0
-1.5
-1.0
70 71 72 73 74 75 76 77
x
w x
%%%%%%%%%%%%%%%%%%%%%%%%%%%%%%%%%%%%%%%%%%%%%%%%%%%%%%%%%%%%%%%%%%%%%%%%%%%%%%%%%%%%%%%%%%%%%%%%%%%%%%%%%%%%%%%%%%%%%%%%%%%%%%%%%%%%%%%%%%
% (c) Sww = 37.5173  ((17.0568)2/8) = 1.1505
% estimate of error variance = [1.1505  6.88692/42]/6 = 0.003538
%  standard error of slope coefficient = (0.003538/42)0.5 = 0.00918
% t6 (0.975) = 2.447
%  95\% CI for slope coefficient is 0.1640  (2.447  0.00918)
% i.e. 0.1640  0.0225 or (0.141, 0.187)
% (d) Fitted value of w71 = 2.5421  fitted value of
% y71 = exp(2.5421) = 0.0787
%  fitted value of n71 = 471  0.0787 = 37.1
% Fitted value of w76 = 1.7222  fitted value of
% y71 = exp(1.7222) = 0.1787
%  fitted value of n76 = 468  0.1787 = 83.6
% \item E(Nx) = Exbcx  E(Nx/Ex) = bcx  E(Yx) = bcx
%  logE(Yx) = 
% x where 
% = logb ,  = logc.
The procedure above is a linear regression analysis of wx = logyx on x , which
is a simple and approximate approach to fitting the stated model .
Note: The method used in  is based on E(Wx) = E(logYx) = c + d x, whereas
the model stated in (Q) is based on logE(Yx) = c + dx.
%%%%%%%%%%%%%%%%%%%%%%%%%%%%%%%%%%%%%%%%%%%%%%%%%%%%%%%%%%%%%%%%%%%%%%%%%%%%%%%%%%%%%%%%%%%%%%%%%%%%%%%%%%%%%%%%%%%%%%%%%%%%%%%%%%%%%%%%%%
    \item 
\end{itemize}
\end{document}
