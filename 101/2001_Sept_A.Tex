\documentclass[a4paper,12pt]{article}

%%%%%%%%%%%%%%%%%%%%%%%%%%%%%%%%%%%%%%%%%%%%%%%%%%%%%%%%%%%%%%%%%%%%%%%%%%%%%%%%%%%%%%%%%%%%%%%%%%%%%%%%%%%%%%%%%%%%%%%%%%%%%%%%%%%%%%%%%%%%%%%%%%%%%%%%%%%%%%%%%%%%%%%%%%%%%%%%%%%%%%%%%%%%%%%%%%%%%%%%%%%%%%%%%%%%%%%%%%%%%%%%%%%%%%%%%%%%%%%%%%%%%%%%%%%%

\usepackage{eurosym}
\usepackage{vmargin}
\usepackage{amsmath}
\usepackage{graphics}
\usepackage{epsfig}
\usepackage{enumerate}
\usepackage{multicol}
\usepackage{subfigure}
\usepackage{fancyhdr}
\usepackage{listings}
\usepackage{framed}
\usepackage{graphicx}
\usepackage{amsmath}
\usepackage{chngpage}

%\usepackage{bigints}
\usepackage{vmargin}

% left top textwidth textheight headheight

% headsep footheight footskip

\setmargins{2.0cm}{2.5cm}{16 cm}{22cm}{0.5cm}{0cm}{1cm}{1cm}

\renewcommand{\baselinestretch}{1.3}

\setcounter{MaxMatrixCols}{10}

\begin{document}
\begin{enumerate}

%%%%%%%%%%%%%%%%%%%%%%%%%%%%%%%%%%%%%%%%%%%%%%%%%%%%%%%%%%%%%%%%%%%%%%%%%%%%%%%%%%%%%%%%%%%%%%%%%%%%%%%%%%%%%%%%%
\item 1 Data were collected on 100 consecutive days for the number of claims, x, arising
from a group of policies. This resulted in the following frequency distribution
\begin{center}
\begin{tabular}{c|c|c|c|c|c|c}
x: & 0 & 1 & 2 & 3 & 4 & $\geq 5$ \\
f: & 14 & 25 & 26 & 18 &  12 & 5 \\
\end{tabular}
\end{center}
Calculate the median and interquartile range for these data. 

%%%%%%%%%%%%%%%%%%%%%%%%%%%%%%%%%%%%%%%%%%%%%%%%%%%%%%%%%%%%%%%%%%%%%%%%%%%%%%%%%%%%%%%%%%%%%%%%%%%%%%%%%
\item 2 Let $X_1$ and $X_2$ be independent Poisson random variables with respective means $\mu_1$ and $\mu_2$.
Assuming the moment generating function of a Poisson random variable, determine the moment generating function of $X_1 + X_2$ and hence state the distribution of $X_1 + X_2$ . 

%%%%%%%%%%%%%%%%%%%%%%%%%%%%%%%%%%%%%%%%%%%%%%%%%%%%%%%%%%%%%%%%%%%%%%%%%%%%%%%%%%%%%%%%%%%%%%%%%%%%%%%%%
\item 3 Let $X1 , X2 , \ldots , X5$ be independent N(0,1) random variables and let
( ) 5 5 2 2
1 1
1 1
= and =
5 4 i i
i i
X X S X X
= =
% - - − .
Calculate ( ) 5 2
1
0 and 9.488 i
i
P X X X
=
% - \lambda
%  > − < 
%  \times
% - . 

%%%%%%%%%%%%%%%%%%%%%%%%%%%%%%%%%%%%%%%%%%%%%%%%%%%%%%%%%%%%%%%%%%%%%%%%%%%%%%%%%%%%%%%%%%%%%%%%%%%%%%%%%
\item 4 Let $\{(x_i , y_i); i = 1, \ldots , n\}$ denote a set of n pairs of points with Assuming the usual expressions for the estimated coefficients, verify that the least squares fitted regression line of y on x passes through the point $(x, y)$. 
\end{enumerate}

\newpage

1 n = 100. So median =
1
50
2
th observation = 2
Q1 =
1
25
2
th observation = 1
Q3 =
1
75
2
th observation = 3 -IQR = 3 - 1 = 2
(same answers using alternative definitions)
%%%%%%%%%%%%%%%%%%%%%%%%%%%%%%%%%%%%%%%%%%%%%%%%%%%%%%%%%%%%%%%%%%%%%%%%%%%%%%%%%%%%%%%%%%%%%%5

\medskip 
2 As X1 and X2 are independent
1 2 1 2
( ) ( ). ( ) X X X X M t M tM t -
-
1 ( 1) . 2 ( 1) e et e et -% \lambda - \lambda
- using formula in Green book
( 1 2 )( 1) e et - %\lambda- 
-
- X1 + X2 is Poisson with mean $(\lambda_1 + \lambda_2)$
%%%%%%%%%%%%%%%%%%%%%%%%%%%%%%%%%%%%%%%%%%%%%%%%%%%%%%%%%%%%%%%%%
\medskip 
3 X ~ N with mean 0 so P(X - 0)% \lambda 0.5
% 4S2/2 ~ 2
with 4 d.f. i.e. 4S2 ~ % 2
with 4 d.f.
%  - \lambda
5 2 2
4
1
9.488 = ( 9.488) 0.95 i
i
P X X P
-
% - \lambda
%
-
X and S2 are independent, so probability required = $0.5 \times 0.95 = 0.475$
%%%%%%%%%%%%%%%%%%%%%%%%%%%%%%%%%%%%%%%%%%%%%%%%%%%%%%%%%%%%%%%%%%%%%%%%%%%%%%%%%%

\medskip 
4 Fitted regression line:
% yˆ = -ˆ \lambda ˆx
% But least squares estimate of  is
% -ˆ \lambda y  ˆx.
% Therefore yˆ - y \lambda ˆx  ˆx - y  ˆ(x \lambda x )
% - yˆ \lambda y when x \lambda x. Hence line passes through point -x, y\lambda.
\end{document}
%%%%%%%%%%%%%%%%%%%%%%%%%%%%%%%%%%%%%%%%%%%%%%%%%%%%%%%%%%%%%%%%%%%%%%%%%%%%%%%%%%%%%%%%%%%%%%%%%%%%%%%%%
