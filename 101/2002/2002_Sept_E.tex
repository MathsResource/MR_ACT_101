\documentclass[a4paper,12pt]{article}

%%%%%%%%%%%%%%%%%%%%%%%%%%%%%%%%%%%%%%%%%%%%%%%%%%%%%%%%%%%%%%%%%%%%%%%%%%%%%%%%%%%%%%%%%%%%%%%%%%%%%%%%%%%%%%%%%%%%%%%%%%%%%%%%%%%%%%%%%%%%%%%%%%%%%%%%%%%%%%%%%%%%%%%%%%%%%%%%%%%%%%%%%%%%%%%%%%%%%%%%%%%%%%%%%%%%%%%%%%%%%%%%%%%%%%%%%%%%%%%%%%%%%%%%%%%%

\usepackage{eurosym}
\usepackage{vmargin}
\usepackage{amsmath}
\usepackage{graphics}
\usepackage{epsfig}
\usepackage{enumerate}
\usepackage{multicol}
\usepackage{subfigure}
\usepackage{fancyhdr}
\usepackage{listings}
\usepackage{framed}
\usepackage{graphicx}
\usepackage{amsmath}
\usepackage{chngpage}

%\usepackage{bigints}
\usepackage{vmargin}

% left top textwidth textheight headheight

% headsep footheight footskip

\setmargins{2.0cm}{2.5cm}{16 cm}{22cm}{0.5cm}{0cm}{1cm}{1cm}

\renewcommand{\baselinestretch}{1.3}

\setcounter{MaxMatrixCols}{10}

\begin{document}  8 Assume that, for a group of insurance policies, the number of claims on each policy
can be modelled by a Poisson distribution with the same rate $\lambda$ per year,
independently for each policy. Such a group of 1500 policies gave rise to a total of
183 claims during the last year.
\begin{enumerate}[(a)]
    \item (i) State the value of the maximum likelihood estimate of $\lambda$.
\item (ii) Hence obtain estimates of the following quantities:
(a) The probability that a subset of 10 of these policies results in no claims
over the next six months.
(b) The probability that a subset of 250 of these policies results in more
than 40 claims over the next year. 
\end{enumerate}
%%%%%%%%%%%%%%%%%%%%%%%%%%%%%%%%%%%%%%%%%%%%%%%%%%%%%%%%%%%%%%%%%%%%%%%%%55
%%%%%%%%%%%%%%%%%%%%%%%%%%%%%%%%%%%%%%%%%%%%%%%%%%%%%%%%%%%%%%%%%%

% Question 8 

8 (i) ˆ = 183 = 0.122
1500

\begin{itemize}
    \item (ii) (a) Number of claims X for 10 policies in six months is Poisson with parameter 10(0.5)(0.122) = 0.61
\item \[P(no claims) = P(X = 0) = exp(	0.61) = 0.543\]
\item Alternative:
\begin{itemize}
\item For a single policy \[P(no claim) = exp(	0.061) = 0.9408\]
\item For 10 policies 
$P(no claims) = (0.9408)^10 = 0.543$
\end{itemize}
\item (b) Number of claims X for 250 policies in one year is Poisson with parameter 250(0.122) = 30.5
\item Outwith scope of the Green tables, so we must use a normal
approximation:
X $\leq$ N(30.5,30.5)
\item Applying a continuity correction:
\end{itemize}

( 40) ( 40.5) = ( 40.5 30.5 =1.81) =1 0.96485
30.5
= 0.035
P X P X P Z $\leq$
    $\leq$

\end{document}
