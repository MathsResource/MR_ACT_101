\documentclass[a4paper,12pt]{article}

%%%%%%%%%%%%%%%%%%%%%%%%%%%%%%%%%%%%%%%%%%%%%%%%%%%%%%%%%%%%%%%%%%%%%%%%%%%%%%%%%%%%%%%%%%%%%%%%%%%%%%%%%%%%%%%%%%%%%%%%%%%%%%%%%%%%%%%%%%%%%%%%%%%%%%%%%%%%%%%%%%%%%%%%%%%%%%%%%%%%%%%%%%%%%%%%%%%%%%%%%%%%%%%%%%%%%%%%%%%%%%%%%%%%%%%%%%%%%%%%%%%%%%%%%%%%

\usepackage{eurosym}
\usepackage{vmargin}
\usepackage{amsmath}
\usepackage{graphics}
\usepackage{epsfig}
\usepackage{enumerate}
\usepackage{multicol}
\usepackage{subfigure}
\usepackage{fancyhdr}
\usepackage{listings}
\usepackage{framed}
\usepackage{graphicx}
\usepackage{amsmath}
\usepackage{chngpage}

%\usepackage{bigints}
\usepackage{vmargin}

% left top textwidth textheight headheight

% headsep footheight footskip

\setmargins{2.0cm}{2.5cm}{16 cm}{22cm}{0.5cm}{0cm}{1cm}{1cm}

\renewcommand{\baselinestretch}{1.3}

\setcounter{MaxMatrixCols}{10}

\begin{document}

\begin{enumerate}

%%%%%%%%%%%%%%%%%%%%%%%%%%%%%%%%%%%%%%%%%%%%%%%%%%%%%%%%%%%%%%%%%%%%%%%%%%%%%%%%%%%%%%%%%%%%%%%%%%%%%%%%%%%%%%%%%%%%%%%%%%%%%%%%%%%%%%%%%%
\item 10 A company wants to estimate the percentage of its customers who are willing to shop on the internet. It decides to do so by calculating a symmetrical 95\% two-sided confidence interval for the unknown percentage.
\begin{enumerate}[(a)]
\item Show that, based on a random sample of 200 of the company's customers, the required confidence interval will have a width which is no greater than 13.9\%.

\item Calculate the sample size required which will ensure that, whatever the true percentage, the width of the confidence interval will be no greater than 10\%.
\end{enumerate}
%%%%%%%%%%%%%%%%%%%%%%%%%%%%%%%%%%%%%%%%%%%%%%%%%%%%%%%%%%%%%%%%%%%%%%%%%%%%%%%%%%%%%%%%%%%%%%%%%%%%%%%%%%%%%%%%%%%%%%%%%%%%%%%%%%%%%%%%%%



10  
\begin{itemize}
\item Width of 95\% CI = 2  1.96  {P(1 - P)/200}0.5 where P is sample
proportion
Max value of P(1 - P) is 0.52 = 0.25
 Max width of CI = 2  1.96  (0.25/200)0.5 = 0.139
In terms of percentages, this is 13.9%.
\item 2  1.96  (0.25/n)0.5  0.1  n  385
\end{itemize}
11 
\begin{itemize}
\item E(SN) = E[E(SN|N)]
Now, $E(SN|N = n) = E[(X1 + \ldots+ Xn)n|N = n)] = E[n(X1 + \ldots+ Xn)]$
= n  nX = n2
X
 E(SN) = E(XN2) = XE(N2)
= X
( 2 2 ) N  N

%%%%%%%%%%%%%%%%%%%%%%%%%%%%%%%%%%%%%%%%%%%%%%%%%%%%%%%%%%%%%%%%%%%%%%%%%%%%%%%%%%%%%%%%%%%%%%%%%%%%%%%%%%%%%%%%%%%%%%%%%%%%%%%%%%%%%%%%%%
\item E(S) = E(NX) = NX
\begin{eqnarray*}
\operatornmae{Cov}(S,N) &=& E(SN) - E(S)E(N)\\
&=& \mu_x (\mu^2_N + \sigma^2_N) \;-\; (\mu_N \mu_X)\mu_N \\
&=& \mu_X \sigma^2_N
\end{eqnarray*}

\end{itemize}
\end{document}
