\documentclass[a4paper,12pt]{article}

%%%%%%%%%%%%%%%%%%%%%%%%%%%%%%%%%%%%%%%%%%%%%%%%%%%%%%%%%%%%%%%%%%%%%%%%%%%%%%%%%%%%%%%%%%%%%%%%%%%%%%%%%%%%%%%%%%%%%%%%%%%%%%%%%%%%%%%%%%%%%%%%%%%%%%%%%%%%%%%%%%%%%%%%%%%%%%%%%%%%%%%%%%%%%%%%%%%%%%%%%%%%%%%%%%%%%%%%%%%%%%%%%%%%%%%%%%%%%%%%%%%%%%%%%%%%

\usepackage{eurosym}
\usepackage{vmargin}
\usepackage{amsmath}
\usepackage{graphics}
\usepackage{epsfig}
\usepackage{enumerate}
\usepackage{multicol}
\usepackage{subfigure}
\usepackage{fancyhdr}
\usepackage{listings}
\usepackage{framed}
\usepackage{graphicx}
\usepackage{amsmath}
\usepackage{chngpage}

%\usepackage{bigints}
\usepackage{vmargin}

% left top textwidth textheight headheight

% headsep footheight footskip

\setmargins{2.0cm}{2.5cm}{16 cm}{22cm}{0.5cm}{0cm}{1cm}{1cm}

\renewcommand{\baselinestretch}{1.3}

\setcounter{MaxMatrixCols}{10}

\begin{document} 

A random variable X which can be used in certain circumstances as a model for claim sizes has cumulative distribution function
\[F(x) = \begin{cases}
0 & \mbox{ where }  x < 0 \\ 
1 - \left(\frac{2}{(2+x)}\right)^2 & \mbox{ where }  x > 0 \\ 
\end{cases}
\]

Calculate the value of the conditional probability $P(X > 3|X > 1)$. 
%%%%%%%%%%%%%%%%%%%%%%%%%%%%%%%%%%%%%%%%%%%%%%%%%%%%%%%%%%%%%%%%%%%%%%%%
%% 2002 September Q 3
3 
\begin{eqnarray*}
P(X > 3| X > 1) &=& \frac{P(X > 3 \mbox{and }X > 1)}{P(X > 1)}\\
&=& \frac{P(X > 3 )}{P(X > 1)}\\
&=& \frac{(2/5)^5 }{(2/3)^3 }\\
&=& 0.216\\
\end{eqnarray*}
\newpage 
Suppose that the sums assured under policies of a certain type are modelled by a distribution with mean £8,000 and standard deviation £3,000. Consider a group of 100 independent policies of this type.
Calculate the approximate probability that the total sum assured under this group of policies exceeds £845,000.


%%%%%%%%%%%%%%%%%%%%%%%%%%%%%%%%%%%%%%%%%%%%%%%%%%%%%%%%%%%%%%%%%%%%%
4 Using units of £1000:
Total sum assured S  N(100  8, 100  9) i.e. S  N(800, 900) approximately, by
Central Limit Theorem.
P(S > 845)  P[Z > (845 – 800)/30] = P(Z > 1.5) where Z  N(0, 1)
= 0.067
\end{document}
