\documentclass[a4paper,12pt]{article}

%%%%%%%%%%%%%%%%%%%%%%%%%%%%%%%%%%%%%%%%%%%%%%%%%%%%%%%%%%%%%%%%%%%%%%%%%%%%%%%%%%%%%%%%%%%%%%%%%%%%%%%%%%%%%%%%%%%%%%%%%%%%%%%%%%%%%%%%%%%%%%%%%%%%%%%%%%%%%%%%%%%%%%%%%%%%%%%%%%%%%%%%%%%%%%%%%%%%%%%%%%%%%%%%%%%%%%%%%%%%%%%%%%%%%%%%%%%%%%%%%%%%%%%%%%%%

\usepackage{eurosym}
\usepackage{vmargin}
\usepackage{amsmath}
\usepackage{graphics}
\usepackage{epsfig}
\usepackage{enumerate}
\usepackage{multicol}
\usepackage{subfigure}
\usepackage{fancyhdr}
\usepackage{listings}
\usepackage{framed}
\usepackage{graphicx}
\usepackage{amsmath}
\usepackage{chngpage}

%\usepackage{bigints}
\usepackage{vmargin}

% left top textwidth textheight headheight

% headsep footheight footskip

\setmargins{2.0cm}{2.5cm}{16 cm}{22cm}{0.5cm}{0cm}{1cm}{1cm}

\renewcommand{\baselinestretch}{1.3}

\setcounter{MaxMatrixCols}{10}

\begin{document}

\begin{enumerate}
%%%%%%%%%%%%%%%%%%%%%%%%%%%%%%%%%%%%%%%%%%%%%%%%%%%%%%%%%%%%%%%%%%%%%%%%%%%%%%%%%%%%%%%%%%%%%%%%%%%%%%%%%%%%%%%%%%%%%%%%%%%%%%%%%%%%%%%%%%
\begin{enumerate}
\item 1 Let t px denote the probability that a life aged x survives for at least a further t years,
and consider three independent lives aged 40, 50 and 60 years such that
10 p40 = 0.95, 10 p50 = 0.85, 10 p60 = 0.70.
\begin{enumerate}[(a)]
\item Determine the probability that exactly one of these three lives survives ten
years. 
\item Determine the probability that it is the youngest life that survives, given that
exactly one life survives ten years. [1]
\end{enumerate}
\item 2 Claim amounts are modelled as an exponential random variable with mean £1,000.
\begin{enumerate}[(a)]
\item Calculate the probability that one such claim amount is greater than £5,000.
[1]
\item Calculate the probability that a claim amount is greater than £5,000 given that
it is greater than £1,000. 
\end{enumerate}

\end{enumerate}
%%%%%%%%%%%%%%%%%%%%%%%%%%%%%%%%%%%%%%%%%%%%%%%%%%%%%%%%%%%%%%%%%%%%%%%%%%%%%%%%%%%%%%%%%%%%%%%%%%%%%%%%%%%%%%%%%%%%%%%%%%%%%%%%%%%%%%%%%%
1 
\begin{itemize} \item P(exactly one) = (0.95)(0.15)(0.30)+(0.05)(0.85)(0.30)+(0.05)(0.15)(0.70)
= 0.04275 + 0.01275 + 0.00525 = 0.06075
\item P(youngest | exactly one) = 0.04275 0.704
0.06075 
2 \item X ~ exponential with  = 1/1000 and density e x 
 .

%%%%%%%%%%%%%%%%%%%%%%%%%%%%%%%%%%%%%%%%%%%%%%%%%%%%%%%%%%%%%%%%%%%%%%%%
%% 2002 April Q 2

(i) $X \sim \mbox{exponential}(\lambda = 1/1000)$ and density 
\[ f(x) = \lambda e^{ -\lambda  x} ,\] wher $x > 0 $.

\[P(X > t) = e^{ -\lambda  t} \](stated or by integration)

\begin{eqnarray*} 
P(X > 5000) &=& e^{ - \frac{1}{1000} 5000} \\
&=& e^{-5}\\
&=& 0.0067 \\
\end{eqnarray*}
(ii)
%%%%%%%%%%

with $t_1 > t_2$ we can say
\begin{eqnarray*} 
P ( X > t_1 | X > t_2 ) &=& 
\frac{P ( X > t_1 \mbox{ and } X > t_2 ) }{P ( X > t_2 ) }\\
&=& \frac{P ( X > t_1  ) }{P ( X > t_2 ) }\\
&=&  \frac{e^{-\lambda t_1}}{e^{-\lambda t_1}}
\end{eqnarray*}

% \begin{framed}
% For two events $A$ and $B$
% If $ B \subseteq A$ then $A \mbox{ and } B \equiv A$
% \end{framed}

\begin{eqnarray*} 
P ( X > 5000| X > 1000 ) 
&=&  \frac{e^{-\frac{1}{1000} \times 5000}}{e^{-\frac{1}{1000} \times 1000}}\\
&=&  \frac{e^{-5}}{e^{-1}}\\
&=&  e^{-4}\\
&=&  0.00182 \\
\end{eqnarray*}


OR: candidates may refer to the memoryless property of the exponential to
obtain $P(X > 5000 | X > 1000) = P(X > 4000)$ = $e^{-4}$ .

%%%%%%%%%%%%%%%%%%%%%%%%%%%%%%%%%%%%%%%%%%%%%%%%%%%%%%%%%%%%%%%%%%%%%%%%%%%%%%%%%%%%%%%%%%%%%%%%%%%%%%%%%%%%%%%%%%%%%%%%%%%%%%%%%%%%%%%%%%

\end{document}
