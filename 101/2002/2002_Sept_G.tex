\item 10 The number of claims N which arise from a portfolio of business is modelled as a
Poisson variable with mean . The claim amounts Xi : i = 1, 2 , ..., N are modelled as
independent gamma variables each with parameters  and  and are independent of
N. Let S be the total claim amount arising from this portfolio.
\begin{enumerate}[(a)]

\item (i) Obtain expressions for the mean and standard deviation of S in terms of , 
and using general results for the mean and variance of S, 
\item (ii) Consider the case where  = 100 and the individual claim amounts have mean
£100 and standard deviation £50.
(a) Calculate the mean and standard deviation of the total claim amount S.
\item (b) Calculate an approximate value for the probability that the total claim
amount S exceeds £12,500, giving a brief justification of your
approach. 
\end{enumerate}
%%%%%%%%%%%%%%%%%%%%%%%%%%%%%%%%%%%%%%%%%%%%%%%%%%%%%%%%%%%%%%%%%%
10 (i) E(S) = E(N)E(X ) = 


2 2
2 2 Var(S) = E(N)Var(X ) Var(N)[E(X )] = ( ) = (1 )   
    
  
2
sd(S) = (1 )  


(ii) (a) 2
2 =100 & = 50 = 4 , = 0.04  
  
 
E(S) = 100(100) = £10,000
2
( ) = 100(4)(1 4) = £1,118
0.04
sd S 
(b) Since N is going to be large, the Central Limit Theorem allows
normality.
( 12500) ( 12500 10000 = 2.236) where ~ (0,1)
1118
=1 0.987 = 0.013
P S P Z Z N 
  

