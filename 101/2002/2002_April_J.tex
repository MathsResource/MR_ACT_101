\documentclass[a4paper,12pt]{article}

%%%%%%%%%%%%%%%%%%%%%%%%%%%%%%%%%%%%%%%%%%%%%%%%%%%%%%%%%%%%%%%%%%%%%%%%%%%%%%%%%%%%%%%%%%%%%%%%%%%%%%%%%%%%%%%%%%%%%%%%%%%%%%%%%%%%%%%%%%%%%%%%%%%%%%%%%%%%%%%%%%%%%%%%%%%%%%%%%%%%%%%%%%%%%%%%%%%%%%%%%%%%%%%%%%%%%%%%%%%%%%%%%%%%%%%%%%%%%%%%%%%%%%%%%%%%

\usepackage{eurosym}
\usepackage{vmargin}
\usepackage{amsmath}
\usepackage{graphics}
\usepackage{epsfig}
\usepackage{enumerate}
\usepackage{multicol}
\usepackage{subfigure}
\usepackage{fancyhdr}
\usepackage{listings}
\usepackage{framed}
\usepackage{graphicx}
\usepackage{amsmath}
\usepackage{chngpage}

%\usepackage{bigints}
\usepackage{vmargin}

% left top textwidth textheight headheight

% headsep footheight footskip

\setmargins{2.0cm}{2.5cm}{16 cm}{22cm}{0.5cm}{0cm}{1cm}{1cm}

\renewcommand{\baselinestretch}{1.3}

\setcounter{MaxMatrixCols}{10}

\begin{document}
%%%%%%%%%%%%%%%%%%%%%%%%%%%%%%%%%%%%%%%%%%%%%%%%%%%%%%%%%%%%%%%%%%%%%%%%%%%%%%%%%%%%%%%%%%%%%%%%%%%%%%%%%%%%%%%%%%%%%%%%%%%%%%%%%%%%%%%%%%
\item The table below gives the numbers of deaths nx in a year in groups of women aged x
years. The exposures of the groups, denoted Ex , are also given (the exposure is
essentially the number of women alive for the year in question). The values of the
death rates yx , where yx = nx/Ex , and the log(death rates), denoted wx , are also given.
age x number of deaths nx exposure Ex yx = nx/Ex wx = logyx
% 70 30 426 0.07042 2.6532
% 71 38 471 0.08068 2.5173
% 72 38 454 0.08370 2.4805
% 73 53 482 0.10996 2.2077
% 74 59 445 0.13258 2.0205
% 75 61 423 0.14421 1.9365
% 76 82 468 0.17521 1.7417
% 77 96 430 0.22326 1.4994
% x = 588 x2 = 43260 w = 17.0568 w2 = 37.5173 xw = 1246.7879
\begin{enumerate}[(a)]
\item A scatter plot of yx against x is shown below.
Draw a scatter plot of wx against x and comment briefly on the two scatter
plots and the relationships displayed. 
\item (a) Calculate the least squares fit regression line in which wx is modelled
as the response and x as the explanatory variable.
(b) Draw the fitted line on your scatter plot of wx against x.
(c) Calculate a 95\% confidence interval for the slope coefficient of the
regression model of wx on x, adopting the assumptions of the usual
“normal regression model”.
(d) Calculate the fitted values for the number of deaths for the group aged
71 years and the group aged 76 years. 
\item Explain briefly the relationship between the fitting procedure used in part \item
and a model which states that the number of deaths Nx is a random variable
with mean Exbcx for some constants b and c. 
\end{enumerate}
0.00
0.05
0.10
0.15
0.20
0.25
69 70 71 72 73 74 75 76 77 78
x
y x


14 

\begin{itemize}
\item Plot
Comments: yx v x relationship is not linear
wx v x relationship appears to be linear, strong
\item (a) Sxx = 43260 – 5882/8 = 42
% Sxw = 1246.7879 – (588  17.0568)/8 = 6.8869
%  slope = 6.8869/42 = 0.1640
% so intercept = 17.0568/8  (6.8869/42)  (588/8) = 14.1842
% Fitted line is wx = 14.184 + 0.1640x
(b) Line on plot
-3.0
-2.5
-2.0
-1.5
-1.0
70 71 72 73 74 75 76 77
x
w x
-3.0
-2.5
-2.0
-1.5
-1.0
70 71 72 73 74 75 76 77
x
w x
%%%%%%%%%%%%%%%%%%%%%%%%%%%%%%%%%%%%%%%%%%%%%%%%%%%%%%%%%%%%%%%%%%%%%%%%%%%%%%%%%%%%%%%%%%%%%%%%%%%%%%%%%%%%%%%%%%%%%%%%%%%%%%%%%%%%%%%%%%
% (c) Sww = 37.5173  ((17.0568)2/8) = 1.1505
% estimate of error variance = [1.1505  6.88692/42]/6 = 0.003538
%  standard error of slope coefficient = (0.003538/42)0.5 = 0.00918
% t6 (0.975) = 2.447
%  95\% CI for slope coefficient is 0.1640  (2.447  0.00918)
% i.e. 0.1640  0.0225 or (0.141, 0.187)
% (d) Fitted value of w71 = 2.5421  fitted value of
% y71 = exp(2.5421) = 0.0787
%  fitted value of n71 = 471  0.0787 = 37.1
% Fitted value of w76 = 1.7222  fitted value of
% y71 = exp(1.7222) = 0.1787
%  fitted value of n76 = 468  0.1787 = 83.6
% \item E(Nx) = Exbcx  E(Nx/Ex) = bcx  E(Yx) = bcx
%  logE(Yx) = 
% x where 
% = logb ,  = logc.
The procedure above is a linear regression analysis of wx = logyx on x , which
is a simple and approximate approach to fitting the stated model .
Note: The method used in  is based on E(Wx) = E(logYx) = c + d x, whereas
the model stated in (Q) is based on logE(Yx) = c + dx.
%%%%%%%%%%%%%%%%%%%%%%%%%%%%%%%%%%%%%%%%%%%%%%%%%%%%%%%%%%%%%%%%%%%%%%%%%%%%%%%%%%%%%%%%%%%%%%%%%%%%%%%%%%%%%%%%%%%%%%%%%%%%%%%%%%%%%%%%%%
    \item 
\end{itemize}


\newpage

\noindent{\bf Regression estimates}

\begin{eqnarray*}
	S_{XY} &=&
	\sum x_iy_i - \frac{\sum x_i\sum y_i}{n}\\
	S_{XX} &=&
	\sum x_i^2 - \frac{(\sum x_i)^2}{n}\\
	S_{YY} &=&
	\sum y_i^2 - \frac{(\sum y_i)^2}{n}\\
\end{eqnarray*}
{\bf Slope Estimate}
\begin{eqnarray*}
	b_1 = \frac{S_{XY}}{S_{XX}}
\end{eqnarray*}
{\bf Intercept Estimate}
\begin{eqnarray*}
	b_0 = \bar{y} -b_1\bar{x}
\end{eqnarray*}
{\bf Pearson's correlation coefficient}

\begin{eqnarray*}
	r = \frac{S_{XY}}{\sqrt{S_{XX} \times S_{YY}}}
\end{eqnarray*}
{\bf Standard error of the Slope}
\begin{eqnarray*}
	S.E.(b1) = \sqrt{\frac{s^2}{S_{XX}}}
\end{eqnarray*}

where $s^2 = \frac{SSE}{n-2}$
and SSE $= S_{YY} - b_1S_{XY}$

\end{document}
