\documentclass[a4paper,12pt]{article}

%%%%%%%%%%%%%%%%%%%%%%%%%%%%%%%%%%%%%%%%%%%%%%%%%%%%%%%%%%%%%%%%%%%%%%%%%%%%%%%%%%%%%%%%%%%%%%%%%%%%%%%%%%%%%%%%%%%%%%%%%%%%%%%%%%%%%%%%%%%%%%%%%%%%%%%%%%%%%%%%%%%%%%%%%%%%%%%%%%%%%%%%%%%%%%%%%%%%%%%%%%%%%%%%%%%%%%%%%%%%%%%%%%%%%%%%%%%%%%%%%%%%%%%%%%%%

\usepackage{eurosym}
\usepackage{vmargin}
\usepackage{amsmath}
\usepackage{graphics}
\usepackage{epsfig}
\usepackage{enumerate}
\usepackage{multicol}
\usepackage{subfigure}
\usepackage{fancyhdr}
\usepackage{listings}
\usepackage{framed}
\usepackage{graphicx}
\usepackage{amsmath}
\usepackage{chngpage}

%\usepackage{bigints}
\usepackage{vmargin}

% left top textwidth textheight headheight

% headsep footheight footskip

\setmargins{2.0cm}{2.5cm}{16 cm}{22cm}{0.5cm}{0cm}{1cm}{1cm}

\renewcommand{\baselinestretch}{1.3}

\setcounter{MaxMatrixCols}{10}

\begin{document}
\begin{enumerate}
\item 5 Show that the probability generating function for a binomial (n, p) distribution is
\[ G_X (t) = (1 − p + pt)^n .\]
Deduce the moment generating function. 
%%%%%%%%%%%%%%%%%%%%%%%%%%%%%%%%%%%%%%%%%%%%%%%%%%%%%%%%%%%%%%%%%%%%%%%%%%%%%%%%%%%%%%%%%%%%%%%%%%%%%%%%%%%%%%%%%%%
\item 6 Let $X$ have a normal distribution with mean μ and standard deviation σ, and let
the ith cumulant of the distribution of X be denoted $\kappa_i$ .
Assuming the moment generating function of X, determine the values of $\kappa_2$ , $\kappa_3$ ,
and $\kappa_4$ . 


\end{enumerate}
\newpage
\[PGF	G ( z ) = {\displaystyle G(z)=[(1-p)+pz]^{n}} \]
5 PGF for binomial (n,p)
%%%% 2001 April Question 5


PGF for binomial (n,p)


\begin{eqnarray*}
G_X(t) 
&=& E(t^X) \\ 
&=& \sum^{n}_{x=0} t^x P(X = x)  \\
&=& \sum^{n}_{x=0} t^x  {n \choose x} p^x \;(1-p)^{n-x} \\
&=& \sum^{n}_{x=0} {n \choose x} pt^x \;(1-p)^{n-x}\\
&=& (1 - p + pt)^{n}
\end{eqnarray*}


\begin{framed}
 Using summation notation, the binomial formula can be written as 
\[ {\displaystyle (x+y)^{n}=\sum _{k=0}^{n}{n \choose k}x^{n-k}y^{k}=\sum _{k=0}^{n}{n \choose k}x^{k}y^{n-k}.} \]
\end{framed}

MGF obtained by replacing $t$ by $e^t$, i.e. $M_X(t) = (1 - p + pe^t)^{n}$
MGF obtained by replacing t by et, i.e. ( ) (1 t )n.
X M t = − p + pe
%%%%%%%%%%%%%%%%%%%%%%%%%%%%%%%%%%%%%%%%%%%%%%%%%%%%%%%%%%%%%%%%%%%%%%%%%%%%%%%%%%%%%%%%%%%%%%%%%%%%%%%%%%%%%%%%%%%
6 \[C(t) = logM(t) = log[exp(μt + σ2t2/2)] \]from Green Book
= μt + σ2t2/2
\kappa_r is the coefficient of tr/r! , r = 2,3,…
∴ \kappa_2 = σ2, \kappa_3 = \kappa_4 = 0



\end{document}
