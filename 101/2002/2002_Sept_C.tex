\documentclass[a4paper,12pt]{article}

%%%%%%%%%%%%%%%%%%%%%%%%%%%%%%%%%%%%%%%%%%%%%%%%%%%%%%%%%%%%%%%%%%%%%%%%%%%%%%%%%%%%%%%%%%%%%%%%%%%%%%%%%%%%%%%%%%%%%%%%%%%%%%%%%%%%%%%%%%%%%%%%%%%%%%%%%%%%%%%%%%%%%%%%%%%%%%%%%%%%%%%%%%%%%%%%%%%%%%%%%%%%%%%%%%%%%%%%%%%%%%%%%%%%%%%%%%%%%%%%%%%%%%%%%%%%

\usepackage{eurosym}
\usepackage{vmargin}
\usepackage{amsmath}
\usepackage{graphics}
\usepackage{epsfig}
\usepackage{enumerate}
\usepackage{multicol}
\usepackage{subfigure}
\usepackage{fancyhdr}
\usepackage{listings}
\usepackage{framed}
\usepackage{graphicx}
\usepackage{amsmath}
\usepackage{chngpage}

%\usepackage{bigints}
\usepackage{vmargin}

% left top textwidth textheight headheight

% headsep footheight footskip

\setmargins{2.0cm}{2.5cm}{16 cm}{22cm}{0.5cm}{0cm}{1cm}{1cm}

\renewcommand{\baselinestretch}{1.3}

\setcounter{MaxMatrixCols}{10}

\begin{document}
 5 Suppose that a line is fitted by least squares to a set of data $\{ ( x_i , y_i ) , i = 1, 2, \ldots , n \}$
which has sample correlation coefficient $r$. Let the fitted value at $x_i$ be denoted $\hat{y}_i$.
Show that the sample correlation coefficient of the data $\{ (  y_i, \hat{y}_i ) , i = 1, 2, \ldots , n \}$, that
is, of the observed and fitted y values, is also equal to $r$. 

\end{enumerate}
%%%%%%%%%%%%%%%%%%%%%%%%%%%%%%%%%%%%%%%%%%%%%%%%%%%%%%%%%%%%%%%%%%
\end{enumerate}
\newpage

5 yˆi = y $\leq$ˆ(xi  x) so$\leq$yˆi = ny so mean of fitted values is y
 Correlation coefft. of observed and fitted values is
$\leq$ $\leq$ 
$\leq$ $\leq$  $\leq$   $\leq$  2 2 1/ 2 2 1/ 2
ˆ ˆ
= =
ˆ ˆ
i i xy
i i yy xx
y y y y S
r
y y y y S S
  $\leq$
  $\leq$
$\leq$
$\leq$ $\leq$
%%%%%%%%%%%%%%%%%%%%%%%%%%%%%%%%%%%%%%%
%%%%%%%%%%%%%%%%%%%%%%%%%%%%%%%%%%%%%%%%%%%%%%%%%%%%%%%%%%%%%%%%%%

6 As part of an investigation an insurance company collected data for the year 2000 on
claims sizes for all claims on a certain type of motor insurance policy. The resulting
data are given below in the form of a grouped frequency distribution.
\begin{verbatim}
Claim size (£) Frequency
$\leq$ 100 862
> 100 and $\leq$ 200 608
> 200 and $\leq$ 300 1253
> 300 and $\leq$ 400 1066
> 400 and $\leq$ 500 558
> 500 1290
Total 5637
\end{verbatim}


\begin{enumerate}[(a)]
    \item (i) Calculate the cumulative frequencies and draw a graph of the claim size
distribution function (i.e. the cumulative frequencies against claim size). 
    \item (ii) You have been asked to determine the proportion of claim sizes which are less
than £250. Use linear interpolation to approximate this proportion to two
decimal places. 
    \item (iii) Use linear interpolation to approximate (to the nearest £) the value of the
median of the claim size distribution. 
\end{enumerate}

%%%%%%%%%%%%%%%%%%%%%%%%%%%%%%%%%%%%%%%%%%%%%%%%%%%%%%%%%%%%%%%%%%

Page 3
6 (i) Cumulative frequency distribution is
x F(x)
0 0
100 862
200 1470
300 2723
400 3789
500 4347
max 5637
(ii) For x = 250 the corresponding F(x) is approximately
1470 (2723 1470) (250 200) 2096.5
100
$\leq$
 $\leq$ 
and the corresponding approximate proportion is 2096.5 0.37
5637
$\leq$
(iii) For the median we need x such that F(x) = 0.5(5637) = 2818.5
median 300 2818.5 2723 (100) £309
1066
$\leq$
  
The maximum claim size is not given, and candidates were expected to exercise common
sense and judgement. One sensible approach is to leave the x-scale open-ended and indicate
the presence of a horizontal asymptote at cumulative frequency 5637.
0 250 500
0
1000
2000
3000
4000
5000
6000
x
F(x)
Claim size distribution function
for median
for P(X<250)

\end{document}
