\documentclass[a4paper,12pt]{article}

%%%%%%%%%%%%%%%%%%%%%%%%%%%%%%%%%%%%%%%%%%%%%%%%%%%%%%%%%%%%%%%%%%%%%%%%%%%%%%%%%%%%%%%%%%%%%%%%%%%%%%%%%%%%%%%%%%%%%%%%%%%%%%%%%%%%%%%%%%%%%%%%%%%%%%%%%%%%%%%%%%%%%%%%%%%%%%%%%%%%%%%%%%%%%%%%%%%%%%%%%%%%%%%%%%%%%%%%%%%%%%%%%%%%%%%%%%%%%%%%%%%%%%%%%%%%

\usepackage{eurosym}
\usepackage{vmargin}
\usepackage{amsmath}
\usepackage{graphics}
\usepackage{epsfig}
\usepackage{enumerate}
\usepackage{multicol}
\usepackage{subfigure}
\usepackage{fancyhdr}
\usepackage{listings}
\usepackage{framed}
\usepackage{graphicx}
\usepackage{amsmath}
\usepackage{chngpage}

%\usepackage{bigints}
\usepackage{vmargin}

% left top textwidth textheight headheight

% headsep footheight footskip

\setmargins{2.0cm}{2.5cm}{16 cm}{22cm}{0.5cm}{0cm}{1cm}{1cm}

\renewcommand{\baselinestretch}{1.3}

\setcounter{MaxMatrixCols}{10}

\begin{document}  11 Consider a group of n males each aged 30 years living in a particular society. The
lives may be assumed to be independent.
Suppose that x of the n men die by the end of a subsequent period of duration t0 years
and that n  x survive the period.
Suppose that the lifetime of these men from age 30 is to be modelled as a random
variable with an exponential distribution with mean 1/ hours.
\begin{enumerate}[(a)]
    \item 
(i) State the distribution of the random variable X whose value x, the number of
men who die within t0 years, has been observed. 
\item (ii) (a) Verify that the log-likelihood of the observation x is given by:
\[  = log L( ) = x log(1 e t0 ) (n x) t0 
       + constant\]
(b) Determine ˆ, the maximum likelihood estimator of .
\item (iii) Show that
\[0
0
2 2
0
2 =
1
t
t
E nt e
e


  
        	
 [4]\]
\item (iv) (a) Consider the case n = 1,000, t0 = 20 years, and x = 320.
Evaluate ˆ, and show that the value of its standard error is
approximately 0.00108.
\item (b) Hence, calculate an approximate 95% confidence interval for the mean
lifetime from age 30 of such men. [7]
\end{enumerate}
%%%%%%%%%%%%%%%%%%%%%%%%%%%%%%%%%%%%%%%%%%%%%%%%%%%%%%%%%%%%%%%%%%
\begin{itemize}
\item 11 (i) P(T < x) = 1 	 exp(	
x) so X ~ bi(n, 1 	 exp(	
t0))
\item (ii) (a) L(
)  (1 	 exp(	
t0))x [exp(	
t0) ]nx
(
) = xlog(1 	 exp(	
t0)) + (n 	 x)log[exp(	
t0)] + constant
= xlog(1 	 exp(	
t0)) 	 (n 	 x)
t0 + constant
(b) 

 = xt0 exp(	
t0) /(1 	 exp(	
t0)) 	 (n 	 x)t0
= xt0[1 + exp(	
t0) /(1 	 exp(	
t0))] 	 nt0
= xt0/[(1 	 exp(	
t0)] 	 nt0
%%%%%%%%%%%%%%%%%%%%%%%%%%%%%%%%%%%%%%%
Page 7
Setting 0 


  exp(	
t0) = 1 	 x/n 
0
ˆ 1 log(1 x)
t n
   
\item [OR: direct from MLE of P(survive) = exp(	
t0) is observed
proportion which survive, namely 1 – x/n ; hence MLE of 
.]
\item (iii) From (ii) (b)
2
2


 = 	xt0(1 	 exp(	
t0))2(t0 exp(	
t0))
= 	 xt0
2 exp(	
t0) (1 	exp(	
t0))2

2
2 E( ) 


 = t0
2 exp(	
t0)(1 	 exp(	
t0))2E(X)
= t0
2exp(	
t0)(1 	 exp(	
t0))2 n(1 	 exp(	
t0))
[using mean of binomial distribution from part (i)]
= n t0
2 exp(	
t0) /[1 	 exp(	
t0)]
\item (iv) (a) ˆ = 	 (1/20) log(0.68) = 0.019283
\item Estimate of $exp(	
t0)$ is 0.68,
so
2
2 E( ) 


  1000  202  0.68/0.32 = 850000
 s.e.( ˆ ) 850000 1/ 2 0.0010847  0.00108     
\item (b) 0.019283 
 (1.96  0.0010847)
i.e. 0.019283 
 0.002126 i.e. 0.01716 to 0.02141
95% CI for mean lifetime 1/
 is (1/0.02141, 1/0.01716)
i.e. 46.7 years to 58.3 years

\item There was a slight error in line 6 of Question 11: the last word should have appeared as
“years” instead of “hours”. The examiners took this into account, and gave full credit to
any candidates who provided an alternative answer because of this.
\end{itemize}
\end{document}
