\documentclass[a4paper,12pt]{article}

%%%%%%%%%%%%%%%%%%%%%%%%%%%%%%%%%%%%%%%%%%%%%%%%%%%%%%%%%%%%%%%%%%%%%%%%%%%%%%%%%%%%%%%%%%%%%%%%%%%%%%%%%%%%%%%%%%%%%%%%%%%%%%%%%%%%%%%%%%%%%%%%%%%%%%%%%%%%%%%%%%%%%%%%%%%%%%%%%%%%%%%%%%%%%%%%%%%%%%%%%%%%%%%%%%%%%%%%%%%%%%%%%%%%%%%%%%%%%%%%%%%%%%%%%%%%

\usepackage{eurosym}
\usepackage{vmargin}
\usepackage{amsmath}
\usepackage{graphics}
\usepackage{epsfig}
\usepackage{enumerate}
\usepackage{multicol}
\usepackage{subfigure}
\usepackage{fancyhdr}
\usepackage{listings}
\usepackage{framed}
\usepackage{graphicx}
\usepackage{amsmath}
\usepackage{chngpage}

%\usepackage{bigints}
\usepackage{vmargin}

% left top textwidth textheight headheight

% headsep footheight footskip

\setmargins{2.0cm}{2.5cm}{16 cm}{22cm}{0.5cm}{0cm}{1cm}{1cm}

\renewcommand{\baselinestretch}{1.3}

\setcounter{MaxMatrixCols}{10}

\begin{document}

\begin{enumerate}
\item
5 Ten independent hypothesis tests are to be conducted, each at the 5% significance
level.
Calculate the probability that at least one of the tests produces a significant result,
assuming that the null hypothesis for each of the 10 tests is true, and comment briefly
on the value. 
\item 6 Show that the slope of the regression line fitted by least squares to the three points
(0,0) , (1,y) , (2,2)
is 1 for all values of y. 
%%%%%%%%%%%%%%%%%%%%%%%%%%%%%%%%%%%%%%%%%%%%%%%%%%%%%%%%%%%%%%%%%%%%%%%%%%%%%%%%%%%%%%%%%%%%%%%%%%
\item 7 Claims arise through time on a portfolio of policies one after another, at random, and
at a constant rate per week. Claim sizes are to be modelled as a N(, 2) random
variable, independent of the times of occurrence and the accumulated numbers of
claims.
The moment generating function of S, the total size of all claims which occur in a
period of k weeks (where k is a positive integer), is to be used in a theoretical report
being written by two students.
One student (A) suggests that the moment generating function of S is given by:
Suggestion A :
\[
M_S ( t ) = exp \left( k\;\lambda \left{ exp \left[ \mu\;t + s 2 t 2 \right] - 1 \right} \; \right)
\]

while the other (B) disagrees and suggests that it is in fact given by:
Suggestion B :

\[M_S ( t ) = exp \left( k\;\lambda exp \left[ \mu\;t + s 2 t 2 \right] - 1 \right)  \]
One of the students is correct. Determine which one this is.
\item 8 A student actuary is about to collect data from a batch of insurance proposal forms. It
is known that on average one in every fifty such forms contains incomplete
information, and the student wants to be 95% sure of having at least 500 forms which
are complete.
Show that he must use a batch of at least 516 forms. (Use a suitable model and
appropriate tables for the distribution of the number of forms containing incomplete
information.) 
\end{enumerate}
%%%%%%%%%%%%%%%%%%%%%%%%%%%%%%%%%%%%%%%%%%%%%%%%%%%%%%%%%%%%%%%%%%%%%%%%%%%%%%%%%%%%%%%%%%%%%%%%%%
\newpage

5 P(at least one test is significant | each null hypothesis is true)\\
= 1 − P(no test is significant | each null hypothesis is true)\\
= 1 − (1 − 0.05)10 as the 10 tests are independent\\
= 0.4\\

Comment: a false-positive is very likely with the 10 multiple tests.

%%%%%%%%%%%%%%%%%%%%%%%%%%%%%%%%%%%%%%%%%%%%%%%%%%%%%%%%%%%%%%%%%%%%%%%%%%%%%%%%%%%%%%%%%%%%%%%%%%
6 Sxx = 5 – 32/3 = 2 , Sxy = y + 4 – 3(y+2)/3 = 2
So fitted slope = 2/2 = 1
%%%%%%%%%%%%%%%%%%%%%%%%%%%%%%%%%%%%%%%%%%%%%%%%%%%%%%%%%%%%%%%%%%%%%%%%%%%%%%%%%%%%%%%%%%%%%%%%%%
7 N is Poisson(kλ) with MN(t) = exp[kλ{exp(t) – 1}].
S has a compound distribution with mgf MS(t) = MN{logMX(t)}
and \[MX(t) = exp(μt + \sigma^2t2/2)]\.
So mgf of S is MN(μt + \sigma^2t2/2) and correct suggestion is A.
OR: by using the result quoted in the Formulae and Tables book
OR: we must have $MS(0) = 1$, so B is wrong.
%%%%%%%%%%%%%%%%%%%%%%%%%%%%%%%%%%%%%%%%%%%%%%%%%%%%%%%%%%%%%%%%%%%%%%%%%%%%%%%%%%%%%%%%%%%%%%%%%%

8 Let X be the number of forms with incomplete information in a batch of n forms.
Then X ~ Poisson(0.02n) approximately
\begin{itemize}
    \item With n = 516, $X \sim P(10.32)$ and $P(X \leq 16) = 0.965$ approx, by linear interpolation
\item With n = 515, $X \sim P(10.3)$ and $P(X \leq 15) = 0.940$ approx, by linear interpolation
\item So he requires 516 forms
\end{itemize}

%%%%%%%%%%%%%%%%%%%%%%%%%%%%%%%%%%%%%%%%%%%%%%%%%%%%%%%%%%%%%%%%%%%%%%%%%%%%%%%%%%%%%%%%%%%%%%%%%%
% Page 5
OR: the distribution of X can be modelled approximately using a normal distribution
with mean 0.02n and variance 0.0196n; we require P(X ≤ n − 500) to be at least 0.95;
the analysis is more awkward, but solving a quadratic in n gives n ≥ 515

\end{document}
