
\documentclass[a4paper,12pt]{article}

%%%%%%%%%%%%%%%%%%%%%%%%%%%%%%%%%%%%%%%%%%%%%%%%%%%%%%%%%%%%%%%%%%%%%%%%%%%%%%%%%%%%%%%%%%%%%%%%%%%%%%%%%%%%%%%%%%%%%%%%%%%%%%%%%%%%%%%%%%%%%%%%%%%%%%%%%%%%%%%%%%%%%%%%%%%%%%%%%%%%%%%%%%%%%%%%%%%%%%%%%%%%%%%%%%%%%%%%%%%%%%%%%%%%%%%%%%%%%%%%%%%%%%%%%%%%

\usepackage{eurosym}
\usepackage{vmargin}
\usepackage{amsmath}
\usepackage{graphics}
\usepackage{epsfig}
\usepackage{enumerate}
\usepackage{multicol}
\usepackage{subfigure}
\usepackage{fancyhdr}
\usepackage{listings}
\usepackage{framed}
\usepackage{graphicx}
\usepackage{amsmath}
\usepackage{chngpage}

%\usepackage{bigints}
\usepackage{vmargin}

% left top textwidth textheight headheight

% headsep footheight footskip

\setmargins{2.0cm}{2.5cm}{16 cm}{22cm}{0.5cm}{0cm}{1cm}{1cm}

\renewcommand{\baselinestretch}{1.3}

\setcounter{MaxMatrixCols}{10}

\begin{document}

%%%%%%%%%%%%%%%%%%%%%

\begin{enumerate}
\item 
[Total 15]9
A stationary second-order autoregressive process X, which may be assumed to
be in equilibrium at time 0, is defined by
X t = μ + α 1 (X t−1 − μ) + α 2 (X t−2 − μ) + e t ,
where {e t : t ≥ 1} is a sequence of independent, zero-mean Normal random
variables, each with variance σ 2 e .
(i)
(ii)
(a) Obtain an equation for γ 1 in terms of γ 0 and γ 2 by substituting for
X t in the equation γ 1 = Cov(X t , X t−1 ).
(b) Derive similar equations for γ 2 and γ 0 .
(c) State the autocorrelation function ρ k of X for k = 0, 1, 2.
[5]
Suppose that the equations derived in (i) for ρ 1 and ρ 2 are used as the
basis of an estimation procedure: estimates α $ 1 and α $ 2 are defined to be
the solutions of those equations when ρ is replaced by a suitably-
defined sample autocorrelation function r.
Solve these equations.

%%%%%%%%%%%%%%%%%%%%%%%%%%%%%%%%%%%%%%%%%%%%%%%555
10
(i)
[3]
[Total 8]
(a) Define standard Brownian motion B t , t ≥ 0 and give its transition
probability density.
(b) Write down the transition probability density of general
Brownian motion W t = σB t + μt.
[4]
Let S t defined represent a share price at time t.
(ii)
Solve the stochastic differential equation
dS t = μS t dt + σS t dB t .
[5]
(iii) Calculate, given the parameters μ = 25% p.a., σ = 20% on an annual
basis, the probability that the share price will exceed 45 in four months’
time given that its current price is 38.
[4]
(iv) Calculate the probability that the share price will exceed 45 at any
stage during the next four months given that its current value is 38.
[You may use the formula
P[max( B s + λ s ) > y ] = G
0≤ s ≤ t
F G λ t − y I J + e
H t K
2 λ y
G
F G − y − λ t I J ,
H t K
where y ≥ 0 and G denotes the normalised Gaussian probability
distribution function.]
[4]
[Total 17]
103—5


%%%%%%%%%%%%%%%%%%%%%%%%%%%%%%%%%%%%%%%%%%%%%%%%%%%%%%%%%%%%%%%%%%%%%%%%%%%%%%%%%%%%%%%%%%%%%%%%%%%%
9
(i)
1249
155
+ b
.
1404
1404
150  ́ 1249 + 600  ́ 155
= 100.32.
1404
(a) g 1 = Cov(X t , X t-1 ) = Cov(a 1 X t-1 + a 2 X t-2 + e t , X t-1 ) = a 1 g 0 + a 2 g 1 + 0, since
e t is independent of X t-1 .
(b) Similarly g 2 = a 1 g 1 + a 2 g 0 and g 0 = a 1 g 1 + a 2 g 2 + Cov(X t , e t ). A further
application of the same technique gives Cov(X t , e t ) = s 2 e .
Thus g 1 =
(c)
(ii)
r k is found by the relation r k = g k / g 0 .
We have =  1 = r 1 ( 1 - a  2 ) and a  2 +
=  1 =
10
(i)
F
G H
(a)
I
J K
a 1
a 12
g 0 and g 2 = a 2 +
g 0 .
1 - a 2
1 - a 2
r 1 ( 1 - r 2 )
,
1 - r 12
=  2 =
a  12
= r 2 , which are solved by
1 - a  2
r 2 - r 12
.
1 - r 12
B t defined by following properties:
· Independent increments: B t - B s independent of B a , 0 £ a £ s
whenever s £ t.
· Stationary Gaussian increments: B t - B s ~ N(0, t - s).
· Continuous sample paths: t ® B t continuous.
Page 7Subject 103 (Stochastic Modelling) — April 2000 — Examiners’ Report
Transition density to go from x at time s to y at time t:
1
g t-s (y - x) =
(b)
2 p( t - s )
e - ( y - x )
2
/ 2 ( t - s )
.
{W s = x, W t = y} = {sB s + ms = x, sB t + mt = y} = {B s =
y - m t
}. Hence transition density of W is
s
1
g t-s
I
(ii)
F G y - x - m ( t - s ) I J .
H
K
s
By Itô’s lemma
d(log S t ) =
I
J K
F
G H
1
1
1
dS t +
- 2 ( dS t ) 2
2
S t
S t
= μdt + sdB t -
I 2
dt.
2
Hence
F
G H
log S t = log S 0 + m -
s 2
2
I t + sB ,
J K
t
and finally
F m - s I t + s B
G 2 J K
S e H
.
2
t
S t =
(iii)
0
L M F s I t > log b O P
a P Q
M N G H 2 J K
L 1 F b F s I t I O P
= P M B > G log - G m -
M N s H a H 2 J K J K P Q
F log b - F m - s I t I
G a G H 2 J K JJ
= 1 - G
J .
G
s t
K
H
P[S t > b1⁄2S 0 = a] = P s B t + m -
2
2
t
2
Page 8
x - m s
, B t =
sSubject 103 (Stochastic Modelling) — April 2000 — Examiners’ Report
Here a = 38, b = 45, m = 0.25, s = 0.2, t =
1
3
year.
So, above quantity is
1 - G(0.800) = 1 - 0.7881 = 0.2119.
(iv)
L M
M N
F
G H
F
G H
O P
I I
J K J K
P Q
F F m - s I t + log b I
G G 2 J K
a J
b U G H
J .
log V G -
a W G
s t
J K
G H
P max S s 3 b 1⁄2 S 0 = a = P max B s + m -
0 £ s £ t
0 £ s £ t
F F m - s I t - log b I
G G H 2 J K
a J
R 2 m - s
J
= G
+ exp S
J T s
G
s t
K
H
2
s 2 s
b
1
3 log
s
a
2 s
2
2
2
The first term is 0.2119 by (iii).
F b I
The second term is the product of G J
H a K
2 m - s 2
s 2
= 6.9893 with G(-2.128)
= 1 - G(2.128) = 1 - 0.9833.
So the result is finally 0.2119 x 0.0167 = 0.3286.


\end{document}

\end{document}
