
\documentclass[a4paper,12pt]{article}

%%%%%%%%%%%%%%%%%%%%%%%%%%%%%%%%%%%%%%%%%%%%%%%%%%%%%%%%%%%%%%%%%%%%%%%%%%%%%%%%%%%%%%%%%%%%%%%%%%%%%%%%%%%%%%%%%%%%%%%%%%%%%%%%%%%%%%%%%%%%%%%%%%%%%%%%%%%%%%%%%%%%%%%%%%%%%%%%%%%%%%%%%%%%%%%%%%%%%%%%%%%%%%%%%%%%%%%%%%%%%%%%%%%%%%%%%%%%%%%%%%%%%%%%%%%%

\usepackage{eurosym}
\usepackage{vmargin}
\usepackage{amsmath}
\usepackage{graphics}
\usepackage{epsfig}
\usepackage{enumerate}
\usepackage{multicol}
\usepackage{subfigure}
\usepackage{fancyhdr}
\usepackage{listings}
\usepackage{framed}
\usepackage{graphicx}
\usepackage{amsmath}
\usepackage{chngpage}

%\usepackage{bigints}
\usepackage{vmargin}

% left top textwidth textheight headheight

% headsep footheight footskip

\setmargins{2.0cm}{2.5cm}{16 cm}{22cm}{0.5cm}{0cm}{1cm}{1cm}

\renewcommand{\baselinestretch}{1.3}

\setcounter{MaxMatrixCols}{10}

\begin{document}
\begin{enumerate}

PLEASE TURN OVER11
The movements of a consumer price index are to be subjected to time series
analysis with the aim of forecasting future behaviour. The index is calculated
monthly.
(i)
Explain whether you would expect to fit a model which included (a) a
trend term, (b) a seasonal effect.
[3]
The values {x t : 1 ≤ t ≤ n} are the residuals which remain once any trend or
seasonal variations have been removed. An ARIMA(1, 1, 1) model is to be fitted
to the {x t }.
(ii)
(iii)
(a) Assuming the ARIMA(1, 1, 1) model is correct, write down an
equation for X n +1 in terms of the white noise process
{e t : 1 ≤ t ≤ n + 1} and the observations {x t : 1 ≤ t ≤ n}.
(b) State the parameters of the model.
[2]
The Box-Jenkins procedure defines the k-step-ahead forecast for X to be
x  n ( k ) = E (X n + k  x n , x n − 1 , ..., x 1 ).
(iv)
(a) Derive the 1-step-ahead and 2-step-ahead forecasts for X for the
ARIMA(1, 1, 1) model, assuming that the values of the parameters
and the value of e 0 are known exactly.
(b) Evaluate the prediction variance Var(X n +1 − x ˆ n (1)) , again assuming
that the values of the parameters are known.
[5]
The most elementary form of the technique known as exponential
smoothing produces at time n a 1-step-ahead forecast x n * defined by
x n * = x n + ξ( x n * − 1 − x n ),
for some ξ ∈ (0, 1) which may be chosen by the user.
Show that, for particular values of the autoregressive and moving average
parameters, the Box-Jenkins forecasts above coincide with the forecasts
produced by exponential smoothing.
[2]
(v)
(vi)
103—8
(a) State whether the ARIMA(1, 1, 1) model is I(0), I(1) or neither.
(b) Discuss whether there is a difference between an I(0) model and an
I(1) model in terms of the conditional distribution of X n + k given {x t :
1 ≤ t ≤ n} for large values of k. [3]
It is suggested that a salaries index might be cointegrated with the
consumer price index.
(a) Explain what is meant by the suggestion.
(b) Comment on whether it is a reasonable suggestion.
[3]
[Total 18]




%%%%%%%%%%%%%%%%%%%%%%%%%%%%%%%%%%%%%%%%%%%%%%%%%%%%%%%%%%%%%%%%%%%%%%%%%%%
11
(i) Consumer prices do tend to exhibit regular seasonal variation, though not
a great deal these days. And, since prices tend to go up rather more than
they come down, it is probably worth including a trend term in any model.
It is certainly possible to test whether the trend term is equal to zero.
(ii) (a)
X n +1 − x n = α(x n − x n −1 ) + e n +1 + βe n .
(b)
The parameters are α, β and σ 2 e . The trend removal process would
have accounted for any μ parameter.
(iii)
x ˆ n (1) = E (X n +1  x n , ..., x 1 ) = x n + α (x n − x n −1 ) + E (e n +1 + β e n  x n , ..., x 1 ). Now
e n +1 has mean 0 and is conventionally supposed independent of everything
that happens before n.
On the other hand, e n can be deduced from past data,
e.g. e n = x n − x n −1 − α (x n −1 − x n −2 ) − β e n −1 , which may be iterated back to get
e n in terms of the known x and the known e 0 .
Thus
x ˆ n (1) = x n + α (x n − x n −1 ) + β e n .
Similarly,
x ˆ n (2) = E (X n +2  F n ) = E (X n +1 + α (X n +1 − x n ) + e n +2 + β e n +1  F n )
= (1 + α ) x ˆ n (1) − α x n .
We see that X n +1 − x ˆ n (1) = e n +1 , so that the prediction variance is just
Var(e n +1 ) = σ 2 e .
Page 8Subject 103 (Stochastic Modelling) — September 2000 — Examiners’ Report
(iv)
Since e n = x n − x ˆ n − 1 (1) , we have
x ˆ n (1) = x n + α (x n − x n−1 ) + β (x n − x ˆ n − 1 (1));
if we set α = 0 and β = −ξ , the equation is identical to the updating
equation for exponential smoothing.
(v)
An ARIMA(p, d, q) model is I(d); in this case, x is I(1).
A stationary (I(0)) model has an equilibrium distribution: the distribution
of the forecast of X n+k would converge to equilibrium for large k. An I(1)
process is the partial sum of an I(0) process, so would have increasing
variance, even if the mean happened to be stable.
(vi)
Two series {x} and {y} are cointegrated if both are I(1) but there are some
constants a and b such that {ax + by} is stationary.
Two processes are likely to be cointegrated if one drives the other, or if
both are driven by the same underlying process. In the given instance the
suggestion is certainly worth investigating.
Page 9
%%%%%%%%%%%%%%%%%%%%%%%%%%%%%%%%%%%%%%%%%%%%%%%%%%%%%%%%%%%%%%%%%%%%%%%%%%%%%%%%%%%%%%%%%%%%%%5555


\end{document}
