\documentclass[a4paper,12pt]{article}

%%%%%%%%%%%%%%%%%%%%%%%%%%%%%%%%%%%%%%%%%%%%%%%%%%%%%%%%%%%%%%%%%%%%%%%%%%%%%%%%%%%%%%%%%%%%%%%%%%%%%%%%%%%%%%%%%%%%%%%%%%%%%%%%%%%%%%%%%%%%%%%%%%%%%%%%%%%%%%%%%%%%%%%%%%%%%%%%%%%%%%%%%%%%%%%%%%%%%%%%%%%%%%%%%%%%%%%%%%%%%%%%%%%%%%%%%%%%%%%%%%%%%%%%%%%%

\usepackage{eurosym}
\usepackage{vmargin}
\usepackage{amsmath}
\usepackage{graphics}
\usepackage{epsfig}
\usepackage{enumerate}
\usepackage{multicol}
\usepackage{subfigure}
\usepackage{fancyhdr}
\usepackage{listings}
\usepackage{framed}
\usepackage{graphicx}
\usepackage{amsmath}
\usepackage{chngpage}

%\usepackage{bigints}
\usepackage{vmargin}

% left top textwidth textheight headheight

% headsep footheight footskip

\setmargins{2.0cm}{2.5cm}{16 cm}{22cm}{0.5cm}{0cm}{1cm}{1cm}

\renewcommand{\baselinestretch}{1.3}

\setcounter{MaxMatrixCols}{10}

\begin{document}
\begin{enumerate}

5
A motor insurance company assumes that a holder of a provisional driver’s
licence will make claims according to a Poisson process with rate X per year,
where X is not fixed but is determined randomly for each driver according to the
density function
f(x) = 2e −2x
6
(x > 0).
%%%%%%%%%%%%%%%%%%%%%%%%%%%%%%%%%%%%%5
\begin{framed}
\large
(i) Describe how to simulate an observation X from the density f using a
single pseudo-random variable U assumed uniformly distributed on [0, 1].

\end{framed}

%%%%%%%%%%%%%%%%%%%%%%%%%%%%%%%%%5
6
(x > 0).
(i) Describe how to simulate an observation X from the density f using a
single pseudo-random variable U assumed uniformly distributed on [0, 1].
[3]
(ii) Explain how, given the value X generated in (i), you would use a sequence
U 1 , U 2 , ... of uniform pseudo-random variables to simulate the number of
claims made in two six-month periods by a provisional driver with mean
claim rate X per year.
[4]
(iii) Describe a simulation-based method for estimating the conditional
probability that a provisional driver makes 2 or more claims in the second
six months of driving given that no claim was made in the first six
months. [Here the value X is to be assumed unknown.]

%%%%%%%%%%%%%%%%%%%%%%%%%%%%%%%%%%%%%%%%%%%%




%%%%%%%%%%%%%%%%%%%%%%%%%%%%%%%%%%%%%%%%%%%%%%%%%%%%%%%%%
5
(i)
Use the inverse distribution function technique.
We have F(x) = 1 - e - 2 x , and we require U = F(X) = 1 - e - 2 X , so that
X = - 12 log(1 - U ) .

%%%%%%%%%%%%%%%%%%%%%%%%%%%%%%%%%%%%%5
\begin{framed}
\large
(ii) Explain how, given the value X generated in (i), you would use a sequence
U 1 , U 2 , ... of uniform pseudo-random variables to simulate the number of
claims made in two six-month periods by a provisional driver with mean
claim rate X per year.
\end{framed}

(ii)
We need two Poisson pseudo-random variables, Y 1 and Y 2 , each with
mean m = 12 X .
There are two possible methods for generating Poisson(m):
6
· Keep generating exponential variables with mean 1 until the
cumulative sum exceeds m, then set Y equal to one less than the
number of variables generated.
· Draw up a table of the cumulative distribution function F m of Poisson (m), use a single uniform U and let Y be the first y such
that F m (y) > U.


%%%%%%%%%%%%%%%%%%%%%%%%%%%%%%%%%%%%%5
\begin{framed}
\large
(iii) Describe a simulation-based method for estimating the conditional
probability that a provisional driver makes 2 or more claims in the second
six months of driving given that no claim was made in the first six
months. [Here the value X is to be assumed unknown.]
\end{framed}


(iii) Carry out the above procedures a large number of times, independently.
Let N 0 be the number of times no claims were made in the first six
months, N 0,2+ the number of times no claims were made in the first six months but two or more in the next six. The required estimate is
N 0,2+ / N 0 .
