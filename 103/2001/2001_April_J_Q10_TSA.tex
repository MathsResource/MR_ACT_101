\documentclass[a4paper,12pt]{article}

%%%%%%%%%%%%%%%%%%%%%%%%%%%%%%%%%%%%%%%%%%%%%%%%%%%%%%%%%%%%%%%%%%%%%%%%%%%%%%%%%%%%%%%%%%%%%%%%%%%%%%%%%%%%%%%%%%%%%%%%%%%%%%%%%%%%%%%%%%%%%%%%%%%%%%%%%%%%%%%%%%%%%%%%%%%%%%%%%%%%%%%%%%%%%%%%%%%%%%%%%%%%%%%%%%%%%%%%%%%%%%%%%%%%%%%%%%%%%%%%%%%%%%%%%%%%

\usepackage{eurosym}
\usepackage{vmargin}
\usepackage{amsmath}
\usepackage{graphics}
\usepackage{epsfig}
\usepackage{enumerate}
\usepackage{multicol}
\usepackage{subfigure}
\usepackage{fancyhdr}
\usepackage{listings}
\usepackage{framed}
\usepackage{graphicx}
\usepackage{amsmath}
\usepackage{chngpage}

%\usepackage{bigints}
\usepackage{vmargin}

% left top textwidth textheight headheight

% headsep footheight footskip

\setmargins{2.0cm}{2.5cm}{16 cm}{22cm}{0.5cm}{0cm}{1cm}{1cm}

\renewcommand{\baselinestretch}{1.3}

\setcounter{MaxMatrixCols}{10}

\begin{document}
\begin{enumerate}
\item A family agrees an expenditure target, $Y_n$ , for year n, in such a way that the annual increase in the expenditure target is proportional to the increase in the
family income over the previous year. The actual expenditure during the year, X n , is assumed to be related to the expenditure target, but incorporating an element of randomness and a factor accounting for the family’s propensity to
overspend. The family income, $I_n$ , is assumed to grow at a constant annual rate,
before randomness is taken into account.

The head of the household believes that the following three equations form an
appropriate representation of the above information:
\begin{eqnarray*}
Y n &=& Y n − 1 + \beta(I n − 1 − I n − 2 )\\
X n &=& (1 + \pi) Y n + e n (1)\\
I n &=& (1 + \alpha) I n − 1 + e n (2)\\
\end{eqnarray*}
where ${( e n (1) , e n (2) ) : n = 1, 2, ...}$ is a sequence of zero-mean bivariate Normal random variables and $\alpha$, $\beta$ and $\pi$ are positive parameters (with \beta < 1).
103 A2001—6
\begin{enumerate}
\item (i) Express the first of the equations in terms of the backshift operator, B, and deduce that a linear relationship exists between Y n and I n − 1 .
\item (ii) Show that the process Z n = (X n , I n ) is a first-order multivariate autoregressive process.
\item (iii) State, with reasons, whether ${I n : n \geq 1}$ is a stationary time series, and hence determine whether {Z n : n \geq 1} is I(0), I(1) or neither.
\item (iv) Find an estimator for the parameter \alpha by minimising the quantity
\[Σ t n = 2 ( e t (2) ) 2 .\]
\item (v)
The head of the household wishes to perform a simulation to investigate whether the propensity to overspend will result in negative net savings.
It is assumed that Var( e n (1) ) = σ 1 2 , Var( e n (2) ) = σ 22 and Cov( e n (1) , e n (2) ) = ρσ 1 σ 2 , where $−1 < ρ < 1$.
\item (vi)
(a) Describe a method of simulating an observation of the pair
( e n (1) , e n (2) ) starting from two uniformly distributed pseudo-random variables U 1 , U 2 .
(b) Describe the role of sensitivity analysis in drawing conclusions from the simulation.
\end{enumerate}
An alternative model is proposed, involving the logarithms of the
quantities I n , X n and Y n :
\begin{eqnarray*}
ln Y n &=& ln Y n−1 + ln I n−1 − ln I n−2\\
ln X n &=& θ + ln Y n + e n (1)\\
ln I n &=& φ + ln I n−1 + e n (2)\\
\end{eqnarray*}

Discuss whether this model is more suitable than the original model. [2]
[Total 19]
%%%%%%%%%%%%%%%%%%%%%%%%%%%%%%%%%%%%55
\newpage
\begin{itemize}
\item 
(i) $(1 − B) Y = \beta B (1 − B) I$,
with solution Y = \beta BI + const
\item (ii) We have the vector equation
years.
æ X n ö æ 0 (1 + \pi ) \beta ö æ X n − 1 ö æ const ö æ e n (1) ö
ç
÷ = ç
÷ + ç
÷ ç
÷ + ç ç (2) ÷ ÷ ,
è I n ø è 0 1 + \alpha ø è I n − 1 ø è 0 ø è e n ø
which clearly represents a vector AR(1).
\item (iii)
I is not stationary: the condition for an AR(1) to be stationary is that the
autoregressive parameter is less than 1 in absolute value.
I is not I(1), either, since ∇I = e (2) + \alpha BI which, as already stated, is not
stationary.
Z is therefore neither I(0) nor I(1).
\item (iv)
The equation for the sum of squares is
SS =
n
å
( e t (2) ) 2 =
t = 2
n
å ( I
t = 2
t
− (1 + \alpha ) I t − 1 ) 2 .
\item Differentiating,
0 = − 2
n
å I
t = 2
t − 1 ( I t
− (1 + \alpha ) I t − 1 ),
%%-- Page 7Subject 103 (Stochastic Modelling) — April 2001 — Examiners’ Report
implying that
\alphâ =
(v)
(a)
Σ nt = 2 I t − 1 ( I t − I t − 1 )
Σ t n = 2 I t 2 − 1
.
\item First we need to obtain N(0,1) variates Z 1 and Z 2 . The core reading
mentions two methods: either
Z 1 =
− 2 lnU 1 sin(2\piU 2 ),
Z 2 =
− 2 lnU 1 cos(2\piU 2 )
or
Z 1 = V 1
− 2 ln S
,
S
where V i = 2U i − 1, S =
Z 2 = V 2
− 2 ln S
,
S
V 1 2 + V 2 2 and any values of U 1 and U 2
which give S > 1 are rejected.
Now define E 1 = σ 1 Z 1 , E 2 = ρσ 2 Z 1 +
(b)
(vi)
%%-Page 8
1 − ρ 2 σ 2 Z 2 .
\item Sensitivity analysis applies mostly to the initial assumptions.
Values for the parameters \alpha, \beta, \pi, σ 1 , σ 2 and ρ must be assumed,
but may not correspond exactly to the actual situation. The head
of household should investigate whether making small changes to
the values used will make large differences to the conclusions.
The revised model still meets the requirements set down at the start of
the problem. It is likely to prove more tractable in that ln I is now a
simple random walk with drift, and is therefore I(1).
\end{itemize}
\end{document}
