\documentclass[a4paper,12pt]{article}

%%%%%%%%%%%%%%%%%%%%%%%%%%%%%%%%%%%%%%%%%%%%%%%%%%%%%%%%%%%%%%%%%%%%%%%%%%%%%%%%%%%%%%%%%%%%%%%%%%%%%%%%%%%%%%%%%%%%%%%%%%%%%%%%%%%%%%%%%%%%%%%%%%%%%%%%%%%%%%%%%%%%%%%%%%%%%%%%%%%%%%%%%%%%%%%%%%%%%%%%%%%%%%%%%%%%%%%%%%%%%%%%%%%%%%%%%%%%%%%%%%%%%%%%%%%%

\usepackage{eurosym}
\usepackage{vmargin}
\usepackage{amsmath}
\usepackage{graphics}
\usepackage{epsfig}
\usepackage{enumerate}
\usepackage{multicol}
\usepackage{subfigure}
\usepackage{fancyhdr}
\usepackage{listings}
\usepackage{framed}
\usepackage{graphicx}
\usepackage{amsmath}
\usepackage{chngpage}

%\usepackage{bigints}
\usepackage{vmargin}

% left top textwidth textheight headheight

% headsep footheight footskip

\setmargins{2.0cm}{2.5cm}{16 cm}{22cm}{0.5cm}{0cm}{1cm}{1cm}

\renewcommand{\baselinestretch}{1.3}

\setcounter{MaxMatrixCols}{10}

\begin{document}
\begin{enumerate}


%%%%%%%%%%%%%%%%%%%%%%%%%%%%%%%5

The evolution of a stock price S t is modelled by
\[S_t = e \mu t +σ B t ,\]
where B t represents a standard Brownian motion, $\mu$ and $\sigma$ are fixed parameters
and the initial value of the stock is S 0 = 1.
\begin{enumerate}[(i)]
\item (i) Derive an expression for $P{S t \leq x}$.% [2]
\item (ii) Derive expressions for the median of S t and the expectation of S t . [4]
\item (iii) (a) Determine an expression for the conditional expectation E(S t F s ),
where $s < t$ and where ${F s : s \geq 0}$ denotes the filtration associated
with the process S.
(b) Find conditions on \mu and σ under which the process {S t : t \geq 0} is a
martingale.
(c) State, with reasons, whether or not the stock would be a good long
term investment in this case.
\end{enumerate}
%%%%%%%%%%%%%%%%%%%%%%%%%%%%%%%%%%%%%%
%%103 A2001—48
%%--Question 8 
\item A company assesses the credit-worthiness of various firms every quarter; the ratings are, in order of decreasing merit, A, B, C and D (default). Historical data support the view that the credit rating of a typical firm evolves as a Markov
chain with transition matrix
æ 1 − \alpha − \alpha 2
\alpha
\alpha 2
ç
\alpha
1 − 2 \alpha − \alpha 2
\alpha
P = ç
2
ç
\alpha
\alpha
1 − 2 \alpha − \alpha 2
ç ç
0
0
0
è
0 ö
÷
\alpha 2 ÷
\alpha ÷
÷
1 ÷ ø
for some parameter \alpha

\begin{enumerate}[(i)]
\item (i) Draw the transition graph of the chain. [2]
\item (ii) Determine the range of values of \alpha for which the matrix P is a valid
transition matrix. [2]
\item (iii) State, with reasons, whether the chain is irreducible and aperiodic.
\item (iv) Derive a stationary probability distribution for the chain and establish
whether it is unique.
[4]
\item (v) For the value \alpha = 0.1, calculate the probability that the company’s rating
in the third quarter, X 3 , is in the default state D:
[2]
(a) in the case where the company’s rating in the first quarter, X 1 , is
equal to A
(b) in the case X 1 = B
(c) in the case X 1 = C
(d) in the case X 1 = D
\end{enumerate}
\newpage

%%%%%%%%%%%%%%%%%%%%%%%%%%%%%%%%%%%
7
(i)
P{S t ≤ x} = P{exp (\mut + σB t ) ≤ x} = P{\mut + σ tN ≤ ln(x)} = Φ
(
ln( x ) − \mu t
σ t
) , where
Φ denotes the standard Normal distribution function.
(ii)
We have to find m so that P{S t ≤ m} = P { e \mu t +σ B t ≤ m } = P{\mut + σB t ≤ ln (m)}
= P{σB t ≤ ln (m) −\mut} =
1
2
. Since σB t is a symmetric normal variable, its
median is 0 and the last equation can only be satisfied if ln(m) − \mut = 0
and m = e \mut .
The expectation is ES t = Ee (\mut+σB(t)) = Ee \mut e σ
Page 4
tN
σ 2 t
= e \mu t e 2 = e
( \mu+ ) t .
σ 2
2Subject 103 (Stochastic Modelling) — April 2001 — Examiners’ Report
(iii)
By the same token, E(S t F s ) = S(s) E(e
\mu ( t−s )+ σ ( B ( t ) −B ( s ))
) = S(s)
( \mu+ ) ( t − s ) .
e
σ 2
2
If S is to be a martingale, the conditional expectation must be equal to
S(s).
This will happen if \mu = − 12 σ 2 .
From part (ii) we see that for this stock with initial value 1, the median of
the distribution at time t goes to 0 exponentially fast for large t hence, a
very bad investment!
8
(i)
Transition Graph
1 − \alpha − \alpha 2
1 − 2\alpha − \alpha 2
\alpha
A
B
\alpha
\alpha 2
\alpha 2
\alpha 2
\alpha
\alpha
C
D
\alpha
1
1 − 2\alpha − \alpha 2
(ii) All transition probabilities must lie in [0,1].
Now 1 − 2\alpha − \alpha 2 ≤ 1 − \alpha − \alpha 2 ≤ 1 for \alpha \geq 0, so it suffices to ensure that 1 − 2\alpha − \alpha 2 \geq 0 i.e. \alpha ≤ 2 − 1. So the range of possible values of
\alpha is [0, 2 − 1].
(iii) The chain is not irreducible since D is a trap state.
The chain is aperiodic by inspection.
(iv) A stationary probability distribution, if it exists, must obey
(1 − \alpha − \alpha 2 ) π A + \alphaπ B + \alpha 2 π C
\alphaπ A + (1 − 2\alpha − \alpha 2 ) π B + \alphaπ C
\alpha 2 π A + \alphaπ B + (1 − 2\alpha − \alpha 2 ) π C
\alpha 2 π B + \alphaπ C + π D
=
=
=
=
π A
π B
π C
π D
The last equation implies π B = π C = 0, and this in turn shows that π A = 0.
Hence the stationary probability distribution is π = (0, 0, 0, 1) T .
Page 5Subject 103 (Stochastic Modelling) — April 2001 — Examiners’ Report
It is unique: there is just one recurrent class and it is aperiodic. (Or point out that there is no other solutions to the equations.)
(v)
With \alpha = 0.1, the transition matrix is
0 ö
æ 0.89 0.1 0.01
ç
÷
ç 0.1 0.79 0.1 0.01 ÷
ç 0.01 0.1 0.79 0.1 ÷
ç ç
÷
0
0
1 ÷ ø
è 0
Its square is
æ 0.8022 0.169 0.0268 0.002 ö
ç
÷
ç 0.169 0.6441 0.159 0.0279 ÷
ç 0.0268 0.159 0.6342 0.18 ÷
ç ç
÷
0
0
1 ÷ ø
è 0
the relevant entries being the last column.
