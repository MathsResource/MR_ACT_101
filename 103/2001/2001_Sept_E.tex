\documentclass[a4paper,12pt]{article}

%%%%%%%%%%%%%%%%%%%%%%%%%%%%%%%%%%%%%%%%%%%%%%%%%%%%%%%%%%%%%%%%%%%%%%%%%%%%%%%%%%%%%%%%%%%%%%%%%%%%%%%%%%%%%%%%%%%%%%%%%%%%%%%%%%%%%%%%%%%%%%%%%%%%%%%%%%%%%%%%%%%%%%%%%%%%%%%%%%%%%%%%%%%%%%%%%%%%%%%%%%%%%%%%%%%%%%%%%%%%%%%%%%%%%%%%%%%%%%%%%%%%%%%%%%%%

\usepackage{eurosym}
\usepackage{vmargin}
\usepackage{amsmath}
\usepackage{graphics}
\usepackage{epsfig}
\usepackage{enumerate}
\usepackage{multicol}
\usepackage{subfigure}
\usepackage{fancyhdr}
\usepackage{listings}
\usepackage{framed}
\usepackage{graphicx}
\usepackage{amsmath}
\usepackage{chngpage}

%\usepackage{bigints}
\usepackage{vmargin}

% left top textwidth textheight headheight

% headsep footheight footskip

\setmargins{2.0cm}{2.5cm}{16 cm}{22cm}{0.5cm}{0cm}{1cm}{1cm}

\renewcommand{\baselinestretch}{1.3}

\setcounter{MaxMatrixCols}{10}

\begin{document}
\begin{enumerate}9
[3]
(a) Using the filtered series {Y t } obtained in (iii), derive an expression
for ∇Y t in terms of $\mu$, $\beta$, X t , X t − 1 , ...
(b) State, with reasons, whether {Y t } is I(0), I(1) or neither.
[4]
[Total 12]
In the Vasicek model, the spot rate of interest is governed by the stochastic differential equation
\[dr t = a(b − r t ) dt + σdB t\]
where B t is a standard Brownian motion and a, b > 0.
(i)
A stochastic process {U t : t ≥ 0} is defined by U t = e at r t .
(a) Derive an equation for dU t .
(b) Solve the equation to find U t .
(c) Show that
r t = b + (r 0 − b) e − at + σ
103 S2001—6
ò
t
0
e a ( s − t ) dB s
(ii) State the probability distribution of r t and its limit for large t.
(iii) Derive, in the case s < t, the conditional expectation E[r t . s ], where {. s : s ≥ 0} is the filtration generated by the Brownian motion B.
10
[4]
[3]
[Total 12]
%%%%%%%%%%%%%%%%%%%%%%%%%
9
dU t = d(e at r t ) = ae at r t dt + e at dr t
(i)
= e at [ar t dt + ab dt - ar t dt + sdB t ]
= e at (ab dt + sdB t ) .
Hence
= U 0 + ab
U t
ò
t
0
e as ds + s
= r 0 + b(e at - 1) + s
ò
t
0
ò
t
0
e as dB s
e as dB s
Thus
= e - at U t = b + (r 0 - b) e - at + s
r t
(ii)
From above, r t is Gaussian
with mean b + (r 0 - b) e - at and variance
s 2
Page 6
ò
t
0
e 2 a ( s - t ) ds = s 2
1 - e - 2 at
.
2 a
ò
t
0
e a ( s - t ) dB s .Subject 103 (Stochastic Modelling) — 
%%%%%%%%%%%%%%%%%%%%%%%%%%%%%%%%

As t ® ¥, the distribution of the spot rate is N(b, s 2 / 2a).
(iii)
r t
= b + (r 0 - b) e - at + se - at
ò
t
0
e au dB u .
Hence, by the martingale property of Itô integrals
E[r t 1⁄2F s ] = b + (r 0 - b) e - at + se - at E é ê
ë
= b + (r 0 - b) e - at + se - at
ò
s
0
ò
t
0
e au dB u 1⁄2 F s ù ú
û
e au dB u
= e a(s - t) r s + b(1 - e a(s - t) ).
\end{document}
