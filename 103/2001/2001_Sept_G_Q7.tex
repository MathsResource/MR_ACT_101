\usepackage{vmargin}
\usepackage{amsmath}
\usepackage{graphics}
\usepackage{epsfig}
\usepackage{enumerate}
\usepackage{multicol}
\usepackage{subfigure}
\usepackage{fancyhdr}
\usepackage{listings}
\usepackage{framed}
\usepackage{graphicx}
\usepackage{amsmath}
\usepackage{chngpage}

%\usepackage{bigints}
\usepackage{vmargin}

% left top textwidth textheight headheight

% headsep footheight footskip

\setmargins{2.0cm}{2.5cm}{16 cm}{22cm}{0.5cm}{0cm}{1cm}{1cm}

\renewcommand{\baselinestretch}{1.3}

\setcounter{MaxMatrixCols}{10}

\begin{document}

[Total 11]7
According to an interest rate model which operates in continuous time, the
interest rate r(t) may change only by upward jumps of fixed size j u or by
downward jumps of fixed size j d (where j d < 0), occurring independently according
to Poisson processes N u (t) (with rate \lambda u ) and N d (t) (with rate \lambda d ).
Let T u denote the time of the first up jump in the interest rate, T d the time of the
first down jump, T = min(T u , T d ) the time of the first jump. Further, let I be
defined as an indicator taking the value 1 if the first jump is an up jump or 0
otherwise.
\begin{enumerate}
\item (i) Determine expressions for the probabilities P{T u > t}, P{T d > t} and P{T > t}
\item 
(ii) Determine the distribution of I.
\item 
(iii) Show, by evaluating P{T > t and I = 1}, that I and T are independent
random variables.
\item 
(iv) Calculate the expectation and variance of the interest rate at time t given
the current rate r(0).
Hint: r(t) = r(0) + j u N u (t) + j d N d (t).
\item 
Show that {r(t) : t ≥ 0} is a process with stationary, independent
increments.
\end{enumerate}
\newpage
%%%%%%%%%%%%%%%%%%%%%%%%%%%%%%%%%%%%%%%%%%%%%%%%%%%%%%%%%%%%%%%%%%
7
q k - 1
q K - 1
.
(i) P{T u > t} = e -l u t , P{T d > t} = e -l d t , P{T > t} = P{T u > t} P{T d > t} = e - ( l u +l d ) t
(ii) P{I = 1} = P{T d > T u } = ò 0 ¥ l u e -l u t P{T d > t} dt =
(iii) P{T > t Ç I = 1} = ò 0 ¥ l u e -l u t P{T d > t} dt =
l u
l u + l d
l u
l u + l d
e - ( l u +l d ) t
This is the product of P{T > t} and P{I = 1}.
8
(iv) Er t = r 0 + (j u l u + j d l d ) t and Var r t = j u 2 l u t + j d 2 l d t
(v) Take 0 < s < t. Then r t+s - r t = j u (N u (t + s) - N u (t)) + j d (N d (t + s) - N d (t)).
Now N u (t + s) - N u (t) has a Poisson(l u s) distribution, not depending on t
and independent of {N u (r) : 0 \leq r \leq s}, and N d (t + s) - N d (t) has a
Poisson(l d s) distribution, also independent of t and of {N d (r) : 0 \leq r \leq s}, so
r t+s - r t has the same distribution as r s - r 0 and is independent of r s - r 0 .
\end{document}
