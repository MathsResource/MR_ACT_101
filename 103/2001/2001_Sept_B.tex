\documentclass[a4paper,12pt]{article}

%%%%%%%%%%%%%%%%%%%%%%%%%%%%%%%%%%%%%%%%%%%%%%%%%%%%%%%%%%%%%%%%%%%%%%%%%%%%%%%%%%%%%%%%%%%%%%%%%%%%%%%%%%%%%%%%%%%%%%%%%%%%%%%%%%%%%%%%%%%%%%%%%%%%%%%%%%%%%%%%%%%%%%%%%%%%%%%%%%%%%%%%%%%%%%%%%%%%%%%%%%%%%%%%%%%%%%%%%%%%%%%%%%%%%%%%%%%%%%%%%%%%%%%%%%%%

\usepackage{eurosym}
\usepackage{vmargin}
\usepackage{amsmath}
\usepackage{graphics}
\usepackage{epsfig}
\usepackage{enumerate}
\usepackage{multicol}
\usepackage{subfigure}
\usepackage{fancyhdr}
\usepackage{listings}
\usepackage{framed}
\usepackage{graphicx}
\usepackage{amsmath}
\usepackage{chngpage}

%\usepackage{bigints}
\usepackage{vmargin}

% left top textwidth textheight headheight

% headsep footheight footskip

\setmargins{2.0cm}{2.5cm}{16 cm}{22cm}{0.5cm}{0cm}{1cm}{1cm}

\renewcommand{\baselinestretch}{1.3}

\setcounter{MaxMatrixCols}{10}

\begin{document}
\begin{enumerate}
%%-- Question 3
\item A stationary stochastic process {Y t : t = 0, 1, ...} satisfies the relationship Y t = \mu + 0.8(Y t−1 − \mu) − 0.4(Y t−2 − \mu) + e t ,
where {e t : t = 0, 1, ...} is a sequence of independent, zero-mean Normal random variables with common variance σ 2 .
%%--Question 4
\item (i) Calculate the autocorrelation function, ρ k , and the partial autocorrelation function, φ k , of Y for k = 1 and 2.

(ii) State, without performing additional calculations, what you would expect to find if you were to calculate ρ k and φ k for larger values of k.
[Total 7]
Consider the simplified model of credit rating of companies in continuous time described below. There are three ratings which a company can have, A, B and D (default) and the possible transitions are as follows:
•
•
•
(i)
from A to B with rate 4\alpha
from B to A with rate \alpha
from B to D with rate 3\alpha
Write down the matrix form of Kolmogorov’s forward equations as it
applies to this model and verify that the transition matrix P(t) = P(0, t)
given below is a solution:
æ 1 2 e − 2 \alpha t + 1 2 e − 6 \alpha t
ç
P(t) = ç 1 4 e − 2 \alpha t − 1 4 e − 6 \alpha t
ç
0
è
(ii)
e − 2 \alpha t − e − 6 \alpha t
1
2
e − 2 \alpha t + 1 2 e − 6 \alpha t
0
1 − 2 3 e − 2 \alpha t + 1 2 e − 6 \alpha t ö
÷
1 − 4 3 e − 2 \alpha t − 1 4 e − 6 \alpha t ÷ .
÷
1
ø

Find the time τ after which a company starting in state A is more likely to
be in state D than in state A.

[Total 7]
\end{document}
