\documentclass[a4paper,12pt]{article}

%%%%%%%%%%%%%%%%%%%%%%%%%%%%%%%%%%%%%%%%%%%%%%%%%%%%%%%%%%%%%%%%%%%%%%%%%%%%%%%%%%%%%%%%%%%%%%%%%%%%%%%%%%%%%%%%%%%%%%%%%%%%%%%%%%%%%%%%%%%%%%%%%%%%%%%%%%%%%%%%%%%%%%%%%%%%%%%%%%%%%%%%%%%%%%%%%%%%%%%%%%%%%%%%%%%%%%%%%%%%%%%%%%%%%%%%%%%%%%%%%%%%%%%%%%%%

\usepackage{eurosym}
\usepackage{vmargin}
\usepackage{amsmath}
\usepackage{graphics}
\usepackage{epsfig}
\usepackage{enumerate}
\usepackage{multicol}
\usepackage{subfigure}
\usepackage{fancyhdr}
\usepackage{listings}
\usepackage{framed}
\usepackage{graphicx}
\usepackage{amsmath}
\usepackage{chngpage}

%\usepackage{bigints}
\usepackage{vmargin}

% left top textwidth textheight headheight

% headsep footheight footskip

\setmargins{2.0cm}{2.5cm}{16 cm}{22cm}{0.5cm}{0cm}{1cm}{1cm}

\renewcommand{\baselinestretch}{1.3}

\setcounter{MaxMatrixCols}{10}

\begin{document}
%% \begin{enumerate}
%% Question 9
A continuous-time Markov sickness and death model has four states: H (healthy),
S (sick), T (terminally ill) and D (dead). From a healthy state transitions are possible to states S and D, each at rate 0.05 per year. A sick person recovers his health at rate 1.0 per year; other possible transitions are to D and T, each with
rate 0.1 per year. Only one transition is possible from the terminally ill state, and that is to state D with transition rate 0.4 per year.
\begin{enumerate}[(a)]
\item Draw the transition graph for this process.
\item Define $P(t) = {p ij (t) : i, j \in H, S, T, D}$ where p ij (t) denotes the probability of being in state j at time t given that the individual was in state $i$ at time 0.
State the Kolmogorov forward equation satisfied by the matrix P(t), making sure that you specify the entries of the matrix A which appears.

\item Calculate the probability of being healthy for at least 10 uninterrupted years given that you are healthy now.

\item Let d j denote the probability that a life which is currently in state j will never suffer a terminal illness. By considering the first transition from
1
state H, show that d H = 1 2 + 1 2 d S and deduce similarly that d S = 12
+ 5 6 d H .
\end{enumerate}

Hence evaluate d H and d S .
(v)
10
Write down the expected duration of a terminal illness, starting from the moment of the first transition into state T. Use the result of (iv) to deduce
the expectation of the future time spent terminally ill by an individual
who is currently healthy.

%%%%%%%%%%%%%%%%%%%%%%%%%%%%%%%%%%%%%%%%%%%%%%%%%%%%%%%%%%%%%%%%%%%%%%%%%%%%%%%%%%%%%%%%%%%
\newpage
9
\begin{itemize}
\item (i)
Transition Graph:
0.05
H
0.1
S
T
1.0
0.05
0.1
0.4
D
\item (ii)
KFE: P ′ ( t ) = P(t) A,
0
0.05 ö
æ − 0.1 0.05
ç
÷
1.0 − 1.2 0.1 0.1 ÷
ç
A =
.
ç 0
0
− 0.4 0.4 ÷
ç ç
÷
0
0
0 ÷ ø
è 0

\item (iii)
Page 6
∞
0.1e − 0.1 x dx = e − 1 .
The probability of staying in state H for 10 years is ò 10
%%-- Subject 103 (Stochastic Modelling) — April 2001 — Examiners’ Report
\item (iv)
First transition from H must be to S or D, each equally likely. If to D, then it is certain that no terminal illness will occur; otherwise, the probability
of avoiding a terminal illness is d S .

From S similarly, except that the transition probabilities are to H with
0.1
1
prob. 1.0
= 5 6 , to D or to T, each with prob. 1.2
= 12
. Once in T it is not
1.2
possible to avoid terminal illness.
Solving the above equations, d S =
d H =
(v)
13
14
1
12
+
5
6
× 1 2 (1 + d S ), implying that d S =
6
7
,
.
The Markov property implies that the time spent in state T has
exponential distribution. The rate is 0.4 per year, so the expectation is 2.5
years.
The expected time spent in terminal illness given current health is 2(ever
hit T X 0 = H) × 2.5 years =
10
2.5
14

\end{itemize}
\end{document}
