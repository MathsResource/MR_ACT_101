\documentclass[a4paper,12pt]{article}

%%%%%%%%%%%%%%%%%%%%%%%%%%%%%%%%%%%%%%%%%%%%%%%%%%%%%%%%%%%%%%%%%%%%%%%%%%%%%%%%%%%%%%%%%%%%%%%%%%%%%%%%%%%%%%%%%%%%%%%%%%%%%%%%%%%%%%%%%%%%%%%%%%%%%%%%%%%%%%%%%%%%%%%%%%%%%%%%%%%%%%%%%%%%%%%%%%%%%%%%%%%%%%%%%%%%%%%%%%%%%%%%%%%%%%%%%%%%%%%%%%%%%%%%%%%%

\usepackage{eurosym}
\usepackage{vmargin}
\usepackage{amsmath}
\usepackage{graphics}
\usepackage{epsfig}
\usepackage{enumerate}
\usepackage{multicol}
\usepackage{subfigure}
\usepackage{fancyhdr}
\usepackage{listings}
\usepackage{framed}
\usepackage{graphicx}
\usepackage{amsmath}
\usepackage{chngpage}

%\usepackage{bigints}
\usepackage{vmargin}

% left top textwidth textheight headheight

% headsep footheight footskip

\setmargins{2.0cm}{2.5cm}{16 cm}{22cm}{0.5cm}{0cm}{1cm}{1cm}

\renewcommand{\baselinestretch}{1.3}

\setcounter{MaxMatrixCols}{10}

\begin{document}
\begin{enumerate}
%%--Question 5

%%%%%%%%%%%%%%%%
\begin{enumerate}[(a)]
\item (i) Derive expressions for $\rho_1$ and $\rho_2$ , the autocorrelation function of X at lags
1 and 2, in the case that X is a stationary process satisfying the recursion:
[1]
\[X t = \alpha X t−1 + e t + \beta e t−1 ,\]
where ${e t : t = 1, 2, ...}$ is a sequence of uncorrelated random variables with
[5]
mean 0, variance $\sigma^2$ .
\item (ii)
A company’s monthly sales figures, corrected for trend and seasonal factors, exhibit sample autocorrelation function at lags 1 and 2 of r 1 = 0.5,
r 2 = 0.4. Find method of moments estimators of \alpha and \beta for the model
in (i).
\end{enumerate}

\newpage
%%%%%%%%%%%%%%%%%%%%%%%%
5
\begin{itemize}
    \item 

(i) Let γ k denote the autocovariance function of X. Then
\begin{eqnarray*}
\operatorname{Cov}(X t , e t ) &=& 0 + \sigma^2 + 0 = \sigma^2 ;\\
\operatorname{Cov}(X t , e t − 1 ) &=& \alpha\; \gamma 0 + 0 + \beta\sigma^2 ;\\
γ 2 &=& \alpha\; \gamma 1\\
γ 1 &=& \alpha\; \gamma 0 + 0 + \beta \operatorname{Cov}(X t − 1 , e t − 1 ) = \alpha\; \gamma 0 + \beta\sigma^2\\
γ 0 &=& \alpha\; \gamma 1 + \operatorname{Cov}(X t , e t ) + \beta \operatorname{Cov}(X t , e t − 1 ) = \alpha 2 γ 0 + (1 + 2\alpha\beta + \beta 2 ) \sigma^2 ,\\
\end{eqnarray*}

implying that
(ii)
    \item γ 0 = \sigma^2
(1 + 2 \alpha\beta + \beta 2 ) ,
2
1 −\alpha
\rho_1= ( \alpha + \beta ) (1 + \alpha\beta )
, \rho_2 = \alpha\rho_1.
1 + 2 \alpha\beta + \beta 2
\item Estimate of \alpha is r 2 / r 1 = 0.8; estimate of \beta is given by
1
2
(1 + 2\alpha\beta + \beta 2 ) = (\alpha + \beta) (1 + \alpha\beta), or 0.3\beta 2 + 0.84\beta + 0.3 = 0, with solution
\beta = −1.4 ±
0.96 .
\item In this case we take the positive square root to ensure invertibility.
\end{itemize}
Page 3Subject 103 (Stochastic Modelling) — April 2001 — Examiners’ Report

\end{document}
