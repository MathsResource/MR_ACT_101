\documentclass[a4paper,12pt]{article}

%%%%%%%%%%%%%%%%%%%%%%%%%%%%%%%%%%%%%%%%%%%%%%%%%%%%%%%%%%%%%%%%%%%%%%%%%%%%%%%%%%%%%%%%%%%%%%%%%%%%%%%%%%%%%%%%%%%%%%%%%%%%%%%%%%%%%%%%%%%%%%%%%%%%%%%%%%%%%%%%%%%%%%%%%%%%%%%%%%%%%%%%%%%%%%%%%%%%%%%%%%%%%%%%%%%%%%%%%%%%%%%%%%%%%%%%%%%%%%%%%%%%%%%%%%%%

\usepackage{eurosym}
\usepackage{vmargin}
\usepackage{amsmath}
\usepackage{graphics}
\usepackage{epsfig}
\usepackage{enumerate}
\usepackage{multicol}
\usepackage{subfigure}
\usepackage{fancyhdr}
\usepackage{listings}
\usepackage{framed}
\usepackage{graphicx}
\usepackage{amsmath}
\usepackage{chngpage}

%\usepackage{bigints}
\usepackage{vmargin}

% left top textwidth textheight headheight

% headsep footheight footskip

\setmargins{2.0cm}{2.5cm}{16 cm}{22cm}{0.5cm}{0cm}{1cm}{1cm}

\renewcommand{\baselinestretch}{1.3}

\setcounter{MaxMatrixCols}{10}

\begin{document}
\begin{enumerate} 103 S2001—35



%%%%%%%%%%%%%%%%%%%%%%%%%%%%%%%%%%%%%%%%%%%%%%%%%%%%%%%%%%%%%%%%

The evolution of a stock price is modelled as a discrete time process S n = Σ i=
1 X i ,
where X 1 , X 2 , ... are independent, identically distributed random variables with
P{X i = 1} = p and P{X i = −1} = q = 1 − p. The investment will be liquidated at
either the bankruptcy time T_{0} (the first time n when the price S n hits 0) or the
first time T_{K} when the price attains a fixed target $K$, whichever occurs first.
Let T = min(T_{0} , T_{K} ) denote the liquidation time (the exit time from [0, K]). Let
A = {T_{K} < T_{0} } denote the event that the target is met before bankruptcy and let
p k = P[AS 0 = k] denote the probability of this event, given that the initial price is
S 0 = k, where k ∈ {1, ..., K − 1}.
\begin{enumerate}
\item (i) By conditioning on the price of the stock at time 1, determine a difference
equation satisfied by p k , k = 1, ..., K − 1.

\item (ii) Assume that p = q =
(iii)
1
2
.
(a) Show that S n is a martingale.
(b) Derive an expression for p k by applying the optional stopping
theorem to this martingale stopped at T.

\item Assume now that p \neq q.
(a)
(b)
103 S2001—4
Determine a value $\theta \neq 1$ such that Y n = \theta S n is a martingale.
[4]
Derive an expression for p k in this case.
\end{enumerate}
%%%%%%%%%%%%%%%%%%%%%%%%%%%%%%%%%%%%%%%%%%%%%%%%%%%%%%%%%55
6
· Keep generating exponential variables with mean 1 until the
cumulative sum exceeds m, then set Y equal to one less than the
number of variables generated.
· Draw up a table of the cumulative distribution function F m of
Poisson (m), use a single uniform U and let Y be the first y such
that F m (y) > U.
(iii) Carry out the above procedures a large number of times, independently.
Let N 0 be the number of times no claims were made in the first six
months, N 0,2+ the number of times no claims were made in the first six
months but two or more in the next six. The required estimate is
N 0,2+ / N 0 .
(i) Let A = {T_{K} < T_{0} }, U = {S 1 - S 0 = 1}, D = {S 1 - S 0 = -1}. Conditioning after
one step, we find:
p k = P[A1⁄2S 0 = k] = pP[A1⁄2S 0 = k Ç U] + qP[A1⁄2S 0 = k Ç D] = p p k +1 + q p k - 1 .
(Together with p K = 1, p 0 = 0 this could be used to solve p k ...)
(ii)
(a)
Let F n = s{X 1 , X 2 , ..., X n }. To show S n is a martingale we need to
show that the conditional expectation of the increments of S n is 0.
E[S n +1 - S n 1⁄2F n ] = E[X n +1 1⁄2F n ] = E[X n +1 ] = 0.
(b)
By the optional stopping theorem,
E k (S T ) = S 0 = k
On the other hand
E k (S T ) = K p k + 0 (1 - p k ) = k,


%%%%%%%%%%%%%%%%%%%%%%%%%%%%%%%%%%%%%%%%%%%%%%%%%%%%%%%%%%%
which gives
p k =
(iii)
k
.
K
The exponential process Y n = q S n is a martingale if and only if the
expectation of the factors is 1, i.e. if and only if Eq X 1 = (pq + qq - 1 ) = 1. This
equation has two roots q = 1, q = qp .
Applying the optional stopping theorem to the martingale yielded by the
second root gives: q k = q K p k + q 0 (1 - p k ) and p k =
7
q k - 1
q K - 1
.

\end{document}
