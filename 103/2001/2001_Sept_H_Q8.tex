

\usepackage{vmargin}
\usepackage{amsmath}
\usepackage{graphics}
\usepackage{epsfig}
\usepackage{enumerate}
\usepackage{multicol}
\usepackage{subfigure}
\usepackage{fancyhdr}
\usepackage{listings}
\usepackage{framed}
\usepackage{graphicx}
\usepackage{amsmath}
\usepackage{chngpage}

%\usepackage{bigints}
\usepackage{vmargin}

% left top textwidth textheight headheight

% headsep footheight footskip

\setmargins{2.0cm}{2.5cm}{16 cm}{22cm}{0.5cm}{0cm}{1cm}{1cm}

\renewcommand{\baselinestretch}{1.3}

\setcounter{MaxMatrixCols}{10}

\begin{document}






[Total 12]8
A company keeps records of quarterly sales figures, {S t : t = 1, 2, ...n} for the most
recent n quarters. It wishes to analyse the records with the aim of predicting the
sales figures in the near future.
The model suggested by the company is:
\[\log S t = \mu + \betat + \theta Q ( t ) + X t ,\]
where {X t : t = 1, 2, ..., n} is a stationary time series, Q(t) takes the value 1, 2, 3 or
4 depending on whether the tth quarter is the first, second, third or fourth
quarter of the financial year, and \theta 1 + \theta 2 + \theta 3 + \theta 4 = 0.
(i) Explain why the company has suggested a linear model for log S t rather
than a linear model for S t .
[1]
(ii) Explain the significance of the parameters \mu, \beta and {\theta q : 1 \leq q \leq 4} and give
[4]
a reason for the assumption that \theta 1 + \theta 2 + \theta 3 + \theta 4 = 0.
(iii) Derive a linear filter Y t = Σ + k 2= − 2 a k log S t + k which has the property that the
filtered series {Y t } does not depend on {\theta q : 1 \leq q \leq 4}.
(iv)
[3]
(a) Using the filtered series {Y t } obtained in (iii), derive an expression
for ∇Y t in terms of $\mu$, $\beta$, X t , X t − 1 , ...
(b) State, with reasons, whether {Y t } is I(0), I(1) or neither.
[4]
[Total 12]
\newpage
%%%%%%%%%%%%%%%%%%%%%%%%%%%%%%%%%%%%%%%%%%%%%%%%%%%%%%%%%%%
We assume that Sq q = 0 because any non-zero value could be subsumed in
m; estimation procedures report an indeterminacy if we do not make this
assumption.
(ii)
We need to ensure that each q q has a coefficient of
1
4 (assuming that the
1
8 X t+2 .
filter coefficients add to 1).
The filter
(iv)
( 1 8 , 1 4 , 1 4 , 1 4 , 1 8 ) will do.
We have
Y t = m + bt +
1
8
X t - 2 +
1
4
X t - 1 +
1
4
X t +
1
4
X t+1 +
Hence
ÑY t = Y t - Y t - 1 = b +
1
8
(X t+2 + X t+1 - X t - 2 - X t - 3 ) .
Clearly Y is not stationary, if only because it has a trend in the mean.
ÑY, however, does look stationary, so it is reasonable to claim that Y is
I(1).


%%%%%%%%%%%%%%%%%%%%%%%
(i) Fluctuations in sales tend to be proportional to sales in the sense that
Var(S t+1 -S t ) \mu S t 2 , company growth tends to be exponential rather than
linear. (Either of these explanations is sufficient.)
(ii) m + bt represents a deterministic linear trend in the main sales volume; b
is related to the average annual percentage increase in sales, whereas m is
related to the initial value.
The q q refer to predictable seasonal fluctuations above and below the
average from one quarter to another due to the weather, timing of
Christmas, etc.
Page 5Subject 103 (Stochastic Modelling) — 
%%%%%%%%%%%%%%%%%%%%%%%%%%%%%%%%
\end{document}
