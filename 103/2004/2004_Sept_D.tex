\documentclass[a4paper,12pt]{article}

%%%%%%%%%%%%%%%%%%%%%%%%%%%%%%%%%%%%%%%%%%%%%%%%%%%%%%%%%%%%%%%%%%%%%%%%%%%%%%%%%%%%%%%%%%%%%%%%%%%%%%%%%%%%%%%%%%%%%%%%%%%%%%%%%%%%%%%%%%%%%%%%%%%%%%%%%%%%%%%%%%%%%%%%%%%%%%%%%%%%%%%%%%%%%%%%%%%%%%%%%%%%%%%%%%%%%%%%%%%%%%%%%%%%%%%%%%%%%%%%%%%%%%%%%%%%

\usepackage{eurosym}
\usepackage{vmargin}
\usepackage{amsmath}
\usepackage{graphics}
\usepackage{epsfig}
\usepackage{enumerate}
\usepackage{multicol}
\usepackage{subfigure}
\usepackage{fancyhdr}
\usepackage{listings}
\usepackage{framed}
\usepackage{graphicx}
\usepackage{amsmath}
\usepackage{chngpage}

%\usepackage{bigints}
\usepackage{vmargin}

% left top textwidth textheight headheight

% headsep footheight footskip

\setmargins{2.0cm}{2.5cm}{16 cm}{22cm}{0.5cm}{0cm}{1cm}{1cm}

\renewcommand{\baselinestretch}{1.3}

\setcounter{MaxMatrixCols}{10}

\begin{document}
\begin{enumerate}

8 (i) Write down the defining equation of an ARMA(1,1) process, identifying the parameters of the process. [2]
(ii) Explain what it means to say that a time series is stationary and state (but do not prove) a condition needed to ensure that an ARMA(1,1) process can be stationary. [2]
(iii) Outline the method of moments parameter estimation technique as it would be applied to estimate the parameters of an ARMA(1,1) process. [3]
(iv) Suppose an individual has fitted the following model to a dataset
xt = 9 12 0 71xt 1 et 0 17et 1
The most recently observed value in the series is x25 = 14 82 , with estimated residual e25 1 98 .
(a) Obtain estimates x25(1) and x25(2) for x26 and x27 .
(b) The simplest form of exponential smoothing used at time 24 gave a forecast for x25 of 12.97. Assuming the smoothing parameter is equal to 0.3, find the forecast for x26 .
[4]
(v) Discuss when the method of exponential smoothing might in practice be
preferred to a method based on the Box-Jenkins technique. [2]
[Total 13]
\newpage


8 (i) The equation is
Xt (Xt 1 ) et et 1
The parameters are (the autoregressive parameter), (the moving average parameter), the mean level and the innovation standard deviation .
%%%%%%%%%%%%%%
\newpage
(ii) A time series process is (weakly) stationary if the mean of the process,
mt E(Xt ) , does not vary with time and the covariance of the process,
Cov(Xt Xs ) depends only on the time difference t s and not on the
particular values t s .
For the model in (1) to be stationary, 1 is needed.
%%%%%%%%%%%%%%
\newpage
(iii) For the method of moments, we calculate the theoretical ACF 1 2 in terms
of the parameters . Then we find the sample ACF, say r1 r2 from the data.
Subsequently we obtain estimates for by equating 1 with r1 and 2
with r2 .
The value of 2 is estimated using the calculated value of 0 and the sample
variance, whereas an estimate for is the sample mean x .
%%%%%%%%%%%%%%
\newpage
(iv) (a) Using the given values we obtain the forecasts
x25(1) 9 12 0 71(14 82) 0 17( 1 98) 19 306
and
x25(2) 9 12 0 71(19 306) 22 827
(b) For exponential smoothing the equation is
x25(1) x24(1) x25 x24(1) 12 97 0 3(14 82 12 97) 13 525
Subject 103 (Stochastic Modelling) September 2004 Examiners Report
Page 10

%%%%%%%%%%%%%%
\newpage
(v) Exponential smoothing is simple to apply and does not suffer from problems
of over-fitting. If the data appear fairly stationary but are not especially well
fitted by any of the Box-Jenkins methods, exponential smoothing is likely to
produce more reliable results. More advanced versions of exponential
smoothing can cope with varying trends and multiplicative variation.
Many candidates omitted to mention as a parameter in part (i). Marks for
this question were not quite as good as for Q7, indicating that the practical
aspects of Time Series analysis are less well understood than the theoretical
ones.
\end{document}
