
6 A model frequently used for interest rates is the Cox-Ingersoll-Ross process, which is
an Itô process Xt satisfying the stochastic differential equation
dXt = (b Xt )dt Xt dBt ,
where Bt is a standard Brownian motion.
(i) Define m1(t) = E(Xt | X0 = x).
(a) Verify that m1 satisfies the ordinary differential equation
1
= ( 1)
dm
b m
dt
(b) Solve the equation to determine m1(t) for all t 0. [3]
103 S2004 5 PLEASE TURN OVER
(ii) Define Yt = Xt2 and 2
m2 (t) = E[Xt | X0 = x] .
(a) Use Itô s Lemma to derive a stochastic differential equation satisfied
by Yt.
(b) Deduce an ordinary differential equation satisfied by m2(t).
(c) Assuming that 2 0
dm
dt
as t , show that
2
lim Var[ | 0 = ] =
2 t t
b
X X x .
[7]
[Total 10]
7 A stationary autoregressive process Xt is defined by the recursive relationship
Xt = 1(Xt 1 ) 2 (Xt 2 ) p (Xt p ) et
where {et t 1} is a sequence of independent, zero-mean Normal variables, each
with variance 2 .
(i) Derive the Yule-Walker equations
2
k = 1 k 1 2 k 2 p k p 1k 0
for 0 k p , where k = Cov(Xt Xt k ) . [2]
(ii) Describe a diagnostic procedure based on a sequence of observations from a
time series for testing whether the underlying time series can be modelled as a
second order autoregressive process. [2]
(iii) Consider the second order autoregressive process
Xt = 0 6Xt 1 0 3Xt 2 et
(a) Determine whether the process can be stationary.
(b) State, with a reason, whether the process possesses the Markov
property.
(c) Assuming that =1, calculate the values of 0 1 2 . [7]
[Total 11]
103 S2004 6


6 (i) (a) 1
1
1
x t x[ t ] x[ t ] ( )
dm d
E X E dX E b X b m
dt dt dt
, where Ex
denotes conditional expectation given X0 = x. This derivation uses the
fact that the increments of Brownian motion have expectation equal to
zero.
(b) [ t 1( )] t d
e m t be
dt
, implying that
1( ) [ ( 1)] ( ) m t e t x b e t b x b e t .
(ii) (a) 2 2
2 2 3/ 2
2 ( ) 2 [ ( ) ]
[2 ] 2 2
t t t t t t t t t
t t t t
dY X dX dX X b X dt X dB X dt
b X dt X dt X dB
(b) 2
2 ( ) [2 ] 1( ) 2 2 ( ).
d
m t b m t m t
dt
Again we have used the fact
that Brownian increments have mean zero.
(c) We do not need to solve the equation, but just to note that since dm2/dt
tends to 0, this implies that 2 limt m2(t) = [2 b + 2] limt m1(t)
= 2 b2 + b 2. Therefore
2
2 2
lim [ | 0 ]
2 t t
b
E X X x b , from which we deduce that
2
lim Var[ | 0 ]
2 t t
b
X X x .
This question was relatively poorly answered, although much of it is based on
the standard theory of Ordinary Differential Equations.
7 (i) Taking covariances with Xt k for k 1 in (1) gives
Cov(Xt Xt k ) 1Cov(Xt 1 Xt k ) 2Cov(Xt 2 Xt k ) pCov(Xt p Xt k )
which gives the Yule-Walker equations since, by definition,
k Cov(Xt Xt k ) for 0 k p .
For k 0 , there is an extra term which accounts for Cov(Xt et ) 2 .
(ii) A diagnostic test is based on the partial ACF and uses the fact that, for an
AR(2) process, the partial autocorrelations, k , are zero for k 2 .
Subject 103 (Stochastic Modelling) September 2004 Examiners Report
Page 8
The values of k are estimated by the partial ACF, k , and for k 2 the
asymptotic variance of k is 1 n . Using a normal approximation, values of the
sample partial ACF outside the range 2 n indicate that the AR(2) model
may be inadequate.
(iii) (a) The process can be written in terms of the backward shift operator as
(1 0 6 0 3 2 ) B B Xt et .
Hence the characteristic polynomial is 1 0 6z 0 3z2 with roots
0 6 (0 6)2 1 2
0 6
i.e. the roots are 1 156 6 .
Since both roots lie outside the unit circle, the process can be
stationary.
(b) Xt is not Markov since the conditional distribution of Xk 1 given the
history up to time k depends on Xk 1 as well as on Xk .
(c) The Yule-Walker equations in this case yield
0 0 6 1 0 3 2 1 (3)
1 0 6 0 0 3 1 (4)
2 0 6 1 0 3 0 (5)
From (4) we have
0
1 0 1
6
0 7 0 6
7
(6)
and substituting into (5) we get
2 0 0 0
36 3 57
70 10 70
(7)
Inserting the last two equations into (3) we obtain
0 0 0
36 171
1
70 700
which gives
0 0
36 171 700
1 1
70 700 169
Subject 103 (Stochastic Modelling) September 2004 Examiners Report
Page 9
Then (6) and (7) yield resp.
0
1 2
6 600 570
7 169 169
The examiners were pleased to note the high quality of answers to this
question. It appears that the theoretical principles of Time Series analysis are
well understood.
