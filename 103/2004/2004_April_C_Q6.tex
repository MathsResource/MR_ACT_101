\documentclass[a4paper,12pt]{article}

%%%%%%%%%%%%%%%%%%%%%%%%%%%%%%%%%%%%%%%%%%%%%%%%%%%%%%%%%%%%%%%%%%%%%%%%%%%%%%%%%%%%%%%%%%%%%%%%%%%%%%%%%%%%%%%%%%%%%%%%%%%%%%%%%%%%%%%%%%%%%%%%%%%%%%%%%%%%%%%%%%%%%%%%%%%%%%%%%%%%%%%%%%%%%%%%%%%%%%%%%%%%%%%%%%%%%%%%%%%%%%%%%%%%%%%%%%%%%%%%%%%%%%%%%%%%

\usepackage{eurosym}
\usepackage{vmargin}
\usepackage{amsmath}
\usepackage{graphics}
\usepackage{epsfig}
\usepackage{enumerate}
\usepackage{multicol}
\usepackage{subfigure}
\usepackage{fancyhdr}
\usepackage{listings}
\usepackage{framed}
\usepackage{graphicx}
\usepackage{amsmath}
\usepackage{chngpage}

%\usepackage{bigints}
\usepackage{vmargin}

% left top textwidth textheight headheight

% headsep footheight footskip

\setmargins{2.0cm}{2.5cm}{16 cm}{22cm}{0.5cm}{0cm}{1cm}{1cm}

\renewcommand{\baselinestretch}{1.3}

\setcounter{MaxMatrixCols}{10}

\begin{document}
\begin{enumerate}

6 (i) Describe a method for generating a sequence U1 U2 of pseudo-random
numbers uniformly distributed in the interval (0 1) and from this a pseudorandom
sequence of numbers uniformly distributed over an arbitrary
range (a b) , where a b . [3]
(ii) Explain the main advantage of using pseudo-random, as opposed to purely
random numbers, for testing the suitability of a model. [1]
(iii) Given a sequence {Un n 1 2 } as in (i), explain how you would simulate
(a) an observation from the Pareto density
1 ( ) = 1 0
(1 )a
a
f x a x
x
(b) a discrete random variable X with probability function given by
1 2
( = ) = for 1 2 ( = 0) =
2 2
P X i i n P X
n n
[4]
(iv) The Pareto distribution in (iii) is often referred to as a heavy-tailed
distribution . Explain the use of this term and discuss under what
circumstances it might offer a suitable model for modelling sizes of claims
arriving at an insurance company. [2]

%%%%%%%%%%%%%%%%%%%%%%%%%%%%%%%%%%%%%%%%%%%%%%%%%%%%%%%%%%%%%%%%%%%%%%%%%%%%%%%%%%%%%%%%
\newpage

6 (i) Use a linear congruential generator (LCG). Specify three positive integers
a c m with m a m c and an initial value x0 , then generate a sequence of
x1 x2 xn in the range {0 1 2 m 1} by the recursive rule
Subject 103 (Stochastic Modelling) April 2004 Examiners Report
Page 7
xn (axn 1 c) (modm) n 1 2 N
Then un xn m is a sequence of pseudo-random numbers in [0 1].
To generate a sequence over the range (a b) , a linear transformation
vn (b a)un a is needed. Thus the procedure is: generate a variable
U U(0 1) using a LCG as above; return V (b a)U a .
(ii) The main advantage of using pseudo-random numbers is reproducibility.
When testing a model, this almost certainly depends on a number of
parameters and assumptions, so it is very often desirable to examine the
sensitivity of the model to these assumptions and the values of the parameters
used. It is thus necessary to reduce the random element to a minimum and
rerun the experiment using the same data .
(iii) (a) We use the inverse transform method. The cdf of the Pareto is
0 1
1
( ) 1
(1 ) (1 )
x
a a
a
F x
x x
so the inverse is
F 1( y) (1 y) 1 a 1
Thus the procedure is
Generate an observation U from the Uniform (0,1) distribution.
Return X (1 U) 1 a 1 (or, equally good, X U 1 a 1) as an
observation from the Pareto.
(b) Since the number of possible values for X is n 1, we need to split
the interval (0 1) into n 1 subintervals, the length of which has to be
proportional to the corresponding probabilities. Given a uniform
variable U , one such procedure is:
For i 1 2 n , if (i 1) (n 2) U i (n 2) , then set X i
If U n (n 2) , then set X 0 .
(iv) From (iii), we see that the tail of the Pareto distribution is 1 F(x) (1 x) a ,
which decreases to zero much more slowly than e.g. the exponential or normal
distributions.
Alternatively, it is called fat-tailed because of the relatively high probability
of producing values which are a long way from the mean/median.
Subject 103 (Stochastic Modelling) April 2004 Examiners Report
In general, the Pareto offers a suitable choice for portfolios where there is a
non-negligible chance of very large claims, which is reasonable in various
forms of general insurance, in particular insurance that deals with natural
catastrophes (floods, earthquakes, hurricanes etc).
This question was generally well answered and caused few difficulties.
\end{document}
