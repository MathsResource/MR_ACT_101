
%%%%%%%%%%%%%%%%%%%%%%%%%%%%%%%%%%%%%%%%%%%%%%%%%%%%%%%%%%%%%%%%%%%%%%%%%%%%%%%%%%%%%%%%%%%%%%
8 A restorer of historic sites travels between Bath, Warwick, Stratford and Caernarvon.
Having arrived at a site, the restorer stays there for a random number of days, then
moves on to one of the others. The restorer s diary for the last two months shows the
following time spent in each place:
Bath (5 days), Warwick (3 days), Caernarvon (7 days), Bath (3 days),
Stratford (7 days), Warwick (5 days), Bath (3 days), Stratford (3 days)
Caernarvon (8 days), Bath (2 days)
A researcher suggests that the path taken by the restorer forms a Markov chain with
state space {Bath, Caernarvon, Stratford, Warwick}.
(i) Suppose {Xt: t = 0, 1, 2, } is a Markov chain with transition matrix P and let
Di,n denote the duration of the nth visit to state i, that is, the number of
consecutive steps spent by the chain in state i. Show that Di,n is a geometric
random variable and is independent of Di,m for m < n. [3]
(ii) Discuss, in the light of the data, whether it is likely that this suggestion is
satisfied. [2]
(iii) Assuming that the researcher s suggestion is correct, write down an estimate
of the transition matrix P of the Markov chain. [2]
(iv) State, giving reasons for your answers, whether the Markov chain model is
(a) irreducible, and (b) aperiodic. [2]
(v) The restorer arrives in Warwick on day t0. Use the estimated transition matrix
to calculate the probability that he is in Warwick on day t0 + 3. [3]
[Total 12]
103 A2004 7 PLEASE TURN OVER#
%%%%%%%%%%%%%%%%%%%%%%%%
8 (i) Suppose at time t X has just arrived in state i. The probability that X remains
in state i until time t + k and then leaves, giving a duration of k + 1 steps in
state i, is k (1 )
pii pii . In other words, 1
( , ) (1 ) d
P Di n d pii pii , which is
the probability function of a geometric random variable.
The fact that Di,n is independent of previous durations follows from the
Markov property: What happens to X after arriving in state i is independent of
anything that happened before that moment.
(ii) A sequence of successive observations of a geometric random variable is
likely to produce 1 more often than any other value. The durations of the
restorer s stays in the various locations are always at least 2 days and often
much more. There is an indication that the geometric distribution is
inappropriate for the data set provided.
Subject 103 (Stochastic Modelling) April 2004 Examiners Report
An alternative indication that the Markov property is dubious is the fact that
the restorer appears to return to Bath at regular intervals, i.e. regardless of the
time spent in each state, the path taken appears to lack randomness.
(iii) pij nij / ni , where nij is the number of transitions from state i to state j and
ni+ is the total number of transitions out of state i. For example, of the 8 days
spent in Warwick, one is followed by a trip to Caernarvon, one by a trip to
Bath and the remaining 6 are followed by another day in Warwick, so that nWC
= 1, nWB = 1 and nWW = 6. We therefore have (using the order B, C, S, W)
9 2 1
0
12 12 12
2 13
0 0
15 15
1 8 1
0
10 10 10
1 1 6
0
8 8 8
P
(iv) The model is irreducible. Starting from Bath it is possible to visit Stratford,
Warwick, Caernarvon and return to Bath, completing the circuit.
It is also aperiodic, since pii > 0 for some (in fact for all) state i.
(v) We need the entry of P3 corresponding to the row and column of Warwick
(4th row, 4th column in this case).
Now 2
.5729 .0271 .2583 .1417
.2156 .7511 .0222 .0111
.0258 .1792 .6400 .1550
.2042 .2021 .0208 .5729
P , so the required answer is
1 1 0 6
.1417 .0111 .1550 .5729
8 8 8 8
= 0.4488.
Each part of this question attracted some unexpected answers as well as some good
ones. The answers to part (i) were on the whole disappointing, but the final parts were
rather better answered in general.
Subject 103 (Stochastic Modelling) April 2004 Examiners Report
Page 11
