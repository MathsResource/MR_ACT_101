\documentclass[a4paper,12pt]{article}

%%%%%%%%%%%%%%%%%%%%%%%%%%%%%%%%%%%%%%%%%%%%%%%%%%%%%%%%%%%%%%%%%%%%%%%%%%%%%%%%%%%%%%%%%%%%%%%%%%%%%%%%%%%%%%%%%%%%%%%%%%%%%%%%%%%%%%%%%%%%%%%%%%%%%%%%%%%%%%%%%%%%%%%%%%%%%%%%%%%%%%%%%%%%%%%%%%%%%%%%%%%%%%%%%%%%%%%%%%%%%%%%%%%%%%%%%%%%%%%%%%%%%%%%%%%%

\usepackage{eurosym}
\usepackage{vmargin}
\usepackage{amsmath}
\usepackage{graphics}
\usepackage{epsfig}
\usepackage{enumerate}
\usepackage{multicol}
\usepackage{subfigure}
\usepackage{fancyhdr}
\usepackage{listings}
\usepackage{framed}
\usepackage{graphicx}
\usepackage{amsmath}
\usepackage{chngpage}

%\usepackage{bigints}
\usepackage{vmargin}

% left top textwidth textheight headheight

% headsep footheight footskip

\setmargins{2.0cm}{2.5cm}{16 cm}{22cm}{0.5cm}{0cm}{1cm}{1cm}

\renewcommand{\baselinestretch}{1.3}

\setcounter{MaxMatrixCols}{10}

\begin{document}
\begin{enumerate}

\item (i) (a) Write down the joint density function of B(½) and B(1), where
{B(t): t 0} is a standard Brownian motion.
(b) Demonstrate that the conditional distribution of B(½) given that
B(1) = b is Normal with mean ½b and variance ¼.

\item 
(ii) James runs around a running track at the same time as an electric hare which
is programmed to go around in exactly one minute. Let Y(t) denote the
distance (in metres) by which James is ahead of the hare t minutes after the
start and suppose that Y(t) can be modelled as Y(t) = 5 B(t), where B(t) is as
in (i).
(a) If James is leading the hare by 5 metres at time t = 0.5, determine the
probability that James eventually finishes the race ahead of the hare.
(b) If James finishes the run 5 metres ahead of the hare, determine the
probability that he was ahead of the hare at time t = 0.5.
\end{enumerate}
%%%%%%%%%%%%%%%%%%%%%
\newpage


5 (i) (a) If X = B(½) and Y = B(1), then the marginal distribution of X is N(0,0.5)
and the conditional distribution of Y given X = x is N(x, 0.5).
This means that fX ,Y (x, y) is given by
1 2 1 ( )2
exp . exp
2 (0.5) 2(0.5) 2 (0.5) 2(0.5)
x y x
1 2 2
exp( ( y 2yx 2x ))
(b) The conditional density function of X given Y is
1 2 2
,
|
2
( , ) exp( ( 2 2 ))
( | )
( ) 1 exp ½
2
X Y
X Y
Y
f x b b bx x
f x b
f b b
2 2
exp 2(x ½b)
This is the density function of the Normal distribution with mean ½b
and variance ¼.
%-------------------------------------------------------------------------%
(ii) (a) Here we seek P(Y(1) > 0 | Y(½) = 5) = P(B(1) > 0 | B(½) = 1)
= 1
2 P B( ) 1
= 1
2 P 2 B( ) 0 2
= 1 2 = 0.9213.
(b) We require 1
2 P Y( ) 0 | Y(1) 5 = 1
2 P B( ) 0 | B(1) 1
Using part (i)(b), this is
0 ½
1 (1) 0.8413.
¼
This question was designed in such a way that candidates could tackle part (ii) even if part (i)
had not worked out right, but many allowed themselves to become discouraged by difficulties
in the early part of the question and gave up.
\end{document}
