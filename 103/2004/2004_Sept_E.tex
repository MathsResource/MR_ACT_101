
9 Vehicles in a certain country are required to be assessed every year for roadworthiness.
At one vehicle assessment centre, drivers wait for an average of 15
minutes before the road-worthiness assessment of their vehicle commences. The
assessment takes on average 20 minutes to complete. Following the assessment, 80%
of vehicles are passed as road-worthy allowing the driver to drive home. A further
15% of vehicles are categorised as a minor fail ; these vehicles require on average
30 minutes of repair work before the driver is allowed to drive home. The remaining
5% of vehicles are categorised as a significant fail ; these vehicles require on
average three hours of repair work before the driver can go home.
A continuous-time Markov model is to be used to model the operation of the vehicle
assessment centre, with states W (waiting for assessment), A (assessment taking
place), M (minor repair taking place), S (significant repair taking place) and H
(travelling home).
(i) Explain what assumption must be made about the distribution of the time
spent in each state. [1]
(ii) Write down the generator matrix for this process. [2]
103 S2004 7
(a) Use Kolmogorov s Forward Equations to write down differential
equations satisfied by pWM(t) and by pWA(t).
(b) Verify that ( ) = 4 t / 20 4 t /15
pWA t e e for t 0, where t is measured in
minutes.
(c) Derive an expression for pWM(t) for t 0.
[7]
(iii) Let Ti be the expected length of time (in minutes) until the vehicle can be
driven home given that the assessment process is currently in state i.
(a) Explain why TW = 15 + TA.
(b) Derive corresponding equations for TA, TM and TS.
(c) Calculate TW.
[4]
[Total 14]


9 (i) The Markov model implies that holding times are exponentially distributed.
(ii) The generator matrix is as follows (in minutes then, equivalently, in hours):
1 1
15 15
1 3 1 1
20 400 400 25
1 1
30 30
1 1
180 180
0 0 0
0
0 0 0
0 0 0
0 0 0 0 0
W A M S H
W
A
M
S
H
1 1
3 3
4 4 0 0 0
0 3 0.45 0.15 2.4
0 0 2 0 2
0 0 0
0 0 0 0 0
W A M S H
W
A
M
S
H
(a) The equations are as follows (first if t is in minutes, then in hours)
1 3
( ) ( ) ( )
30 400
1 1
( ) ( ) ( )
20 15
WM WM WA
WA WA WW
d
p t p t p t
dt
d
p t p t p t
dt
( ) 2 ( ) 0.45 ( )
( ) 3 ( ) 4 ( )
WM WM WA
WA WA WW
d
p t p t p t
dt
d
p t p t p t
dt
Subject 103 (Stochastic Modelling) September 2004 Examiners Report
Page 11
(b) First note that pWW(t) = e t/15. Then try inserting the given formula in
the second equation above:
/ 20 /15 1 4
( )
5 15
t t
WA
d
LHS p t e e
dt
and
/ 20 /15 /15 1 1 1 1 1
( ) ( ) ,
20 15 5 5 15
t t t
WA WW RHS p t p t e e e
which is equal to the RHS, as required.
We should also check that the formula gives pWA(0) = 0, which it does.
(c) / 30 3 / 30 / 20 /15
( ) .4
400
t t t t
WM
d
e p t e e e
dt
with pWM(0) = 0
implies that t /30 ( ) 0.9 1.8 t / 60 0.9 t /30
e pWM t e e , which simplifies
to ( ) 0.9 t /30 1.8 t / 20 0.9 t /15
pWM t e e e .
(iii) (a) The expected length of time spent in state W is 15 mins, after which
there is a transition to state A with probability 1.
(b) The other equations are:
TA = 20 + 0.15 TM + 0.05 TS
TM = 30
TS = 180
(c) Solving these equations gives TA = 20 + 4.5 + 9 = 33.5 and TW = 48.5
mins
A number of students suggested that ( ) 1 ( ) WW WA p t p t , which works in the
two-state case. In this example, however, it is only possible to state that
( ) ( ) ( ) ( ) ( ) 1, WW WA WM WS WH p t p t p t p t p t which is not the same.
It was disappointing that not many candidates made substantial progress with
solving the differential equations, but the last part of the question was in
general well done.


