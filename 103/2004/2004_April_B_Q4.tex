\documentclass[a4paper,12pt]{article}

%%%%%%%%%%%%%%%%%%%%%%%%%%%%%%%%%%%%%%%%%%%%%%%%%%%%%%%%%%%%%%%%%%%%%%%%%%%%%%%%%%%%%%%%%%%%%%%%%%%%%%%%%%%%%%%%%%%%%%%%%%%%%%%%%%%%%%%%%%%%%%%%%%%%%%%%%%%%%%%%%%%%%%%%%%%%%%%%%%%%%%%%%%%%%%%%%%%%%%%%%%%%%%%%%%%%%%%%%%%%%%%%%%%%%%%%%%%%%%%%%%%%%%%%%%%%

\usepackage{eurosym}
\usepackage{vmargin}
\usepackage{amsmath}
\usepackage{graphics}
\usepackage{epsfig}
\usepackage{enumerate}
\usepackage{multicol}
\usepackage{subfigure}
\usepackage{fancyhdr}
\usepackage{listings}
\usepackage{framed}
\usepackage{graphicx}
\usepackage{amsmath}
\usepackage{chngpage}

%\usepackage{bigints}
\usepackage{vmargin}

% left top textwidth textheight headheight

% headsep footheight footskip

\setmargins{2.0cm}{2.5cm}{16 cm}{22cm}{0.5cm}{0cm}{1cm}{1cm}

\renewcommand{\baselinestretch}{1.3}

\setcounter{MaxMatrixCols}{10}

\begin{document}
\begin{enumerate}

4 The value, St, of a stock exchange index is observed at hourly intervals over the
course of a week, generating observations s1, s2, , sn.
Three different models for St are being considered:
\begin{enumerate}
\item (a) St = S0 t Bt , where Bt is a standard Brownian motion.
\item (b) = 0 t Bt
St S e , where Bt is as in (a).
\item (c) St ( St 1 ) et , where {et: t = 1, 2, } is a sequence of
uncorrelated, zero-mean random variables.
\end{enumerate}
%--------------------------------------------------%

\begin{enumerate}
\item (i) Outline the processes of model fitting and model validation, including a
description of the role of simulation. 
\item (ii) For EACH of the three models above, outline the steps required during the
process of fitting the model. 
\item (iii) Choose ONE of the above models and describe TWO tests you would apply as
part of the model validation process. 
\end{enumerate}


%%%%%%%%%%%%%%%%%%%%%
4 (i) Model fitting involves first choosing a suitable family of models, then a
suitable state space and finally estimating the values of the model parameters
to isolate a single member of the family.
Having performed the model fitting process, we have the best-fitting model
from the given family. Model validation involves checking whether this is an
adequate explanation of the observations.
Simulation can be used in model validation to test that the paths typically
followed by a simulated version of the process are broadly similar to the paths
observed in practice.

(ii) (a) Under the given model, the successive increments St form a
sequence of i.i.d. N( , 2) random variables, so the parameters can be
estimated using the sample mean and sample variance of the observed
increments.
(b) The first thing to do is to take the log of the observations, obtaining x1,
, xn say. Then apply the same technique as in (a).

(c) This is a time series model, so a time series technique (such as Box-
Jenkins) would be appropriate.
Parameters are estimated by Method of Moments or Maximum
Likelihood, or simply by applying a computer package to do the
estimation for you.

%%%%%%%%%%%%%%%%%%%%%%%%%%%%%%%%%%%%%%%%%%%%%%%%%%%%%%%%%%%%%%%%%%%%%%%%%%%
\newpage

(iii) Any of the tests described here are equally valid.
(a) The model implies that the increments are Normally distributed, so a
standard test of normality, such as the Anderson-Darling test, the
Kolmogorov-Smirnov test or a chi-squared goodness-of-fit test can be
employed to test this.

Another testable property of the model is independence of increments:
it is possible to investigate the lag-1 sample ACF and determine
whether it is significantly different from zero, or use the Ljung-Box
chi-square statistics to perform a portmanteau test on all lags of the
sample ACF simultaneously.
A similar test is to inspect the sample ACF and PACF of the residuals
to see whether the observed values fall outside the 5% significance
band.
We could test whether there was a relationship between the increment
st and st; this is a test for homoscedasticity, though in the situation
described in the question it is unlikely to provide useful results.

A turning points test, a run test or a sign change test could also be
used, in each case comparing a calculated statistic with tables of a
reference distribution. These are tests for serial independence (usually
applied to a sequence of residuals).
(b) Apply the same tests as in (a) to the logarithm of the data.
(c) Tests can be applied to the sequence of residuals arising from the
fitting process. Since, if the model is accurate, these should form a
sequence of uncorrelated Normal observations, the same tests as in (a)
can be applied.

A procedure which might be applied to the model in part (c) but not to
the models in (a) or (b) is to inspect the sample partial ACF of the
increment process St; if the process really is a first-order
autoregression as stated in the model, then the sample PACF should
have no significant values at lags higher than 1.
This question differentiated well between candidates, with some very good answers indicating
that the candidates understood the principles and practice of modelling, and others
highlighting deficiencies of understanding.
\end{document}
