\documentclass[a4paper,12pt]{article}

%%%%%%%%%%%%%%%%%%%%%%%%%%%%%%%%%%%%%%%%%%%%%%%%%%%%%%%%%%%%%%%%%%%%%%%%%%%%%%%%%%%%%%%%%%%%%%%%%%%%%%%%%%%%%%%%%%%%%%%%%%%%%%%%%%%%%%%%%%%%%%%%%%%%%%%%%%%%%%%%%%%%%%%%%%%%%%%%%%%%%%%%%%%%%%%%%%%%%%%%%%%%%%%%%%%%%%%%%%%%%%%%%%%%%%%%%%%%%%%%%%%%%%%%%%%%

\usepackage{eurosym}
\usepackage{vmargin}
\usepackage{amsmath}
\usepackage{graphics}
\usepackage{epsfig}
\usepackage{enumerate}
\usepackage{multicol}
\usepackage{subfigure}
\usepackage{fancyhdr}
\usepackage{listings}
\usepackage{framed}
\usepackage{graphicx}
\usepackage{amsmath}
\usepackage{chngpage}

%\usepackage{bigints}
\usepackage{vmargin}

% left top textwidth textheight headheight

% headsep footheight footskip

\setmargins{2.0cm}{2.5cm}{16 cm}{22cm}{0.5cm}{0cm}{1cm}{1cm}

\renewcommand{\baselinestretch}{1.3}

\setcounter{MaxMatrixCols}{10}

\begin{document}
%%%%%%%%%%%%%%%%%%%%%%%%%%%%%%%%%%%%%%%%%%%%%%%%%%%%%%%%%%%%%%%%%%%%%%%%%%%%%%%%%%
2 

\large 
A Box-Jenkins model-fitting procedure suggests that the best fitting model for a set of normalised share price data x1, , xn is ARMA(1,2), with equation

Xt 0.63Xt 1 et 0.45et 1 0.34et 2 ,
where {e1, e2, } is a sequence of uncorrelated, zero-mean random variables with
variance 2.

(i) Determine whether the model is stationary and invertible. 
(ii) Calculate 0, 1, 2, the autocovariance function of the fitted model at lags 0, 1, and 2, in terms of 2. 


%%%%%%%%%%%%%%%%%%%%%%%%%%%%%%%%%%%%%%%%%%%%%%%%%%%%%%%%%%%%%%%%%%%%%%%%%%%%%%%%%%
2 (i) The solution to 1 0.63z 0 is z = 1/0.63, which is greater than 1. Therefore the model is stationary.
For invertibility, we should check that the roots of 1 0.45z 0.34z2 0 are outside the unit circle. They are ( 0.45 1.25)/( 2*0.34) = 2.5 or 1.18, both OK.

%%% Subject 103 (Stochastic Modelling) April 2004 Examiners Report
Page 3
(ii) Let 0 = Cov(Xt, et), 1 = Cov(Xt, et 1), 2 = Cov(Xt, et 2). Then
0 = 0.63 1 + 0 + 0.45 1 0.34 2
1 = 0.63 0 + 0.45 0 0.34 1
2 = 0.63 1 0.34 0
0 = 2 (this may be regarded as obvious and not stated explicitly)
1 = 0.63 0 + 0.45 2 = 1.08 2
2 = 0.63 1 0.34 2 = 0.3404 2
An alternative expression for 0 may be obtained using
0 1 1 2
2 2 2 2
0 0 1
Var(0.63 0.45 0.34 )
0.63 (1 0.45 0.34 ) 2 0.63[0.45 0.34 ]
t t t t X e e e
which removes the need to calculate 2.
Having derived the equations, we need to solve them.
1 = 0.63 0 + 0.0828 2
0 = 0.63(0.63 0 + 0.0828 2) + 1.370 2 = 0.3969 0 + 1.422 2,
implying that
0 = 2.358 2, 1 = 1.5684 2, 2 = 0.6481 2.
%%%%%%%%%%%%%%%%%%%%%%%%%%%%%%%%%%%%%%%%%%%%%%%%%%%%%%%%%%%%%%%%%%%%
\newpage

Candidates were often unsure of the procedure required to derive the equations for the k but did rather better at solving them. In particular, many candidates did not take correct account of the relationship between Xt and past values of et. Marks were awarded for correct methodology when deriving the solutions, even if the equations being solved were not the
right ones.

\end{document}
