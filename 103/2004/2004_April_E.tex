
9 A water company is developing a time series model to model the supply of water, Xt,
in its reservoirs at the end of month t. The model takes the form
Xt = Xt 1 Rt Dt ,
where Rt represents rainfall in month t and Dt represents demand in month t. Rt and
Dt are themselves modelled by
= ( 1 1)
=
t t t t t
t t
R R e
D R
where , , are positive constants, t is a deterministic function of time and
{et: t = 1, 2, } is a sequence of independent Normal random variables with mean 0
and variance 2.
(i) (a) Comment on whether it is reasonable to assume that demand is
negatively related to rainfall.
It is suggested that t should be cyclical with period 12, i.e. that
t+12 = 12 for all t. Comment on whether this suggestion is sensible.
[2]
(ii) State, with reasons, whether Rt is stationary. [1]
(iii) (a) Suppose that R0 is Normally distributed with mean 0 and variance 2
R.
Show that Rt is also Normally distributed for each t > 0 and derive an
expression for its mean.
(b) Calculate a value of 2
R which ensures that Var(Rt) = 2
R for all t. [4]
(iv) Given a sequence of observations R1, , R120, D1, , D120 covering 10 years
of data, suggest simple estimators for t (1 t 12), , and . [4]
(v) Assuming that the values of the parameters are known exactly, and that R120,
D120 and X120 are known, derive an estimate x120 (1) for the supply of water at
the end of month 121 and give an expression forVar(X121 x120 (1)) . [3]
[Total 14]
103 A2004 8
%%%%%%%%%%%%%%%%%%%%%%%%%%%%%%%%%%%%%%%%%%%%%%%%%%%%%%%%%%%%%%%%%%

9 (i) (a) In one sense it is reasonable, because in the absence of rainfall
gardeners might increase their demand for water. But the relationship
does not work in reverse: in particularly rainy weather the demand is
likely to be no less than in normally rainy weather.
(b) Yes. It reflects the seasonal pattern of rainfall.
(ii) No. It satisfies an equation which is explicitly dependent on t, so it will not be
stationary unless all the t are equal to one another. (Arguments based on
whether | | > 1 are only relevant if all the t are equal.)
(iii) (a) Normality: by induction. If Rt 1 and et are Normal, then Rt must be
Normal, as it is a linear combination of these two. Since R0 is Normal,
normality follows for all t.
E(Rt ) t [E(Rt 1) t 1] . This can be iterated backwards to
zero, showing that E(Rt ) t .
An alternative approach is to solve explicitly for Rt and read off the
answers from there.
1
0 0
0
( )
t
t k
t t t k
k
R R e
Now we can see that Rt is a linear combination of Normal random
variables with some constants, hence Normally distributed, and that the
mean is t.
(b) Var(Rt) is equal to 2 Var(Rt 1) + 2. If Rr has constant variance R
2
then R
2 = 2
R
2 + 2, implying that R
2 = 2 / (1 2).
If instead the above expression for Rt is used, we can see that
1 2
2 2 2 2 2 2 2
2 2
0
1 1
, so that
1 1
t t
t k
R R R t
k
(iv) t represents the mean rainfall in one particular month of the year. Estimate
this by calculating the average rainfall in that month over the 10-year period,
12 24 108
1
( )
