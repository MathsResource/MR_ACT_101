
1 A sequence x1, x2, , xn of observations is provided, each of which takes one of a
finite set, S, of possible values. A researcher wishes to fit a discrete-time Markov
chain model to these data.
(i) Write down estimates of the transition probabilities of the Markov chain. [1]
(ii) Outline the purpose of model validation. [1]
(iii) Describe the role of simulation in sensitivity analysis in the given context. [3]
[Total 5]
2 A person who catches a particular virus for the first time immediately falls ill with
Disease A. The illness lasts for a random length of time, whose mean depends on the
age of the person. Having recovered from Disease A, the person will not fall ill with it
again. However, if the person catches the same virus again after the age of 18, the
result will be Disease B, which will affect the person for an average of three months
regardless of age. After suffering from Disease B the person will not catch the virus
again.
It is proposed that a continuous-time Markov model can be used to model the person s
medical history, with states:
0: not currently ill with Disease A or Disease B
1: currently ill with Disease A
2: currently ill with Disease B
(i) Explain why the proposed state space is inadequate and suggest an enlarged
state space which can support a Markov model. [2]
(ii) Draw a diagram to illustrate the possible transitions. [2]
(iii) State, with reasons, whether a time-homogeneous or a time-inhomogeneous
model is more suitable to model the medical history of a single person. [1]
(iv) Explain why a national medical service might find that a time-homogeneous
model was just as good as a time-inhomogeneous one for planning the
provision of treatment for Diseases A and B. [1]
[Total 6]
103 S2004 3 PLEASE TURN OVER
3 You are given a stream of standard identically and independently distributed uniform
[0,1] random variables U1, U2, U3, ..
Describe how to use this random variable stream to generate random variates with the
following distributions:
(i) The discrete distribution with possible values A, B and C with respective
probabilities 1/5, 1/4 and 11/20. [2]
(ii) The continuous distribution with probability density function
(10 ) /18 for 4 10
( ) =
0 otherwise
x x
f x [3]
(iii) The continuous distribution function with probability density function
2 2
sin for 0
( ) =
0 otherwise
x x
f x [3]
[Hint: Use the Acceptance-Rejection method.]
[Total 8]

%%%%%%%%%%%%%%%%%%%%%%%%%
%%%%%%%%%%%%%%%%%%%%%%%%%%%%%%%%%
Faculty of Actuaries Institute of Actuaries
EXAMINATIONS
September 2004
Subject 103 Stochastic Modelling
EXAMINERS REPORT
Faculty of Actuaries
Institute of Actuaries
Subject 103 (Stochastic Modelling) September 2004 Examiners Report
Page 2
he examiners were pleased to note that the overall quality of answers on this
final sitting of subject 103 was high and that many of the candidates
demonstrated a good knowledge of the principles and practice of stochastic
modelling. As always, credit was awarded for comments which showed that
candidates had an understanding of the topic covered in a question, even if the
calculations gave the wrong answer due to some mathematical error.
Question 7 was particularly well answered, with Questions 2 and 4 not far
behind. Questions 6 and 10 had the lowest proportion of good answers; it is
possible that time pressure played a role in the case of Question 10.
1 (i) Let nij denote the number of direct transitions from state i to state j, with ni+
the total number of transitions out of state i. Then pij nij / ni .
(ii) Model fitting aims to find the best-fitting model in a given class. But it is
conceivable that even the best-fitting model in the class does not fit very well.
Model validation is a set of procedures to test the adequacy of the fit.
(iii) Sensitivity analysis is part of model validation. The purpose is to determine
whether the behaviour of the fitted model would be substantially different if
the parameter values were slightly different from the estimates already
obtained.
The technique involves simulating the fitted process a large number of times,
using several simulations for each of a number of slightly different parameter
values, then examining the output of the simulation to attempt to identify
systematic differences.
It is important that the same sequence of random numbers be used in each of
the sets of simulations to ensure comparability.
Many candidates failed to mention the importance of using the same sequence
of pseudo-random numbers. Apart from that, most answers showed good
knowledge of the principles of modelling.
2 (i) It is inadequate because someone who has never suffered from disease A or B
is not in the same position as someone who has suffered from one or both in
the past but is currently healthy.
The state space should be extended by splitting state 0 into 3: 0: Has never
suffered from A or B , A: Has suffered from A but is now healthy ,
AB: Has suffered both A and B but is now healthy.
(ii)
0
1
A
2
AB
Subject 103 (Stochastic Modelling) September 2004 Examiners Report
Page 3
(iii) Only a time-inhomogeneous model can properly reflect the dependence of
both the recovery rate for A and the infection rate for B on the age of the
person.
(iv) If, in a population taken as a whole, the number of people in each age group is
roughly constant over time, then the age-dependent transition rates of the
individuals who make up the population can be averaged out to give a timehomogeneous
model which works perfectly adequately given that national
medical services are generally only concerned with total numbers falling ill.
This question was answered well in general. Where candidates lost marks it
was often due to mis-specifying the additional states in part (i). Splitting state
A into Has recovered from Disease A and is aged below 18 and Has
recovered from Disease A and is aged 18 or more is reasonable when
modelling an entire population, but does not lead to a time-homogeneous
Markov model when applied to a single individual, since one s 18th birthday
does not occur at a random time. However, answers along these lines with
good explanations were given full marks.
3 (i) Set Xi as follows:
if 0 1/ 5
if 1/ 5 9 / 20
if 9 / 20
i
i i
i
A U
X B U
C U
(ii) Use the inverse transformation method.
The distribution function is
4
( ) 10
18
x t F x dt
2 10
1
36
x
for 4 x 10 .
Solving the equation gives F 1(u) 10 6 1 u
So we can set Xi 10 6 1 Ui
or alternatively we could use Xi 10 6 Ui
(iii) Use acceptance-rejection method:
Let V1 = U1 , so that V1 is uniformly distributed on [0, ] and has density
function g(x) = 1/ over that range.
We define
Subject 103 (Stochastic Modelling) September 2004 Examiners Report
Page 4
2
0 0
sup ( ) / ( ) sup 2sin 2.
x x
C f x g x x
If U2 < sin2 V1 let X1 = V1; otherwise reject this value and select a new pair
U1, U2. Repeat for other Xi
Answers to parts (i) and (ii) were generally good. For part (iii) many
candidates only described the general theory without specifying g(x) or
calculating the constant C.
