



10 An insurance company has an initial capital of u and receives premium income at
constant rate c per unit time. Claims, for one unit each, occur according to a Poisson
process {Nt, t 0} with rate . The company s surplus at time t is defined as
St = u ct Nt .
(i) A Lévy process is defined as a stochastic process with stationary, independent
increments. Verify that St satisfies this definition. [2]
(ii) For s > 0, write down the conditional distribution of Nt+s Nt given Nt. [1]
(iii) Define = St
Yt e .
(a) Derive an equation which must be satisfied by if Yt is to be a
martingale.
(b) Show, by means of a sketch or otherwise, that the equation derived in
(a) has a negative solution as long as c > and a positive solution
if c < .
(c) Comment on whether the condition c > is likely to be satisfied in
practice. [6]
(iv) LetT = inf{t 0 : St 0 or St K}, where K is a positive constant and assume
that c > .
(a) Explain why either ST = K or 1 ST < 0.
(b) Verify that the conditions of the optional stopping theorem apply to the
stopped martingaleYmin(t,T ) .
(c) Define to be the probability that S goes below zero before it hits K,
that is, the probability that ST < 0. Prove that
( ) (1 ) K
E YT e
(d) Derive a lower bound for using the result in (iv)(c) and state the
limit of this lower bound as K . [6]
[Total 15]
END OF PAPER


10 t t t t t r r r r .
is best estimated using the lag-1 autocorrelation of R ,
120
1 1
2
120
2
1
( )( )
( )
t t t t
t
t t
t
r r
r
.
Noticing that Cov(Dt, Rt) = Var(Rt), we suggest minus the ratio of the
sample covariance of D with R divided by the sample variance of the R, i.e.
Subject 103 (Stochastic Modelling) April 2004 Examiners Report
120
1
120
2
1
( )( )
( )
t t
t
t
t
r r d d
r r
, although in fact a better estimate would be obtained
if account was taken of the variable mean of the Rt.
(Less appropriate is using the fact that Var(Dt ) 2 Var(Rt ) to suggest
2 / 2 sD sR , since the sign of is lost in this operation.)
Now we have an estimate for , we can use Rt Dt .
(v) X121 X120 R121 D121 . In addition, R121 1 (R120 0 ) e121 and
D121 R121 .
Therefore X121 X120 (1 )[ 1 (R120 0 )] (1 )e121 , which
implies that x120 (1) X120 (1 )[ 1 (R120 0 )]
and 2 2
Var(X121 x120 (1)) (1 ) .
Most candidates made encouraging progress with this question, though part (iv) proved
difficult for most. In part (v), as well, the relationships between the variables were not
always well understood.
%%%%%%%%%%%%%%%%%%%%%%%%%%%%%%%%%%%%%%%%%%%%%%%%%%%%%%%%%%%%%%%%%%%%%%%%%%%%%%%%%%%%%%%%%%%%%%%%%%
10 (i) The increment St+s St is equal to cs (Nt+s Nt), which is independent of
anything that happened before time t and has the same distribution as
cs (Ns N0), using the independent increment property of the Poisson
process.
Alternatively, it is possible to use the Decomposition Theorem, to state that a
Lévy process is any sum of three components: a deterministic linear function
of time, a purely continuous random component (a multiple of Brownian
motion) and a purely discontinuous random component (a compound Poisson
process). In this case the coefficient of the Brownian component is zero and
the jumps of the compound Poisson process all have size 1.
(ii) By the independent increments property, Nt s Nt ~ Po( s) given Nt.
(iii) (a) [ | ] [u c(t s)] [ Nt s | ] [u c(t s) Nt ] s[e 1]
E Yt s Ft e E e Ft e .
This is equal to Yt if c [1 e ] .
(b) If c < , the line y = c increases slowly from 0 as increases,
whereas [1 e ] increases more rapidly to start with, but never
Subject 103 (Stochastic Modelling) April 2004 Examiners Report
Page 13
exceeds . Therefore there must be a crossing point at a positive value
of .
If, on the other hand, c > , then there is no positive crossing point; the
line y = c goes slowly to as , whereas [1 e ] tends to
exponentially fast as .
(c) Yes; this is the condition for the company to remain solvent.
(iv) (a) If S hits K first, then ST = K exactly, since increases in S occur
continuously. But if S becomes negative before hitting K, then the
negative movement must have been caused by a downward jump;
jumps are of magnitude 1, so in this case ST can be no less than 1.
(b) Since St is between 1 and K for all 0 t T, it follows that Ymin(t,T ) is
bounded above and below, from which we deduce that the optional
stopping theorem applies.
(c) ST = K with probability 1 , or with probability , ST takes a value
somewhere between 1 and 0. Therefore YT takes the value e K with
probability 1 , or with probability it takes some value between 1
and e . Hence the result.
(d) Clearly,
[ ]
1 1
K u K
T
K K
E Y e e e
e e
.
As K becomes larger, the fact that < 0 means that e K 0, so in the
limit we have e u.
There was an error in the final part of the question; where the question asked for a lower
bound, the quantity which could be derived from the previous part of the question was in fact
an upper bound. Candidates who reached the last part and were confused by the error were
treated generously.
It appeared that many candidates tried this question when they were short on time. Those
candidates who attempted part (iv) often did quite well on it, even if they had omitted earlier
parts of the question.
