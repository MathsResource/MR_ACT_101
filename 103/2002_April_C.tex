\documentclass[a4paper,1pt]{article}

%%%%%%%%%%%%%%%%%%%%%%%%%%%%%%%%%%%%%%%%%%%%%%%%%%%%%%%%%%%%%%%%%%%%%%%%%%%%%%%%%%%%%%%%%%%%%%%%%%%%%%%%%%%%%%%%%%%%%%%%%%%%%%%%%%%%%%%%%%%%%%%%%%%%%%%%%%%%%%%%%%%%%%%%%%%%%%%%%%%%%%%%%%%%%%%%%%%%%%%%%%%%%%%%%%%%%%%%%%%%%%%%%%%%%%%%%%%%%%%%%%%%%%%%%%%%

\usepackage{eurosym}
\usepackage{vmargin}
\usepackage{amsmath}
\usepackage{graphics}
\usepackage{epsfig}
\usepackage{enumerate}
\usepackage{multicol}
\usepackage{subfigure}
\usepackage{fancyhdr}
\usepackage{listings}
\usepackage{framed}
\usepackage{graphicx}
\usepackage{amsmath}
\usepackage{chngpage}

%\usepackage{bigints}
\usepackage{vmargin}

% left top textwidth textheight headheight

% headsep footheight footskip

\setmargins{.0cm}{.5cm}{16 cm}{cm}{0.5cm}{0cm}{1cm}{1cm}

\renewcommand{\baselinestretch}{1.}

\setcounter{MaxMatrixCols}{10}

\begin{document}

\begin{enumerate}
\item
[Total 9]

%%%%%%%%%%%%%%%%%%%%%%%%%%%%%%%%%%%%%
5
The ratio of claims to premium income is calculated annually by a division of a large
insurance company over a period of 0 years. The observed values are plotted against
time in Figure 1a, with the sample autocorrelation function (ACF) plotted in
Figure 1b; the dotted lines indicate the cutoff points for significance at the 5% level.
Figure 1a
Ratio of claims to premium income
1.00
1.100
1.000
0.900
0.800
15
0.700
Year, t
Figure 1b
Autocorrelation Function for Ratio
0.6
0.4
0.
0
-0.
0
1


4
5
6
7
-0.4
-0.6
Lag, k
(i) Explain which feature of the Figures indicates that differencing is not required
in order to obtain a stationary series.
[1]
(ii) On the basis of the sample ACF, r k , the company’s analyst decides to fit a
first-order autoregressive model to the data. State, with reasons, whether you
consider this to be a reasonable decision and indicate what additional plot you
would require in order to make a firmer recommendation.
[]
10 A00—4(iii)
The model is fitted and the residuals calculated. The sample ACF of the
residuals is shown in Figure 1c. State what conclusions you would draw from
the plot.
[1]
Figure 1c
Autocorrelation Function for Residuals
0.6
0.4
0.
0
-0.
0
1


4
5
6
7
-0.4
-0.6
Lag, k
(iv)
The fitted model is
X t = 0.49541 + 0.5118X t - 1 + e t .
6
(a) Derive forecasts x ˆ 0 (1) and x ˆ 0 () for the next two values in the
series.
(b) Comment on the reliability of the forecasts.
[4]
[Total 9]
A disability benefit scheme is modelled in continuous time by a Markov jump process
with states A (active), T (temporarily disabled), P (permanently disabled) and
D (dead). The transition rates are as follows:
A ® T:  l
A ® P: l
A ® D: a
T ® A: 5 l
T ® P:  l
T ® D: a
P ® D:  a
(i) Write down the generator matrix of the process.
(ii) Calculate the probability that, having started in state A, the process has visited
neither T nor P by time t.
[]
(iii) Write down the matrix form of the Kolmogorov backward differential
equations and use this to derive a differential equation for p PD (t), the
probability that a scheme member in state P at time 0 will be in state D at
time t.
[]
(iv) Solve the equation for p PD (t), the probability that a policyholder, initially
permanently disabled, is dead by t.
[]
[Total 9]
10 A00—5
[]


5
(i) If r 1 , r  , r  , ... were all close to 1, that would indicate a need for differencing.
That is not the case here.
(ii) The ACF of an AR(1) is geometrically decreasing. That is approximately the
case here, so AR(1) is not completely unreasonable, but we need a plot of the
sample PACF to check.
(iii) There is a fairly significant departure from a white noise process. The model
does not appear to fit.
(iv) (a)
x ˆ 0 (1) = 0.49541 + 0.5118 x 0 ,
x ˆ 0 () = 0.49541 + 0.5118 x ˆ 0 (1) = 0.74896 + 0.619 x 0 .
[Numerical values are acceptable here: x 0 is in fact equal to 1.01, so
the forecasts would be x ˆ 0 (1) = 1.01, x ˆ 0 () = 1.014: a margin of
error is acceptable, as x 0 must be read off the graph.]
(b)
The forecasts are unreliable:
There are only 0 observations in the series, so the confidence intervals
would be quite wide in any case.
In addition, we have seen that the model is inadequate, casting further
doubt on them.
Many candidates provided interesting and sensible comments on the
data provided, though the problems caused by the small sample size
were not generally recognised.
6
(i)
The generator is
æ -a - 4 l
l
a ö
 l
ç
÷
-a - 7 l  l
a ÷
ç 5 l
.
ç 0
-  a  a ÷
0
ç ç
÷ ÷
0
0
0 ø
è 0
Page 5Subject 10 (Stochastic Modelling) — 
%%%%%%%%%%%%%%%%%%%%%%%%%%%%%%%%%%%%%

(ii)
The required event is that either no jump out of A has taken place by t or that a
jump to D has taken place. This has probability
e - ( a +4 l )t + (1 - e - ( a +4 l )t )
(iii)
a
.
a + 4 l
The backward equations state P ¢ ( t ) = AP ( t ).
We have
d
p PD ( t ) = -  a p PD ( t ) +  a
dt
(iv)
The solution to p ¢ PD ( t ) = -  a p PD ( t ) +  a with p PD (0) = 0 is p PD ( t )
= 1 - e -  a t .
Well done, on the whole. Where there were difficulties they may have been
due to practising more with the time-inhomogeneous version of the equation
rather than the time-homogeneous one.
