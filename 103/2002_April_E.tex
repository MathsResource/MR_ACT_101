\documentclass[a4paper,1pt]{article}

%%%%%%%%%%%%%%%%%%%%%%%%%%%%%%%%%%%%%%%%%%%%%%%%%%%%%%%%%%%%%%%%%%%%%%%%%%%%%%%%%%%%%%%%%%%%%%%%%%%%%%%%%%%%%%%%%%%%%%%%%%%%%%%%%%%%%%%%%%%%%%%%%%%%%%%%%%%%%%%%%%%%%%%%%%%%%%%%%%%%%%%%%%%%%%%%%%%%%%%%%%%%%%%%%%%%%%%%%%%%%%%%%%%%%%%%%%%%%%%%%%%%%%%%%%%%

\usepackage{eurosym}
\usepackage{vmargin}
\usepackage{amsmath}
\usepackage{graphics}
\usepackage{epsfig}
\usepackage{enumerate}
\usepackage{multicol}
\usepackage{subfigure}
\usepackage{fancyhdr}
\usepackage{listings}
\usepackage{framed}
\usepackage{graphicx}
\usepackage{amsmath}
\usepackage{chngpage}

%\usepackage{bigints}
\usepackage{vmargin}

% left top textwidth textheight headheight

% headsep footheight footskip

\setmargins{.0cm}{.5cm}{16 cm}{cm}{0.5cm}{0cm}{1cm}{1cm}

\renewcommand{\baselinestretch}{1.}

\setcounter{MaxMatrixCols}{10}

\begin{document}

\begin{enumerate}
\item
[4]
[Total 11]

%%%%%%%%%%%%%%%%%%%%%%%%%%%%%%%%%%%%%
9
A motor insurer operates a no claims discount system that has five levels. The
percentage of the basic premium paid by the insured in each level is as follows:
Level % premium charged
5
4


1 100
90
80
70
60
Insured motorists move between levels depending on the number of claims in the
previous year. For each policyholder, the number of claims per year follows a
Poisson distribution with mean 0.5.
For those in Levels , , 4 and 5 at the start of the previous year:
· if no claims are made during the previous year, the insured moves down one
level (e.g. from Level 4 to Level )
· if one claim is made during the previous year, the insured moves up one level
(except those in Level 5 at the start of the previous year, who will remain in
Level 5)
· if two claims are made during the previous year, the insured moves up two
levels (except those in Level 5 at the start of the previous year, who will
remain in Level 5 and those in Level 4, who will move to Level 5)
· if three or more claims are made during the previous year, the insured moves
to Level 5
For those in Level 1 at the start of the start of the previous year, a no claims discount
protection policy applies whereby they remain in Level 1 if they make one claim. If
they make two claims, they move to Level . If they make three or more claims, they
move to Level 5. If they make no claims, they remain in Level 1.
(i) Determine the transition matrix for the no claims discount system (assuming
that all motorists continue their policies).
[]
(ii) A policyholder is in Level  for the first year of the policy. Assuming that the
policy is maintained, calculate the probability that at the start of the third year
the policyholder will be (a) in Level 1, (b) in Level .
[]
(iii) (a)
State conditions under which the probability of being in a particular
state after n years converges as n ® ¥ to some limit which is
independent of the initial state.
(b)
Verify that the conditions are satisfied in this instance.
(c)
Determine the ultimate probability that the insured will be in Level 1.
[8]
10 A00—8(iv)
10 A00—9
The insurer suspects that the model used for its calculations may be too
simplistic. Given annual data listing numbers of claims per policy, broken
down by discount level, state which test would be most appropriate to test the
assumption that the distribution of the number of claims per policy per year is
Poisson with mean 0.5.
[1]
[Total 15]











9
(i)
Level at start of this year after:
Level at start
of prev yr 0 claims in
previous yr 1 claims in
previous yr  claims in
previous yr  or more claims
in previous yr
5
4


1 4


1
1 5
5
4

1 5
5
5
4
 5
5
5
5
5
For each policyholder, the number of claims in each year has a Poisson (0.5)
distribution. So
P (0 claims) = e - 0.5 = 0.7788
P (1 claim) = 0.5 e - 0.5
e - 0.5
= 0.5  .
 = 0.1947
P ( claims)
= 0.04
P ( or more claims) = 1 – (0.7788 + 0.1947 + 0.04) = 0.00
0
0
0.00 ö
æ 0.975 0.04
ç
÷
0
0.1947 0.04 0.00 ÷
ç 0.7788
Transition matrix P = ç 0
0.7788
0
0.1947 0.065 ÷
ç
÷
0
0.7788
0
0.1 ÷
ç 0
ç 0
0
0
0.7788 0.1 ÷ ø
è
(ii)
Page 8
In order to be in level 1 in year  the policyholder requires two consecutive
claim-free years. The probability of this is 0.7788  = 0.6065.Subject 10 (Stochastic Modelling) — 
%%%%%%%%%%%%%%%%%%%%%%%%%%%%%%%%%%%%%

A similar argument can be used for the probability of being in level  in
year , but it may be simpler to calculate the whole vector of probabilities x  .
x 1 =
x  =
x  =
=
( 0 0 1 0
( 0 0 1 0
( 0 0.7788
( 0.6065 0
0 )
0 ) . P =
( 0
0.7788 0 0.1947 0.065 )
0 0.1947 0.065 ) . P
0.0 0.096 0.0506 )
Probability of being in level  is 0.%
(iii)
(a) The required conditions are that the chain is irreducible and aperiodic.
(b) Irreducibility: level i can be reached from level j in | j - i | steps;
Aperiodicity: p ii > 0 for some i.
(c) The stationary distribution p will not depend on the starting position.
Require
( p 1
p 
p 
p 4
p 5 ) P =
( p 1
p 
p 
p 4
p 5 )
This gives the following equations:
0.975 p 1 + 0.7788 p  = p 1 (1)
0.04 p 1 + 0.7788 p  = p  ()
0.1947 p  + 0.7788 p 4 = p  ()
0.04 p  + 0.1947 p  + 0.7788 p 5 = p 4 (4)
and
p 1 + p  + p  + p 4 + p 5 = 1
(5)
Solving the simultaneous equations:
from (1) Þ p  = 1 - 0.975 p 1 = 0.040 p 1 ;
0.7788
substitute for p  into () Þ p  = 0.040 - 0.04 p 1 = 0.0144 p 1 ;
0.7788
substitute for p  and p  into ()
Þ p 4 = 0.0144 - 0.1947  ́ 0.040 p 1 = 0.00747 p 1 ;
0.7788
Page 9Subject 10 (Stochastic Modelling) — 
%%%%%%%%%%%%%%%%%%%%%%%%%%%%%%%%%%%%%

substitute for p  , p  and p 4 into (4)
Þ p 5 = 0.00747 - 0.1947  ́ 0.0144 - 0.04  ́ 0.040 p 1 = 0.00541 p 1
0.7788
and substituting for p  , p  , p 4 and p 5 into (5) Þ p 1 = 0.9440
(and p  = 0.01, p  = 0.0117, p 4 = 0.0071 and p 5 = 0.0051).
(iv)
A chi-squared goodness-of-fit test is best here.
Very good answers on the whole. Some confusion was caused by the fact that
the states were presented in the reverse of the standard order, but most
candidates coped with this pretty well.
