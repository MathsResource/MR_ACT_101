

5
A motor insurance company assumes that a holder of a provisional driver’s
licence will make claims according to a Poisson process with rate X per year,
where X is not fixed but is determined randomly for each driver according to the
density function
f(x) = 2e −2x
6
(x > 0).
(i) Describe how to simulate an observation X from the density f using a
single pseudo-random variable U assumed uniformly distributed on [0, 1].
[3]
(ii) Explain how, given the value X generated in (i), you would use a sequence
U 1 , U 2 , ... of uniform pseudo-random variables to simulate the number of
claims made in two six-month periods by a provisional driver with mean
claim rate X per year.
[4]
(iii) Describe a simulation-based method for estimating the conditional
probability that a provisional driver makes 2 or more claims in the second
six months of driving given that no claim was made in the first six
months. [Here the value X is to be assumed unknown.]

[Total 9]

%%%%%%%%%%%%%%%%%%%%%%%%%%%%%%%%%5
6
(x > 0).
(i) Describe how to simulate an observation X from the density f using a
single pseudo-random variable U assumed uniformly distributed on [0, 1].
[3]
(ii) Explain how, given the value X generated in (i), you would use a sequence
U 1 , U 2 , ... of uniform pseudo-random variables to simulate the number of
claims made in two six-month periods by a provisional driver with mean
claim rate X per year.
[4]
(iii) Describe a simulation-based method for estimating the conditional
probability that a provisional driver makes 2 or more claims in the second
six months of driving given that no claim was made in the first six
months. [Here the value X is to be assumed unknown.]

[Total 9]
n
The evolution of a stock price is modelled as a discrete time process S n = Σ i=
1 X i ,
where X 1 , X 2 , ... are independent, identically distributed random variables with
P{X i = 1} = p and P{X i = −1} = q = 1 − p. The investment will be liquidated at
either the bankruptcy time T 0 (the first time n when the price S n hits 0) or the
first time T K when the price attains a fixed target K, whichever occurs first.
Let T = min(T 0 , T K ) denote the liquidation time (the exit time from [0, K]). Let
A = {T K < T 0 } denote the event that the target is met before bankruptcy and let
p k = P[AS 0 = k] denote the probability of this event, given that the initial price is
S 0 = k, where k ∈ {1, ..., K − 1}.
(i) By conditioning on the price of the stock at time 1, determine a difference
equation satisfied by p k , k = 1, ..., K − 1.

(ii) Assume that p = q =
(iii)
1
2
.
(a) Show that S n is a martingale.
(b) Derive an expression for p k by applying the optional stopping
theorem to this martingale stopped at T.
6
(x > 0).
(i) Describe how to simulate an observation X from the density f using a
single pseudo-random variable U assumed uniformly distributed on [0, 1].
[3]
(ii) Explain how, given the value X generated in (i), you would use a sequence
U 1 , U 2 , ... of uniform pseudo-random variables to simulate the number of
claims made in two six-month periods by a provisional driver with mean
claim rate X per year.
[4]
(iii) Describe a simulation-based method for estimating the conditional
probability that a provisional driver makes 2 or more claims in the second
six months of driving given that no claim was made in the first six
months. [Here the value X is to be assumed unknown.]

[Total 9]
n
The evolution of a stock price is modelled as a discrete time process S n = Σ i=
1 X i ,
where X 1 , X 2 , ... are independent, identically distributed random variables with
P{X i = 1} = p and P{X i = −1} = q = 1 − p. The investment will be liquidated at
either the bankruptcy time T 0 (the first time n when the price S n hits 0) or the
first time T K when the price attains a fixed target K, whichever occurs first.
Let T = min(T 0 , T K ) denote the liquidation time (the exit time from [0, K]). Let
A = {T K < T 0 } denote the event that the target is met before bankruptcy and let
p k = P[AS 0 = k] denote the probability of this event, given that the initial price is
S 0 = k, where k ∈ {1, ..., K − 1}.
(i) By conditioning on the price of the stock at time 1, determine a difference
equation satisfied by p k , k = 1, ..., K − 1.

(ii) Assume that p = q =
(iii)
1
2
.
(a) Show that S n is a martingale.
(b) Derive an expression for p k by applying the optional stopping
theorem to this martingale stopped at T.


Assume now that p \neq q.
(a)
(b)
103 S2001—4
Determine a value \theta \neq 1 such that Y n = \theta S n is a martingale.
[4]
Derive an expression for p k in this case.
[Total 11]

%%%%%%%%%%%%%%%%%%%%%%%%%%%%%%%%%%%%%%%%%%%%%%%%%%%%%%%%%
5
(i)
Use the inverse distribution function technique.
We have F(x) = 1 - e - 2 x , and we require U = F(X) = 1 - e - 2 X , so that
X = - 12 log(1 - U ) .
(ii)
We need two Poisson pseudo-random variables, Y 1 and Y 2 , each with
mean m = 12 X .
There are two possible methods for generating Poisson(m):
6
· Keep generating exponential variables with mean 1 until the
cumulative sum exceeds m, then set Y equal to one less than the
number of variables generated.
· Draw up a table of the cumulative distribution function F m of Poisson (m), use a single uniform U and let Y be the first y such
that F m (y) > U.
(iii) Carry out the above procedures a large number of times, independently.
Let N 0 be the number of times no claims were made in the first six
months, N 0,2+ the number of times no claims were made in the first six months but two or more in the next six. The required estimate is
N 0,2+ / N 0 .
(i) Let A = {T K < T 0 }, U = {S 1 - S 0 = 1}, D = {S 1 - S 0 = -1}. Conditioning after
one step, we find:
p k = P[A1⁄2S 0 = k] = pP[A1⁄2S 0 = k Ç U] + qP[A1⁄2S 0 = k Ç D] = p p k +1 + q p k - 1 .
(Together with p K = 1, p 0 = 0 this could be used to solve p k ...)
(ii)
(a)
Let F n = s{X 1 , X 2 , ..., X n }. To show S n is a martingale we need to
show that the conditional expectation of the increments of S n is 0.
E[S n +1 - S n 1⁄2F n ] = E[X n +1 1⁄2F n ] = E[X n +1 ] = 0.
(b)
By the optional stopping theorem,
E k (S T ) = S 0 = k
On the other hand
E k (S T ) = K p k + 0 (1 - p k ) = k,
Page 4Subject 103 (Stochastic Modelling) — 
%%%%%%%%%%%%%%%%%%%%%%%%%%%%%%%%

which gives
p k =
(iii)
k
.
K
The exponential process Y n = q S n is a martingale if and only if the
expectation of the factors is 1, i.e. if and only if Eq X 1 = (pq + qq - 1 ) = 1. This
equation has two roots q = 1, q = qp .
Applying the optional stopping theorem to the martingale yielded by the
second root gives: q k = q K p k + q 0 (1 - p k ) and p k =

