
\documentclass[a4paper,12pt]{article}

%%%%%%%%%%%%%%%%%%%%%%%%%%%%%%%%%%%%%%%%%%%%%%%%%%%%%%%%%%%%%%%%%%%%%%%%%%%%%%%%%%%%%%%%%%%%%%%%%%%%%%%%%%%%%%%%%%%%%%%%%%%%%%%%%%%%%%%%%%%%%%%%%%%%%%%%%%%%%%%%%%%%%%%%%%%%%%%%%%%%%%%%%%%%%%%%%%%%%%%%%%%%%%%%%%%%%%%%%%%%%%%%%%%%%%%%%%%%%%%%%%%%%%%%%%%%

\usepackage{eurosym}
\usepackage{vmargin}
\usepackage{amsmath}
\usepackage{graphics}
\usepackage{epsfig}
\usepackage{enumerate}
\usepackage{multicol}
\usepackage{subfigure}
\usepackage{fancyhdr}
\usepackage{listings}
\usepackage{framed}
\usepackage{graphicx}
\usepackage{amsmath}
\usepackage{chngpage}

%\usepackage{bigints}
\usepackage{vmargin}

% left top textwidth textheight headheight

% headsep footheight footskip

\setmargins{2.0cm}{2.5cm}{16 cm}{22cm}{0.5cm}{0cm}{1cm}{1cm}

\renewcommand{\baselinestretch}{1.3}

\setcounter{MaxMatrixCols}{10}

\begin{document}
1
There are N individuals in a population, some of whom have a certain infection that spreads as follows. Contacts between two members of this population occur in accordance with a Poisson process having rate $\lambda$. When a contact occurs, it is
equally likely to involve any of the P =
N
e 2 j
pairs of individuals in the population.
If a contact involves an infected and a non-infected individual, then with probability $p$ the noninfected individual becomes infected. Once infected, an individual remains infected throughout. Let $X(t)$ denote the number of infected members of the population at time t.
2


%%%%%%%%%%%%%%%%%%%%%%
3

\begin{enumerate}[(i)]
\item %(i) 
State whether X(t), t > 0 is a continuous-time Markov jump process. If so, write down its state space and transition rates; if not, explain how it can be expressed in terms of a different process which is Markov.

\item %(ii) 
Derive an expression for the expected time until all members are infected, starting from a single infected individual
\end{enumerate}

%%%%%%%%%%%%%%%%%%%%%%%%%%%%%%%%%%%%%%%%%%%%%%%%%%%%%%%%%%%%%%%%%%%%%%%%%%%%%%%%
1
(i)
This is clearly a Markovian birth process. The state space is $\{0, 1, \ldots , N\}$.
Given that we have m infected and N − m healthy individuals, the
number of “dangerous” pair contacts is m(N − m). Thus, the rate
m ( N − m )
σ m,m+1 = λp
.
P
(ii)
The expected total infection time is the sum of the reciprocal rates
N − 1
P
1
.
m
=
1
m ( N − m )
λ p
å
1
1 é 1
1 ù
=
+
ê
ú , this time may also be expressed as
m ( N − m )
N ë m N − m û
N − 1 1
N − 1
. )
m
= 1 m
λ p
(Since
å
2



\end{document}
