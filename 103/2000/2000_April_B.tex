
\documentclass[a4paper,12pt]{article}

%%%%%%%%%%%%%%%%%%%%%%%%%%%%%%%%%%%%%%%%%%%%%%%%%%%%%%%%%%%%%%%%%%%%%%%%%%%%%%%%%%%%%%%%%%%%%%%%%%%%%%%%%%%%%%%%%%%%%%%%%%%%%%%%%%%%%%%%%%%%%%%%%%%%%%%%%%%%%%%%%%%%%%%%%%%%%%%%%%%%%%%%%%%%%%%%%%%%%%%%%%%%%%%%%%%%%%%%%%%%%%%%%%%%%%%%%%%%%%%%%%%%%%%%%%%%

\usepackage{eurosym}
\usepackage{vmargin}
\usepackage{amsmath}
\usepackage{graphics}
\usepackage{epsfig}
\usepackage{enumerate}
\usepackage{multicol}
\usepackage{subfigure}
\usepackage{fancyhdr}
\usepackage{listings}
\usepackage{framed}
\usepackage{graphicx}
\usepackage{amsmath}
\usepackage{chngpage}

%\usepackage{bigints}
\usepackage{vmargin}

% left top textwidth textheight headheight

% headsep footheight footskip

\setmargins{2.0cm}{2.5cm}{16 cm}{22cm}{0.5cm}{0cm}{1cm}{1cm}

\renewcommand{\baselinestretch}{1.3}

\setcounter{MaxMatrixCols}{10}

\begin{document}
\begin{enumerate}
%%%%%%%%%%%%%%%%%%%%%
%%%%%%%%%%%%%%%%%%%%%%%%%%%%%%%%%%%%%%%%%%%%%%%%%%%%%%%%%%%%%%%%%%%%%%%%%%%%%%%%%%%%%%%%%%%%%%%%%%%%%%%%%%%
3
\item The total number of claims received by an insurance company is described by an inhomogeneous Poisson process with rate $\lambda(t)$.
Write down the Kolmogorov forward equations for this process and show that, as in the homogeneous case, the solution is of the form
P 0j (s, t) =
where m(s, t) =
4
( m ( s , t )) j e − m ( s , t )
j !
z λ(x)dx.
t
s
[5]
According to a model used in econometrics, three economic time series, X, Y
and Z, are related to one another by the equations
%%%%%%%%%% 

\begin{eqnarray*}
X_N &=& X_{n-1} + \theta_X Z_{n-1} + e_{1,n) \\
Y_N &=& Y_{n-1} + \theta_Y Z_{n-1}+ e_{2,n) \\
Z_n &=& \theta_Z Z_{n-1} + + e_{3,n) \\
\end{eqnarray*}
where each of θ X , θ Y and θ Z lies in the interval (−1, +1) and the random
variables {e i,n : n ≥ 1, i = 1, 2, 3} may be assumed to be uncorrelated and to have
mean zero.
State, giving your reasons:
(a) whether X, Y and Z are I(0) or I(1)
(b) whether each of X, Y and Z individually satisfies the Markov property
(c) whether the vector-valued process {(X n , Y n , Z n ) : n ≥ 1} satisfies the
Markov property
(d) whether X and Y are cointegrated
103—2
[5]

\end{document}
