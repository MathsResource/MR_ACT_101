
\documentclass[a4paper,12pt]{article}

%%%%%%%%%%%%%%%%%%%%%%%%%%%%%%%%%%%%%%%%%%%%%%%%%%%%%%%%%%%%%%%%%%%%%%%%%%%%%%%%%%%%%%%%%%%%%%%%%%%%%%%%%%%%%%%%%%%%%%%%%%%%%%%%%%%%%%%%%%%%%%%%%%%%%%%%%%%%%%%%%%%%%%%%%%%%%%%%%%%%%%%%%%%%%%%%%%%%%%%%%%%%%%%%%%%%%%%%%%%%%%%%%%%%%%%%%%%%%%%%%%%%%%%%%%%%

\usepackage{eurosym}
\usepackage{vmargin}
\usepackage{amsmath}
\usepackage{graphics}
\usepackage{epsfig}
\usepackage{enumerate}
\usepackage{multicol}
\usepackage{subfigure}
\usepackage{fancyhdr}
\usepackage{listings}
\usepackage{framed}
\usepackage{graphicx}
\usepackage{amsmath}
\usepackage{chngpage}

%\usepackage{bigints}
\usepackage{vmargin}

% left top textwidth textheight headheight

% headsep footheight footskip

\setmargins{2.0cm}{2.5cm}{16 cm}{22cm}{0.5cm}{0cm}{1cm}{1cm}

\renewcommand{\baselinestretch}{1.3}

\setcounter{MaxMatrixCols}{10}

\begin{document}
\begin{enumerate}
%%-- Question 5
\item 
\begin{enumerate}[(i)]
\item (i) State Itô’s lemma as it applies to a stochastic process $\{X_t : t \geq 0\}$ and a
function $f(X_t)$ which is not explicitly dependent on t.

\item (ii) Apply Ito’s Lemma with $f(x) = x_4$ to calculate the stochastic differential
d ( B t 4 ), where B t is standard Brownian motion.
[3]
\item (iii) Hence express the Itô integral
integral involving B s .
t
z 0
B s 3 dB s in terms of B t and of an ordinary
\end{itemize}
\item Consider a homogeneous Markov chain with state space $S = \{1, 2, 3\}$ and
transition matrix

%\begin{pmatrix}
P =
F 1⁄4
G 1⁄2
G
H 3⁄4
1⁄2 1⁄4 I
0 1⁄2 J .
J
1⁄4 0 K
\begin{enumerate}[(i)]
\item (i) Calculate the 3-step transition matrix.
\item (ii) Calculate, for each of the following initial conditions, the probability that the chain will be in state 3 when it is observed at time n = 3 given that:
(iii)
103—3

(a) the chain is in state 1 at time zero
(b) the chain is in state 1 at time zero and in state 2 at time 1
(c) the probabilities of being in states 1, 2, and 3 at time zero are 8 , 9 and 31
respectively
given by 14
31 31
\end{enumerate}
How would your answers to (a), (b) and (c) change if the time of the observation were $n = 300$ instead of $n = 3$?

%%%%%%%%%%%%%%%%%%%%%%%%%%%%%%%%%%%%%%%%%%%%%%%%%%%%%%%%%%%%%%%%%%%%%%%%%%%%%%%%%%%%%%%%55%%
%%- Question 6
The daily closing price of a share is observed every trading day for a year, yielding a sequence of values ${s_1 , \ldots, s_n }$. A model is required for the purposes of predicting future variability of the share price. The model suggested is a
Brownian one.

(i)
Explain briefly, on purely theoretical grounds, which of the two models

\begin{description}
\item[I:] S t = \mu + \alpha t + σB t
\item[II:] log(S t ) = \mu + \alpha t + σB t
\end{description}
%-----------------------------------%
you would expect to provide a better fit.
(ii)

Refer to Figures 1 and 2 below.
(a) Explain briefly whether your chosen model appears to provide a good fit to the data.
(b) State one of the tests you could carry out on the data to ascertain whether the model fits adequately.

(iii)
(a) Describe how a Lévy process model differs from a Brownian model.
(b) Outline the difficulties would you encounter in practice if you were
fitting a Lévy process model to the data provided.
%---------------------------------------------%
Figure 1: the share price S n
Figure 2: log-transformed share price log S n

%%%%%%%%%%%%%%%%%%%%%%%%%%%%%%%%%%%%%%%%%%%%%%%%%%%%%%%%%%%%%%%%%%%%%%%
5
1
2
t
ò 0
B s 3 dB s =
1
4
B t 4 −
3 t
ò
2 0
B s 2 ds .
(i) æ 1 2 1 ö
1 ç
÷
P = ç 2 0 2 ÷ ,
4 ç
÷
è 3 1 0 ø
(ii) (a) P 13 3 =
14
7
=
= 0.21875.
64
32
(b) 2
P 23
=
2
1
=
= 0.125.
16
8
(c) 14 3
9 3
8 3
14 × 14 + 9 × 20 + 8 × 17
512
8
P 13 +
P 23 +
P 33 =
=
=
.
31 × 64
31 × 64
31
31
31
31
(iii)
æ 8 3 5 ö
1 ç
÷
P =
8 6 2 ÷ ,
16 ç ç
÷
è 5 6 5 ø
2
æ 29 21 14 ö
1 ç
÷
P =
26 18 20 ÷ .
64 ç ç
÷
è 32 15 17 ø
3
By time n = 300 the effects of the starting point have worn off. The answer is therefore indistinguishable from the stationary probability π 3 in all three cases.
It is easily observed that the distribution in (c) is stationary, so that
8
π 3 =
.
31
%--------------------------------------------%
6
(i) The daily change in value of a share is generally on a scale consistent with the value of the share: this tends to indicate that model II is preferable.
(ii) (a)
Model II does appear to fit better than model I; the S dataset does indeed exhibit large variations when it is at a high level, and smaller ones when low.
However, the fit does not appear all that good, as Brownian increments are normally distributed, so are seldom as large as
some of the jumps which appear in this dataset.

(b)
If the Brownian model were accurate, the day-to-day increments s n − s n− 1 should be independently normally distributed with
constant mean and variance:
a test of normality (Anderson-Darling, Kolmogorov-Smirnov, χ 2
goodness-of-fit) would do fine; a test of independence (based on sample ACF, or the Durbin-Watson statistic) would also be a good
suggestion.
(iii)
\end{document}
