\documentclass[a4paper,12pt]{article}

%%%%%%%%%%%%%%%%%%%%%%%%%%%%%%%%%%%%%%%%%%%%%%%%%%%%%%%%%%%%%%%%%%%%%%%%%%%%%%%%%%%%%%%%%%%%%%%%%%%%%%%%%%%%%%%%%%%%%%%%%%%%%%%%%%%%%%%%%%%%%%%%%%%%%%%%%%%%%%%%%%%%%%%%%%%%%%%%%%%%%%%%%%%%%%%%%%%%%%%%%%%%%%%%%%%%%%%%%%%%%%%%%%%%%%%%%%%%%%%%%%%%%%%%%%%%

\usepackage{eurosym}
\usepackage{vmargin}
\usepackage{amsmath}
\usepackage{graphics}
\usepackage{epsfig}
\usepackage{enumerate}
\usepackage{multicol}
\usepackage{subfigure}
\usepackage{fancyhdr}
\usepackage{listings}
\usepackage{framed}
\usepackage{graphicx}
\usepackage{amsmath}
\usepackage{chngpage}

%\usepackage{bigints}
\usepackage{vmargin}

% left top textwidth textheight headheight

% headsep footheight footskip

\setmargins{2.0cm}{2.5cm}{16 cm}{22cm}{0.5cm}{0cm}{1cm}{1cm}

\renewcommand{\baselinestretch}{1.3}

\setcounter{MaxMatrixCols}{10}

\begin{document}
\begin{enumerate}
[2]
[Total 5]
(i) Give a definition of the spectral density of a stationary time series,
expressed in terms of the autocovariance function {γ k : k ∈ Z} of the
process. Use this definition to derive the spectral density of a first-order
moving average process and of a first-order autoregression.
[5]
(ii) Suppose the “inverse” of a time series model with spectral density f(ω) is
1
. Using part (i), state
defined to be the model with spectral density
f ( ω )
the form of the inverse of a first-order moving average and state the way
in which the inverse of an invertible MA(1) differs from the inverse of a
non-invertible MA(1).
[2]
[Total 7]
103 A2001—24
Let X n denote an autoregressive high frequency time series modelled by:
X n+1 = (1 − \alpha) X n + θ + τe n ,
where e n = ±1 with equal probabilities and \alpha, θ, τ are constant parameters. An
analyst wishes to investigate whether this series may be approximated by some
continuous time diffusion, i.e. X n ≈ Y nh , where Y t satisfies a stochastic differential
equation
dY t = \mu(Y t ) dt + dB t
and B t denotes standard Brownian motion.

