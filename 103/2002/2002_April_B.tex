\documentclass[a4paper,1pt]{article}

%%%%%%%%%%%%%%%%%%%%%%%%%%%%%%%%%%%%%%%%%%%%%%%%%%%%%%%%%%%%%%%%%%%%%%%%%%%%%%%%%%%%%%%%%%%%%%%%%%%%%%%%%%%%%%%%%%%%%%%%%%%%%%%%%%%%%%%%%%%%%%%%%%%%%%%%%%%%%%%%%%%%%%%%%%%%%%%%%%%%%%%%%%%%%%%%%%%%%%%%%%%%%%%%%%%%%%%%%%%%%%%%%%%%%%%%%%%%%%%%%%%%%%%%%%%%

\usepackage{eurosym}
\usepackage{vmargin}
\usepackage{amsmath}
\usepackage{graphics}
\usepackage{epsfig}
\usepackage{enumerate}
\usepackage{multicol}
\usepackage{subfigure}
\usepackage{fancyhdr}
\usepackage{listings}
\usepackage{framed}
\usepackage{graphicx}
\usepackage{amsmath}
\usepackage{chngpage}

%\usepackage{bigints}
\usepackage{vmargin}

% left top textwidth textheight headheight

% headsep footheight footskip

\setmargins{.0cm}{.5cm}{16 cm}{cm}{0.5cm}{0cm}{1cm}{1cm}

\renewcommand{\baselinestretch}{1.}

\setcounter{MaxMatrixCols}{10}

\begin{document}

\begin{enumerate}
\item
10 A00—
(i) Explain briefly what is meant by a linear trend and by seasonal variation in
respect of a sequence of observed values {x 1 , x  , ..., x n } forming a time series.
[]
(ii) Describe an operation which can be applied to the data in order to remove
additive seasonal variation of period .
[]
(iii) Derive an appropriate operation to perform on the process
X t = exp(a + bt + Z t ),
where Z is I(1), in order to obtain a stationary process Y t .
4
[]
[Total 6]
Consider the second-order autoregressive process
Y t = -  a Y t - 1 + a  Y t -  + Z t
where {Z t } is a zero-mean white noise process with Var(Z t ) = s  .
(i) Determine the range of values of a for which the process Y can be stationary.
[]
(ii) Derive the autocovariances γ 1 and γ  of Y in terms of a and s .
10 A00—
[6]

%%%%%%%%%%%%%%%%%%%%%%%%%%%%%%%%%%%%

(i)
A linear trend means that the line of best fit to the data plotted against time
would have a non-zero slope or that there is evidence from the observations
that there is an underlying tendency for the values to increase or decrease with
time at a constant rate.
Seasonal variation is another deterministic component of the mean which
causes E(X t ) to depend on the remainder when t is divided by the period, d; to
spot it from the data, look for recurring patterns in the data or check the
sample ACF.
(ii)
Either use moving averages: set Y t = 1⁄(X t + X t - 1 ), which has had the seasonal
variation smoothed out.
Or use seasonal differencing: set Y t = Ñ  X t = X t - X t -  (from the Box-
Jenkins armoury)
Or use any linear filter Y t = å a j X t - j as long as
j
å a = å a
j
even j
j
(any such
odd j
filter does answer the question, though it may look very strange)
Or method of seasonal means: estimate a mean for the even-numbered
observations and another for the odd-numbered ones, then subtract these from
the corresponding observations to obtain a set of residuals, which can then be
analysed.
(iii)
Set Y = Ñ (log X ) = Ñ ( a + bt + Z t ) = b + Ñ Z t .
Since it is stated that Z is I (1) it follows that Ñ Z is stationary.
Question  was generally well answered.
Page Subject 10 (Stochastic Modelling) — 
%%%%%%%%%%%%%%%%%%%%%%%%%%%%%%%%%%%%%

4
(i)
In terms of the backwards shift operator we have
(1 +  a B - a  B  ) Y = Z .
We must find the values of a such that the roots of the polynomial
1 +  a x - a  x  lie outside the unit circle.
The roots are
(
)
1
1 ±  , so we require that
a
 + 1
> 1 and
a
 - 1
> 1 , in
a
other words that a <  - 1 .
(ii)
Y t = - α Y t - 1 + α  Y t -  + Z t
Cov[ Y t , Y t ] = γ 0 = - αγ 1 + α  γ  + σ  (1)
Cov[ Y t , Y t - 1 ] = γ 1 = - αγ 0 + α  γ 1 ()
Cov[ Y t , Y t -  ] = γ  = - αγ 1 + α  γ 0 ()
From (); γ 1 = -
 ag 0
(4)
1 - a 
Substitute for γ 1 from (4) into ()
γ  =  a .
 ag 0
1 - a 
æ 5 a  - a 4 ö
+ α  γ 0 = g 0 . ç
ç 1 - a  ÷ ÷
è
ø
(5)
substitute for γ 1 from (4) and γ  from (5) into (1)
Þ
γ 0 =
( )
)( 1 - 6 a + a )
s  1 - a 
(
1 + a 

4
substitute for γ 0 from (6) into (4) and (5) to find γ 1 and γ 
Þ
γ 1 =
-as 
( 1+ a )( 1 - 6 a

Page 4
+ a 4
)
( 5 a - a ) . s
γ =
( 1 + a )( 1 - 6 a + a )

and



4


4
(6)Subject 10 (Stochastic Modelling) — 
%%%%%%%%%%%%%%%%%%%%%%%%%%%%%%%%%%%%%

(Alternative form for the denominator: 1 - 5 a  - 5 a 4 + a 6 .)
Generally well answered, although the exact range of permitted values for a
in (i) caused difficulties.


\end{document}
