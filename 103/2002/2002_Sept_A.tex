
1 Consider the Poisson process Xt defined as a Markov jump process with state space
{0, 1, 2, …} and transition probabilities
Pij(t) = ( )
( )!
j i
t e t
j i

 

(j  i).
(i) Prove that Xt has independent increments. [3]
(ii) Hence prove that Xt t is a martingale with respect to the natural filtration t
associated with Xt. [3]
[Total 6]
2 Yt, t = 1, 2, …., is a time series defined by
Yt Yt 1 Zt 1  Zt 1  
   
where Zt, t = 0, 1, , is a sequence of independent zero-mean variables with
common variance 2 and where || < 1.
(i) State, giving your reasons, the values of p, d and q for which Y is an
ARIMA(p, d, q) process. [2]
(ii) Derive the autocorrelation function 	k, k = 0, 1, 2, . [6]
[Total 8]
3 Consider the accident proneness model in which the cumulative number of accidents
Xt suffered by a driver is a Markovian birth process with linear transition rates given
by
i,j =
( 1) if = 1
0 otherwise
j i i    
(i) Denoting by ai(t) the probability that a driver who has had no accidents at
time 0 has had at least i accidents by time t, explain why
ai(t + dt) = ai(t) + (ai1(t)  ai(t))i
dt + o(dt)
as dt  0 and hence derive a differential equation satisfied by ai(t). [3]
(ii) Show that
ai(t) = (1  et)i
and deduce the value of P[X(t) = i|X(0) = 0]. [4]
103 S2002—3 PLEASE TURN OVER
(iii) A colleague suggests that a better model would involve transition rates i,i+1(t)
which are dependent on t as well as on i. Comment on this suggestion. [1]
(iv) The same colleague proposes the model i,i+1(t) = (i 1)
t
 . Comment on the
proposed model. [1]
[Total 9]


%%%%%%%%%%%%%%%%%%%%%%%%%%%%%%
%%%%%%%%%%%%%%%%%%%%%%%%%%%%%%%%%%%%%%%%%
Faculty of Actuaries Institute of Actuaries
EXAMINATIONS
September 2002
Subject 103 — Stochastic Modelling
EXAMINERS’ REPORT
Introduction
The attached subject report has been written by the Principal Examiner with the aim of
helping candidates.  The questions and comments are based around Core Reading as the
interpretation of the syllabus to which the examiners are working.  They have however
given credit for any alternative approach or interpretation which they consider to be
reasonable.
K G Forman
Chairman of the Board of Examiners
12 November 2002
 Faculty of Actuaries
 Institute of Actuaries
Subject 103 (Stochastic Modelling) — September 2002 — Examiners’ Report
Page 2
Questions involving straightforward applications of Markov Chains and Time Series were
well answered, and the standard of answers to questions about martingales and Brownian
motion is improving from year to year, but candidates appeared to experience unexpected
difficulties in relation to the questions on Markov jump processes.
1 (i) Prove this using the Markov property. (Note that this mark can be earned for
use of the property even if the word “Markov” is not mentioned.)
If s0 < s1, … < sn < s < t, then
P[Xt  Xs = j
Xs0 = i0, …, Xsn = in, Xs = i]
= P[Xt = i + jXs = i] = ( ( )) ( ) ,
!
j
t s e t s
j
   
independent of i0, i1, …, in.
(ii) We need to prove that E[Xt  s ] = Xs .
With s < t we have
E[Xt  s ] = E[Xt  Xs  s ] E[Xs  s ]
= E[Xt  Xs  Xs = (t  s)  Xs .
Thus E[Xt  t  s ] = Xs  s.
The key to part (i) was to use the Markov property; only a few candidates managed to do this
part. Part (ii) was generally well answered.
2 (i) Since the equation can be written (1  B) Y = (1 + (1  )B)Z, the process is
ARMA(1,1), or ARIMA(1,0,1).
(ii) There are a number of possible ways to calculate the k, and one way which
sidesteps the k and calculates the k directly. The solution presented here is
one of the possible answers; other methods were marked on their merits.
k = CovYt ,Yt k    .
Rearrange the time series equation to give Yt = Yt 1 Zt 1  Zt 1      
Now Cov[Yt, Zt] = σ2
Subject 103 (Stochastic Modelling) — September 2002 — Examiners’ Report
Page 3
and Cov[Yt, Zt1] = α.Cov[Yt1, Zt1] + Cov[Zt, Zt1] + (1 – α).Cov[Zt1, Zt1]
= α.σ2 + 0 + (1 – α).σ2 = σ2
Therefore
0 = CovYt ,Yt  = .CovYt ,Yt 1 CovYt ,Zt  1 .CovYt ,Zt 1      
= 2   2
.1    1 .
   2
0 = .1  2  . (1)
1 = CovYt ,Yt 1 
= .CovYt 1,Yt 1 CovZt ,Yt 1 1 .CovZt 1,Yt 1     
   
=   2
.0  0  1  .
   2
1 = .0  1  . (2)
substitute for γ0 from (1) into (2)
    2    2
1 = . .1  2     1 .

    2
1 2
2 . 1 .
=
1
         	

2
2
2
= 1 .
1
    
       	 

substitute for γ1 back into (1)

 
 
2 2
2
0 2
1 .
. 2 .
1
  
    

2
2
2
= 2 .
1
  
       	
For k  2,
k = CovYt ,Yt k  
= .CovYt 1,Yt k  CovZt ,Yt k  1 .CovZt 1,Yt k      
   
= . k 1 0 0     
 1
= . 1 k
k

  
Subject 103 (Stochastic Modelling) — September 2002 — Examiners’ Report
Page 4
The autocorrelation function is:
0
= k
k



. Therefore
0 =1
2
1 2
= 1
2
 


1
= 1 2 k
k k     
Part (i) was generally well answered, using a variety of different approaches. Candidates
generally made good attempts at part (ii), the main problems occurring being a failure to
correctly specify γ0 and algebraic errors in solving the simultaneous equations.
3 (i) If the driver is to have had at least i accidents by time t + dt, either there must
have been i accidents by time t or there must have been exactly i 	 by time t
and another between t and t + dt.
P(exactly i 1) = P(at least i 1) P(at least i).   
Therefore
dai
dt
= i
(ai1  ai).
(ii) Verification: d
dt
{(1  et)i}= i
et(1  et)i1 .
ai1  ai = (1  et)i1(1 [1  et]).
We should also verify that ai(0) is correct: the value should be 0 for i > 0,
which it is.
Then, defining Ti as the time the process first hits i, we have
P0,i(t) = P{Ti t} P{Ti 1 
    t} = (1  et)i  (1  et)i+1 = et(1 et)i
(iii)  Probably a good suggestion. A driver who has had 2 accidents in 10 years is
less likely to have another than a driver who has had 2 accidents in a month.
(iv) The proposed model does address the issue raised in (iii) but leads to a very
high accident rate when t is close to 0, so is unsuitable.
Many candidates attempted part (i), but were unable to give a sufficiently clear description to
convincingly display their understanding and so did not earn full marks. In some cases,
Subject 103 (Stochastic Modelling) — September 2002 — Examiners’ Report
Page 5
candidates did not correctly interpret the definition of ai(t). In part (ii) many candidates tried
to solve the differential equation, where they could instead have simply shown the solution
given to be valid. Candidates generally came up with sensible suggestions for part (iii), but
only a few candidates were able to make suitable comments on part (iv).
