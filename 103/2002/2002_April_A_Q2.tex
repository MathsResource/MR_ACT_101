\documentclass[a4paper,12pt]{article}

%%%%%%%%%%%%%%%%%%%%%%%%%%%%%%%%%%%%%%%%%%%%%%%%%%%%%%%%%%%%%%%%%%%%%%%%%%%%%%%%%%%%%%%%%%%%%%%%%%%%%%%%%%%%%%%%%%%%%%%%%%%%%%%%%%%%%%%%%%%%%%%%%%%%%%%%%%%%%%%%%%%%%%%%%%%%%%%%%%%%%%%%%%%%%%%%%%%%%%%%%%%%%%%%%%%%%%%%%%%%%%%%%%%%%%%%%%%%%%%%%%%%%%%%%%%%

\usepackage{eurosym}
\usepackage{vmargin}
\usepackage{amsmath}
\usepackage{graphics}
\usepackage{epsfig}
\usepackage{enumerate}
\usepackage{multicol}
\usepackage{subfigure}
\usepackage{fancyhdr}
\usepackage{listings}
\usepackage{framed}
\usepackage{graphicx}
\usepackage{amsmath}
\usepackage{chngpage}

%\usepackage{bigints}
\usepackage{vmargin}

% left top textwidth textheight headheight

% headsep footheight footskip

\setmargins{2.0cm}{2.5cm}{16 cm}{22cm}{0.5cm}{0cm}{1cm}{1cm}

\renewcommand{\baselinestretch}{1.3}

\setcounter{MaxMatrixCols}{10}

\begin{document}
[Total 6]
2 Yt, t = 1, 2, …., is a time series defined by
Yt Yt 1 Zt 1  Zt 1  
   
where Zt, t = 0, 1, , is a sequence of independent zero-mean variables with
common variance 2 and where || < 1.
(i) State, giving your reasons, the values of p, d and q for which Y is an
ARIMA(p, d, q) process. [2]
(ii) Derive the autocorrelation function 	k, k = 0, 1, 2, . [6]

%%%%%%%%%%%%%%%%%%%%%%%%%%%%%%%%%%%%%%%%%%%%%%%%%%%%%%%%%%%%%%%%%%%%%%%%%%%%%%%%%%%%5

2 (i) Since the equation can be written (1  B) Y = (1 + (1  )B)Z, the process is
ARMA(1,1), or ARIMA(1,0,1).
(ii) There are a number of possible ways to calculate the k, and one way which
sidesteps the k and calculates the k directly. The solution presented here is
one of the possible answers; other methods were marked on their merits.
k = CovYt ,Yt k    .
Rearrange the time series equation to give Yt = Yt 1 Zt 1  Zt 1      
Now Cov[Yt, Zt] = σ2
Subject 103 (Stochastic Modelling) — September 2002 — Examiners’ Report
Page 3
and Cov[Yt, Zt1] = α.Cov[Yt1, Zt1] + Cov[Zt, Zt1] + (1 – α).Cov[Zt1, Zt1]
= α.σ2 + 0 + (1 – α).σ2 = σ2
Therefore
0 = CovYt ,Yt  = .CovYt ,Yt 1 CovYt ,Zt  1 .CovYt ,Zt 1      
= 2   2
.1    1 .
   2
0 = .1  2  . (1)
1 = CovYt ,Yt 1 
= .CovYt 1,Yt 1 CovZt ,Yt 1 1 .CovZt 1,Yt 1     
   
=   2
.0  0  1  .
   2
1 = .0  1  . (2)
substitute for γ0 from (1) into (2)
    2    2
1 = . .1  2     1 .

    2
1 2
2 . 1 .
=
1
         	

2
2
2
= 1 .
1
    
       	 

substitute for γ1 back into (1)

 
 
2 2
2
0 2
1 .
. 2 .
1
  
    

2
2
2
= 2 .
1
  
       	
For k  2,
k = CovYt ,Yt k  
= .CovYt 1,Yt k  CovZt ,Yt k  1 .CovZt 1,Yt k      
   
= . k 1 0 0     
 1
= . 1 k
k

  
Subject 103 (Stochastic Modelling) — September 2002 — Examiners’ Report
Page 4
The autocorrelation function is:
0
= k
k



. Therefore
0 =1
2
1 2
= 1
2
 


1
= 1 2 k
k k     
Part (i) was generally well answered, using a variety of different approaches. Candidates
generally made good attempts at part (ii), the main problems occurring being a failure to
correctly specify γ0 and algebraic errors in solving the simultaneous equations.
