\documentclass[a4paper,12pt]{article}

%%%%%%%%%%%%%%%%%%%%%%%%%%%%%%%%%%%%%%%%%%%%%%%%%%%%%%%%%%%%%%%%%%%%%%%%%%%%%%%%%%%%%%%%%%%%%%%%%%%%%%%%%%%%%%%%%%%%%%%%%%%%%%%%%%%%%%%%%%%%%%%%%%%%%%%%%%%%%%%%%%%%%%%%%%%%%%%%%%%%%%%%%%%%%%%%%%%%%%%%%%%%%%%%%%%%%%%%%%%%%%%%%%%%%%%%%%%%%%%%%%%%%%%%%%%%

\usepackage{eurosym}
\usepackage{vmargin}
\usepackage{amsmath}
\usepackage{graphics}
\usepackage{epsfig}
\usepackage{enumerate}
\usepackage{multicol}
\usepackage{subfigure}
\usepackage{fancyhdr}
\usepackage{listings}
\usepackage{framed}
\usepackage{graphicx}
\usepackage{amsmath}
\usepackage{chngpage}

%\usepackage{bigints}
\usepackage{vmargin}

% left top textwidth textheight headheight

% headsep footheight footskip

\setmargins{2.0cm}{2.5cm}{16 cm}{22cm}{0.5cm}{0cm}{1cm}{1cm}

\renewcommand{\baselinestretch}{1.3}

\setcounter{MaxMatrixCols}{10}

\begin{document}

1 Consider the Poisson process Xt defined as a Markov jump process with state space
{0, 1, 2, …} and transition probabilities
Pij(t) = ( )
( )!
j i
t e t
j i

 

(j  i).
(i) Prove that Xt has independent increments.
(ii) Hence prove that Xt t is a martingale with respect to the natural filtration t
associated with Xt. 
%%%%%%%%%%%%%%%%%%%%%%%%%%%%%%%%%%%%%%%%%

1 (i) Prove this using the Markov property. (Note that this mark can be earned for
use of the property even if the word “Markov” is not mentioned.)
If s0 < s1, … < sn < s < t, then
P[Xt  Xs = j
Xs0 = i0, …, Xsn = in, Xs = i]
= P[Xt = i + jXs = i] = ( ( )) ( ) ,
!
j
t s e t s
j
   
independent of i0, i1, …, in.
(ii) We need to prove that E[Xt  s ] = Xs .
With s < t we have
E[Xt  s ] = E[Xt  Xs  s ] E[Xs  s ]
= E[Xt  Xs  Xs = (t  s)  Xs .
Thus E[Xt  t  s ] = Xs  s.
The key to part (i) was to use the Markov property; only a few candidates managed to do this
part. Part (ii) was generally well answered.





