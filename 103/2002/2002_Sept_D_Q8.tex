\documentclass[a4paper,12pt]{article}

%%%%%%%%%%%%%%%%%%%%%%%%%%%%%%%%%%%%%%%%%%%%%%%%%%%%%%%%%%%%%%%%%%%%%%%%%%%%%%%%%%%%%%%%%%%%%%%%%%%%%%%%%%%%%%%%%%%%%%%%%%%%%%%%%%%%%%%%%%%%%%%%%%%%%%%%%%%%%%%%%%%%%%%%%%%%%%%%%%%%%%%%%%%%%%%%%%%%%%%%%%%%%%%%%%%%%%%%%%%%%%%%%%%%%%%%%%%%%%%%%%%%%%%%%%%%

\usepackage{eurosym}
\usepackage{vmargin}
\usepackage{amsmath}
\usepackage{graphics}
\usepackage{epsfig}
\usepackage{enumerate}
\usepackage{multicol}
\usepackage{subfigure}
\usepackage{fancyhdr}
\usepackage{listings}
\usepackage{framed}
\usepackage{graphicx}
\usepackage{amsmath}
\usepackage{chngpage}

%\usepackage{bigints}
\usepackage{vmargin}

% left top textwidth textheight headheight

% headsep footheight footskip

\setmargins{2.0cm}{2.5cm}{16 cm}{22cm}{0.5cm}{0cm}{1cm}{1cm}

\renewcommand{\baselinestretch}{1.3}

\setcounter{MaxMatrixCols}{10}

\begin{document}
\begin{enumerate}

8 An analyst is investigating the extent to which the price of a stock at the beginning and end of a time interval can be used to provide information about its price during
the interval. The model used is
St = S0 + t + Bt,
where Bt is a standard Brownian motion. The analyst wishes to investigate the difference between St and the price Sˆt which would be predicted given only S0 and ST, given by
ˆSt = (T t)S0 tST , 0 t T.
T
 
 
\begin{enumerate}
\item (i) Write down E(St  S0 ), Var(St  S0), E(ST St ),  Var(ST St) and
Cov(St  S0, ST  St), for 0 
 t 
 T. 
\item (ii) Calculate the expectation and variance of St  Sˆt . 
\item (iii) Find the value of t  [0, T] where Var(St  Sˆt ) is greatest. Comment on your
answer. 

\item (iv) Give two reasons why the model
St = S0exp(t + Bt)
might be more suitable than the one used by the analyst and discuss whether the results of (iii) might have been substantially different if the analyst had used this model in the first place. 
\end{enumerate}

%%%%%%%%%%%%

8 (i) E(St  S0 ) = t, Var(St  S0 ) =	
2t.
E(ST St )  = (T  t), Var(ST St )  = 	2(T  t).
Cov(St  S0, ST  St) = 0, because of the independent increment property of
Brownian motion.

%-------------------------------------------------------%
(ii) The expectation is t  (t/T)T = 0.
Var(St  Sˆt ) = Var ( t 0 ) ( T t )
T t S S t S S
T T
    
      
   
=
2 2 2
T t 2t t 2 (T t) = t(T t) .
T T T
      
       
 	  	

%-------------------------------------------------------%
(iii) The maximum of t(T  t) is attained at t = ½T.
This is not surprising. The graph of St  ˆSt against t is “tied down” at the ends,
as the function is constrained to be equal to zero. The greatest scope for
variation is bound to be in the middle.
%-------------------------------------------------------%
(iv) Two possible reasons might be that a Brownian motion can become negative,
which a stock price cannot, and that fluctuations in the value of a stock price
are usually proportional to the price. Other reasons could also apply.
Under the revised model ln(St) has the same structure as St in the original
model, so ln(St) will have its greatest variability at ½T. The result will not
differ greatly from the result above.

%%%%%%%%%%%%%%%%%%%%%%%%%%%%%%%%%%%%%%%%%%%%%%%%%%%%%%%%%%%%%%%%%%%%%%%%%%%%%%
\newpage

Candidates generally gave good answers for part (i). However, answers to part (ii) and (iii)
were disappointing, particularly for part (iii) where the answer can be derived very simply
from general reasoning. Part (iv) in general showed better answers, although many
candidates failed to get full marks because they did not discuss how their response to (iii)
would differ under the new model.
Subject 103 (Stochastic Modelling) — September 2002 — Examiners’ Report
Page 10
