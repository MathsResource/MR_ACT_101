
10 The ticket office at a train station has a single ticket machine that is used by travellers
to purchase tickets.
The machine has a tendency to break down, at which point it must be repaired. The
time until breakdown and the time required to effect repairs both follow the
exponential distribution.
Let P1i(t), i = 0, 1, be the probability that at time t (t > 0) there are i ticket machines
working at the ticket office, given that the ticket machine is working at time t = 0.
(i) Derive the Kolmogorov forward differential equations for P1i(t) i = 0, 1 in
terms of:
  , where 1/ is the mean time to breakdown for a machine; and
  ρ, where 1/ρ is the mean time to repair a machine [3]
(ii) Show that     P10 t = 1 e t    


deduce the value of P11(t). [4]
(iii) The station manager is considering adding a second identical ticket machine,
though there is only one repair team to work on the machines in the event that
both are out of action simultaneously. Assuming that a second machine is
added and operates independently of the first one:
(a) Write down the generator matrix of the Markov jump process Xt which
counts the number of working ticket machines at time t.
(b) Derive the Kolmogorov forward differential equations for pi(t), i = 0,
1, 2, the probability that i ticket machines are working.
(c) Given that, for some t,
2
0 2 2
( ) = 2 ,
2 2
p t 
  
2
1 2 2 2 2 2
( ) = 2 and ( ) = ,
2 2 2 2
p t p t  
     
show that i ( )
d p t
dt
=0 for i = 0, 1, 2.
(d) State what conclusions you draw from part (c). [6]
[Total 13]


10 (i) P10 t  h = 1 h.P10 t   .h.P11 t 
 10   = . 10   . 11   d P t P t P t
dt   (1)
Subject 103 (Stochastic Modelling) — September 2002 — Examiners’ Report
Page 11
and
P11 t  h = .h.P10 t   1 .h.P11 t 
 11   = . 10   . 11   d P t P t P t
dt   (2)
[(2) also follows from the fact that P10 t   P11 t  =1.]
(ii) P10 t   P11 t  =1
so from (1) d P10 t   .P10 t  =
dt   

     
10 = . d e tP t e t C
dt
 
 
      
10 = . . P t e t C e t    


     P10 t .1 e t    
 

since P10(0) = 0.
[Alternatively, instead of solving the DE, just verify that the function
proposed as the solution does indeed satisfy the DE and also check that the
value is correct at t = 0.]
Therefore
   
 
11( ) =1 . 1 =
t
P t e t e
 
     
 
 
.
(iii) (a) The generator matrix is now
0
( )
0 2 2
   
         
    	 

(b) Hence the Forward Equations are
0   = . 0   . 1   d p t p t p t
dt  
1   = . 0   1   2. . 2 ( ) d p t p p t p t
dt     
Subject 103 (Stochastic Modelling) — September 2002 — Examiners’ Report
Page 12
2   = . 1   2 . 2   d p t p t p t
dt   
(c) Simply substituting in the suggested values gives the required result.
(d) The implication is that the given distribution is stationary. By the
standard properties of Markov processes, it follows that it is the
equilibrium distribution, so that the long-term probabilities of being in
each of the three states are known.
Answers to this question were on the whole disappointing.
Problems in part (i) included transposing the parameters. Marks were available for deriving
p11 from p10 (using the information given in the question) even where p10 could not be
correctly identified.
Only a small proportion of candidates managed to provide a solution for the differential
equation in part (ii).
Better answers were given for the generator matrix in part (iii).
Candidates that gave a generator matrix by and large were able to score marks by applying
their matrix to produce the forward equations. Very few candidates provided any
conclusions under part (iv).
