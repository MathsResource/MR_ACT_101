\documentclass[a4paper,12pt]{article}

%%%%%%%%%%%%%%%%%%%%%%%%%%%%%%%%%%%%%%%%%%%%%%%%%%%%%%%%%%%%%%%%%%%%%%%%%%%%%%%%%%%%%%%%%%%%%%%%%%%%%%%%%%%%%%%%%%%%%%%%%%%%%%%%%%%%%%%%%%%%%%%%%%%%%%%%%%%%%%%%%%%%%%%%%%%%%%%%%%%%%%%%%%%%%%%%%%%%%%%%%%%%%%%%%%%%%%%%%%%%%%%%%%%%%%%%%%%%%%%%%%%%%%%%%%%%

\usepackage{eurosym}
\usepackage{vmargin}
\usepackage{amsmath}
\usepackage{graphics}
\usepackage{epsfig}
\usepackage{enumerate}
\usepackage{multicol}
\usepackage{subfigure}
\usepackage{fancyhdr}
\usepackage{listings}
\usepackage{framed}
\usepackage{graphicx}
\usepackage{amsmath}
\usepackage{chngpage}

%\usepackage{bigints}
\usepackage{vmargin}

% left top textwidth textheight headheight

% headsep footheight footskip

\setmargins{2.0cm}{2.5cm}{16 cm}{22cm}{0.5cm}{0cm}{1cm}{1cm}

\renewcommand{\baselinestretch}{1.3}

\setcounter{MaxMatrixCols}{10}

\begin{document}
\begin{enumerate}6 The members of a disability insurance scheme are classified as “active” (A),
“temporarily disabled” (T), “permanently disabled” (P) or “dead” (D). Members are
entitled to benefits when they are in state T or P. For the purpose of analysis the state
of each member is recorded on 1 April each year. It is found that the history of a
typical member evolves in time as a discrete-time Markov chain with transition matrix
A
T
P
D
0.75 0.1 0.05 0.1
0.5 0.3 0.1 0.1
0 0 0.8 0.2
0 0 0 1
A T P D
 
 
 
 
 
 
.
(i) Draw the transition graph of the chain and find its stationary probability
distribution. [3]
(ii) Calculate the mean duration (in years) of a permanent disability benefit. [2]
(iii) Calculate the probability that a member, initially active, is either temporarily
or permanently disabled three years after the start of the scheme. [3]
(iv) Calculate the probability that a member, initially active, will never draw any
benefit from the scheme. [2]
[Total 10]
103 S2002—5 PLEASE TURN OVER
7 (i) Describe the three elements of a linear congruential generator, and set out the
recursive relationship used to generate the pseudo-random number sequence.
[3]
(ii) The first three numbers produced by a linear congruential generator are 0.954,
0.462 and 0.628. Use these to generate three pseudo-random numbers from
the Pareto distribution with  = 2 and  = 1.
[The density of the Pareto distribution is f(x) = 1 ( 0).]
( )
x
x




 
[4]
(iii) An exponential random variable T with rate parameter  may be simulated
using the formula
T =  1 logU

where U is uniformly distributed on [0, 1]. Explain how this can be used to
simulate a path of a Markov jump process {Xt: t  0} which has two states,
H (healthy) and S (sick), with transition rates  from H to S and 	 from S to H.
Assume X0 = H. [4]
[Total 11]
103 S2002—6



6 (i)
(For full credit the probabilities should all be included on the diagram.)
Stationary distribution A = T = P = 0, D = 1. Derivation not required: the
answer is obvious to anyone who understands.
(ii) The duration of a permanent disability benefit is a geometric r.v., T.
Since P[Xn 1 = P Xn = P]   = 0.8, the parameter is 0.2:
P[T = n] = (0.8)n10.2. Accordingly its mean is 1
1 0.8 
= 5.
(iii) The required probability is obtained from (1 0 0 0)P3.
(1 0 0 0)P = (.75 .1 .05 .1); (1 0 0 0)P2 = (.6125 .105 .0875 .195);
(1 0 0 0)P3 = (.511875 .09275 .111125 .28425).
The solution is .09275 + .111125 = .203875.
(The summation does not have to be performed to earn the mark: giving the
two probabilities separately is a reasonable interpretation of the question.)
(iv) The probability of never visiting T or P starting from A is
0.1 + 0.75  0.1 + (0.75)2  0.1 + …
Subject 103 (Stochastic Modelling) — September 2002 — Examiners’ Report
Page 8
=
0
0.1 (0.75) = 0.1 = 0.4.
1 0.75
n
n




Most candidates correctly gave the transition graph in part (i), although there were a few
cases where the transition probabilities were not included in the solution. Part (iii) was well
answered. Solutions to (ii) and (iv) were mixed, with perhaps less than a third of students
immediately recognising the required approach and thereby gaining full marks.
7 (i) The three elements are:
  the multiplier — usually denoted a
  the increment — usually denoted c; the increment is often set to zero
(without any loss in the quality of the pseudo-random sequence) to speed
up the generation process
  the modulus — usually denoted m — where m > a and m > c; the
generator will produce a series of pseudo-random numbers with period no
more than m, so the modulus is usually set to as high a number as possible.
The recursive relationship is: Xn+1 = (aXn + c) (mod m) for n = 0, 1, 2, …,
N  1. We then set xk = Xk/m for each k = 1, 2, …, N.
(ii) Use the inverse transform method:
f(x) = 1 3
= 2
( x) (1 x)



 
, so that F(x) = 1 
1 2
1 x
 
    
 x = 1 1.
1 F(x) 

, so
  if F(x) = 0.954, then x = 3.6625
  if F(x) = 0.462, then x = 0.3634
  if F(x) = 0.628, then x = 0.6396
(iii) The structure of a Markov jump process implies that the time until the next
jump has exponential distribution, with rate 	 if the current state is H, 
 if S.
The even-numbered inter-jump times will have one distribution, the oddnumbered
ones a different one.
Obtain numbers y0, y2, y4, …, y2n by the above procedure, being simulated
outcomes of independent random variables Y0, Y2, Y4, …, Y2n exponentially
distributed with parameter 	. Similarly, obtain y1, y3, …, y2n1 using a
parameter 
 instead of 	.
Subject 103 (Stochastic Modelling) — September 2002 — Examiners’ Report
Page 9
Put tj = y0 + y1 + … + yj. Find j such that tj1  t < tj and return
xt =
if is even
if is odd
H j
S j

Candidates made reasonable attempts at part (i), although in many cases insufficient detail
was provided to score full marks. Part (ii) was generally well answered, the only problem
here being simple algebraic errors. Part (iii) was not answered well, with candidates on the
whole failing to recognise that the transitions from healthy to sick and from sick to healthy
should be modelled separately and then combined to provide the required process.
