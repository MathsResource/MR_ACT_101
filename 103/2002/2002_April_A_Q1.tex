\documentclass[a4paper,1pt]{article}

%%%%%%%%%%%%%%%%%%%%%%%%%%%%%%%%%%%%%%%%%%%%%%%%%%%%%%%%%%%%%%%%%%%%%%%%%%%%%%%%%%%%%%%%%%%%%%%%%%%%%%%%%%%%%%%%%%%%%%%%%%%%%%%%%%%%%%%%%%%%%%%%%%%%%%%%%%%%%%%%%%%%%%%%%%%%%%%%%%%%%%%%%%%%%%%%%%%%%%%%%%%%%%%%%%%%%%%%%%%%%%%%%%%%%%%%%%%%%%%%%%%%%%%%%%%%

\usepackage{eurosym}
\usepackage{vmargin}
\usepackage{amsmath}
\usepackage{graphics}
\usepackage{epsfig}
\usepackage{enumerate}
\usepackage{multicol}
\usepackage{subfigure}
\usepackage{fancyhdr}
\usepackage{listings}
\usepackage{framed}
\usepackage{graphicx}
\usepackage{amsmath}
\usepackage{chngpage}

%\usepackage{bigints}
\usepackage{vmargin}

% left top textwidth textheight headheight

% headsep footheight footskip

\setmargins{.0cm}{.5cm}{16 cm}{cm}{0.5cm}{0cm}{1cm}{1cm}

\renewcommand{\baselinestretch}{1.}

\setcounter{MaxMatrixCols}{10}

\begin{document}

\begin{enumerate}
\item
ã Institute of Actuaries1
A stock price {S t : t = 1, , 1⁄4} is modelled as
æ t
ö
S t = S 0 exp ç å X j ÷ ,
ç j = 1 ÷
è
ø
where X 1 , X  , 1⁄4 is a sequence of independent Normal random variables, with $E(X j ) = m j , Var(X j ) = s  j$ .
In order to perform a martingale analysis it is necessary to find a sequence of constants a 1 , a  , 1⁄4 such that Y t = a t S t is a martingale.
%-----------------------------------------------------------------%
\begin{enumerte}[(a)]
\item 
Derive a recurrence equation satisfied by the constants { a t : t = 1, , 1⁄4 }.
[You may use the fact that the moment generating function of N( m , s  ) is
M(s) = exp( m s + 1⁄ s  s  ).]

\item State, with a brief explanation, whether the same answer would be obtained if

X j had a non-normal distribution with mean m j , variance s  j .
\end{enumerate}

An analyst wishes to use a model which is based on Brownian motion, but which does not become too large and positive for large t. The model proposed is
\[X t = B t e - cB t ,\\]
where $B_t$ is standard Brownian motion, and c is a posi

%%%%%%%%%%%%%%%%%%%%%%%%%%%%%%%%%%%%%%%%%%%%%%%%%%%%%%%%%%%%%%%%%%%%%%%%%%%%%%%%%%%%%%%%%%%
\newpage
1
(i)
\begin{itemize}
\item Y k is a martingale, and the values Y 1 , ..., Y k are determined by X 1 , ....., X k , so
\[E[Y k+1 1⁄X 1 , X  , ..., X k ] = E[Y k+1 1⁄Y 1 , Y  , ..., Y k ] = Y k .\]
\item Now
k + 1
é
ù
E[Y k+1 |X 1 , X  , ....., X k ] = E ê a k + 1 .exp å X j | X 1 , X  ,..., X k ú
ê ë
ú û
j = 1
k
é
ù
= a k+1 . E ê e X k + 1 .exp å X j | X 1 , X  ,..., X k ú
ê ë
ú û
j = 1
Y
= a k + 1 . E é e X k + 1 ù . k
ë
û a
k
\item 
We therefore require
a k + 1 =
(ii)
a k
1
é
ù
= a k exp ê -m k + 1 - s  k + 1 ú
\end{itemize}
%%%%%%%%%%%%%%%%%%%%%%%%%%%%5

ë
û
E é e X k + 1 ù
ë
û
The solution depends critically on the moment generating function of a Normal variable. Non-normal variables have different mgfs, so the answer obtained would be different.
Answers to Question 1 were disappointing considering how straightforward it was, suggesting that candidates lacked practice at applications of martingales.

%--------------------------------------------------------%
(i)
d
( be - cb ) = (1 - cb)e - cb = 0 when b = 1/c.
db
Check that this is a maximum:
d 

( be - cb ) = ( -  c + c  b ) e - cb = - ce - 1 when
db
b = 1/c. This is negative, so X is indeed maximised at b = 1/c, giving a
maximum value of e - 1 /c.
(ii)
Itô's Lemma has a number of possible forms:
df ( X t ) = f '( X t ) dX t + 1⁄ f "( X t )( dX t )  , or
¶ f
¶ f
¶  f
df ( X t , t ) = ( X t , t ) dt + ( X t , t ) dX t + 1⁄  ( X t , t )( dX t ) 
¶ t
¶ x
¶ x

are OK, with ( dX t ) = dt.
%%---- Page Subject 10 (Stochastic Modelling) — 
%%%%%%%%%%%%%%%%%%%%%%%%%%%%%%%%%%%%%

d 
db

( be - cb ) = ( - c + c  b)e - cb . Therefore
\[dX t = (1 - cB t ) e - cB t dB t + 1⁄( - c + c  B t ) e - cB t dt .\]
(iii)
Quite the reverse. If B is large and negative X takes enormously negative values. X is not in the least stationary.
A common cause of lost marks was failing to check that the turning point was a maximum. tive constant.
(i) Verify that there is an upper bound which X never exceeds.
[]
(ii) Use Itô’s Lemma to find dX t .
[]
(iii) State, with a brief explanation, whether the suggested model is appropriate for
a process which is asymptotically stationary.
[1]
[Total 6]

\end{document}
