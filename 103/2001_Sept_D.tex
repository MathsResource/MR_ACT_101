103 S2001—35
A motor insurance company assumes that a holder of a provisional driver’s
licence will make claims according to a Poisson process with rate X per year,
where X is not fixed but is determined randomly for each driver according to the
density function
f(x) = 2e −2x
6
(x > 0).
(i) Describe how to simulate an observation X from the density f using a
single pseudo-random variable U assumed uniformly distributed on [0, 1].
[3]
(ii) Explain how, given the value X generated in (i), you would use a sequence
U 1 , U 2 , ... of uniform pseudo-random variables to simulate the number of
claims made in two six-month periods by a provisional driver with mean
claim rate X per year.
[4]
(iii) Describe a simulation-based method for estimating the conditional
probability that a provisional driver makes 2 or more claims in the second
six months of driving given that no claim was made in the first six
months. [Here the value X is to be assumed unknown.]

[Total 9]
n
The evolution of a stock price is modelled as a discrete time process S n = Σ i=
1 X i ,
where X 1 , X 2 , ... are independent, identically distributed random variables with
P{X i = 1} = p and P{X i = −1} = q = 1 − p. The investment will be liquidated at
either the bankruptcy time T 0 (the first time n when the price S n hits 0) or the
first time T K when the price attains a fixed target K, whichever occurs first.
Let T = min(T 0 , T K ) denote the liquidation time (the exit time from [0, K]). Let
A = {T K < T 0 } denote the event that the target is met before bankruptcy and let
p k = P[AS 0 = k] denote the probability of this event, given that the initial price is
S 0 = k, where k ∈ {1, ..., K − 1}.
(i) By conditioning on the price of the stock at time 1, determine a difference
equation satisfied by p k , k = 1, ..., K − 1.

(ii) Assume that p = q =
(iii)
1
2
.
(a) Show that S n is a martingale.
(b) Derive an expression for p k by applying the optional stopping
theorem to this martingale stopped at T.

Assume now that p \neq q.
(a)
(b)
103 S2001—4
Determine a value \theta \neq 1 such that Y n = \theta S n is a martingale.
[4]
Derive an expression for p k in this case.
[Total 11]7
According to an interest rate model which operates in continuous time, the
interest rate r(t) may change only by upward jumps of fixed size j u or by
downward jumps of fixed size j d (where j d < 0), occurring independently according
to Poisson processes N u (t) (with rate \lambda u ) and N d (t) (with rate \lambda d ).
Let T u denote the time of the first up jump in the interest rate, T d the time of the
first down jump, T = min(T u , T d ) the time of the first jump. Further, let I be
defined as an indicator taking the value 1 if the first jump is an up jump or 0
otherwise.
(i) Determine expressions for the probabilities P{T u > t}, P{T d > t} and P{T > t}

(ii) Determine the distribution of I.

(iii) Show, by evaluating P{T > t and I = 1}, that I and T are independent
random variables.
[3]
(iv) Calculate the expectation and variance of the interest rate at time t given
the current rate r(0).
Hint: r(t) = r(0) + j u N u (t) + j d N d (t).
(v)
Show that {r(t) : t ≥ 0} is a process with stationary, independent
increments.
103 S2001—5

[3]
[Total 12]8
A company keeps records of quarterly sales figures, {S t : t = 1, 2, ...n} for the most
recent n quarters. It wishes to analyse the records with the aim of predicting the
sales figures in the near future.
The model suggested by the company is:
log S t = \mu + \betat + \theta Q ( t ) + X t ,
where {X t : t = 1, 2, ..., n} is a stationary time series, Q(t) takes the value 1, 2, 3 or
4 depending on whether the tth quarter is the first, second, third or fourth
quarter of the financial year, and \theta 1 + \theta 2 + \theta 3 + \theta 4 = 0.
(i) Explain why the company has suggested a linear model for log S t rather
than a linear model for S t .
[1]
(ii) Explain the significance of the parameters \mu, \beta and {\theta q : 1 ≤ q ≤ 4} and give
[4]
a reason for the assumption that \theta 1 + \theta 2 + \theta 3 + \theta 4 = 0.
(iii) Derive a linear filter Y t = Σ + k 2= − 2 a k log S t + k which has the property that the
filtered series {Y t } does not depend on {\theta q : 1 ≤ q ≤ 4}.
(iv)
7
q k - 1
q K - 1
.
(i) P{T u > t} = e -l u t , P{T d > t} = e -l d t , P{T > t} = P{T u > t} P{T d > t} = e - ( l u +l d ) t
(ii) P{I = 1} = P{T d > T u } = ò 0 ¥ l u e -l u t P{T d > t} dt =
(iii) P{T > t Ç I = 1} = ò 0 ¥ l u e -l u t P{T d > t} dt =
l u
l u + l d
l u
l u + l d
e - ( l u +l d ) t
This is the product of P{T > t} and P{I = 1}.
8
(iv) Er t = r 0 + (j u l u + j d l d ) t and Var r t = j u 2 l u t + j d 2 l d t
(v) Take 0 < s < t. Then r t+s - r t = j u (N u (t + s) - N u (t)) + j d (N d (t + s) - N d (t)).
Now N u (t + s) - N u (t) has a Poisson(l u s) distribution, not depending on t
and independent of {N u (r) : 0 \leq r \leq s}, and N d (t + s) - N d (t) has a
Poisson(l d s) distribution, also independent of t and of {N d (r) : 0 \leq r \leq s}, so
r t+s - r t has the same distribution as r s - r 0 and is independent of r s - r 0 .
(i) Fluctuations in sales tend to be proportional to sales in the sense that
Var(S t+1 -S t ) \mu S t 2 , company growth tends to be exponential rather than
linear. (Either of these explanations is sufficient.)
(ii) m + bt represents a deterministic linear trend in the main sales volume; b
is related to the average annual percentage increase in sales, whereas m is
related to the initial value.
The q q refer to predictable seasonal fluctuations above and below the
average from one quarter to another due to the weather, timing of
Christmas, etc.
Page 5Subject 103 (Stochastic Modelling) — 
%%%%%%%%%%%%%%%%%%%%%%%%%%%%%%%%

We assume that Sq q = 0 because any non-zero value could be subsumed in
m; estimation procedures report an indeterminacy if we do not make this
assumption.
(ii)
We need to ensure that each q q has a coefficient of
1
4 (assuming that the
1
8 X t+2 .
filter coefficients add to 1).
The filter
(iv)
( 1 8 , 1 4 , 1 4 , 1 4 , 1 8 ) will do.
We have
Y t = m + bt +
1
8
X t - 2 +
1
4
X t - 1 +
1
4
X t +
1
4
X t+1 +
Hence
ÑY t = Y t - Y t - 1 = b +
1
8
(X t+2 + X t+1 - X t - 2 - X t - 3 ) .
Clearly Y is not stationary, if only because it has a trend in the mean.
ÑY, however, does look stationary, so it is reasonable to claim that Y is
I(1).
