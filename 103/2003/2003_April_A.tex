\documentclass[a4paper,12pt]{article}

%%%%%%%%%%%%%%%%%%%%%%%%%%%%%%%%%%%%%%%%%%%%%%%%%%%%%%%%%%%%%%%%%%%%%%%%%%%%%%%%%%%%%%%%%%%%%%%%%%%%%%%%%%%%%%%%%%%%%%%%%%%%%%%%%%%%%%%%%%%%%%%%%%%%%%%%%%%%%%%%%%%%%%%%%%%%%%%%%%%%%%%%%%%%%%%%%%%%%%%%%%%%%%%%%%%%%%%%%%%%%%%%%%%%%%%%%%%%%%%%%%%%%%%%%%%%

\usepackage{eurosym}
\usepackage{vmargin}
\usepackage{amsmath}
\usepackage{graphics}
\usepackage{epsfig}
\usepackage{enumerate}
\usepackage{multicol}
\usepackage{subfigure}
\usepackage{fancyhdr}
\usepackage{listings}
\usepackage{framed}
\usepackage{graphicx}
\usepackage{amsmath}
\usepackage{chngpage}

%\usepackage{bigints}
\usepackage{vmargin}

% left top textwidth textheight headheight

% headsep footheight footskip

\setmargins{2.0cm}{2.5cm}{16 cm}{22cm}{0.5cm}{0cm}{1cm}{1cm}

\renewcommand{\baselinestretch}{1.3}

\setcounter{MaxMatrixCols}{10}

\begin{document}


1 A science student is observing a mouse in the following simple maze:
4
2
1 5
3
6
During each period under observation, the mouse moves to any adjacent, accessible
compartment with equal probability. Successive moves have the Markov property.
Let Xt be the compartment number that the mouse is in at time t.
\begin{enumerate}
    \item (i) Draw a transition diagram, including on your diagram the probabilities of the
possible transitions. 
    \item (ii) Evaluate the probability that a mouse which starts in compartment 1 at time 0
is also in compartment 1 (a) at time 1, (b) at time 2, and (c) at time 3. 
    \item (iii) State with reasons whether:
(a) the Markov chain X possesses a stationary distribution
(b) P(Xt = j | X0 = i) converges to some limit j as t   
[Total 6]
\end{enumerate}


%%%%%%%%%%%%%%%%%%%%%%%%%%%%%%%%%%%%%%%%%%%%

1 (i) Transition diagram:
(ii) (a) and (c) are both zero, as it is not possible to return to 1 in an odd number of
steps.
For (b): (2)
1,1 12 21 13 31
= = 1 1 = 1 .
4 4 2
p p p  p p 
(iii) (a) Yes every chain with finite state space has a stationary distribution.
(b) No. The chain is periodic, so the probabilities do not converge.

% It was possible to read the question as implying the possibility that the mouse could stay where it was: candidates who did this were still able to obtain full marks for the question.
% Every candidate was able to draw the diagram correctly and most of them correctly evaluated the required probabilities.
% Some candidates were not clear about the distinction between a stationary distribution and a limiting distribution.

\end{document}
