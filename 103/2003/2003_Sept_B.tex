\documentclass[a4paper,12pt]{article}

%%%%%%%%%%%%%%%%%%%%%%%%%%%%%%%%%%%%%%%%%%%%%%%%%%%%%%%%%%%%%%%%%%%%%%%%%%%%%%%%%%%%%%%%%%%%%%%%%%%%%%%%%%%%%%%%%%%%%%%%%%%%%%%%%%%%%%%%%%%%%%%%%%%%%%%%%%%%%%%%%%%%%%%%%%%%%%%%%%%%%%%%%%%%%%%%%%%%%%%%%%%%%%%%%%%%%%%%%%%%%%%%%%%%%%%%%%%%%%%%%%%%%%%%%%%%

\usepackage{eurosym}
\usepackage{vmargin}
\usepackage{amsmath}
\usepackage{graphics}
\usepackage{epsfig}
\usepackage{enumerate}
\usepackage{multicol}
\usepackage{subfigure}
\usepackage{fancyhdr}
\usepackage{listings}
\usepackage{framed}
\usepackage{graphicx}
\usepackage{amsmath}
\usepackage{chngpage}

%\usepackage{bigints}
\usepackage{vmargin}

% left top textwidth textheight headheight

% headsep footheight footskip

\setmargins{2.0cm}{2.5cm}{16 cm}{22cm}{0.5cm}{0cm}{1cm}{1cm}

\renewcommand{\baselinestretch}{1.3}

\setcounter{MaxMatrixCols}{10}

\begin{document}
\begin{enumerate}
%%%%%%%%%%%%%%%%%%%%%%%%%%%%%%%%%%%%%%%%%%%%%%%%%%%%%%%%%%%%%%%%%%
103 S2003—3 PLEASE TURN OVER
4 Assume that the yield rate Yt of a certain stock may either take a high constant value
yu or a low constant value yd, during alternating random exponential periods with
respective intensities u, d.
(i) Write down Kolmogorov’s forward equation for the probability Pu,u(t) that Y
is in state yu at time t, given that it starts in state yu. [1]
(ii) Show that
Pu,u = d u ( u d )t .
u d u d
e    

   
[2]
(iii) Let Ut denote the total amount of time spent in state yu up until time t. Derive
and expression for E[UtY(0) = yu], the expected occupation time in state yu
by time t for the two-state continuous-time Markov chain starting in state yu.
Hint: Note that Ut =
0
t
 Isds , where Is =
1 if =
0 if
s u
s u
Y y
Y y

 
and use the previous
result. [2]
(iv) Derive an expression for the expected total yield Xt of the stock by time t.
Hint: Xt = yuUt + yd(t  Ut). [2]
[Total 7]

5 Let X1, X2, X3, … be independent, identically distributed random variables with
P(X1 = +1) = p > ½ and P(X1 = 1) = 1  p < ½. Let S0 = 0, Sn = X1 + X2 + … + Xn
for n  1 be the associated random walk.
(i) State a necessary condition which must be satisfied by the constants 	 and c if
the process
Mn = e Sn cn  
is to be a martingale. [1]
(ii) Use the condition in (i) above to solve for 	 as a function of c. [2]
(iii) Let T1 be the first time that Sn hits 1, i.e. T1 = min{n : Sn = 1}. State the
conditions on c and on 	(c) under which it is valid to use the Optional
Stopping Theorem to evaluate 1 ( ) T E M. [2]
(iv) Derive the moment generating function E(e cT1 )  for c > 0. [3]
[Total 8]
103 S2003—4#

%%%%%%%%%%%%%


4 (i) Pu,u (t) uPu,u (t) d Pu,d (t) .
(ii) Note that Pu,d (t) 1 Pu,u (t) . Therefore we have
Pu,u (t) d ( u d )Pu,u (t) ,
implying that
( ) ( )
, ( ) u d t u d t
u u d
d
e P t e
dt
.
Together with the boundary condition Pu,u(0) = 1, this gives the required
solution.
(iii)
0 0 0 0 0 0 ,
| | | ( )
t t t
E Ut Y yu E Is Y yu ds P Ys yu Y yu ds Pu u s ds .
Applying the previous part, this is equal to
( )
2 1
( )
d u u d t
u d u d
t e .
(iv) E Xt |Y0 yu = ydt ( yu yd )E Ut |Y0 yu
= ( )
2
( )
( ) (1 ).
( )
d u u d u d t
d u d
d u u d
y y
y y y t e
Subject 103 (Stochastic Modelling) September 2003 Examiners Report
Page 6
In general candidates were able to score well on the first two parts, although a common mistake here
was to have the exponential parameters d and u transposed in the Kolmogorov Forward Equations.
In such cases appropriate credit was given for valid attempts at part (ii).
5 (i) Mn = e Sn cn will be a martingale if ( Sn 1 c(n 1) ) = Sn cn.
E e Fn e
(ii) This will happen if ec = Ee Xn 1 = pe qe , i.e. if pe2 e +c + q = 0.
Therefore
=
2 4
ln .
2
ec e c pq
p
(iii) For the OST to hold we require that M is bounded or T1 is bounded or
n T1 M
is bounded.
In this instance, if c 0 and 0 then Mn < e for all n T1. But if c < 0 or
0 then there is no such upper bound and it is not safe to assume that the
OST can be applied.
(iv) When c > 0, one root for is positive, the other negative, since
pe2 e +c + q < 0 when = 0. We need the positive root.
Applying the OST, E(MT) = M0 = 1 as long as c 0 and 0. This implies
that
1 = ( ST1 cT1 ) E e = e E(e cT1 ).
Thus
E(e cT1 ) = e =
2
2
c c 4
p
e e pq
=
2 4
.
2
ec e c pq
q
In a number of cases, candidates covered some of part (ii) under part (i) credit was given in these
cases. In general candidates were able to score well on the bookwork required for part (iii) although
candidates were less successful in tackling the final part.
Subject 103 (Stochastic Modelling) September 2003 Examiners Report
Page 7
