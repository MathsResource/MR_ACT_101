\documentclass[a4paper,12pt]{article}

%%%%%%%%%%%%%%%%%%%%%%%%%%%%%%%%%%%%%%%%%%%%%%%%%%%%%%%%%%%%%%%%%%%%%%%%%%%%%%%%%%%%%%%%%%%%%%%%%%%%%%%%%%%%%%%%%%%%%%%%%%%%%%%%%%%%%%%%%%%%%%%%%%%%%%%%%%%%%%%%%%%%%%%%%%%%%%%%%%%%%%%%%%%%%%%%%%%%%%%%%%%%%%%%%%%%%%%%%%%%%%%%%%%%%%%%%%%%%%%%%%%%%%%%%%%%

\usepackage{eurosym}
\usepackage{vmargin}
\usepackage{amsmath}
\usepackage{graphics}
\usepackage{epsfig}
\usepackage{enumerate}
\usepackage{multicol}
\usepackage{subfigure}
\usepackage{fancyhdr}
\usepackage{listings}
\usepackage{framed}
\usepackage{graphicx}
\usepackage{amsmath}
\usepackage{chngpage}

%\usepackage{bigints}
\usepackage{vmargin}

% left top textwidth textheight headheight

% headsep footheight footskip

\setmargins{2.0cm}{2.5cm}{16 cm}{22cm}{0.5cm}{0cm}{1cm}{1cm}

\renewcommand{\baselinestretch}{1.3}

\setcounter{MaxMatrixCols}{10}

\begin{document}
[Total 8]
\begin{enumerate}
\item 5 (i) Let W(t) be defined by W(t) = 	4B(kt), where B(t) is a standard Brownian
motion.
(a) Calculate the value of k which gives W(t) the same expectation and
covariance function as B(t).
(b) Prove that, for this value of k, W is a standard Brownian motion.

\item 
(ii) Prove that exp[2B(t)	2t] is a martingale. 
\end{enumerate}

%%%%%%%%%%%%%%%%%%%%%%%%%%%%%%%%%%%%%%%%%%%%%%%%%%%%%%%%%%%%%%%%%%%%%%%%%%%%%%%%%%%%%%%%%%%%
\newpage

5
\begin{itemize}
\item (i) (a) EW(t) = 0, Cov(W(s), W(t)) = 16Cov(B(ks), B(kt)) = 16k min(s, t).
Thus k = 1/16 is the required value.
To prove that a process is a BM, it is necessary to check the
covariance, not just the variance.
\item (b) In addition to the expectation and covariance function, we need to show
  that W is Normally distributed. Any linear transformation of a Normal random variable is itself Normal.
  that W has continuous sample paths. For small h, we have W(t + h)  W(t) =  4(B(kt + kh)  B(kt)), which clearly tends to 0
as h approaches 0.
Subject 103 (Stochastic Modelling) — April 2003 — Examiners’ Report
Page 5
  An alternative to proving continuity is to state that the increments
of W are independent of past values and are also Normally
distributed.
\item (ii) Consider exp[2B(t)2t] with respect to the filtration Ft.
B(t) – B(s) is independent of Ft and B(s) is Fs-measurable. Then  2B(t) |  =  2[B(t) B(s)] 2B(s) | 
E e Fs E e e Fs 
= 2B(s)  2[B(t) B(s)] | 
e E e Fs 
= e2B(s)Ee2[B(t) B(s)]  
\item The increment B(t) – B(s) has the normal distribution with mean 0 and
variance t 	s, so the expectation of e2[B(t) – B(s)] is equal to M(2) = exp(2(t  s)), where M(
) is the moment generating function of the N(0, t  s) distribution.
\item It follows that
 2B(t) 2t |  = 2B(s) 2s
E e Fs e  
and therefore e2[B(t) t] is a martingale.
\end{itemize}
% For part (i), in (a) many candidates just checked variance rather than the covariance, but got 1/16 correct. Not so many people got part (b)  generally people assumed that (a) implied (b), rather than considering other properties of Brownian motion
% Most candidates fared well on part (ii), identifying where necessary the mgf of a normal random variable to successfully demonstrate the given process is a martingale.
% Subject 103 (Stochastic Modelling) — April 2003 — Examiners’ Report
% Page 6
\end{document}
