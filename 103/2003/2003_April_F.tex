\documentclass[a4paper,12pt]{article}

%%%%%%%%%%%%%%%%%%%%%%%%%%%%%%%%%%%%%%%%%%%%%%%%%%%%%%%%%%%%%%%%%%%%%%%%%%%%%%%%%%%%%%%%%%%%%%%%%%%%%%%%%%%%%%%%%%%%%%%%%%%%%%%%%%%%%%%%%%%%%%%%%%%%%%%%%%%%%%%%%%%%%%%%%%%%%%%%%%%%%%%%%%%%%%%%%%%%%%%%%%%%%%%%%%%%%%%%%%%%%%%%%%%%%%%%%%%%%%%%%%%%%%%%%%%%

\usepackage{eurosym}
\usepackage{vmargin}
\usepackage{amsmath}
\usepackage{graphics}
\usepackage{epsfig}
\usepackage{enumerate}
\usepackage{multicol}
\usepackage{subfigure}
\usepackage{fancyhdr}
\usepackage{listings}
\usepackage{framed}
\usepackage{graphicx}
\usepackage{amsmath}
\usepackage{chngpage}

%\usepackage{bigints}
\usepackage{vmargin}

% left top textwidth textheight headheight

% headsep footheight footskip

\setmargins{2.0cm}{2.5cm}{16 cm}{22cm}{0.5cm}{0cm}{1cm}{1cm}

\renewcommand{\baselinestretch}{1.3}

\setcounter{MaxMatrixCols}{10}

\begin{document}
\begin{enumerate}

10 The price of a stock through time, {Xt : t  0} is thought to be modelled by the
following relationship:
Xt = 1.7Xt1 	 0.4Xt2 	0.3Xt3 + et 	0.7et1 + 0.12et2
For this model,
(i) Write the equation in terms of the backward shift operator B in the form
(B)(1 	 B)d Xt = (B)et,
where (B) and (B) are polynomials in B. [3]
(ii) Identify the values of p, d and q for which X is an ARIMA(p, d, q) process. [1]
(iii) Explain whether the process {Xt : t  0} is stationary. [1]
(iv) For the value of d from (ii), put Wt = (1 	 B)dXt. Explain why the model can
be written in the equivalent form
Wt = iet i  
and calculate i for i = 0, 1, 2. [3]
(v) Another representation of the model is
(B)Wt = et
where (B) = 1 	 1 i.
i iB 

  Calculate i for i = 1, 2. [1]
(vi) Define two vector-valued stochastic processes Y and Z as
Yt = (Xt, Xt1, Xt2)T, Zt = (Xt, Xt1, Xt2, Xt3, Xt4)T.
Explain which, if any, of the processes {Xt : t  0}, {Yt : t  0} and {Zt : t  0}
has the Markov property. [2]
(vii) Given a set of observations (x1, x2, …, xn) from an ARIMA(1, 1, 1) process
with unknown parameter values, outline the main steps that need to be taken
so that one can obtain forecasts for future values of the process. [3]
[Total 14]

%%%%%%%%%%%%%%%%%%%%%%%%%%%%%%%%%%%%%%%%%%%%%%%%%%5


10 (i) The model can be written as
(1  1.7B + 0.4B2 + 0.3B3) Xt = (1  0.7B + 0.12B2) et.
The term in the brackets on the LHS above is divisible by 1  B. We have
(1  1.7B + 0.4B2 + 0.3B3) = (1  B)(1  0.7B  0.3B2).
The last term is also divisible by 1  B, giving
(1  1.7B + 0.4B2 + 0.3B3) = (1  B)(1  0.7B  0.3B2) = (1  B)2(1 + 0.3B).
(ii) The model can be identified as an ARIMA(1, 2, 2) process.
(iii) {Xt : t 
 0} is clearly a non-stationary process, as the AR operator
(1  B)2(1  0.3B) has two roots with modulus one; any ARIMA(p, d, q)
process with d > 0 is non-stationary.
(iv) (1 + 0.3B)Wt = (1  0.7B + 0.12B2)et, which can be expressed as
Wt = (1 + 0.3B)1(1 	0.7B + 0.12B2)et.
Expanding the first term, Wt = (1  0.3B + 0.09B2  )(1  0.7B + 0.12B2)et
= (1  B + 0.42B2  …)et.
Therefore 0 = 1, 1 = 1, 2 = 0.42.
(v) Just invert the previous representation: et = (1  B + 0.42B2  …)1 Wt = (1 +
B  0.42B2 + B2  …) Wt, so that 1 = 1, 2 = 0.58.
(vi) None of the three processes is Markov; we know that if {Xt : t 
 0} is an
ARIMA(p, d, q) process with q > 0, then any finite collection (Xn, Xn1, …,
Xnm+1)T is non-Markov.
Subject 103 (Stochastic Modelling) — April 2003 — Examiners’ Report
Page 12
(vii) First transform the data, so that the differenced observations yt = (1  B)xt may
be thought of as realisations of a stationary ARIMA(1, 1) model.
Estimate the parameters of the model (by Maximum Likelihood or Method of
Moments) and obtain forecasts for future values of y.
Transform these back to obtain forecasts for the x values.
Part (i) was done very well; of people with the wrong answer, most stopped after
taking out one (1B) factor rather than trying for the 2nd. (ii) and (iii) were also well
answered, following on from (i).
In (iv) there were difficulties with the expansion of the denominator, meaning that few
candidates did well. Similar problems were encountered with part (v).
For the verification or otherwise of the Markov property in (vi) it would have helped
to write out {Yt} and {Zt} explicitly in terms of lagged terms.
Most candidates who attempted (vii) got at least part of the method, but few wrote
down all the main steps in the correct sequence.
