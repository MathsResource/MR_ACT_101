\documentclass[a4paper,12pt]{article}

%%%%%%%%%%%%%%%%%%%%%%%%%%%%%%%%%%%%%%%%%%%%%%%%%%%%%%%%%%%%%%%%%%%%%%%%%%%%%%%%%%%%%%%%%%%%%%%%%%%%%%%%%%%%%%%%%%%%%%%%%%%%%%%%%%%%%%%%%%%%%%%%%%%%%%%%%%%%%%%%%%%%%%%%%%%%%%%%%%%%%%%%%%%%%%%%%%%%%%%%%%%%%%%%%%%%%%%%%%%%%%%%%%%%%%%%%%%%%%%%%%%%%%%%%%%%

\usepackage{eurosym}
\usepackage{vmargin}
\usepackage{amsmath}
\usepackage{graphics}
\usepackage{epsfig}
\usepackage{enumerate}
\usepackage{multicol}
\usepackage{subfigure}
\usepackage{fancyhdr}
\usepackage{listings}
\usepackage{framed}
\usepackage{graphicx}
\usepackage{amsmath}
\usepackage{chngpage}

%\usepackage{bigints}
\usepackage{vmargin}

% left top textwidth textheight headheight

% headsep footheight footskip

\setmargins{2.0cm}{2.5cm}{16 cm}{22cm}{0.5cm}{0cm}{1cm}{1cm}

\renewcommand{\baselinestretch}{1.3}

\setcounter{MaxMatrixCols}{10}

\begin{document}


8 A branch of a bank has three cash dispensers. If at time t a cash dispenser is working,
the probability that it will break down in (t, t + dt) is independent of the state of the
other cash dispensers and is equal to
\[\alpha dt + o(dt).\]
When a cash dispenser breaks down, repair work begins immediately. The time taken
to repair a broken machine is exponentially distributed with mean 1/\beta. There are
enough repair teams to repair all three cash dispensers at the same time, if necessary.
Define Xt as the number of machines not working at time t.
\begin{enumerate}
\item (i) Write down the state space for Xt. [1]
\item (ii) (a) If a cash dispenser is working at time 0, prove that the time until its
first breakdown is exponentially distributed with mean 1/α.
(b) If all three cash dispensers are working at time 0, derive the
distribution of the time until the first breakdown. [5]
\item (iii) Define Pm(t) as the probability that m machines are not working at time t.
Show that the forward equations for the process X imply that:
P0(t) = 3P0 (t) P1(t)
P3(t) =  3P3(t)  P2 (t)
and for m = 1, 2,
Pm(t) = ((3 m) m )Pm(t) (4 m) Pm 1(t) (m 1) Pm 1(t)              .
\item 
(iv) At time 0 all three cash dispensers are working. Show that the forward
equations have a solution
3 3
( ) = ( )m[1 ( )] m, = 0,1, 2,3
Pm t t t m
m

 
   
 
where (t) satisfies a differential equation which you should identify. [5]
\end{enumerate}
\newpage
%%%%%%%%%%%%%%%%%%%%%
8 (i) State space for Xt is {0, 1, 2, 3}.
(ii) (a) Let T be the time taken for a cash dispenser to break down. From the
question:
PT t,t  dt  | T  t  = dt  o(dt)
or in other words
   
 
,
= ()
P T t t dt
dt o dt
P T t
 
 

If we set F(t) = P(T ≤ t), then
 
 
= ()
1
F t dt
dt o dt
F t

 

and letting dt → 0,
 
 
=
1
F t
F t



Integrate this to get ln 1 F t  =  t  const
\begin{itemize}
\item Since F(0) = 0, we can set const = 0, and hence F(t) = 1 – eαt
That is, T has an exponential distribution with mean 1/α.
(b) P(no breakdowns by time t) = (et)3 = e3t. Thus the time until the
first breakdown is exponentially distributed with parameter 3.
\item (iii) P(X(t + h) = m) = P(X(t) = m, X(t + h) = m) + P(X(t) = m  1, X(t + h) = m)
+ P(X(t) = m + 1, X(t + h) = m) + o(h).
Thus in general
1 0
1 3
( ) = ( )[1 ((3 ) ) ] ( )(4 ) 1
( )( 1) 1 ( ),
m m m m
m m
P t h P t m m h P t m h
P t m h oh
 
 
        
   
%% Subject 103 (Stochastic Modelling) — April 2003 — Examiners’ Report
%%  Page 9
where 1A is the indicator function of event A.
\item Rearranging and letting h → 0 gives the required equations.
\item (iv) First look at the LHS: if Pm has the given form, then
’ 3 1 2
( ) = ( )m (1 ( )) m ’( )[ (1 ( )) (3 ) ( )]
Pm t t t t m t m t
m
   

 	  	 	  	   	
 

On the other hand, the RHS is equal to
3  
1 4 1 2
3
( ) (1 ( )) (3 )
3 3
(4 ) ( ) (1 ( )) ( 1) ( ) (1 ( ))
1 1
m m
m m m m
t t m m
m
m t t m t t
m m

   
 
 	      

 
   
  	      	   
   
  
  1 2 2 2 3
= (t)m (1 (t)) m [m (3 m) ] (1 ) m (1 ) (3 m)
m
   

 	  	     	  	    	   	
 

1 2   3
= (t)m (1 (t)) m(m 3 ) (1 )
m
   
         
	 

Thus the LHS and RHS are equal as long as ’ (t)    ( )(t) .

% Almost everyone got part (i). In part (ii), stating the time until first breakdown is exponential in (a) is not sufficient: it was necessary to derive this from first principles.
% Similarly, a reasonable number got the answer to (b) without deriving it properly.
% The most successful strategy for part (iii) was writing the generator matrix and getting the equations from there.
Only the exceptional candidate attempted (iv); the algebra was tough, but did not involve any tricks.
\end{itemize}
\end{document}
