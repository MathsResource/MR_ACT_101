\documentclass[a4paper,12pt]{article}

%%%%%%%%%%%%%%%%%%%%%%%%%%%%%%%%%%%%%%%%%%%%%%%%%%%%%%%%%%%%%%%%%%%%%%%%%%%%%%%%%%%%%%%%%%%%%%%%%%%%%%%%%%%%%%%%%%%%%%%%%%%%%%%%%%%%%%%%%%%%%%%%%%%%%%%%%%%%%%%%%%%%%%%%%%%%%%%%%%%%%%%%%%%%%%%%%%%%%%%%%%%%%%%%%%%%%%%%%%%%%%%%%%%%%%%%%%%%%%%%%%%%%%%%%%%%

\usepackage{eurosym}
\usepackage{vmargin}
\usepackage{amsmath}
\usepackage{graphics}
\usepackage{epsfig}
\usepackage{enumerate}
\usepackage{multicol}
\usepackage{subfigure}
\usepackage{fancyhdr}
\usepackage{listings}
\usepackage{framed}
\usepackage{graphicx}
\usepackage{amsmath}
\usepackage{chngpage}

%\usepackage{bigints}
\usepackage{vmargin}

% left top textwidth textheight headheight

% headsep footheight footskip

\setmargins{2.0cm}{2.5cm}{16 cm}{22cm}{0.5cm}{0cm}{1cm}{1cm}

\renewcommand{\baselinestretch}{1.3}

\setcounter{MaxMatrixCols}{10}

\begin{document}
%%%%%%%%%%%%%%%%%%%%%%%%%%%%%%%%%%%%%%%%%%%%%%%%%%%%%%%%%%%%%%%%%%%%%%%%%%%%%%%%%%%%%%%%%%%%%%%%%%%%%%5
3 

\begin{enumerate}
    \item (i) Let Yt denote Brownian motion with drift \mu and variance rate $\sigma^2$, starting at 0.
Write down the expectation $E(Y_t)$ and variance Var(Yt) of this process. [1]
\item (ii) Let Xt denote a biased random walk which moves up or down by D at time
intervals of size h, i.e.
Xt = 1
t
h
i i Z
 
 
  ,
where Z1, Z2, … are independent random variables, each with distribution $P[Z_i = D] = p$, P[Zi = 	D] = 1 	 p. Calculate the expectation and variance of Xnh and derive an expression for the expectation and variance of $X_t$. 
[Notation: x denotes the integer part of x.]
\item (iii) (a) Use the approximation x  x to derive conditions on p and D such
that the expectation and variance of the process $X_t$ approximate for
large t those of a Brownian motion with drift $\mu$ and variance rate $\sigma^2$.
(b) Comment on the values of h for which such an approximation might be
appropriate.
\end{enumerate}
%%%%%%%%%%%%%%%%%%%%%%%%%%%%%%%%%%%%%%%%%%%%%%%%%%%%%%%%%%%
\newpage
3 (i) E(Yt) = t, Var(Yt) = 2t.
(ii) E(Xnh) = n(2p  1) D, Var(Xnh) = nD2  4p(1  p).
Therefore E(Xt) = (2p  1)D t ,
h
 
 
Var(Xt) = 4p(1  p)D2 t .
h
 
 
(iii) (a) We require  = (2p 1)D/h and 2 = 4p(1  p)D2/h.
(b) This implies that p = 1 1
2
h
D
     
\mu \sigma
, D2 = 2 22.  h   h
This applies only when 0  |h| < 1. Small values of h should be used
because the random walk model converges to Brownian motion /
diffusion as h \mu 0.

%% Parts (i) and (ii) were well answered  maybe half the people successfully equated the random walk moments with the Brownian motion ones  although many candidates failed to see the derivation of the conditions on p and D through to completion. The biggest problem was the last part: "h small" is on the whole a bit too vague to get the full credit.

\end{document}
