\documentclass[a4paper,12pt]{article}

%%%%%%%%%%%%%%%%%%%%%%%%%%%%%%%%%%%%%%%%%%%%%%%%%%%%%%%%%%%%%%%%%%%%%%%%%%%%%%%%%%%%%%%%%%%%%%%%%%%%%%%%%%%%%%%%%%%%%%%%%%%%%%%%%%%%%%%%%%%%%%%%%%%%%%%%%%%%%%%%%%%%%%%%%%%%%%%%%%%%%%%%%%%%%%%%%%%%%%%%%%%%%%%%%%%%%%%%%%%%%%%%%%%%%%%%%%%%%%%%%%%%%%%%%%%%

\usepackage{eurosym}
\usepackage{vmargin}
\usepackage{amsmath}
\usepackage{graphics}
\usepackage{epsfig}
\usepackage{enumerate}
\usepackage{multicol}
\usepackage{subfigure}
\usepackage{fancyhdr}
\usepackage{listings}
\usepackage{framed}
\usepackage{graphicx}
\usepackage{amsmath}
\usepackage{chngpage}

%\usepackage{bigints}
\usepackage{vmargin}

% left top textwidth textheight headheight

% headsep footheight footskip

\setmargins{2.0cm}{2.5cm}{16 cm}{22cm}{0.5cm}{0cm}{1cm}{1cm}

\renewcommand{\baselinestretch}{1.3}

\setcounter{MaxMatrixCols}{10}

\begin{document}
\begin{enumerate}
6 (i) Calculate the autocovariance function {
k : k  0} and the autocorrelation
function {k : k  0} for the mth order Moving Average process
Xt =  + 1
m1
(et + et + … + etm),
where {et : t  0} is a sequence of uncorrelated, zero-mean random variables
with common variance e2. [4]
(ii) Explain whether or not the process is invertible in the case where m = 2. [4]
[Total 8]
%%%%%%%%%%%%%%%%%%%%%%%%%%%%%%%%%%%%%%%%%%%%%%%%%%%%%%%%%%

7 The data set plotted in Figure 1a represents the number of applications, xt, for travel
insurance received by an insurance company’s web site, measured for a total of 60
consecutive months. Figure 1b displays the logarithm of the same data set, yt = ln(xt).
(i) A statistician decides, on the basis of these plots, to fit a linear time series
model to yt rather than to xt. State, giving reasons for your answer, whether
you agree with this decision. [2]
(ii) Explain what is meant by the terms “seasonal variation” and “linear trend”.
Outline one method of compensating for seasonal variation and linear trend in
a data set which exhibits both. [3]
(iii) The data set zt is a seasonally adjusted, detrended version of yt. Figures 2a and
2b below display respectively the sample autocorrelation function and sample
partial autocorrelation function of zt.
(a) Explain what feature of the graphs enables you to conclude that it is
reasonable to fit a stationary model to the data.
(b) Suggest values of p, d and q such that an ARIMA(p, d, q) model is
likely to provide a good fit to the data set zt. Give reasons for your
suggestion. [3]
(iv) Explain how to produce a forecast xˆ60 (1) for the value of x61 given the Box-
Jenkins forecast zˆ60 (1) for the value of z61. [2]
[Total 10]
103 S2003—5 PLEASE TURN OVER
Figure 1a Figure 1b
Figure 2a
Figure 2b
Partial autocorrelation function of z t
-0.5
-0.4
-0.3
-0.2
-0.1
0
0.1
0.2
0.3
0 1 2 3 4 5 6 7 8 9 10 11
Lag, k
PACF
Volume of travel insurance, x t
0 10 20 30 40 50 60
Month, t
Log(Volume), y t = ln(x t )
0 10 20 30 40 50 60
Month, t
Autocorrelation function of z t
-0.6
-0.4
-0.2
0
0.2
0.4
0.6
0 1 2 3 4 5 6 7 8 9 10 11
Lag, k
ACF, r k
103 S2003—6

%%%%%%%%%%%%%%%%%%%%%%%%%%%%%%%%%

6 (i) We have
k = Cov(Xt, Xt k) = 2
0 0
1
Cov ,
( 1)
m m
t r t k r
r r
e e
m
= 2
0 0
1
Cov( , ).
( 1)
m m
t r t k r
r r
e e
m
Clearly if k > m, all terms are zero, so that k = 0.
For 0 k m, there are exactly (m k + 1) non-zero terms, and each of these
covariance terms equals 2e . Thus
k =
2
1 2
( 1)
0
m k 0,1, 2,..., .
m e
k m
k m
The autocorrelation function is
k = 1
1
1 0
1,2,...,
0
m k
m
k
k m
k m
(ii) For the process to be invertible, we require that the roots of the characteristic
equation should be greater than 1 in absolute value.
We can rewrite the MA model with the aid of the backward shift operator B as
follows:
Xt =
1
3
(1 + B + B2) et.
The roots of the characteristic equation
1 + B + B2 = 0
are B = ½ + ½i 3 , B = ½ ½i 3 .
In both cases B = 1. Thus the process is not invertible.
In general candidates made reasonable attempts along the right lines, although this did not always
result in the correct autocorrelation function being calculated. Where possible, some credit was
given for attempts at the second part based on an incorrect part (i).
Subject 103 (Stochastic Modelling) September 2003 Examiners Report
Page 8
7 (i) The original data are clearly subject to seasonal variation, and the size of the
seasonal fluctuations is increasing in line with the value of the underlying
quantity. This suggests that the seasonal variation is multiplicative rather than
additive, in which case taking the logarithm is the sensible thing to do. In
addition to this, a look at the plot of yt against time confirms that the variation
is much more regular.
(ii) Seasonal variation is a predictable pattern of deterministic variation in the
mean of the process which is cyclic, i.e it repeats after a fixed number of time
periods, usually corresponding to a year of elapsed time.
A linear trend is a deterministic pattern of variation in the mean of the process
which is linearly dependent on the time variable, i.e. is of the form a + bt.
There are various possible answers. Possible methods include:
(a) Estimate the trend by linear regression and remove it, then, for each
month, calculate the sample mean value for that particular month over
all five years. From every detrended observation subtract off the
appropriate seasonal mean to obtain seasonally adjusted data.
(b) Remove the linear trend by differencing the data once, then remove the
seasonal variation by seasonal differencing. In other words,
= (1 12 )(1 ) zt B B yt .
(iii) (a) The fact that the sample ACF is not near 1 for small lags is the most
obvious pointer to the stationarity of the adjusted data set.
(b) The clue here is the highest lag for which the ACF or PACF is
significantly different from 0. Looking at the sample ACF we might
suggest that a MA(3) might fit, as the sample ACF is roughly zero for
k > 3. Similarly, a look at the sample PACF seems to indicate an
AR(3). But it might well be possible to find an ARMA(1,1) which
would fit adequately. In other words, d = 0 and either p = 0, q = 3 or
p = 3, q = 0 or p = 1, q = 1.
(iv) We have yt = mt + zt, where mt represents deterministic variation and zt is a
purely random component. The process of seasonal adjustment and
detrending has produced an estimate mt for mt which can be extrapolated into
the future. Thus we have y60 (1) m61 z60 (1) , which in turn leads to
x60 (1) exp( y60 (1)).
Neither the discussion of how to deal with seasonal variation nor the practical part to do with model
identification was especially well tackled.
Subject 103 (Stochastic Modelling) September 2003 Examiners Report
Page 9
