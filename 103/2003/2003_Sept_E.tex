10 (i) Define a geometric Brownian motion and write down a stochastic differential
equation which it satisfies. [2]
(ii) Using the substitution = t  
Yt e Xt c 
 , or otherwise, find a solution for the
stochastic differential equation
dXt = (Xt  c)dt  dBt
with the initial condition X0 = x0, where Bt is standard Brownian motion. [5]
(iii) Write down a condition satisfied by the stationary density function, 
(x), of a
diffusion process, assuming that such a stationary density exists. [2]
(iv) Verify that the stationary distribution of the diffusion process in (i) is Normal,
and identify the mean and variance. [2]
(v) When the equation in (iii) is applied to the geometric Brownian motion, the
general solution is found to be
(x) A Bxk ,
x   
where k = 2 + 2/2 and A and B are arbitrary constants. Show that there is
no probability density function 
(x) which solves the equation in (iii) in the
case of the geometric Brownian motion and explain the implication of this. [2]
[Total 13]
103 S2003—8


10 (i) A geometric Brownian motion can be defined as a process
St S0 exp( t Bt ) , where Bt is a standard Brownian motion.
It satisfies the SDE dSt Stdt StdBt .
Alternatively, use the definition St S0 exp( t Bt ) , which satisfies the SDE
1 2
t 2 t t t dS S dt S dB .
(ii) We use the Itô formula (we must have the form where F is a function of t as
well as x):
dF t, Xt 1 2
2 = Ft t, Xt Fx t, Xt dt Fx t, Xt dXt
Let = t ( ) ,
Yt e Xt c F t Xt . Then Y(0) = x0 and
Subject 103 (Stochastic Modelling) September 2003 Examiners Report
Page 12
= t ( )
dYt d e Xt c
= t [ ] t [ ] t
e Xt c e Xt c dt e dBt
= t
e dBt
Hence 0
0
=
t
s
s
s
Y t x c e dB
and so 0
0
= ( )
t
t t s
t s
s
X c e x c e e dB .
(iii) The required condition for the stationary density of the diffusion Y solving
dYt (Yt )dt (Yt )dBt , from the Core Reading, is
d
dy
[ (y) (y)] = ½
2
2
d
dy
[ 2(y) (y)]
(iv) Two ways to do this. From (i),
2
2 2 ( ) 2
0 0 0
~ ( ( ), ) = ( ), 1
2
t t t s t t
Xt N c e x c e ds N c e x c e
The limiting distribution, as t , is N(c, 2/(2 )). A limiting distribution is
always stationary.
Alternatively, in this instance we have (x) = (x c), (x) = , so the
condition given in (ii) is that
1 2
2
(x) (x c) (x) (x) .
The solution to this DE is
2
2
( )
( ) .exp
x c
x const ,
which is the density function of N(c, 2/(2 )).
(v) A function of the form Axb for all x > 0 cannot integrate to 1, no matter what
the values of A and b. This means that there is no stationary density function
for the geometric Brownian motion.
Subject 103 (Stochastic Modelling) September 2003 Examiners Report
Page 13
It would be surprising if there had been: it is well known that Brownian
motion is non-stationary, and therefore that geometric Brownian motion is
also a non-stationary process, so cannot possess a stationary density.
Marks were poor for Question 10. Candidates should note that the reference in part (iv) of the
question to "the equation in (i)" was incorrect and should have read "the equation in (ii)". Whilst at
least some candidates were still able to complete this part of the question, allowance was made in the
marking of scripts for this error.
It seemed that candidates had not committed to memory the formula for the equilibrium density of a
diffusion process, presumably because it had not been asked before.


