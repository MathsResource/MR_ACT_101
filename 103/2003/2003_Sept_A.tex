\documentclass[a4paper,12pt]{article}

%%%%%%%%%%%%%%%%%%%%%%%%%%%%%%%%%%%%%%%%%%%%%%%%%%%%%%%%%%%%%%%%%%%%%%%%%%%%%%%%%%%%%%%%%%%%%%%%%%%%%%%%%%%%%%%%%%%%%%%%%%%%%%%%%%%%%%%%%%%%%%%%%%%%%%%%%%%%%%%%%%%%%%%%%%%%%%%%%%%%%%%%%%%%%%%%%%%%%%%%%%%%%%%%%%%%%%%%%%%%%%%%%%%%%%%%%%%%%%%%%%%%%%%%%%%%

\usepackage{eurosym}
\usepackage{vmargin}
\usepackage{amsmath}
\usepackage{graphics}
\usepackage{epsfig}
\usepackage{enumerate}
\usepackage{multicol}
\usepackage{subfigure}
\usepackage{fancyhdr}
\usepackage{listings}
\usepackage{framed}
\usepackage{graphicx}
\usepackage{amsmath}
\usepackage{chngpage}

%\usepackage{bigints}
\usepackage{vmargin}

% left top textwidth textheight headheight

% headsep footheight footskip

\setmargins{2.0cm}{2.5cm}{16 cm}{22cm}{0.5cm}{0cm}{1cm}{1cm}

\renewcommand{\baselinestretch}{1.3}

\setcounter{MaxMatrixCols}{10}

\begin{document}
\begin{enumerate}

1 A wheel is divided into 37 equal sections, labelled from 0 to 36. A ball is rolled
repeatedly around the wheel, landing in sections X1, X2, … Let Mn be the highest
outcome achieved from the first n rolls, i.e. Mn = max1in Xi.
(i) Show by general reasoning that Mn is a Markov chain. [1]
(ii) Derive the transition probabilities for Mn. [2]
(iii) Determine whether Mn is irreducible and aperiodic. [1]
(iv) Determine the equilibrium probability distribution of Mn. [1]
[Total 5]
%%%%%%%%%%%%%%%%%%%%%%%%%%%%%%%%%%%%%%%%%%%%%%%%%%%%%%%%5
2 (i) Calculate the covariance between the values X(t), X(t + s) taken by a Poisson
process X(t) with constant rate  at the two times t and t + s, where s > 0. [2]
(ii) Calculate the covariance between the values B(t), B(t + s) taken by a standard
Brownian motion B(t) at the two times t and t + s, where s > 0. [1]
(iii) A share price, St, is modelled as St = exp(Y(t)), where
Y(t) = Y(0) + B(t) + (X1(t)  X2(t)).
Here B(t) is a standard Brownian motion, X1 and X2 are rate  Poisson
processes, independent of each other and of B(t), and  and  are constants.
(a) Give a definition of a Lévy process.
(b) Show that Y(t) is a Lévy process.
(c) Calculate the covariance of Y(t) and Y(t + s) for s > 0. [4]
[Total 7]
%%%%%%%%%%%%%%%%%%%%%%%%%%%%%%%%%%%%%%%%%%%%%%%%%%%%%%%%%%%%%%
3 An ARIMA process X satisfies the recursion
Xt = 1Xt 1 2Xt 2 et et 1        ,
where et is white noise with variance σ2.
(i) (a) Write down a condition in terms of the roots of an equation for X to be
stationary.
(b) Show, in the case where 2 = 0.5, that X is stationary as long as
1< 1.5.
[4]
(ii) Derive the spectral density function of X. [3]
[Total 7]
%%%%%%%%%%%%%%%%%

1 (i) Suppose that Mn = i. Then the value of Mn+1 depends entirely on the outcome
of the next spin of the roulette wheel (this being independent of Mn) and the
value of Mn. Hence Mn is a Markov chain.
(ii) Let Xn be the winning number from the nth spin of the roulette wheel. Then
P(Xn = i) = 1/37 for all i = 0, 1, 2, .,36.
If j > i then P(Mn+1 = j | Mn = i) = P(Xn+1 = j) = 1/37;
If j < i then P(Mn+1 = j | Mn = i) = 0;
If j = i then P(Mn+1 = j | Mn = i) = P(Xn+1 i) = (i + 1)/37.
i.e. pij =
1
37
( 1)
37
if
if
0 if
i
j i
j i
j i
(iii) M is aperiodic, as it can stay in the same state with positive probability. It is
not, however, irreducible, since it is not possible to return to state j from any
state k > j.
(iv) State 36 is absorbing, so the only stationary distribution, which is also the
limiting distribution, is 36 = 1, i = 0 for all other i.
The general reasoning type answers required for parts (i), (iii) and (iv) were quite well done in
general. There were some difficulties with part (ii) However candidates fared less well on part (ii),
and in many cases candidates struggling on part (ii) then failed to attempt the later parts.
2 (i) The Poisson process and the standard Brownian motion both possess the
independent increments property.
Cov(X(t), X(t + s)) = Cov(X(t), X(t)) + Cov(X(t), X(t + s) X(t)) = t + 0, by
the independent increments property.
(ii) Cov(B(t), B(t + s)) = Cov(B(t), B(t)) + Cov(B(t), B(t + s) B(t) = t.
(iii) (a) A Lévy process is a continuous-time process with stationary,
independent increments. Alternatively, a Lévy process can be defined
as a sum of three (indpendent) components: a constant drift, a multiple
of Brownian motion and a purely discontinuous random component
such as a compound Poisson process.
(b) The increments of Y are the weighted sum of the increments of B, X1
and X2, so are stationary and independent.
Subject 103 (Stochastic Modelling) September 2003 Examiners Report
Page 4
(c) Cov(Y(t), Y(t + s)) = 2Cov(B(t), B(t), B(t + s)) + 2Cov(X1(t),
X1(t + s)) + 2Cov(X2(t), X2(t + s)) = ( 2 + 2 2 )t. All other terms
vanish by independence.
The question demanded straightforward manipulation of the independent increments
property and the covariance function.
3 (i) (a) The condition for X to be stationary is that the roots of the equation
2
1 1z 2z = 0
should lie outside the unit circle.
(b) The roots are
2
1 1 2
2
4
2
.
In the given instance these are 2
1 1 2 .
If 1
2 > 2 then we require that
2
1 1 2 1 and 2
1 1 2 1
which is equivalent to
2 2
1 2 1 1
implying that 1 1.5.
If, on the other hand, 1
2 < 2, then the roots are imaginary and satisfy
2 2 2
z = 1 (2 1 ) = 2
so that the condition is automatically satisfied.
(ii) The spectral density satisfies
H1( ) fX = H2 ( ) fe ( ),
where H1 is the transfer function associated with (1 1B 2B2), H2 the
transfer function associated with (1 + B), fX( ) is the spectral density of X
and fe( ) is the spectral density of e.
Subject 103 (Stochastic Modelling) September 2003 Examiners Report
Page 5
We have
H1( ) = |1 1ei
2e2i |2, H2( ) = |1 + ei |2 and fe( ) = 2/(2 ).
Therefore
2 2
2 2
1 2 1 1 2 2
1 2 cos( )
=
2 1 2 cos 2 cos 2
fX
There was good understanding of the ARIMA process, which meant that candidates
successfully derived the quadratic equation in part (i), though some were let down by their
knowledge of complex numbers. Part (ii) was a straightforward application of the transfer
function: the fact that marks were on average slightly lower seems to indicate that it had not
been learned especially well.
