




11 A continuous-time process with three states is observed from time 0 up until the time
of the 20th transition. The results may be summarised as follows:
No. of transitions
from state i to:
State, i No. of visits
to state i
Minutes spent
in state i
State 1 State 2 State 3
1 8 48  3 5
2 4 160 1  3
3 8 240 7 1 
(i) Describe the stages of model fitting and model verification in the modelling
process. [2]
(ii) Suppose that a Markov jump process model is to be fitted to the data set
above. List all the parameters of the model and discuss the assumptions made
when such a model is fitted to a data set. [4]
(iii) Estimate the parameters of the model in (ii) above and write down the
estimated generator matrix. [4]
(iv) Suggest one test which could be applied as part of the model verification
process. State the null hypothesis, H0, identify the test statistic and name its
distribution under H0. [2]
(v) The 20th transition of the observed process takes it into state 1. Use the
estimated parameter values to give point estimates of the times until the 21st
and 22nd transitions. [3]
[Total 15]
%%%%%%%%%%%%%%%%%%%%%%%%%%%%%%%%%%%%%%%%%%%%%%%%%%%%%%%%%%%%%%%%%%%%%%%%%%%%%%%%%%%%%%%%%%%%%%%%%%%%%%%%%%%%%%%%%

11 (i) Model fitting: this occurs after the family of model has been decided and
concerns the estimation of the values of parameters. The set of parameters to
be estimated is determined by the choice of model family.
Model verification: once the model has been fitted we need to check that the
fitted process resembles what has been observed. Generally we produce
simulations of the process, using the estimated parameter values, and compare
them with the observations.
(ii) The parameters are the rate of leaving state i, i, for each i, and also the jumpchain
transition probabilities, rij for j i, where rij is the conditional
probability that the next transition takes the chain to state j given that it is now
in state i. Alternatively, one may regard the parameters as being ij, where ii
= i and, for j i, ij = i rij.
Assumptions of the Markov model are that the duration of holding time in
state i has exponential distribution with parameter determined only by i and is
independent of anything that happened before the current arrival in state i, and
that the destination of the next jump after leaving state i is independent of the
holding time in state i and of anything that happened before the chain arrived
in state i.
(iii) 1
i is the average duration of each stay in state i. Thus 1
1 6
per minute, or
10 per hour, 1
2 40
per minute or 1.5 per hour, 1
3 30
per minute, or 2 per
hour.
3
12 8
r , 5
13 8
r , 1
21 4
r , 3
23 4
r , 7
31 8
r and 1
32 8
r .
Thus the generator matrix, in units of hr 1, is
Subject 103 (Stochastic Modelling) September 2003 Examiners Report
Page 14
10 3.75 6.25
0.375 1.5 1.125
1.75 0.25 2
(iv) One test should test whether the holding times in each state are exponentially
distributed. If Ti,k denotes the kth holding time in state i, then the hypothesis is
that Ti,1, Ti,2, , Ti,ni is a sample from an exponential distribution with
parameter i : sort the observations into categories, calculate expected number
in each category and hence find the X2 statistic by summing (O E)2/E. This
should be compared with the critical value of the 2 distribution with m 2
d.f., where m is the number of categories.
(v) Estimate of expected duration of a visit to state 1 is 6 mins, so this is the
estimated time until the 21st transition.
Estimate of expected time between 21st and 22nd transitions is
E(Time | transition is to 2) P(transition is to 2) + E(Time | to 3) P(to 3)
= 40 (3/8) + 30 (5/8) = 33.75 mins.
Attempts to fit the model to the observed data were generally sensible and encouraging,
although in a minority of cases the calculation of the parameter estimates betrayed evidence
of some confusion. There was a tendency to be less successful as the question continued,
with the result that attempts at the final part were of a noticeably lower standard.
This document was created with Win2PDF available at http://www.daneprairie.com.
The unregistered version of Win2PDF is for evaluation or non-commercial use only.
