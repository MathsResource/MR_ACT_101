
\documentclass[a4paper,12pt]{article}

%%%%%%%%%%%%%%%%%%%%%%%%%%%%%%%%%%%%%%%%%%%%%%%%%%%%%%%%%%%%%%%%%%%%%%%%%%%%%%%%%%%%%%%%%%%%%%%%%%%%%%%%%%%%%%%%%%%%%%%%%%%%%%%%%%%%%%%%%%%%%%%%%%%%%%%%%%%%%%%%%%%%%%%%%%%%%%%%%%%%%%%%%%%%%%%%%%%%%%%%%%%%%%%%%%%%%%%%%%%%%%%%%%%%%%%%%%%%%%%%%%%%%%%%%%%%

\usepackage{eurosym}
\usepackage{vmargin}
\usepackage{amsmath}
\usepackage{graphics}
\usepackage{epsfig}
\usepackage{enumerate}
\usepackage{multicol}
\usepackage{subfigure}
\usepackage{fancyhdr}
\usepackage{listings}
\usepackage{framed}
\usepackage{graphicx}
\usepackage{amsmath}
\usepackage{chngpage}

%\usepackage{bigints}
\usepackage{vmargin}

% left top textwidth textheight headheight

% headsep footheight footskip

\setmargins{2.0cm}{2.5cm}{16 cm}{22cm}{0.5cm}{0cm}{1cm}{1cm}

\renewcommand{\baselinestretch}{1.3}

\setcounter{MaxMatrixCols}{10}

\begin{document}
PLEASE TURN OVER11
Patients arriving at the Accident and Emergency department (state A) wait for
an average of one hour before being classified by a junior doctor as requiring
in-patient treatment (I), out-patient treatment (O) or further investigation (F).
Only one new arrival in ten is classified as an in-patient, five in ten as out-
patients.
If needed, further investigation takes an average of 3 hours, after which 50% of
cases are discharged (D), 25% are sent to receive out-patient treatment and
25% admitted as in-patients.
Out-patient treatment takes an average of 2 hours to complete, in-patient
treatment an average of 60 hours. Both result in discharge.
It is suggested that a time-homogeneous Markov process with states A, F, I, O
and D could be used to model the progress of patients through the system,
with the ultimate aim of reducing the average time spent in the hospital.
(i) Write down the matrix of transition rates, {σ ij : i, j = A, F, I, O, D}, of
such a model.
[2]
(ii) Calculate the proportion of patients who eventually receive in-patient
treatment.
[1]
(iii) Derive expressions for the probability that a patient arriving at time
t = 0 is:
(iv)
(a) yet to be classified by the junior doctor at time t, and
(b) undergoing further investigation at time t
[4]
Let m i denote the expectation of the time until discharge for a patient
currently in state i.
(a)
Explain in words why m i satisfies the following equation:
1
+
λ i
m i =
∑
j ∉ { i , D }
σ ij
λ i
m j
where λ i = ∑ σ ij .
j
(b)
(v)
Hence calculate the expectation of the total time until discharge for a newly-arrived patient.
[4]
State the distribution of the time spent in each state visited according to this model.
[1]
The average times listed above may be assumed to be the sample mean waiting times derived from tracking a large sample of patients through the system.
(vi)
103—6
Describe briefly what additional feature of the data might be used to
check that this simple model matches the situation being modelled. [2](vii)
103—7
The hospital management committee believes that replacing the junior
doctor with a more senior doctor will save resources by reducing the
proportion of cases sent for further investigation. Alternatively, the
same resources could go towards reducing out-patient treatment time.
(a) Outline briefly the calculations that would need to be performed
to compare the options.
(b) Discuss whether the current model is suitable as a basis for
making decisions of this nature.
[4]
[Total 18]
\end{document}
