\documentclass[a4paper,1pt]{article}

%%%%%%%%%%%%%%%%%%%%%%%%%%%%%%%%%%%%%%%%%%%%%%%%%%%%%%%%%%%%%%%%%%%%%%%%%%%%%%%%%%%%%%%%%%%%%%%%%%%%%%%%%%%%%%%%%%%%%%%%%%%%%%%%%%%%%%%%%%%%%%%%%%%%%%%%%%%%%%%%%%%%%%%%%%%%%%%%%%%%%%%%%%%%%%%%%%%%%%%%%%%%%%%%%%%%%%%%%%%%%%%%%%%%%%%%%%%%%%%%%%%%%%%%%%%%

\usepackage{eurosym}
\usepackage{vmargin}
\usepackage{amsmath}
\usepackage{graphics}
\usepackage{epsfig}
\usepackage{enumerate}
\usepackage{multicol}
\usepackage{subfigure}
\usepackage{fancyhdr}
\usepackage{listings}
\usepackage{framed}
\usepackage{graphicx}
\usepackage{amsmath}
\usepackage{chngpage}

%\usepackage{bigints}
\usepackage{vmargin}

% left top textwidth textheight headheight

% headsep footheight footskip

\setmargins{.0cm}{.5cm}{16 cm}{cm}{0.5cm}{0cm}{1cm}{1cm}

\renewcommand{\baselinestretch}{1.}

\setcounter{MaxMatrixCols}{10}

\begin{document}

\begin{enumerate}
\item

%%%%%%%%%%%%%%%%%%%%%%%%%%%%%%%%%%%%%
7
(i)
Demonstrate that the random variable
X = - 1⁄4log(U)
has an exponential distribution with mean 1⁄4 when U is uniformly distributed
over the range (0, 1).
[]
(ii)
8
(a) Explain how the above can be used to simulate a path of the Poisson
process with intensity l = 4.
(b) Derive a method for simulating a Poisson random variable with
mean 4.
[4]
(iii) Describe an alternative method for generating a Poisson random variable with
mean 4 based on the cumulative distribution function.
[]
(iv) State, giving reasons, which of the two methods in (ii) and (iii) would be more
efficient if a simulation calls for a large number of Poisson random variables
with mean 4.
[]
[Total 10]
A company is studying the health records of its longest serving employee in order to
improve its provision for health insurance. Let X(t) = H if the employee is healthy at
time t, X(t) = S otherwise. The available information includes the value of X(t) for all
0 £ t £ T.
(i) Explain the principal stages in the formulation and verification of a stochastic
model for this process.
[]
(ii) The company chooses to model X as a two-state time-homogeneous Markov
jump process with transition rates s HS = s , s SH = r .
10 A00—6
(a) State the distribution of a typical holding time in state H and of a
typical holding time in state S assuming the model is valid.
(b) Write down estimates for the parameters s and r of the model in terms
of quantities which may be derived from the available data.
(c) Indicate one test which could be used to determine whether the data
support the assumption that the Markov jump process model is
suitable.
[4](iii)
10 A00—7
Having fitted the model the company discovers that the observed distribution
of holding times in state H does not fit the predictions of the time-
homogeneous Markov model; in particular, the mean holding time in state H
between visits to state S appears to be decreasing with t.
(a) Describe the principal difference between a time-inhomogeneous
model and a time-homogeneous one and indicate whether a time-
inhomogeneous model might provide a better fit to the observations.
(b) Explain why the original model could still be used if the company is
large and has a roughly constant age profile.
%%%%%%%%%%%%%%%%%%%%%%%%%%%%%%%%%%%%%%%%%%%%%%%%%%%%%%%%%%%%%%%%%%%%%%%%%%%%
7
(i) Either P ( X > x ) = P ( U < e - 4x ) = e - 4x , so that f X ( x ) = 4 e - 4x ,
Or probability distribution function F ( x ) = 1 - e -l x has inverse
1
F - 1 ( y ) = - log(1 - y ). Hence the inversion method reads:
l
(1)
Generate y from U (0, 1).
1
()
Return x = - log(1 - y ).
l
1
1
- log y is as good as - log(1 - y ) here, since 1 - Y is also U (0, 1).
l
l
(ii) (a)
Obtain numbers y 1 , y  , ..., y n by the above procedure, being outcomes
of random variables Y 1 , Y  , ..., Y n independent exponentially
distributed with parameter l = 4.
Put t j = y 1 + y  + ... + y j and return
x t = 0 if t < t 1 ,
x t = j if t j £ t < t j+1 .
(b)
(iii)
Page 6
Take the value of the Poisson process at time 1.
Calculate F ( i ) for each i = 0, 1, .... Then
(1) Generate u from U (0, 1).
() Return the smallest i such that u £ F ( i ).Subject 10 (Stochastic Modelling) — 
%%%%%%%%%%%%%%%%%%%%%%%%%%%%%%%%%%%%%

(iv)
The method in (ii) requires an average of four uniform r.v.s per Poisson r.v.;
the method in (iii) requires only one. Since the distribution function needs to
be calculated only once, (iii) should be much more efficient.
Very poorly answered, considering that it deals only with simulating standard
exponential and Poisson random variables. Even the inverse distribution
function method was largely misunderstood. There must be the suspicion that
many candidates are not getting as far as Unit 7 in the Core Reading.
8
(i)
First choose a class of model which might be supposed, for physical reasons, to provide a reasonable fit to the data. Identify the parameters of the model.
Next estimate the values of the parameters from the data.
See if the observed data match the pattern which would be expected if the model were accurate and if the parameters had the values given by their
estimates. If not, the model should be revised.
(ii)
(a) Exponential distribution in each case, with rate s in H , r in S.
(b) The time spent in state H before the next visit to S has mean s - 1 .
Therefore a reasonable estimate for s is the reciprocal of the mean length of each visit: ŝ = (Number of transitions from H to S )/(Total
time spent in state H up until the last transition from H to S ), although
it would be equally valid to use the Maximum Likelihood Estimator,
which is (Number of transitions from H to S )/(Total time spent in state H ).
Similarly for r̂ .
(iii)
(c) Testing whether the successive holding times are independent
exponential variables would be best, and any procedure which does test this is acceptable. Something like using the c  goodness-of-fit test
on the even-numbered holding times, then again on the odd-numbered ones, springs to mind, but there may be other, equally reasonable,
answers.
(a) For a time-inhomogenous model the transition rates s and r are functions of t .
It is certainly possible to improve the fit by using a time-inhomogenous model in this instance.
Page 7Subject 10 (Stochastic Modelling) — 
%%%%%%%%%%%%%%%%%%%%%%%%%%%%%%%%%%%%%

(b)
If the age profile is represented by a density function f ( a ); then the
overall average rate at which a healthy employee falls sick is s =
ò f ( a ) s ( a ) da , roughly constant for all t . The same of course applies
to the overall average rate of recovery. (It is not necessary to write
down the integral to obtain full marks: any explanation which covers the basic principle will suffice.)

Generally good answers. A number of candidates answered part (i) in the
context of this particular model rather than in general terms and so were only
awarded a reduced number of marks.
