\documentclass[a4paper,12pt]{article}

%%%%%%%%%%%%%%%%%%%%%%%%%%%%%%%%%%%%%%%%%%%%%%%%%%%%%%%%%%%%%%%%%%%%%%%%%%%%%%%%%%%%%%%%%%%%%%%%%%%%%%%%%%%%%%%%%%%%%%%%%%%%%%%%%%%%%%%%%%%%%%%%%%%%%%%%%%%%%%%%%%%%%%%%%%%%%%%%%%%%%%%%%%%%%%%%%%%%%%%%%%%%%%%%%%%%%%%%%%%%%%%%%%%%%%%%%%%%%%%%%%%%%%%%%%%%

\usepackage{eurosym}
\usepackage{vmargin}
\usepackage{amsmath}
\usepackage{graphics}
\usepackage{epsfig}
\usepackage{enumerate}
\usepackage{multicol}
\usepackage{subfigure}
\usepackage{fancyhdr}
\usepackage{listings}
\usepackage{framed}
\usepackage{graphicx}
\usepackage{amsmath}
\usepackage{chngpage}

%\usepackage{bigints}
\usepackage{vmargin}

% left top textwidth textheight headheight

% headsep footheight footskip

\setmargins{2.0cm}{2.5cm}{16 cm}{22cm}{0.5cm}{0cm}{1cm}{1cm}

\renewcommand{\baselinestretch}{1.3}

\setcounter{MaxMatrixCols}{10}

\begin{document}

\begin{enumerate}
%%%%%%%%%%%%%%%%%%%%%%%%%%%%%%%%%%%%%%%%%%%%%%%%%%%%%%%%%%%%%%%%%%
\item 12 A psychologist conducted an investigation into the effect of alcohol on reaction times
using 10 male and 10 female subjects. Each subject was given two tests on different
days, during which his/her reaction times were recorded.
Before each of the tests, the subject drank a glass of liquid. Some glasses contained a
fixed quantity of alcohol and others contained a liquid which had a similar colour and
taste but no alcohol. Each subject drank one glass of each type. The order of
presentation was randomized, independently for each subject.
The data below give the reaction times, in units of 0.01 seconds. Also given is the
difference between the reaction time with alcohol and the reaction time without
alcohol for each subject (reaction time with alcohol minus reaction time without).
Males
Subject No. 1 2 3 4 5 6 7 8 9 10
With alcohol 45 51 35 43 51 54 51 49 44 52
Without alcohol 40 54 21 31 44 47 39 33 32 56
Difference 5 3 14 12 7 7 12 16 12 4
Females
Subject No. 1 2 3 4 5 6 7 8 9 10
With alcohol 47 54 58 48 60 46 55 74 56 49
Without alcohol 39 40 42 30 51 41 55 68 47 40
Difference 8 14 16 18 9 5 0 6 9 9
(i) (a) Construct a 95% confidence interval for the mean difference between
the reaction times with and without alcohol for the males, using the 10
difference values.
(b) Construct a similar 95% confidence interval based on the female
difference values.
(c) Comment briefly on the two confidence intervals. [8]
(ii) (a) Perform a two-sample t-test to investigate whether the alcohol effect
differs between males and females.
(b) Show that the variances in the male and female samples are not
significantly different at the 5% level, and comment briefly with
reference to the test conducted in (ii)(a). [8]
%%%%%%%%%%%%%%%%%%%%%%%%%%%%%%%%%%%%%%%%%%%%%%%%%%%%%%%%%%%%%%%%%%
\item 13 The table below gives the frequency of coronary heart disease by age group. The
table also gives the age group midpoint (x) and
ˆ
= log 1 ˆ
y
  
     
, where ˆ denotes the
proportion in an age group with coronary heart disease.
Coronary Heart
Disease
Age group x Yes No n y
20–29 25 1 9 10 2.19722
30–34 32.5 2 13 15 1.87180
35–39 37.5 3 9 12 1.09861
40–44 42.5 5 10 15 0.69315
45–49 47.5 6 7 13 0.15415
50–54 52.5 5 3 8 0.51083
55–59 57.5 13 4 17 1.17865
60–69 65 8 2 10 1.38629
x = 360; x2 =17437.5;y =  2.9392; y2 =13.615; xy =  9.0429
(i) (a) Calculate an estimate of the probability of having coronary heart
disease under the assumption that the probability does not differ over
the age groups.
(b) Construct an 8 	 2 contingency table with marginal totals and conduct
a 2
 test for differences in the probability of having coronary heart
disease for the different age groups. [8]
(ii) Consider the regression model y =  x .
(a) Draw a scatterplot of y against x, and comment on the appropriateness
of the suggested model.
(b) Calculate the least squares fitted regression line of y on x.
(c) Calculate a 99% confidence interval for the slope parameter.
(d) Interpret the result obtained in (i)(b) with reference to the confidence
interval obtained in (ii)(c). 
\end{enumerate}
%%%%%%%%%%%%%%%%%%%%%%%%%%%%%%%%%%%%%%%%%%%%%%%%%%%%%%%%%%%%%%%%%%

\newpage

12 (i) (a) Males: n1 = 10 x1= 7.8 s1 = 6.8605
95% CI for male data:
%%%%%%%%%%%%%%%%%%%%%%%%%%%%%%%%%%%%%%%
Page 8
x1 
 t9(0.025) 1
1
s
n
= 7.8 
 2.262 6.8605
10
= 7.8 
 4.907
= (2.89, 12.71)
(b) Females: n2 = 10 x2 = 9.4 s2 =5.37897
95% CI for female data:
2
2 9
2
x t (0.025) s
n

= 9.4 2.262 5.37897
10
 = 9.4 
 3.848
= (5.55, 13.25)
(c) Neither of the intervals include zero, and therefore there is evidence
that the alcohol has an effect on reaction times, i.e. it increases the
reaction time.
(ii) (a) Two sample t-test
1 2
2
1 2
1 1
p
t x x
s
n n


 
  
 	
2 2
2 1 1 2 2
1 2
( 1) ( 1)
p 2
s n s n s
n n
  

 
9(6.8605)2 9(5.37897)2
18

 = 37.9999
sp = 6.1644
2
10
7.8 9.4 1.6 0.58
6.1644 2.7568
t  
   
t18(0.25) = 0.688, and probability value is > 0.5.
There is no evidence to reject the null hypothesis that the means for
males and females are the same.
%%%%%%%%%%%%%%%%%%%%%%%%%%%%%%%%%%%%%%%%%%%%%%%%%%%%%%%%%%%%%%%%%%
We can conclude that alcohol has a similar effect.
(b)
2
1
2
2
s 1.626
s

The F9,9 distribution has upper 5\% critical point at 3.179. Our
observed value (1.626) is well within the main body of the distribution
and is not significant at the 10\% level of testing (and therefore not at
the 5\% level). There is no evidence to suggest that the variances
differ. The assumption of common variance was made when
conducting the test in (ii)(a).
%%%%%%%%%%%%%%%%%%%%%%%%%%%%%%%%%%%%%%%%%%%%%%%%%%%%%%%%%%%%%%%%%%
13 (i) ni = number in group i , ri = number with coronary heart disease in group i.
(a) ri = 43 ni =100
Estimate of the probability of having coronary heart disease is given by
ˆ = = 43 = 0.43.
100
i
i
r
n



(Assuming constant probability of having coronary heart disease over
age groups.)
(b)
Coronary heart
disease
Yes No Total
Age groups 20–29 1
(4.30)
9
(5.70)
10
30–34 2
(6.45)
13
(8.55)
15
35–39 3
(5.16)
9
(6.84)
12
40–44 5
(6.45)
10
(8.55)
15
45–49 6
(5.59)
7
(7.41)
13
50–54 5
(3.44)
3
(4.56)
8
55–59 13
(7.31)
4
(9.69)
17
60–69 8
(4.30)
2
(5.70)
10
Total 43 57 100
%%%%%%%%%%%%%%%%%%%%%%%%%%%%%%%%%%%%%%%%%%%%%%%%%%%%%%%%%%%%%%%%%%
The expected values assuming a constant probability of having coronary heart disease
are given in parentheses (= row total ˆ ).
2 2 2
2 = ( ) = (1 4.30) ... (2 5.7)
4.30 5.7
i i
i
f e
e
  
    = 26.6 on 7 d.f.
2
7 (0.01) 18.48.
Strongly reject null hypothesis of constant probability over the
different age groups. [Note: If one decides to combine cells to
safeguard against very low expected frequencies, one should combine
adjacent cells.]
(ii) (a)
Linear model seems appropriate, but extremes (x = 25 and x = 65) are
not as good as 32.5–57.5 age range.
(b) Sxx = 17437.5
3602
8  = 1237.5
Syy =
( 2.9392)2 13.615
8

 = 12.535
Sxy = 9.0429 (360)( 2.9392)
8

  = 123.22
Least squares estimates:
ˆ = = 123.22 = 0.09957
1237.5
xy
xx
S
S 
25 35 45 55 65
1
0
-1
-2
x
y
%%%%%%%%%%%%%%%%%%%%%%%%%%%%%%%%%%%%%%%
Page 11
ˆ = y ˆ x =  0.3674  0.09957(45) =  4.85
(c)
2
ˆ 2 = xy /( 2)
yy
xx
S
S n
S
 
    
 
 
(123.22)2 = 12.535 / 6
1237.5
 
    
 
= 0.04430
 ˆ = 0.210 on 6 d.f.
s.e.(ˆ
) =
ˆ 0.210
Sxx 1237.5

 = 0.0060
99% CI for 6 ˆ ˆ : (0.005) t   s.e. (ˆ
)
= 0.0996  3.707 (0.0060)
= 0.0996  0.0222 i.e. (0.0774, 0.1218)
(d) In (i)(b), the probability of having coronary heart disease was found to vary with age. The 99\% confidence interval for the slope parameter in the regression obtained in (ii)(c) also shows that the probability of having coronary heart disease depends (linearly) on age as zero is not within the interval.

\end{document}
