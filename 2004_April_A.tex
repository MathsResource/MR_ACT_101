Faculty of Actuaries Institute of Actuaries
EXAMINATIONS
26 April 2004 (pm)
Subject 101 Statistical Modelling
Time allowed: Three hours
%%%%%%%%%%%%%%%%%%%%%%%%%%%%%%%%%%%%%%%%%%%%%%%%%%%%%%%%%%%%%%%%%%%%%%%%%%%%%%%%%%%%%%%%%%%%%%%%%%%%%%%%%%%
1 The probability that a claim is made on a certain type of policy in a particular year is
0.04. Five hundred policies are selected at random.
Use a suitable normal approximation to calculate the probability that no more than 30
of these will result in a claim during the year. [3]
2 The interquartile range of a continuous distribution with distribution function F(x) is
defined as IQR = x2 x1 where F(x1) = 0.25 and F(x2) = 0.75.
Show that for a normal distribution with variance 2, the interquartile range is
IQR = 1.349 . [3]
3 The cumulant generating function of a random variable X is given by:
10 CX (t) = logMX (t) = 2 1 t 1
where MX (t) is the moment generating function.
Determine the mean and variance of the distribution of X. [3]
4 Claim sizes in a certain insurance situation are modelled by an exponential
distribution with mean $20,000. The insurer defines a claim to be a large claim if the
claim size exceeds $35,000.
State, with a reason, the expected size of a large claim. [2]

%%%%%%%%%%%%%%%%%%%%%%%%%%%%%%%%%%%%%%%%%%%%%%%%%%%%%%%%%%%%%%%%%%%%%%%%%%%%%%%%%%%%%%%%%%%%%%%%%%%%%%%%%%%
Faculty of Actuaries Institute of Actuaries
EXAMINATIONS
April 2004
Subject 101 Statistical Modelling
EXAMINERS REPORT
Faculty of Actuaries
Institute of Actuaries
%%%%%%%%%%%%%%%%%%%%%%%%%%%%%%%%%%%%%
Page 2

1 X = number of policies where a claim is made.
X bi (500, 0.04)
X N((500)(0.04), 500(0.04)(0.96)) N(20, 19.2)
P(X 30) P(X 30.5) using continuity correction
30.5 20 10.5
(2.40)
19.2 19.2
= 0.9918
2 The required values of x1 and x2, the lower and upper quartiles, are such that:
F(x1) = 0.25 and F(x2) = 0.75,
i.e. 1 0.25
x
and 2 0.75,
x
where is the standard normal distribution function.
From the statistical table of the percentage points of the standard normal distribution,
we have
x1
= 0.6745 x1 = 0.6745
x2
= 0.6745 x2 = + 0.6745
where is the population mean and is the population standard deviation.
IQR = x2 x1
= 0.6745 + 0.6745
= 1.349 .
%%%%%%%%%%%%%%%%%%%%%%%%%%%%%%%%%%%%%
Page 3
3 C´(t) = 20(1 t) 11 , C´´(t) = 220(1 t) 12
E[X] = C´(0) = 20
V[X] = C´´(0) = 220
[OR as coefficients of t and of t2/2! in expansion of C(t)]
4 Answer: $35,000 + 20,000 = $55,000
Reason: the memoryless property of the exponential distribution (the excess above
35,000 itself has an exponential distribution with mean 20,000).
[Note: relatively few candidates were able to exploit the memoryless property of the
exponential distribution to advantage.]
