\documentclass[a4paper,12pt]{article}

%%%%%%%%%%%%%%%%%%%%%%%%%%%%%%%%%%%%%%%%%%%%%%%%%%%%%%%%%%%%%%%%%%%%%%%%%%%%%%%%%%%%%%%%%%%%%%%%%%%%%%%%%%%%%%%%%%%%%%%%%%%%%%%%%%%%%%%%%%%%%%%%%%%%%%%%%%%%%%%%%%%%%%%%%%%%%%%%%%%%%%%%%%%%%%%%%%%%%%%%%%%%%%%%%%%%%%%%%%%%%%%%%%%%%%%%%%%%%%%%%%%%%%%%%%%%

\usepackage{eurosym}
\usepackage{vmargin}
\usepackage{amsmath}
\usepackage{graphics}
\usepackage{epsfig}
\usepackage{enumerate}
\usepackage{multicol}
\usepackage{subfigure}
\usepackage{fancyhdr}
\usepackage{listings}
\usepackage{framed}
\usepackage{graphicx}
\usepackage{amsmath}
\usepackage{chngpage}

%\usepackage{bigints}
\usepackage{vmargin}

% left top textwidth textheight headheight

% headsep footheight footskip

\setmargins{2.0cm}{2.5cm}{16 cm}{22cm}{0.5cm}{0cm}{1cm}{1cm}

\renewcommand{\baselinestretch}{1.3}

\setcounter{MaxMatrixCols}{10}

\begin{document}
% 18 September 2000 (pm)
% Subject 101 — Statistical Modelling
% Faculty of Actuaries Institute of Actuaries
% EXAMINATIONS
%%%%%%%%%%%%%%%%%%%%%%%%%%%%%%%%%%%%%%%%%%%%%%%%%%%%%%%%%%%%%%%%%%%%%%%%
\begin{enumerate}
\item 5 The number of claims which arise under a policy of a particular type in a year is
to be modelled as a Poisson($\lambda$) random variable. A random sample of 500 such policies gave rise to a total of 84 claims in 1999.
Calculate a 95\% confidence interval for $\lambda$. 
%%%%%%%%%%%%%%%%%%%%%%%%%%%%%%%
\item Suppose that the linear regression model
\[Y = \alpha + \beta x + e\]
is fitted to data {(yi , xi) : i = 1, 2, … , n}, where y is the salary (£) of a company
manager and x (years) is the number of years of relevant experience of that
manager.
State the units of measurement (if any) of
\begin{enumerate}
    \item  $\hat{\alpha}$ , the estimate of $\alpha$ ,
\item $\hat{\beta}$ , the estimate of $\beta$ ,
\item $R^2$ , the coefficient of determination of the fit. 
\end{enumerate}

\newpage

\item In a correlation analysis based on a random sample of 10 values from a bivariate
normal distribution, a t-test of
H0 : ρ = 0 v. H1 : ρ > 0
results in a probability-value of 0.025.
Calculate the value of the sample correlation coefficient. 
\newpage
\item  Claims on a certain class of policy are classified as being of two types, I and II.
Past experience has shown that:
25\% of claims are of type I and 75\% are of type II;
Type I claim amounts have mean £500 and standard deviation £100;
Type II claim amounts have mean £300 and standard deviation £70.
Calculate the mean and the standard deviation of the claim amounts on this
class of policy. [6]
\end{enumerate}

%%%%%%%%%%%%%%%%%%%%%%%%%%%%%%%%%%%%%%%%%%%%%%%%%%%%%%%%%%%%%%%%%%%%%%%%%%%%%%%%%%%%

8 Let Y = amount
Let X = 1, 2 for types I, II
∴ P(X = 1) = 0.25, P(X = 2) = 0.75
E(YX = 1) = 500, Var(YX = 1) = 1002
all given
E(YX = 2) = 300, Var(YX = 2) = 702
E(Y) = E(E(YX)) = 500(0.25) + 300(0.75)
= 125 + 225 = £350

%%%%%%%%%%%%%%%%%%%%%%%%%%%%%%%%%%%%%%%%%%%%%%%%%%%%%%%%%
Page 4
V(Y) = E(V(YX)) + V(E(YX))
E(V(YX)) = 1002(0.25) + 702(0.75)
= 2500 + 3675 = 6175
V(E(YX)) = 5002(0.25) + 3002(0.75) − 3502
= 62500 + 67500 − 122500 = 7500
∴ V(Y) = 6175 + 7500 = 13675
∴ s.d.(Y) = £116.9
OR: V(E(YX)) = 0.25(500 − 350)2 + 0.75(300 − 350)2 = 7500
alternative method for V(Y):
E(Y2X = 1) = 1002 + 5002 = 260000
E(Y2X=2) = 702 + 3002 = 94900
∴ E(Y2) = 0.25(260000) + 0.75(94900)
= 136175
∴ V(Y) = 136175 − 3502 = 13675
∴ s.d.(Y) = 116.9
Q8 Comment: Very few candidates seemed to be aware of the result V(Y) = E[V(Y|X)]
+ V[E(Y|X)] and how and when it should be used.
9 C(t) = log M(t) = −$\alpha$log(1− θt)
C′(t) = $\alpha$θ(1 − θt)−1 , C′′(t) = $\alpha$θ2(1 − θt)−2 , C′′′(t) = 2$\alpha$θ3(1 − θt)−3
∴ κ2 = C′′(0) = $\alpha$θ2 , κ3 = C′′′(0) = 2$\alpha$θ3 so coefficient is
κ3 / (κ2)3/2 = 2$\alpha$θ3 / ($\alpha$θ2)3/2 = 2 / √$\alpha$

13 ( ) i E Y⋅ = μ + τi and E(Y⋅⋅ ) = μ
( ) i E Y⋅ E(μˆ ) = E(Y⋅⋅ ) = μ ∴ unbiased
(ˆ ) i E τ = ( ) i E Y⋅ −Y⋅⋅ = μ + τi − μ = τi ∴ unbiased
%%%%%%%%%%%%%%%%%%%%%%%%%%%%%%%%%%%%%%%%%%%%%%%%%%%%%%%%%
Page 7
V(μˆ ) = V(Y⋅⋅ ) =
2
kr
σ
as Y⋅⋅ is mean of kr r.v.’s each with var σ2
(ˆ ) i V τ = ( ) i V Y⋅ −Y⋅⋅
= ( ) ( ) 2 ( , ) i i V Y⋅ + V Y⋅⋅ − Cov Y⋅ Y⋅⋅
=
2 2
1 1 2
2 . .r
r kr r kr
σ + σ − σ
=
2 2
r kr
σ − σ =
(k 1) 2
kr
− σ
%%%%%%%%%%%%%%%%%%%%%%%%%%%%%%%%%%%%%%%%%%%%%%%%%%
\medskip 
Alternative method:
( ) i V Y⋅ −Y⋅⋅ =
1 1
1 i j
j i
V Y Y
k ⋅ k ≠ ⋅
    −  − Σ 
   
=
1 2 2 1 2 2
1 (k 1)
k r k r
 σ  σ
 −  + −  −
   
=
2
k2r
σ {(k − 1)2 + (k − 1)} =
(k 1) 2
kr
− σ
Q13 Comment: This material was unfamiliar to most candidates.

\end{document}
