
9 An insurance company specifies in its advertising literature that 75% of all small
claims (less than £200) on household policies are fully settled within one month. As
part of an internal audit a random sample of 200 small claims is examined.
(i) It is found that 146 of these small claims were fully settled within one month.
Obtain an approximate one-sided 99% confidence interval for the true
percentage of all small claims which are fully settled within one month. [3]
(ii) The mean and standard deviation of the individual claim amounts were
calculated as x  £112.41and s = £51.62 respectively. Obtain an approximate
two-sided 99% confidence interval for the population mean of all small claim
amounts. [2]
[Total 5]
%%%%%%%%%%%%%%%%%%%%%%%%%%%%%%%%%%%%%%%%%%%%%%%%%%%%%%%%%%%%%%%%%%%%%%%%%%%%%%%%%%%%%%%%%%%%%%%%
10 In an investigation into the comparison of claim amounts between three different
regions, a random sample of 10 independent claim amounts was taken from each
region and an analysis of variance was performed. The resulting ANOVA table is
given below with some entries omitted:
Source of variation d.f. SS MSS
Between regions 2 4439.7 2219.9
Residual * * *
Total 29 15153.2
(i) Calculate the missing entries in this table and perform the appropriate F-test to
determine whether there are significant differences between the mean claim
amounts for the three regions. [3]
(ii) The three sample means were given by:
Region A B C
Sample mean 147.47 154.56 125.95
Given that it was of particular interest to compare regions A and B, calculate a
95% confidence interval for the difference in their means and comment on
your answer in the light of the F-test performed in part (i). [4]
%%%%%%%%%%%%%%%%%%%%%%%%%%%%%%%%%%%%%%%%%%%%%%%%%%%%%%%%%%%%%%%%%%%%%%%%%%%%%%%%%%%%%%%%%%%%%%%%
11 Let S  X1  X2 ... XN (and S = 0 if N = 0) where the Xi ’s are independent and
identically distributed as exponential random variables with mean  and are also
independent of N, which has a Poisson distribution with mean .
(i) Prove, from first principles (that is without quoting any general results for
compound distributions), that the moment generating function of S, MS(t), is
given by    ( ) = exp 1 1 1 MS t t        	
. [4]
(ii) Using the expression for the moment generating function of S in (i) (and
without quoting any general results for compound distributions), derive an
expression for the variance of S. [4]
[Total 8]
101 A2003—5 PLEASE TURN OVER


9 (i) ˆ 146 0.73
200
p 
The usual two-sided 99% confidence interval for the proportion p would be
pˆ  2.58 s.e.( pˆ ) to pˆ  2.58 s.e.( pˆ )
Given the nature of the claim, the appropriate one-sided 99% confidence
interval for the proportion p is of the form (0, pU),
where 0.73 2.326 (0.73)(0.27) 0.73 0.073
U 200 p     giving (0, 0.803)
For percentage: (0, 80.3%)
(ii) 99% confidence interval for the mean  is
x 2.58 s
n
 for large n
112.41 2.58 51.62 £112.41 9.42
200
    or (£102.99, £121.83)
Note: using 2.576 gives £112.41 9.40 or (£103.01, £121.81)

%%%%%%%%%%%%%%%%%%%%%%%%%%%%%%%%%%%%%%%%%%%%%%%%%%%%%%%%%%%%%%%%%%%%%%%%%%%%%%%%%%%%%%%%%%%%%%%%
10 (i) Missing entries are:
d.f. = 27 by subtraction
SS = 10713.5 by subtraction
MS = 396.8 being 10713.5/27.
Observed F = 5.59 on 2 and 27 d.f.
From tables the 1% critical point is 5.488
Significant at 1%, so there is quite strong evidence of a difference in the mean
claim amounts for the three regions.
(ii) 95% confidence interval for (A  B ) , or equivalently (A  B ) , is
0.025,27
( ) ˆ 1 1
A B 10 10 y  y  t  
giving
(147.47 154.56) 2.052 396.8 1 1
10 10
  
	7.09  18.28 or (	25.37, 11.19)
This comfortably contains zero indicating no difference between the
underlying means for regions A and B.
The significant result of the F-test clearly comes from region C mean being
much lower than the region A and B means.
11 (i) MS(t) = E[etS] = E[E(etS|N)]
E[etS|N=n] = E[exp{t(X1 + X2 +…+ Xn}|N=n]
= E[exp{t(X1 + X2 +…+ Xn}] (since the Xi’s are independent of N)
= 
 E[exp(tXi)] (since the Xi’s are iid)
= {MX(t)}n
 MS(t) = E[{MX(t)}N] = E[exp{NlogMX(t)}] = MN{logMX(t)}
Here MN(t) = exp{(et – 1)} and MX(t) = (1 	t)1 and so
     exp 1 1 1 MS t t       	 

(ii) MS′(t) = MS(t)  (1 	 t)2
Subject 101 (Statistical Modelling) — April 2003 — Examiners’ Report
Page 6
MS′′(t) = MS(t)  22 (1 	 t)3 + MS′(t)  (1 	 t)2
So E[S] = MS′(0) = 
E[S2] = MS′′(0) = 22 + (  
22 + 22
V[S] = 22 + 22	22
22
Alternative solution using the cumulant generating function
Let CS(t) = logMS(t) = {(1 - t)-1 – 1}
CS (t) = (1 - t)-2 , CS(t) = 2(1 - t)-3
So E[S] = CS (0) =  and V[S] = CS(0) = 22
