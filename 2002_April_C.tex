\documentclass[a4paper,12pt]{article}

%%%%%%%%%%%%%%%%%%%%%%%%%%%%%%%%%%%%%%%%%%%%%%%%%%%%%%%%%%%%%%%%%%%%%%%%%%%%%%%%%%%%%%%%%%%%%%%%%%%%%%%%%%%%%%%%%%%%%%%%%%%%%%%%%%%%%%%%%%%%%%%%%%%%%%%%%%%%%%%%%%%%%%%%%%%%%%%%%%%%%%%%%%%%%%%%%%%%%%%%%%%%%%%%%%%%%%%%%%%%%%%%%%%%%%%%%%%%%%%%%%%%%%%%%%%%

\usepackage{eurosym}
\usepackage{vmargin}
\usepackage{amsmath}
\usepackage{graphics}
\usepackage{epsfig}
\usepackage{enumerate}
\usepackage{multicol}
\usepackage{subfigure}
\usepackage{fancyhdr}
\usepackage{listings}
\usepackage{framed}
\usepackage{graphicx}
\usepackage{amsmath}
\usepackage{chngpage}

%\usepackage{bigints}
\usepackage{vmargin}

% left top textwidth textheight headheight

% headsep footheight footskip

\setmargins{2.0cm}{2.5cm}{16 cm}{22cm}{0.5cm}{0cm}{1cm}{1cm}

\renewcommand{\baselinestretch}{1.3}

\setcounter{MaxMatrixCols}{10}

\begin{document}

\begin{enumerate}
\item 9 Let X and Y be independent random variables. Let V and W be the random variables
defined by
\[V = max {X, Y}, W = min {X, Y}\]
i.e. V is the larger, and W is the smaller, of the observations of X and Y.
Let FX , FY , FV , FW denote the distribution functions of X, Y, V, W respectively.
\begin{enumerate}[(a)]
\item Show that FV(t) = FX(t)FY(t) . 
\item Show that FW(t) = FX(t) + FY(t)  FX(t)FY(t) . [3]
\item The random variable X has an exponential distribution with parameter 4 and,
independently, Y has an exponential distribution with parameter 4. Obtain the
distribution function of the minimum of X and Y and state its mean. [3]
\end{enumerate}

%%%%%%%%%%%%%%%%%%%%%%%%%%%%%%%%%%%%%%%%%%%%%%%%%%%%%%%%%%%%%%%%%%%%%%%%%%%%%%%%%%%%%%%%%%%%%%%%%%%%%%%%%%%%%%%%%%%%%%%%%%%%%%%%%%%%%%%%%%
\item 10 A company wants to estimate the percentage of its customers who are willing to shop
on the internet. It decides to do so by calculating a symmetrical 95% two-sided
confidence interval for the unknown percentage.
\begin{enumerate}[(a)]
\item Show that, based on a random sample of 200 of the company’s customers, the
required confidence interval will have a width which is no greater than 13.9%.
[4]
\item Calculate the sample size required which will ensure that, whatever the true
percentage, the width of the confidence interval will be no greater than 10%.
\end{enumerate}
%%%%%%%%%%%%%%%%%%%%%%%%%%%%%%%%%%%%%%%%%%%%%%%%%%%%%%%%%%%%%%%%%%%%%%%%%%%%%%%%%%%%%%%%%%%%%%%%%%%%%%%%%%%%%%%%%%%%%%%%%%%%%%%%%%%%%%%%%%
\item 11 Consider the following model for aggregate claim amounts S:
S = X1 + X2 + …+ XN
where the Xi are independent, identically distributed random variables representing
individual claim amounts and N is a random variable, independent of the Xi , and
representing the number of claims. Let X have mean X and let N have mean N and
variance 2
N .
\begin{enumerate}[(a)]
\item Show that
E(SN) = X (2N  2N )
by considering expected values conditional on the value of N. [3]
\item Hence derive an expression for the covariance between S and N. 
\end{enumerate}
\end{enumerate}
%%%%%%%%%%%%%%%%%%%%%%%%%%%%%%%%%%%%%%%%%%%%%%%%%%%%%%%%%%%%%%%%%%%%%%%%%%%%%%%%%%%%%%%%%%%%%%%%%%%%%%%%%%%%%%%%%%%%%%%%%%%%%%%%%%%%%%%%%%

9
\begin{eqnarray*}
F_V (t)\\ &=& P(V \leq t)\\ &=& P(max \{X, Y\} \leq t)
\\ &=& P(X \leq t \mbox{ and } Y \leq t)\\ &=& P(X \leq t) P(Y \leq t)\\ &=& F_X (t)F_Y (t)\mbox{ as X  and  Y are independent} 
\end{eqnarray*}

\begin{eqnarray*}
F W (t)\\ &=& P(W \leq t)\\ &=& 1 - P(min{X, Y} > t)
\\ &=& 1 – P(X > t \mbox{ and } Y > t)\\ &=& 1 - P(X > t) P(Y > t) \mbox{ as X  and  Y are independent} 
\\ &=& 1 – [1 – P(X \leq t)] [1 – P(Y \leq t)]
\\ &=& 1 – (1 – F_X (t)) (1 – F_Y (t))
\\ &=& F_X (t) + F_Y (t) – F_X (t)F_Y (t) .
\end{eqnarray*}

\begin{itemize}

\item
FX(t) = FY(t) = 1 – e4t , so
FW(t) = 1 – e4t + 1 – e4t – (1 – e4t)2
= 2 – 2e4t – 1 – e8t + 2e4t
= 1 – e8t
This is the distribution function of an exponential distribution with parameter 8, and
therefore the mean is 1/8.
\end{itemize}

10  
\begin{itemize}
\item Width of 95% CI = 2  1.96  {P(1 – P)/200}0.5 where P is sample
proportion
Max value of P(1 – P) is 0.52 = 0.25
 Max width of CI = 2  1.96  (0.25/200)0.5 = 0.139
In terms of percentages, this is 13.9%.
\item 2  1.96  (0.25/n)0.5  0.1  n  385
\end{itemize}
11 
\begin{itemize}
\item E(SN) = E[E(SN|N)]
Now, E(SN|N = n) = E[(X1 + …+ Xn)n|N = n)] = E[n(X1 + …+ Xn)]
= n  nX = n2
X
 E(SN) = E(XN2) = XE(N2)
= X
( 2 2 ) N  N

%%%%%%%%%%%%%%%%%%%%%%%%%%%%%%%%%%%%%%%%%%%%%%%%%%%%%%%%%%%%%%%%%%%%%%%%%%%%%%%%%%%%%%%%%%%%%%%%%%%%%%%%%%%%%%%%%%%%%%%%%%%%%%%%%%%%%%%%%%
\item E(S) = E(NX) = NX
 Cov(S,N) = E(SN) – E(S)E(N) = X
( 2 2 ) N  N – (NX)N = X
2
N

\end{itemize}
\end{document}
