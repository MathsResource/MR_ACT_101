
\documentclass[a4paper,12pt]{article}

%%%%%%%%%%%%%%%%%%%%%%%%%%%%%%%%%%%%%%%%%%%%%%%%%%%%%%%%%%%%%%%%%%%%%%%%%%%%%%%%%%%%%%%%%%%%%%%%%%%%%%%%%%%%%%%%%%%%%%%%%%%%%%%%%%%%%%%%%%%%%%%%%%%%%%%%%%%%%%%%%%%%%%%%%%%%%%%%%%%%%%%%%%%%%%%%%%%%%%%%%%%%%%%%%%%%%%%%%%%%%%%%%%%%%%%%%%%%%%%%%%%%%%%%%%%%

\usepackage{eurosym}
\usepackage{vmargin}
\usepackage{amsmath}
\usepackage{graphics}
\usepackage{epsfig}
\usepackage{enumerate}
\usepackage{multicol}
\usepackage{subfigure}
\usepackage{fancyhdr}
\usepackage{listings}
\usepackage{framed}
\usepackage{graphicx}
\usepackage{amsmath}
\usepackage{chngpage}

%\usepackage{bigints}
\usepackage{vmargin}

% left top textwidth textheight headheight

% headsep footheight footskip

\setmargins{2.0cm}{2.5cm}{16 cm}{22cm}{0.5cm}{0cm}{1cm}{1cm}

\renewcommand{\baselinestretch}{1.3}

\setcounter{MaxMatrixCols}{10}

\begin{document}
\begin{enumerate}
%%%%%%%%%%%%%%%%%%%%%
CT3 A2005
2
[5]8
9
The distribution of claim size under a certain class of policy is modelled as a normal
random variable, and previous years records indicate that the standard deviation is
£120.
(i) Calculate the width of a 95% confidence interval for the mean claim size if a
sample of size 100 is available.
[2]
(ii) Determine the minimum sample size required to ensure that a 95% confidence
interval for the mean claim size is of width at most £10 .
[2]
(iii) Comment briefly on the comparison of the confidence intervals in (i) and (ii)
with respect to widths and sample sizes used.
[1]
[Total 5]
Let X 1 , , X n denote a large random sample from a distribution with unknown
population mean and known standard deviation 3. The null hypothesis H 0 : = 1 is
to be tested against the alternative hypothesis H 1 : > 1, using a test based on the
sample mean with a critical region of the form X
k , for a constant k.
It is required that the probability of rejecting H 0 when = 0.8 should be
approximately 0.05, and the probability of not rejecting H 0 when = 1.2 should be
approximately 0.1.
(i)
Show that the test requires
k 0.8
3/ n
where
(ii)
0.95 and
k 1.2
3/ n
0.10
is the standard normal distribution function.
[4]
The values for the sample size n and the critical value k which satisfy the
requirements of part (i) are n = 482 and k = 1.025 (you are not asked to verify
these values).
Calculate the approximate level of significance of the test, and comment on
the value.
[3]
[Total 7]
CT3 A2005
3
PLEASE TURN OVER10
A model used for claim amounts (X, in units of £10,000) in certain circumstances has
the following probability density function, f(x), and cumulative distribution function,
F(x):
f ( x ) =
5(10 5 )
(10 x )
, x
6
0 ; F ( x ) = 1
10
10 x
5
.
You are given the information that the distribution of X has mean 2.5 units (£25,000)
and standard deviation 3.23 units (£32,300).
(i) Describe briefly the nature of a model for claim sizes for which the standard
deviation can be greater than the mean.
[2]
(ii) (a)
Show that we can obtain a simulated observation of X by calculating
x = 10 (1 r )
0.2
1
where r is an observation of a random variable which is uniformly
distributed on (0,1).
(b)
Explain why we can just as well use the formula
x = 10 r
0.2
1
to obtain a simulated observation of X.
(c)
Calculate the missing values for the simulated claim amounts in the
table below (which has been obtained using the method in (ii)(b)
above):
r Claim (£)
0.7423
0.0291
0.2770
0.5895
0.1131
0.9897
0.6875
0.8525
0.0016
0.5154 6,141
10,2872
29,272
11,148
54,635
207
7,782
3,243
?
?
[5]
%%%%%%%%%%%%%%%%%%%%%%%%%%%%%%%%%%%%%%%%%%%%%%%%%%%%%%%%%%%%%%%%%%%%%%%%%%%%%%%%%%%%%

Width of 95% confidence interval: 1.96
120
120
[or 2 1.96
n
n
3.92
120
]
n
120
23.52
100
120
[or 2 1.96
£47.04]
100
1.96
(ii)
For the width of a 95% confidence interval to be at most
120
1.96
10
n
n
1.96 120
10
23.52 , n
10 we require
553.19
i.e., take the sample size as 554.
Page 3Subject CT3 (Probability and Mathematical Statistics Core Technical)
9
(iii) The confidence interval in (ii) in narrower
much larger sample size.
(i) X approx
0.05
N
,
3 2
n
April 2005
Examiners Report
to achieve this we require a
for large n by the central limit theorem.
P(reject H 0 | = 0.8) = P( X > k| = 0.8)
k 0.8
3/ n
= 1
k 0.8
3/ n
0.95
and
0.1
P(do not reject H 0 | = 1.2)
= P ( X
1.2)
k 1.2
3/ n
=
(ii)
k |
Significance level
1
= P(reject H 0 when H 0 is true) = P ( X
1.025 1
3 / 482
1
k |
1)
(0.18) 1 0.57 0.43.
The significance level of the test is very high (43%).
10
(i) X takes positive values only so to have such a relatively high standard
deviation the distribution must be positively skewed with sizeable probability
associated with high values (i.e. the model embraces high claim sizes; the
density has a long or heavy tail).
(ii) (a) Solving r = F(x)
(b) R ~ U(0,1) 1 R ~ U(0, 1) so (1 r) is also a random number from
(0, 1), so we can use 1 r in place of r , giving the formula
x 10 r
(c)
Page 4
0.2
r = 0.0016
r = 0.5154
(1 + x/10) = (1 r)
1
claim = 262390
claim = 14175
0.2
x = 10[(1
r)
0.2
1]Subject CT3 (Probability and Mathematical Statistics Core Technical)
(i)
SS T = 77249 1203 2 /20 = 4888.55
SS B = 387 2 /8 + 254 2 /4 + 270 2 /4 + 292 2 /4
SS R = 2857.875
Examiners Report
1203 2 /20 = 2030.675
H 0 : no treatment effects (i.e. population means are equal) v H 1 : not H 0
Analysis of Variance
Source
DF
SS
Factor
3
2031
Error
16
2858
Total
19
4889
MS
677
179
F
3.79
F 3,16 (0.05) = 3.239, F 3,16 (0.01) = 5.292
P-value is lower than 0.05 (but higher than 0.01), so we can reject H 0 at least
at the 5% level of testing (Note: actually P-value is 0.032). The data do
indicate significant differences amongst the treatment means.
(ii)
(a)
Residual = observed value treatment mean
Treatment means are: Control 48.375, A 63.5, B 67.5, C 73.0
Missing values are:
Control
Preparation A
Preparation B
Preparation C
5.4
9.5
16.5
27
8.4
8.5
4.5
18
16.6
2.5
16.5
11
2.6
1.5
4.5
2
15.4
9.4
5.6
13.6
(b)
20
10
11
April 2005
0
-10
-20
-30
means
treatment
50
Control
60
70
A
B
C
Page 5Subject CT3 (Probability and Mathematical Statistics Core Technical)
(iii)
April 2005
Examiners Report
(c) Observations Y ij (j th value for treatment i) are independent and
normally distributed with variance 2 which is constant across
treatments.
(d) The assumptions seem reasonable
with the exception of the constant
variance assumption, which is questionable
the data for preparation
A appear to be less variable than the data for the other treatments.
The control mean is lower than all three treatment means
(48.4 v 63.5, 67.5, 73.0) so there is prima facie evidence to support the
suggestion.
One could perform a two-sample t-test of control mean = treatment mean
by combining the data for the 3 preparations (and using samples of sizes 8
and 12).
12
(i)
(a)
The probability function for the zero-truncated Poisson distribution is
given by
P ( Y
y | Y
P ( Y
0)
y and Y
P ( Y 0)
y
e
y !(1 P ( Y
y
0))
e
( y 1, 2, ).
y !(1 e )
(b)
Expectation of Y:
y
E [ Y ]
y
y 1
e
y !(1 e )
z
(1 e )
(1 e
Page 6
)
z 0
[1]
e
z !
( z
.
(1 e
)
y 1)
0)Subject CT3 (Probability and Mathematical Statistics Core Technical)
(ii)
(a)
The log likelihood function for
April 2005
Examiners Report
is:
n
log L ( )
y i log
n
n log(1 e )
constant
i 1
d log L ( )
d
ny
n n
e
1 e
the ML estimate is determined by the solution of the equation
d log L ( )
d
0
e
y
0
1 e
As this equation may be rewritten as
y
and E [ Y ]
1 e
1 e
the ML estimate is the same as the method of moments estimate.
(b)
d 2
d
2
ny
log L ( )
n
2
and since E [ Y ] E [ Y ]
e
(1 e ) 2
(1 e )
, the Cramer-Rao lower bound is
given by,
1
CR lb
E
or
d
d
2
log L ( )
(1 e ) 2
n (1 e
1
2
e )
n
1
(1 e )
e
(1 e ) 2
.
Page 7Subject CT3 (Probability and Mathematical Statistics Core Technical)
(iii)
(a)
April 2005
Examiners Report
The expected frequencies for the fitted zero-truncated Poisson model
are given by
y
n
e
( y 1, 2, ) where
0.8925 and n
2423
y !(1 e )
y
e i
f i
1
1500.48
1486
2
669.59
694
3
199.20
195
4
44.45
37
( f i e i ) 2
(1486 1500.48) 2
=
e i
1500.48
(on 4 df).
2
5
7.93
10
6
1.35
1
Total
2423.00
2423
(1 1.35) 2
= 2.99
1.35
The Yellow Book gives that the probability value is greater than 50%,
therefore there is no evidence to reject the null hypothesis, i.e. the
model seems appropriate for the data.
[OR
(b)
2
= 2.68 on 3 df if
5 combined rather than 6.]
As approx. ~ N ( , CR lb) for large n, a 95% confidence interval for
is given by
1.96 CR lb
0.8925 1.96 5.711574 10
at
4
, since CR lb
5.711574
= 0.8925,
= 0.8925 1.96(0.0238989) = 0.8925
= (0.84566, 0.93934)
0.04684
Then the 95% confidence interval for the mean of Y ,
by
0.84566
1 e
Page 8
10 -4
0.84566
,
0.93934
1 e
0.93934
= (1.48, 1.54).
1
, is given
1 eSubject CT3 (Probability and Mathematical Statistics Core Technical)
13
(i)
April 2005
S mm = 129853.03 (1136.1) 2 /10 = 780.709
S ss = 377700.62 (1934.2) 2 /10 = 3587.656
S ms = 221022.58 (1136.1)(1934.2)/10 = 1278.118
r 1278.118
(780.709)(3587.656)
H 0 : = 0 v. H 1 :
t
r 8
1 r 2
3.35
Examiners Report
0.764
> 0
Prob-value = P(t 8 > 3.35) = 0.005 from tables.
[OR use Fisher s transformation]
(ii)
Given the issue of whether mortality can be used to predict sickness, we
require a plot of sickness against mortality:
There seems to be an increasing linear relationship such that mortality could
be used to predict sickness.
Page 9Subject CT3 (Probability and Mathematical Statistics Core Technical)
1278.118
1.6371 and
780.709
(iii)
2
1
[1934.2
10
April 2005
Examiners Report
(1136.1)] 7.426
1
(1278.118) 2
{3587.656
} 186.902
8
780.709
2
Var [ ]
Test H 0 :
t
0.2394
780.709
= 2 v. H 1 :
1.6371 2
0.2394
< 2
0.74 on 8 df
Prob-value large; no evidence to reject H 0 : = 2
So we can accept that the slope is as large as 2.
(iv)
For a region with m = 115:
estimated expected s = 7.426 + 1.6371(115) = 195.69
with variance =
2
1
{
10
(115 113.61) 2
} 19.1528
780.709
95% confidence limits are:
195.69
t 8 (s.e.)
195.69
2.306(4.376)
195.69
10.09 or (185.60, 205.78)
END OF EXAMINERS REPORT
Page 10
