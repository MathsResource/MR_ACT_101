
\documentclass[a4paper,12pt]{article}

%%%%%%%%%%%%%%%%%%%%%%%%%%%%%%%%%%%%%%%%%%%%%%%%%%%%%%%%%%%%%%%%%%%%%%%%%%%%%%%%%%%%%%%%%%%%%%%%%%%%%%%%%%%%%%%%%%%%%%%%%%%%%%%%%%%%%%%%%%%%%%%%%%%%%%%%%%%%%%%%%%%%%%%%%%%%%%%%%%%%%%%%%%%%%%%%%%%%%%%%%%%%%%%%%%%%%%%%%%%%%%%%%%%%%%%%%%%%%%%%%%%%%%%%%%%%

\usepackage{eurosym}
\usepackage{vmargin}
\usepackage{amsmath}
\usepackage{graphics}
\usepackage{epsfig}
\usepackage{enumerate}
\usepackage{multicol}
\usepackage{subfigure}
\usepackage{fancyhdr}
\usepackage{listings}
\usepackage{framed}
\usepackage{graphicx}
\usepackage{amsmath}
\usepackage{chngpage}

%\usepackage{bigints}
\usepackage{vmargin}

% left top textwidth textheight headheight

% headsep footheight footskip

\setmargins{2.0cm}{2.5cm}{16 cm}{22cm}{0.5cm}{0cm}{1cm}{1cm}

\renewcommand{\baselinestretch}{1.3}

\setcounter{MaxMatrixCols}{10}

\begin{document}
\begin{enumerate}
11
An actuary has been advised to use the following positively-skewed claim size
distribution as a model for a particular type of claim, with claim sizes measured in
units of £100,
f ( x ; )
x 2
2
3
exp
x
: 0
x
,
with moments given by E[X] = 3 , E[X 2 ] = 12
0
2
and E[X 3 ] = 60 3 .
(i) Determine the variance of this distribution and calculate the coefficient of
skewness.
[4]
(ii) Let X 1 , X 2 ,
, X n be a random sample of n claim sizes for such claims.
Show that the maximum likelihood estimator (MLE) of
is given by
and show that it is unbiased for .
(iii)
A sample of n = 50 claim sizes yields x i
X
3
[5]
313.6 and x i 2
2, 675.68 .
(a) Calculate the MLE
.
(b) Calculate the sample variance and comment briefly on its comparison
with the variance of the distribution evaluated at .
(c) Given that the sample coefficient of skewness is 1.149, comment
briefly on its comparison with the coefficient of skewness of the
distribution.
[4]
(iv)
12
(a) Write down a large-sample approximate 95% confidence interval for
the mean of the distribution in terms of the sample mean x and the
sample variance s 2 . Hence obtain an approximate 95% confidence
interval for and evaluate this for the data in part (iii) above.
(b) Evaluate the variance of the distribution at both the lower and upper
limits of this confidence interval and comment briefly with reference to
your answer in part (iii)(b) above.
[5]
[Total 18]
In an experiment to compare the effects of vaccines of differing strengths intended to
give protection to children against a particular condition, twelve batches of vaccine
were tested in twelve equal-sized groups of children. The percentages of children
who subsequently remained healthy after exposure to the condition, named the PRH
values, were recorded. The strength of each batch of vaccine was measured by an
independent test and recorded as the SV value.
CT3 A2006 6The recorded values are:
Batch:
1
PRH (y): 16
SV (x):
0.9
x 38.4 ;
(i)
2
3
4
68 23 35
1.6 2.3 2.7
y
528 ;
5
6
7
8
9
42 41 46 48 52
3.0 3.3 3.7 3.8 4.1
x 2 137.16 ;
y 2
10 11 12
50 54 53
4.2 4.3 4.5
25, 428 ;
xy 1, 778.4
Draw a rough plot of the data to show the relationship between the SV and
PRH values.
[2]
It is evident that one of the observations is out of line and so may have an undue
effect on any regression analysis. You are asked to investigate this as follows.
(ii)
(a) Calculate the total, regression, and error sums of squares for a least-
squares linear regression analysis for predicting PRH values from SV
values using all 12 data observations.
(b) Determine the coefficient of determination R 2 .
(c) Determine the equation of the fitted regression line.
(d) Examine whether or not there is evidence, at the 5% level of testing, to
enable one to conclude that the slope of the underlying regression
equation is non-zero.
[11]
The details of the regression analysis after removing the data for batch 2 are given in
the box below.
Regression equation: y = 3.76 + 11.4 x
Coef
Stdev
t-ratio p-val
Intercept
x 0.255
0.000
3.757
11.377
Analysis of Variance
Source
df
SS
Regression 1
Error
9
Total
10
(iii)
1486.9
80.8
1567.6
3.092
0.8838
MS
1486.9
8.98
1.22
12.9
F p-val
165.69 0.000
(a) Comment on the main differences in the results of the regression
analysis resulting from removing the data for batch 2.
(b) Calculate a 95% confidence interval for the expected (mean) PRH
value for a batch of vaccine with SV value 3.5.
[9]
[Total 22]
END OF PAPER
CT3 A2006 7

%%%%%%%%%%%%%%%%%%%%%%%%%%%%%%%%%%%%%%%%%%%%%%%%%%%%%%%%%%%%%%%%%%%%%%%%%%%%%%%%%%%%%%%%%%%%%%%%%
Page 8Subject CT3 (Probability and Mathematical Statistics Core Technical)
11
(i)
2
= E[X 2 ]
[E(X)] 2 = 12
2
= E[X 3 ] 3 E[X 2 ] + 2
3
= 60
(3 ) 2 = 3
April 2006
Examiners Report
2
3
3(3 )(12 2 ) + 2(3 ) 3
= (60
108 + 54)
3
3
= 6
coefficient of skewness =
3
3
3
6
1.155
2 3
( 3
)
[OR: note that X ~ gamma(3,1/ ) and use formulae in tables
2
so var = 3 2 and coef. of skew. =
]
3
n
(ii)
L ( )
i 1
x i 2
2
exp(
3
x i
x i 2
)
2
n 3 n
log L ( ) log( x i 2 ) n log 2 3 n log
x i
x i
3 n
log L ( )
x i
exp
2
equate to zero:
x i
3 n
x i
3 n
2
2
this clearly maximises L( )
X i
3 n
So MLE is
(iii)
E 1
E X
3
(a) x
(b) s 2
2
2
log L ( ) ]
X
3
1
E X
3
313.6
50
[or consider
6.272
1
3
3
unbiased
6.272
3
2.091
1
313.6 2
(2675.68
) 14.465
49
50
3
2
and 3
2
13.117
s 2 is a bit larger but still quite close
Page 9Subject CT3 (Probability and Mathematical Statistics Core Technical)
(iv)
April 2006
Examiners Report
(c) sample coefficient 1.149 is very close to the distribution value 1.155
(a) s 2
is x 1.96
n
approximate 95% CI for
as
= 3 , divide by 3 for an approximate 95% CI for
1
s 2
x 1.96
3
n
for data:
1
14.465
6.272 1.96
3
50
2.091 0.351
(b)
2
= 3
2
or
(1.740, 2.442)
= 9.083 at lower limit of 1.740
= 17.890 at upper limit of 2.442
s 2 = 14.465 is well within these values
confirming that s 2 is quite close to 3 2 .
%%%%%%%%%%%%%%%%%%%%%%%%%%%%%%%%%%%%%%%%%%%%%%%%%%%%%%%%%%%%%%%%%%%%%%%%%%%%%%%%%%%%%%%%%%%%%%%%%%%%
12
(i)
70
60
50
40
30
20
1
2
3
4
SV
(ii)
Page 10
(a)
SSTOT = S yy = 25428
528 2 /12 = 2196
S xx = 137.16 38.4 2 /12 = 14.28 , S xy = 1778.4 (38.4 528)/12 = 88.8
SSREG = 88.8 2 /14.28 = 552.20, SSRES = 2196 552.20 = 1643.80Subject CT3 (Probability and Mathematical Statistics Core Technical)
(b) R 2 = 552.2/2196 = 0.251 (25.1%)
(c) y = a + bx:
April 2006
b 88.8 /14.28 6.2185
a 528 /12 88.8 /14.28 (38.4 /12)
Examiners Report
24.101
Fitted line is y = 24.101 + 6.2185x
(d)
s . e . b
1643.8 /10
14.28
1/ 2
3.3928
Observed t = (6.2185 0)/3.3928 = 1.833 < t 10 (0.025) = 2.228
so we do not have evidence at the 5% level of testing to justify
rejecting b = 0 and concluding that the underlying slope is non-zero.
(iii)
(a)
Large change in slope (and intercept) of fitted line.
The total and error sums of squares are much reduced.
The fit of the linear regression model is much improved (R 2 is much
increased from 25.1% to 94.9%).
We have overwhelming evidence to justify concluding that the slope is
non-zero.
(b)
Fitted PRH value at SV = 3.5 is 3.757 + (11.377 3.5) = 43.577
n 11,
x 38.4 1.6 36.8,
x 2 137.16 1.6 2 134.6
S xx = 11.4873
s.e. of estimation =
1
11
3.5 36.8 /11
11.4873
1/ 2
2
8.98
0.9138
t 9 (0.025) = 2.262
95% CI for expected PRH is 43.577
i.e. 43.577
2.067
(2.262
0.9138)
or (41.51, 45.64)
%%%%%%%%%%%%%%%%%%%%%%%%%%%%%%%%%%%%%%%%%%%%%%%%%%%%%%%%%%%%%%%%%%%%%%%%%%%%%%%%%%%%%%%%%%%%%%5555


\end{document}
