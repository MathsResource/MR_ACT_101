\documentclass[a4paper,12pt]{article}

%%%%%%%%%%%%%%%%%%%%%%%%%%%%%%%%%%%%%%%%%%%%%%%%%%%%%%%%%%%%%%%%%%%%%%%%%%%%%%%%%%%%%%%%%%%%%%%%%%%%%%%%%%%%%%%%%%%%%%%%%%%%%%%%%%%%%%%%%%%%%%%%%%%%%%%%%%%%%%%%%%%%%%%%%%%%%%%%%%%%%%%%%%%%%%%%%%%%%%%%%%%%%%%%%%%%%%%%%%%%%%%%%%%%%%%%%%%%%%%%%%%%%%%%%%%%

\usepackage{eurosym}
\usepackage{vmargin}
\usepackage{amsmath}
\usepackage{graphics}
\usepackage{epsfig}
\usepackage{enumerate}
\usepackage{multicol}
\usepackage{subfigure}
\usepackage{fancyhdr}
\usepackage{listings}
\usepackage{framed}
\usepackage{graphicx}
\usepackage{amsmath}
\usepackage{chngpage}

%\usepackage{bigints}
\usepackage{vmargin}

% left top textwidth textheight headheight

% headsep footheight footskip

\setmargins{2.0cm}{2.5cm}{16 cm}{22cm}{0.5cm}{0cm}{1cm}{1cm}

\renewcommand{\baselinestretch}{1.3}

\setcounter{MaxMatrixCols}{10}

\begin{document}


%%%%%%%%%%%%%%%%%%%%%%%%%%%%%%%%%%%%%%%%%%%%%%%%%%%%%%%
13 Suppose that the distribution of a physical coefficient, X, can be modelled
using a uniform distribution on (0, 1). A researcher is interested in the
distribution of Y, an adjusted form of the reciprocal of the coefficient, where Y
= (1 / X) - 1.
(i) Show that the probability density function of Y is given by
fY(y) = 1/(1 + y)2 , y > 0 
(ii) Show that the mean of Y does not exist. 
[Total 5]
%%%%%%%%%%%%%%%%%%%%%%%%%%%%%%%%%%%%%%%%%%%%%%%%%%%%%%%
14 An insurance company issues house buildings policies for houses of similar
size in four different post-code regions A, B, C and D.
(i) An insurance agent takes independent random samples of 10 house
buildings policies for houses of similar size in regions A and B. The
annual premiums (£) were as follows:
Region A: 229 241 270 256 241 247 261 243 272 219
(Sx = 2,479 ; Sx2 = 617,163)
Region B: 261 269 284 268 249 255 237 270 269 257
(Sx = 2,619 ; Sx2 = 687,467)
\begin{enumerate}[(a)]
\item Perform a two-sample t-test at the 5% level to compare the
premiums for these two regions.
\item Present the data in a simple diagram and hence comment briefly
on the validity of the assumptions required for the above t-test.
\item Calculate a 95% confidence interval for the underlying common
standard deviation s of such premiums. 
\end{enumerate}
%------------------------%
(ii) The agent takes further independent random samples of 10 such
policies from the other two regions C and D. The annual premiums
were as follows:
Region C: 253 247 244 245 221 229 245 256 232 269
(Sx = 2,441 ; Sx2 = 597,607)
Region D: 279 268 290 245 281 262 287 257 262 246
(Sx = 2,677 ; Sx2 = 718,973)
\item Perform a one-way analysis of variance at the 5% level to
compare the premiums for all four regions.
\item Present the new data in a simple diagram and hence comment
briefly on the validity of the assumptions required for the
analysis of variance.
\item Calculate a 95% confidence interval for the underlying common
standard deviation s of such premiums in the four regions. [10]
(iii) Comment briefly on your two confidence intervals in (i)\item and (ii)\item
above. [1]
%%%%%%%%%%%%%%%%%%%%%%%%%%%%%%%%%%%%%%%%%%%%%%%%%%%%%%%
15 An engineer is interested in estimating the probability that a particular
electrical component will last at least 12 hours before failing. In order to do
this, a random sample of n components is tested to destruction and their
failure times, , ,..., , 1 2 n x x x are recorded. The engineer models failure times by
assuming that they come from a distribution with distribution function, F, and
probability density function, f, given below.
F(x) = 1 - 1
(1 ) 1
,
+ x a- f(x) = a
a
-
+
1
(1 x)
, a > 1, x > 0.
(i) Determine aˆ , the maximum likelihood estimator of a, and, assuming n
is large, use asymptotic theory to show that an approximate 95%
confidence interval for a is given by
ˆ 1 ˆ 1.96
n
a -
a ± . [8]
(ii) A sample of size n = 80 leads to a maximum likelihood estimate of a of
1.56. Use this figure to
%------------------------%
\begin{enumerate}[(a)]
\item estimate the probability a component will fail before 12 hours,
\item determine an approximate upper 95% one-sided confidence
interval for a, and
\item hence determine an approximate 95% one-sided confidence
interval which provides an upper bound for the probability in
part (ii) above. 
\end{enumerate}
%------------------------%
(iii) Sixty-one of the eighty components tested in part (ii) failed before 12
hours, so a second engineer estimates the failure probability by
61/80 = 0.7625, and constructs an upper 95% confidence interval based
on the binomial distribution.
\item Construct this interval, and
\item comment on the advantages and disadvantages of this method
when compared to the method of part (ii). 
%%%%%%%%%%%%%%%%%%%%%%%%%%%%%%%%%%%%%%%%%%%%%%%%%%%%%%%
16 The table below contains measurements on the strengths of beams. The width
and height of each beam was fixed but the lengths varied. Data are available
on the length (cm) and strength (Newtons) of each beam.
Length l x = log l Strength
p
y = log p Fitted value Residual
7 1.946 11775 9.374 9.379 -0.005
7 1.946 11275 9.330 9.379 -0.049
9 2.197 8400 9.036 9.055 -0.019
9 2.197 8200 9.012 9.055 -0.043
12 2.485 6100 8.716 8.684 0.032
12 2.485 6050 8.708 8.684 0.024
14 2.639 5200 8.556 8.486 0.070
18 2.890 3750 8.230 8.162 0.068
18 2.890 3650 8.202 8.162 0.040
20 2.996 3275 8.094 8.026 0.068
20 2.996 3175 8.063 8.026 0.037
24 3.178 2200 7.696 7.791 -0.095
24 3.178 2125 7.662 7.791 -0.129
Sx = 34.023, Sx2 = 91.3978, Sy = 110.679, Sxy = 286.6299
It is thought that P and L satisfy the law P = k/L where k is a constant, so
log P = log k - log L, i.e. Y = log k - X.
A graph of log P against log L is included.
The simple linear regression model y = a + bx has been fitted to the data, and
the fitted values and residuals are recorded in the table above.
(i) Use the data summaries above to calculate the least squares estimates
$ a of a and ˆb of b . 
(ii) Assuming the usual normal linear regression model
\begin{emumerate}
\item estimate the error variance s^2,
\item calculate a 95% confidence interval for b, and
\item discuss briefly whether the data are consistent with the
relationship P = k L . 
\end{enumerate}

(iii) Plot the residuals of the model against X and comment on the
information contained in the plot. 
[Total 13]
7.5
7.7
7.9
8.1
8.3
8.5
8.7
8.9
9.1
9.3
9.5
1.9 2.1 2.3 2.5 2.7 2.9 3.1 3.3
x = log l
y = log p
%%%%%%%%%%%%%%%%%%%%%%%%%%%%%%%%%%%%%%%%%%%%%%%%%%%%%%%
\newpage

%%%%%%%%%%%%%%%%%%%%%%%%%%%%%%%%%%%%%%%%%%%%%%%%%%%%%%%%%%%%%%%%%%%%%
Page 5
13 (i) FY(y) = P(Y < y) = P(1/X < y + 1) = P[X > 1/(y + 1)] = 1  1/(y + 1)
so fY(y) = dFY(y)/dy = 1/(y+1)2 , y > 0
(ii) E(Y) = 2 1 2
0 0 0
y(1 y) dy (1 y) dy (1 y) dy
  
  
       
The integral of (1 + y)1 gives log(1 + y) which 
  as y 
  so this
integral is not finite. So E(Y) does not exist.
14 (i) (a) A
x = 247.9, 2
A
s =
2 1 2479
617163
9 10
 
  
 
= 290.99
B
x = 261.9, 2
B
s =
2 1 2619
687467
9 10
 
  
 
= 172.32
2
p
s =
290.99 172.32
2

= 231.66
Obs t =
247.9 261.9
1 1
231.66
10 10

 
  
 
= 2.06
For two-sided test, t18(2.5%) = 2.101.
As 2.06 < 2.101, there is no evidence at the 5% level of a difference
between regions A and B.
(b)
Normality — OK in both cases.
Equal variances — OK.
A
B
210 220 230 240 250 260 270 280 290
     
     

%%%%%%%%%%%%%%%%%%%%%%%%%%%%%%%%%%%%%%%%%%%%%%%%%%%%%%%%%%%%%%%%%%%%%
Page 6
(c)
2
2
2 18
18
~ p
S


 95% CI for 2 is
2 2
2 2
0.975,18 0.025,18
18 18
, p p
 S S 
     
=
18(231.66) 18(231.66)
,
31.53 8.231
 
 
 
= (132.25, 506.6)
 95% CI for  is (11.5, 22.5)
(ii) (a) x = 10216, x2 = 2621210
SST = 2621210 
2 10216
40
= 12043.6
SSB =
2 2 2 2 2 2479 2619 2441 2677 10216
10 40
  
 = 3774.8
 SSR = 8268.8 by subtraction.
Source df SS MS
Regions
Residual
3
36
3774.8
8268.8
1258.3
229.7
Total 39 12043.6
F =
1258.3
229.7
= 5.48 on (3,36) df
F3,36(5%) 

2.9 by interpolation
Clearly reject H0 : A = B = C = D at the 5% level
 Strong evidence of a difference between regions A–D.
(b)
[same scale!]
Normality — OK.
Equal variances for A, B, C, D — OK.
 C
D
210 220 230 240 250 260 270 280 290
 
 
 
   
  

%%%%%%%%%%%%%%%%%%%%%%%%%%%%%%%%%%%%%%%%%%%%%%%%%%%%%%%%%%%%%%%%%%%%%
Page 7
(c)
2
2
2 36
36ˆ
~



where 2 36ˆ = SSR
 95% CI for 2 is
2 2
0.975,36 0.025,36
,
R R  SS SS 
     
=
8268.8 8268.8
,
54.4 21.37
 
 
 
= (152.0, 386.9) [interpolate in tables]
 95% CI for  is (12.3, 19.7)
(iii) Second CI is narrower as it is based on more data.



%%%%%%%%%%%%%%%%%%%%%%%%%%%%%%%%

15 (i) Start by writing down the likelihood function
L() =
( 1)
(1 )
n
i x

 
 
.
The log-likelihood function is
l() = log L() = n log( 1)   log(1 + xi).
Differentiating gives
l

=
1
n
 
 log(1 + xi).
It is easy to see that the log-likelihood has only one turning point, so this
can be found by equating the derivative to zero. This gives that the
maximum likelihood estimate is
ˆ
= 1 + .
log(1 ) i
n
  x
The second derivative is
2
2
 l

= 2 .
( 1)
n
 
So an approximate 95% confidence interval for  is
ˆ 1
ˆ 1.96 .
n
 
 
%%%%%%%%%%%%%%%%%%%%%%%%%%%%%%%%%%%%%%%%%%%%%%%%%%%%%%%%%%%%%%%%%%%%%
Page 8
(ii) (a) The probability a component will last less than 12 hours before
failing can be estimated by the point estimate
1  ˆ 1
1
13
= 1  0.56
1
13
= 0.762.
(b) An approximate 95% confidence upper bound for  is
ˆ 1
ˆ 1.645
n
 
  = 1.56 + 1.645
0.56
80
= 1.663.
(c) This gives an upper bound for the failure probability of
0.663
1
1
13
 = 0.817.
(iii) (a) The endpoint of the binomial confidence interval is
0.7625 + 1.645 
0.7625 (1 0.7625)
80
 
= 0.841.
(b) There is no single answer to this part. The main points are:
The second engineer’s (binomial) method will be valid and doesn’t
need any parametric assumptions about the data.
The first engineer’s method needs the data to follow the
distribution specified, but in that case it will be more powerful
than the binomial method, which does not use the data efficiently.
This is illustrated in this data as the two point estimates are very
close, but the first engineer’s confidence interval is narrower.
16 (i) Sxx = 91.3978 
2 34.023
13
= 2.354, and
Sxy = 286.6299 
110.679 34.023
13

= 3.0341.
The least squares estimates are ˆ =
3.0341
2.354

= 1.289 and
ˆ
= ˆ
y  x = 11.887.
(ii) (a) The sum of squares of the residuals is 0.049019, so 2 is estimated
by the residual mean square 0.049019
11 = 0.004456.
(b) The estimated variance of ˆ is 0.004456
2.3544 = 0.00189.
%%%%%%%%%%%%%%%%%%%%%%%%%%%%%%%%%%%%%%%%%%%%%%%%%%%%%%%%%%%%%%%%%%%%%
Page 9
This leads to the following 95% confidence interval for :
1.289  2.201 0.00189 = (1.384, 1.193).
(c) If the relationship P = k/L is correct the slope parameter of the
regression line should be –1. As the upper end of the interval is
less than –1, the data do not support the suggested relationship.
(iii)
The residuals show that the line underfits in the centre. A straight line
doesn’t fit the data very well.
-0.15
-0.10
-0.05
0.00
0.05
0.10
1.9 2.1 2.3 2.5 2.7 2.9 3.1 3.3
log L
