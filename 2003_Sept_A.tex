\documentclass[a4paper,12pt]{article}

%%%%%%%%%%%%%%%%%%%%%%%%%%%%%%%%%%%%%%%%%%%%%%%%%%%%%%%%%%%%%%%%%%%%%%%%%%%%%%%%%%%%%%%%%%%%%%%%%%%%%%%%%%%%%%%%%%%%%%%%%%%%%%%%%%%%%%%%%%%%%%%%%%%%%%%%%%%%%%%%%%%%%%%%%%%%%%%%%%%%%%%%%%%%%%%%%%%%%%%%%%%%%%%%%%%%%%%%%%%%%%%%%%%%%%%%%%%%%%%%%%%%%%%%%%%%

\usepackage{eurosym}
\usepackage{vmargin}
\usepackage{amsmath}
\usepackage{graphics}
\usepackage{epsfig}
\usepackage{enumerate}
\usepackage{multicol}
\usepackage{subfigure}
\usepackage{fancyhdr}
\usepackage{listings}
\usepackage{framed}
\usepackage{graphicx}
\usepackage{amsmath}
\usepackage{chngpage}

%\usepackage{bigints}
\usepackage{vmargin}

% left top textwidth textheight headheight

% headsep footheight footskip

\setmargins{2.0cm}{2.5cm}{16 cm}{22cm}{0.5cm}{0cm}{1cm}{1cm}

\renewcommand{\baselinestretch}{1.3}

\setcounter{MaxMatrixCols}{10}

\begin{document}

\begin{enumerate}
\item

%%%%%%%%%%%%%%%%%%%%%%%%%%%%%%%%%%%%%%%%%%%%%%%%%%%%%%%%%%%%%%%%%%%%%%%%%%%%%%%%%%%%%%%%%%%%%%%%%%
1 In a large corporation 50 new employees joined the company’s pension scheme
during the last year. It is assumed that each new employee has a probability of 0.40
of remaining in the scheme for at least 10 years, independently for each new
employee.
Calculate an approximate value for the probability that more than half of last year’s
50 new employees remain in the pension scheme for at least 10 years. 
\item 2 Let (X1, X2, … , X9) be a random sample from a N(0,2) distribution. Let X and
S2 denote the sample mean and variance respectively.
Find the approximate value of P X  S  by referring to an appropriate statistical
table. 
%%%%%%%%%%%%%%%%%%%%%%%%%%%%%%%%%%%%%%%%%%%%%%%%%%%%%%%%%%%%%%%%%%%%%%%%%%%%%%%%%%%%%%%%%%%%%%%%%%
\item 3 A random sample of size 200 is taken from a large group of motor policies. The
proportion of policies on which claims arose during the last year is ˆ  0.16 .
Calculate approximate 95% confidence limits for the true proportion  for the whole
group of policies. 
\item 4 A random sample of 16 values, x1, x2,  , x16, was drawn from a normal population
and gave the following summary statistics:
16 16
2
1 1
i 51.2, i 243.19.
i i
x x
 
   
Calculate a 95% confidence interval for the population mean. 
%%%%%%%%%%%%%%%%%%%%%%%%%%%%%%%%%%%%%%%%%%%%%%%%%%%%%%%%%%%%%%%%%%%%%%%%%%%%%%%%%%%%%%%%%%%%%%%%%%

\end{enumerate}

%%%%%%%%%%%%%%%%%%%%%%%%%%%%%%%%%%%%%%%%%%%%%%%%%%%%%%%%%%%%%%%%%%%%%%%%%%
Page 3

1 Let X be the number remaining in the scheme for at least 10 years.
X ~ binomial(50, 0.4)
So, approximately X ~ N(20, 12)
We require P(X > 25).
Using a continuity correction,
P(X > 25.5) �� ( 25.5 20) ( 1.59) 1 0.944 0.056
12
P Z P Z −
> = > = − =
2 ( ) ( 8 )
P X S P 3X 3 P t 3
S
⎛ ⎞
> = ⎜ > ⎟ = >
⎝ ⎠
which is between 0.005 and 0.01.
3 ˆ(1 ˆ) 1.96
200
θ −θ
± where ˆθ = 0.16
1.96 (0.16)(0.84) 1.96(0.026) 0.051
200
⇒ ± ⇒ ± ⇒ ±
or, as an interval: 0.16 ± 0.051⇒(0.109,0.211)
4 16 16 2
1 1 16, i i 51.2, i i 243.19; n x x = = = Σ = Σ =
2
2
243.19 51.2 3.20, 16 79.35 5.29.
15 15
x s
−
= = = =
s = 5.29 = 2.3
%%%%%%%%%%%%%%%%%%%%%%%%%%%%%%%%%%%%%%%%%%%%%%%%%%%%%%%%%%%%%%%%%%%%%%%%%%
Page 4
The 95% confidence interval is given by:
0.025, 1
3.2 2.131 2.3 3.2 1.23
n 4
x t s
n ± − = ± = ±
i.e. (1.97, 4.43)
\end{document}
